\documentclass[../main.tex]{subfiles}

\begin{document}
\setcounter{chapter}{35}
\chapter{Fonctions de deux variables}
\tableofcontents
\clearpage

\setsection{14}
\section{Exemple}
\begin{tcolorbox}[title=Exemple 36.15, title filled=false, colframe=darkgreen, colback=darkgreen!10!white]
    Les projections $(x, y)\mapsto \sqrt{x^2 + y^2}$ est continue sur $\mathbb{R}^2$. 
\end{tcolorbox}

\noindent Soit $a = (x_0, y_0)\in \mathbb{R}^2$. On note : 
\begin{align*}
    p_1:&\mathbb{R}^2\to \mathbb{R} \\
    &(x, y)\mapsto x
\end{align*}
Soit $\epsilon > 0$, pour tout $(x, y)\in B(a, \epsilon)$. 
\begin{align*}
    |p_1(x, y) - p_1(x_0, y_0)| = |x - x_0| \leq \| (x, y) - (x_0, y_0)\|
\end{align*}
Donc $p_1$ est bien continue en $a$, donc sur $\mathbb{R}^2$. 

\setsection{16}
\section{Exemple}
\begin{tcolorbox}[title=Exemple 36.17, title filled=false, colframe=darkgreen, colback=darkgreen!10!white]
    Soi t$I$ et $J$ deux intervalles et $f\in \mathcal{C}(I, \mathbb{R})$ et $g\in \mathcal{C}(J, \mathbb{R})$. Alors : 
    \begin{align*}
        I\times J\to \mathbb{R}^2; (x, y)\mapsto f(x) + g(y) \quad \text{et} \quad I\times J\to \mathbb{R}^2; (x, y)\mapsto f(x)g(y)
    \end{align*}
    sont continues sur $I\times J$. 
\end{tcolorbox}

\begin{itemize}
    \item Soit $(x_0, y_0)\in I\times J$. \\
    Soit $\epsilon > 0$. \\
    Par continuité de $f$ et $g$ on choisit $\alpha > 0$ tel que : 
    \begin{itemize}
        \item $\forall x\in I, |x - x_0| < \alpha \Rightarrow |f(x) - f(x_0)| < \epsilon$
        \item $\forall y\in J, |y - y_0| < \alpha \Rightarrow |g(y) - g(y_0)| < \epsilon$
    \end{itemize}
    Soit $(x, y)\in B((x_0, y_0), \alpha)$. 
    \begin{align*}
        |f(x) + g(y) - f(x_0) - g(y_0)| &\leq |f(x) - f(x_0)| + |g(y) - g(y_0)|
    \end{align*}
    \item De la même manière (voir le produit de fonctions de $\mathcal{C}^0$, chap. 15). 
\end{itemize}


\end{document}