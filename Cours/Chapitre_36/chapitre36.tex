\documentclass[../main.tex]{subfiles}

\begin{document}
\setcounter{chapter}{35}
\chapter{Fonctions de deux variables}
\tableofcontents
\clearpage

\setsection{14}
\section{Exemple}
\begin{tcolorbox}[title=Exemple 36.15, title filled=false, colframe=darkgreen, colback=darkgreen!10!white]
    Les projections $(x, y)\mapsto \sqrt{x^2 + y^2}$ est continue sur $\mathbb{R}^2$. 
\end{tcolorbox}

\noindent Soit $a = (x_0, y_0)\in \mathbb{R}^2$. On note : 
\begin{align*}
    p_1:&\mathbb{R}^2\to \mathbb{R} \\
    &(x, y)\mapsto x
\end{align*}
Soit $\epsilon > 0$, pour tout $(x, y)\in B(a, \epsilon)$. 
\begin{align*}
    |p_1(x, y) - p_1(x_0, y_0)| = |x - x_0| \leq \| (x, y) - (x_0, y_0)\|
\end{align*}
Donc $p_1$ est bien continue en $a$, donc sur $\mathbb{R}^2$. 

\setsection{16}
\section{Exemple}
\begin{tcolorbox}[title=Exemple 36.17, title filled=false, colframe=darkgreen, colback=darkgreen!10!white]
    Soi t$I$ et $J$ deux intervalles et $f\in \mathcal{C}(I, \mathbb{R})$ et $g\in \mathcal{C}(J, \mathbb{R})$. Alors : 
    \begin{align*}
        I\times J\to \mathbb{R}^2; (x, y)\mapsto f(x) + g(y) \quad \text{et} \quad I\times J\to \mathbb{R}^2; (x, y)\mapsto f(x)g(y)
    \end{align*}
    sont continues sur $I\times J$. 
\end{tcolorbox}

\begin{itemize}
    \item Soit $(x_0, y_0)\in I\times J$. \\
    Soit $\epsilon > 0$. \\
    Par continuité de $f$ et $g$ on choisit $\alpha > 0$ tel que : 
    \begin{itemize}
        \item $\forall x\in I, |x - x_0| < \alpha \Rightarrow |f(x) - f(x_0)| < \epsilon$
        \item $\forall y\in J, |y - y_0| < \alpha \Rightarrow |g(y) - g(y_0)| < \epsilon$
    \end{itemize}
    Soit $(x, y)\in B((x_0, y_0), \alpha)$. 
    \begin{align*}
        |f(x) + g(y) - f(x_0) - g(y_0)| &\leq |f(x) - f(x_0)| + |g(y) - g(y_0)|
    \end{align*}
    \item De la même manière (voir le produit de fonctions de $\mathcal{C}^0$, chap. 15). 
\end{itemize}

\setsection{23}
\section{Exemple}
\begin{tcolorbox}[title=Exemple 36.24, title filled=false, colframe=darkgreen, colback=darkgreen!10!white]
    Quelles sont les dérivées partielles de la fonction $f:(x, y)\mapsto e^{xy^2}$ ?
\end{tcolorbox}

\noindent Sous réserve d'existence : 
\begin{align*}
    \forall (x, y)\in \mathbb{R}^2, \partial_1f(x, y) &= y^2 e^{xy^2} \\
    \partial_2f(x, y) &= 2xy e^{xy^2}
\end{align*}
Démonstration de la première : \\
Soit ($x, y)\in \mathbb{R}^2$. 
\begin{align*}
    \forall t\in \mathbb{R}, f((x, y) + t(1, 0)) &= f(x + t, y) \\
    &= e^{(x + t)y^2} \\
    &= e^{xy^2 + ty^2} \\
    &= g(t)
\end{align*}
Donc $g\in \mathcal{D}'(\mathbb{R}, \mathbb{R})$ et : 
\begin{align*}
    \forall t\in \mathbb{R}, g'(t) &= y^2 e^{xy^2 + ty^2} \\
    g'(0) &= y^2 e^{xy^2}
\end{align*}

\setsection{34}
\section{Exemple}
\begin{tcolorbox}[title=Exemple 36.35, title filled=false, colframe=darkgreen, colback=darkgreen!10!white]
    La fonction $(x, y)\mapsto e^{-x}\ln y$ est de classe $\mathcal{C}^1$ sur $\mathbb{R} \times \mathbb{R}_+^*$. 
\end{tcolorbox}

\begin{align*}
    \forall (x, y)\in \mathbb{R}^2, \partial_1f(x, y) &\mapsto -e^{-x}\ln y \in \mathcal{C}^0(\Omega, \mathbb{R}) \\
    \partial_2f(x, y) &\mapsto \frac{e^{-x}}{y} \in \mathcal{C}^0(\Omega, \mathbb{R})
\end{align*}
Par opérations. 

\setsection{37}
\section{Existence de développement limité à l'ordre 1 pour les fonctions de classe $\mathcal{C}^1$}
\begin{tcolorbox}[title=Théorème 36.38, title filled=false, colframe=orange, colback=orange!10!white]
    Soit $\Omega$ un ouvert de $\mathbb{R}^2$ et $f \in \mathcal{C}^1(\Omega, \mathbb{R})$. Soit $a \in \Omega$. La fonction $f$ possède un développement limité à l'ordre 1 au voisinage de $a$ :
    $$f(a+v) \underset{v \rightarrow 0}{=} f(a)+\langle\nabla f(a), v\rangle+\mathrm{o}(\|v\|)$$
    ce qui est équivalent à :
    $$f(x) \underset{x \rightarrow a}{=} f(a)+\langle\nabla f(a), x-a\rangle+\mathrm{o}(\|x-a\|)$$
    Si les variables de $f$ sont notées $x$ et $y$ et si $a=\left(x_a, y_a\right)$, cela revient à :
    $$f\left(x_a+h, y_a+k\right) \underset{(h, k) \rightarrow(0,0)}{=} f(a)+\partial_1 f(a) \cdot h+\partial_2 f(a) \cdot k+\mathrm{o}(\|(h, k)\|)$$
\end{tcolorbox}

\noindent Par translation, on le montre pour $a = 0$. \\
On pose $v = (h, k)\in \mathbb{R}^2$. \\
Par hypothèse, $f\in \mathcal{C}^1(\Omega, \mathbb{R})$, $\partial_1f$ et $\partial_2f$ sont continues. \\
Siot $\epsilon > 0$, il existe $r > 0$ tel que :
\begin{align*}
    \forall v\in B(0, r), &|\partial_1f(v) - \partial_1f(0)| < \epsilon \\
    &|\partial_2f(v) - \partial_2f(0)| < \epsilon
\end{align*}
On a : 
\begin{align*}
    |f(v) - f(0) - \langle \nabla f(0), v\rangle| &= |f(v) - f(0) - \partial_1f(0)h - \partial_2f(0)k| \\
    &= |f(h, k) - f(h, 0) + f(h, 0) - f(0, 0) - \partial_1f(0)h - \partial_2f(0)k| \\
    &= \left| \int_{0}^{k} \partial_2 f(h, t) \,dt - \partial_2 f(0) k + \int_{0}^{h} \partial_{1} f(t, 0) \,dt - \partial_1 f(0) h \right| \\
    \text{(Inégalité triangulaire x3)}\quad &\leq \int_{0}^{k} |\partial_2 f(h, t) - \partial_2 f(0)| \,dt+ \int_{0}^{h} |\partial_{1} f(t, 0) - \partial_1 f(0)| \,dt
\end{align*}
Pour $v\in B(0, r)$ : 
\begin{align*}
    \forall t\in [0, k], (0, t) &\in B(0, r) \\
    \forall t\in [0, h], (h, t) &\in B(0, r)
\end{align*}
Ainsi : 
\begin{align*}
    |f(v) - f(0) - \langle \nabla f(0), v\rangle| &\leq \epsilon \times h + \epsilon \times k \\
    &\leq \epsilon |h| + \epsilon |k| \\
    &\leq 2\epsilon \|v\|
\end{align*}
En conlusion, $f(v) - f(0) - \langle \nabla f(0), v\rangle = o(\|v\|)$. 

\setsection{41}
\section{Règle de la chaîne}
\begin{tcolorbox}[title=Théorème 36.42, title filled=false, colframe=orange, colback=orange!10!white]
    Soit $\Omega$ un ouvert de $\mathbb{R}^2$, $I$ un intervalle et $f\in \mathcal{C}^1(\Omega, \mathbb{R})$. Soit $x$ et $y$ dans $\mathcal{C}^1(I, \mathbb{R})$ deux fonctions pour lesquelles pour tout $t\in I$, $(x(t), y(t))\in \Omega$. \\
    La fonction $F:t f(x(t), y(t))$ est de classe $\mathcal{C}^1(I, \mathbb{R})$ sur $I$ et pour tout $t\in I$ : 
    \begin{align*}
        F'(t) &= x'(t) \partial_1 f(x(t), y(t)) + y'(t) \partial_2 f(x(t), y(t))
    \end{align*}
\end{tcolorbox}

\noindent $f\in \mathcal{C}^1(\Omega, \mathbb{R})$ donc possède un $\operatorname{DL}_1(t)$ en tout point. \\
Soit $t\in I$ et $h\in \mathbb{R}$ : 
\begin{align*}
    F(t + h) &= f(x((t + h), y(t + h))) \\
    &= \underbrace{f(x(t + h), y(t + h)) - (x(t), y(t))}_{\underset{h \to +\infty}{\longrightarrow} 0} + (x(t), y(t)) \\
    \text{(36.36) } &= f(x(t), y(t)) + \langle \nabla f(x(t), y(t)), (x(t + h), y(t + h)) - (x(t), y(t))\rangle + o(\|(x(t + h) - x(t), y(t + h) - y(t))\|)
\end{align*}
On a : 
\begin{align*}
    \begin{cases}
        x(t + h) - x(t) &\underset{h\to 0}{=} h x'(t) + o(h) \underset{h\to 0}{=} O(h) \\
        y(t + h) - y(t) &\underset{h\to 0}{=} h y'(t) + o(h) \underset{h\to 0}{=} O(h)
    \end{cases}
\end{align*}
Donc : 
\begin{align*}
    o(\| (x(t + h) - x(t), y(t + h) - y(t) \|) &\underset{h\to 0}{=} o(\|O(h), O(h)\|) \\
    &\underset{h\to 0}{=} o(h)
\end{align*}
Et : 
\begin{align*}
    \langle \nabla f(x(t), y(t)), (x(t + h) - x(t), y(t + h) - y(t))\rangle &= \partial_1 f(x(t), y(t)) h x'(t) + \partial_2 f(x(t), y(t)) h y'(t) + o(h)
\end{align*}
En conclusion : 
\begin{align*}
    F(t + h) &\underset{h\to 0}{=} F(t) + h [ \partial_1 f(x(t), y(t)) x'(t) + \partial_2 f(x(t), y(t)) y'(t) ] + o(h)
\end{align*}
Donc $F$ possède un $\operatorname{DL}_1(t)$, donc $F\in \mathcal{D}^1(I, \mathbb{R})$ et :
\begin{align*}
    \forall t, F'(t) &= \partial_1 f(x(t), y(t)) x'(t) + \partial_2 f(x(t), y(t)) y'(t) \\
    &\in \mathcal{C}^0(I, \mathbb{R}) \text{ par opérations}
\end{align*}
Donc $F\in \mathcal{C}^1(I, \mathbb{R})$. 

\section{Exemple}
\begin{tcolorbox}[title=Exemple 36.43, title filled=false, colframe=darkgreen, colback=darkgreen!10!white]
    Soit $f\in \mathcal{C}^1(\mathbb{R}^2, \mathbb{R})$. Montrer que $f(t^2, \sin t)$ est de classe $\mathcal{C}^1(\mathbb{R}, \mathbb{R})$ et déterminer sa dérivée.
\end{tcolorbox}

\noindent $f\in \mathcal{C}^1(\mathbb{R}^2, \mathbb{R})$ .\\
$t\mapsto t^2, \sin \in \mathcal{C}^1(\mathbb{R}, \mathbb{R})$. \\
D'après la règle de la chaîne, $F\in \mathcal{C}^1(\mathbb{R}^2, \mathbb{R})$ et : 
\begin{align*}
    \forall t\in \mathbb{R}, F'(t) &= 2t \partial_1 f(t^2, \sin t) + \cos t \partial_2 f(t^2, \sin t)
\end{align*}

\setsection{45}
\section{Règle de la chaîne, dérivées directionnelle et gradient}
\begin{tcolorbox}[title=Théorème 36.46, title filled=false, colframe=orange, colback=orange!10!white]
    Soit $\Omega$ un ouvert de $\mathbb{R}^2$ et $f \in \mathcal{C}^1(\Omega, \mathbb{R})$.
    \begin{enumerate}
        \item Soit $I$ un intervalle de $\gamma \in \mathcal{C}^1\left(I, \mathbb{R}^2\right)$ une fonction pour laquelle $\gamma(I) \subset \Omega$. La fonction $f \circ \gamma$ est de classe $\mathcal{C}^1$ sur $I$ et pour tout $t \in I$ :
        $$(f \circ \gamma)^{\prime}(t)=\left\langle\nabla f(\gamma(t)), \gamma^{\prime}(t)\right\rangle$$

        \item La fonction $f$ possède une dérivée directionnelle en tout point de $\Omega$ et dans toutes les directions. Plus précisément, pour tout $a \in \Omega$, tout $v=(h, k) \in \mathbb{R}^2$ :
        $$\mathrm{D}_v f(a)=\langle\nabla f(a), v\rangle=\partial_1 f(a) . h+\partial_2 f(a) . k$$

        \item Interprétation du gradient : le gradient de $f$ est orthogonal aux lignes de niveau de $f$ et dirigé dans le sens des pentes croissantes.
    \end{enumerate}
\end{tcolorbox}

\begin{enumerate}
    \item RAF
    \item Soit $a\in \Omega$ et $v\in \mathbb{R}^2$. Comme $\Omega$ est ouvert, il existe $I = ]-\alpha, \alpha[$ tel que : 
    \begin{align*}
        &I\to \mathbb{R} \\
        &t\mapsto f(a + tv) = f(x_a + th, y_a + tk)
    \end{align*}
    est bien définie et est de classe $\mathcal{C}^1$ d'après la règle de la chaîne ($t\mapsto x_a + th\in \mathcal{C}^1(I), t\mapsto y_a + tk\in \mathcal{C}^1(I)$). 
    \begin{align*}
        \forall t\in I, g'(t) &= h \partial_1 f(a + tv) + k \partial_2 f(a + tv)
    \end{align*}
    Ainsi : 
    \begin{align*}
        D_v f(a) &= g'(a) \\
        &= h \partial_1 f(a) + k \partial_2 f(a) \\
        &= \langle \nabla f(a), v\rangle
    \end{align*}

    \item Notons $P_\lambda: z = \lambda$ pour $\lambda \in \mathbb{R}$. 
    \begin{align*}
        S = \{ (x, y, f(x, y)) \mid (x, y)\in \Omega \}
    \end{align*}
    Une line de niveau $\lambda$ est l'ensemble des solutions de $f(x, y) = \lambda$. \\
    Soit $P_\lambda\cap S$. On suppose que $P_\lambda\cap S \neq \emptyset$ et que $P_\lambda\cap S$ n'est pas un singleton. \\
    Soit $a\in P_\lambda\cap S$. On admet qu'il existe $I = ]-\alpha, \alpha[$ et $\gamma\in \mathcal{C}^1(I, \mathbb{R})$ tel que :
    \begin{align*}
        \begin{cases}
            \gamma(0) = a \\
            \gamma(I) \subset P_\lambda\cap S
        \end{cases}
    \end{align*}
    On a donc : 
    \begin{align*}
        \forall t\in I, f\circ \gamma(t) = \lambda
    \end{align*}
    D'après la règle de la chaîne ($f\in \mathcal{C}^1, \gamma\in \mathcal{C}^1$) : 
    \begin{align*}
        \forall t\in I, \langle \nabla f(\gamma(t)), \gamma'(t)\rangle (f\circ \gamma)'(t) &= 0
    \end{align*}
    Donc $\nabla f(\gamma(a))$ est bien orthogonal à $\gamma'(a)$ (vecteur tangent à $P_\lambda\cap S$ en $a$). \\
    D'après (2) pour tout $v\in \mathbb{R}^2$ : 
    \begin{align*}
        |D_v f(a)| &= |\langle \nabla f(a), v\rangle| \\
        &\leq \| \nabla f(a) \| \|v\|
    \end{align*}
    avec égalité si et seulement si $v$ est colinéaire à $\nabla f(a)$ (et positif si te seulement si $v$ est positivement linéaire à $\nabla f(a)$). 
\end{enumerate}

\setsection{50}
\section{Points critiques et extrema locaux}
\begin{tcolorbox}[title=Théorème 36.51, title filled=false, colframe=orange, colback=orange!10!white]
    Soit $\Omega$ un ouvert de $\mathbb{R}^2, f \in \mathcal{C}^1(\Omega, \mathbb{R})$ et $a \in \Omega$. Si $f$ possède un extremum local en $a$ (i.e. un maximum ou minimum local) alors $a$ est un point critique, i.e. vérifie $\nabla f(a)=0_{\mathbb{R}^2}$. \\
    Par conséquent, le plan tangent de $f$ en un point critique est parallèle au plan abscisse et les dérivées directionnelles de $f$ en $a$ sont toutes nulles.
\end{tcolorbox}

\noindent On suppose que $a$ est un maximum local. \\
Soit $v\in \mathbb{R}^2$. \\
On pose $g:I\to \mathbb{R}; t\mapsto f(a + tv)$. \\
$f(a)$ est nécessairement un maximum local pour $g$. \\
Donc $g'(0) = 0$. \\
Soit $D_v f(a) = 0$. 


\end{document}