\documentclass[../main.tex]{subfiles}

\begin{document}
\setcounter{chapter}{35}
\chapter{Fonctions de deux variables}
\tableofcontents
\clearpage

\setsection{14}
\section{Exemple}
\begin{tcolorbox}[title=Exemple 36.15, title filled=false, colframe=darkgreen, colback=darkgreen!10!white]
    Les projections $(x, y)\mapsto \sqrt{x^2 + y^2}$ est continue sur $\mathbb{R}^2$. 
\end{tcolorbox}

\noindent Soit $a = (x_0, y_0)\in \mathbb{R}^2$. On note : 
\begin{align*}
    p_1:&\mathbb{R}^2\to \mathbb{R} \\
    &(x, y)\mapsto x
\end{align*}
Soit $\epsilon > 0$, pour tout $(x, y)\in B(a, \epsilon)$. 
\begin{align*}
    |p_1(x, y) - p_1(x_0, y_0)| = |x - x_0| \leq \| (x, y) - (x_0, y_0)\|
\end{align*}
Donc $p_1$ est bien continue en $a$, donc sur $\mathbb{R}^2$. 


\end{document}