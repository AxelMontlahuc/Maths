\documentclass[../main.tex]{subfiles}

\begin{document}
\setcounter{chapter}{22}
\chapter{Sous-espaces affines}
\tableofcontents
\clearpage

\section{Sous-espace affine}
\begin{tcolorbox}[title=Définition, title filled=false, colframe=lightpurple, colback=lightpurple!10!white]
    Soit $E$ un $\mathbb{K}$-espace vectoriel. \\
    \begin{itemize}
        \item On appelle \textbf{sous-espace affine de $E$} toute partie $\mathcal{F}$ de $E$ de la forme : 
        \begin{align*}
            \mathcal{F} = x + F = \{f + x \mid f \in F\}
        \end{align*}
        où $F$ est un sous-espace vectoriel de $E$ et $x$ un vecteur de $E$. \\
        
        \item Le sous-espace vectoriel $F$ associé au sous-espace affine $\mathcal{F}$ est unique. On l'appelle \textbf{direction de $\mathcal{F}$} et ses éléments sont appelés les \textbf{vecteurs directeurs de $\mathcal{F}$}. 
    \end{itemize}
\end{tcolorbox}

\noindent On suppose que $\mathcal{F} = x_1 + F_1 = x_2 + F_2$. \\
Soit $y\in F_1$. \\
On a $y + x_1 \in \mathcal{F}$ donc $y + x_1 = x_2 + y_2$ avec $y_2 \in F_2$. \\
Or $x_1 \in \mathcal{F}$ donc $x_1 = x_2 + g_2$ avec $g_2 \in F_2$. \\
Donc : 
\begin{align*}
    y &= x_2 - x_1 + y_2 \\
    &= y_2 - g_2 \\
    &\in F_2
\end{align*}
avec $F_1 \subset F_2$. \\
Par symétrie : 
\begin{align*}
    F_1 = F_2
\end{align*}


\end{document}