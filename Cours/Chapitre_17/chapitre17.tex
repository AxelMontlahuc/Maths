\documentclass[../main.tex]{subfiles}

\begin{document}
\setcounter{chapter}{16}
\chapter{Fractions rationnelles}
\tableofcontents
\clearpage

\setsection{1}
\section{Addition, multiplication et produit par un scalaire}
\begin{tcolorbox}[title=Définition 17.2, title filled=false, colframe=orange, colback=orange!10!white]
    Soit $\frac{P}{Q}$ et $\frac{R}{S}$ deux fractions rationnelles et soit $\lambda \in \mathbb{K}$. On pose
    $$\frac{P}{Q} + \frac{R}{S} = \frac{PS + QR}{QS} \text{, } \frac{P}{Q} \times \frac{R}{S} = \frac{PR}{QS} \text{ et } \lambda \times \frac{P}{Q} = \frac{\lambda P}{Q}.$$
\end{tcolorbox}

\noindent Montrons que l'addition est bien définie. \\
Soit $\frac{P_1}{Q_1} = \frac{P}{Q}$ et $\frac{R}{S}$ dans $\mathbb{K}(X)$. \\
Montrons que : 
\begin{align*}
    \frac{PS + QR}{QS} = \frac{P_1S + Q_1R}{Q_1S}
\end{align*}
On a : 
\begin{align*}
    (PS + QR)Q_1S - (P_1S + Q_1R)QS &= S^2(\underbrace{PQ_1 - P_1Q}_{= 0}) + RS(\underbrace{QQ_1 - Q_1Q}_{= 0}) \\
    &= 0
\end{align*}
On raisonne de la même manière pour $\frac{R}{S} = \frac{R_1}{S_1}$ et ainsi, l'opération est bien définie. 

\setsection{9}
\section{Degré d'une fraction}
\begin{tcolorbox}[title=Définition 17.10, title filled=false, colframe=orange, colback=orange!10!white]
    Soit $F = \frac{P}{Q}$ une fraction. On pose $\deg(F) = -\infty$ si $F = 0$ et $\deg(F) = \deg(P) - \deg(Q)$ sinon. Le degré d'une fraction est donc un élément de $\mathbb{Z} \cup\{ -\infty \}$. 
\end{tcolorbox}

\noindent Si $\frac{P_1}{Q_1} = \frac{P}{Q}$, alors : 
\begin{align*}
    P_1Q &= PQ_1 \\
    \text{donc } \deg(P_1Q) &= \deg(PQ_1) \\
    \text{donc } \deg(P_1) + \deg(Q) &= \deg(P) + \deg(Q_1) \text{ ($\mathbb{K}$ intègre)} \\
    \text{donc } \deg(P_1) - \deg{Q_1} &= \deg(P) - \deg(Q)
\end{align*}

\setsection{12}
\section{Propriété du degré}
\begin{tcolorbox}[title=Théorème 17.13, title filled=false, colframe=orange, colback=orange!10!white]
    Soit $F$ et $G$ deux fractions rationnelles. On a
    $$\deg(F + G) \leq \max(\deg(F), \deg(G)) \text{ et } \deg(F \times G) = \deg(F) + \deg(G).$$
    On retrouve les mêmes propriétés que pour les polynômes. 
\end{tcolorbox}

\noindent Soit $F = \frac{P}{Q}$ et $G = \frac{R}{S}$. 
\begin{itemize}
    \item \begin{align*}
        \deg(F + G) &= \deg(\frac{PS + QR}{QS}) \\
        &= \deg(PS + QR) - \deg(QS) \\
        &\leq \max(\deg(PS), \deg(QR)) - \deg(QS) \\
        &= \max(\deg(PS) - \deg(QS), \deg(QR) - \deg(QS)) \\
        &= \max\left(\deg \left(\frac{P}{Q}\right), \deg \left(\frac{R}{Q}\right)\right) \\
        &= \max(\deg(F), \deg(G))
    \end{align*}

    \item RAS
\end{itemize}

\setsection{18}
\section{Théorème}
\begin{tcolorbox}[title=Théorème 17.19, title filled=false, colframe=orange, colback=orange!10!white]
    Soit $F$ et $G$ deux fractions rationnelles. Si les fonctions rationnelles $\tilde F$ et $\tilde G$ sont égales sur une partie infinie $\mathcal D_F \cap \mathcal D_G$ alors les fractions rationnelles sont égales, i.e. $F = G$. 
\end{tcolorbox}

\noindent On note $F = \frac{P}{Q}$ et $G = \frac{R}{S}$ avec $P \wedge Q = 1$ et $R \wedge S = 1$. \\
On a : 
\begin{align*}
    \forall x \in \mathcal{D} \subset \mathcal{D}_F \cap \mathcal{D}_G, \tilde F(x) = \tilde G(x)
\end{align*}
Soit : 
\begin{align*}
    \forall x \in \mathcal{D}, \tilde{P(x)} \times \tilde{S(x)} = \tilde{R(x)} \times \tilde{Q(x)}
\end{align*}
Comme $\mathcal{D}$ est infini, d'après le théorème de rigidité, $PS = RQ$, donc $F = G$.

\section{Fraction dérivée}
\begin{tcolorbox}[title=Définition 17.20, title filled=false, colframe=orange, colback=orange!10!white]
    Soit $F = \frac{P}{Q} \in \mathbb{K}(X)$. On appelle \textbf{fraction dérivée} de $F$ la fraction notée $F'$ (ou $\frac{dF}{dX}$) définie par
    $$F' = \frac{P'Q - PQ'}{Q^2}.$$
    Le résultat ne dépend pas du représentant de $F$ choisi. On définit également les dérivées successives de $F$ en posant $F^(0) = F$ et pour tout $n \in \mathbb{N}, F^{(n+1)} = (F^{(n)})'$.
\end{tcolorbox}

\noindent On écrit $F = \frac{P}{Q} = \frac{R}{S}$. \\
Montrons que $\frac{P'Q - Q'P}{Q^2} = \frac{R'S - RS'}{S^2}$. \\
Comme $\frac{P}{Q} = \frac{R}{S}$, on a $PS = RQ$. \\
Donc $P'S + S'P = R'Q + Q'R$. \\
Ainsi : 
\begin{align*}
    [P'Q - PQ']S^2 - [R'S - RS']Q^2 &= P'SQ^2 + S'PQ^2 - R'QS^2 - Q'RS^2 \\
    &= QS(P'S - R'Q) + Q^2 RS' - S^2 Q'P \\
    &= QS (Q'R - S'P) + PSQS' - SQRQ' \\
    &= 0
\end{align*}

\setsection{23}
\section{Dérivée logarithmique d'un produit}
\begin{tcolorbox}[title=Théorème 17.24, title filled=false, colframe=orange, colback=orange!10!white]
    Si $F$ est une fraction non nulle qui se facotorise en $F = F_1 \times \ldots \times F_n$ dans $\mathbb{K}(X)$ avec $n \in \mathbb{N}$ alors
    $$\frac{F'}{F} = \frac{F_1'}{F_1} + \ldots + \frac{F_n'}{F_n}.$$
\end{tcolorbox}

\noindent Pour $n = 2$ seulement. \\
\begin{align*}
    F = F_1 \times F_2 \neq 0
\end{align*}
Donc :
\begin{align*}
    F' = F_1'F_2 + F_1F_2'
\end{align*}
Donc :
\begin{align*}
    \frac{F'}{F} = \frac{F_1'F_2}{F_1F_2} + \frac{F_1F_2'}{F_1F_2} = \frac{F_1'}{F_1} + \frac{F_2'}{F_2}
\end{align*}


\end{document}