\documentclass[../main.tex]{subfiles}

\begin{document}
\setcounter{chapter}{24}
\chapter{Comparaison locale des fonctions}
\tableofcontents
\clearpage

\setsection{5}
\section{Caractérisation séquentielle}
\begin{tcolorbox}[title=Théorème 25.6, title filled=false, colframe=orange, colback=orange!10!white]
    Soit $f$ et $g$ deux fonctions sur $X$ et $a\in \overline{X}$. Alors : 
    \begin{enumerate}
        \item $f =_a O(g)$ si et seulement si pour toute suite $(u_n) \underset{n \to +\infty}{\longrightarrow} a$ à valeurs dans $X$, alors $f(u_n) = O(g(u_n))$. 
        \item $f =_a o(g)$ si et seulement si pour toute suite $(u_n) \underset{n \to +\infty}{\longrightarrow} a$ à valeurs dans $X$, alors $f(u_n) = o(g(u_n))$.
    \end{enumerate}
\end{tcolorbox}

\begin{enumerate}
    \item \begin{align*}
        f =_a O(g) &\text{ ssi } \text{il existe $h$ bornée au voisinage de $a$ tel que } f = g \cdot h \\
        &\text{ ssi } \text{Pour toute suite $(u_n) \in X^{\mathbb{N}}$ avec $u_n\to a$, } f(u_n) = g(u_n) \times w_n \text{ où $(w_n)$ est une suite bornée. } \\
        &\text{ } \boxed{\Rightarrow} \quad w_n = h(u_n) \text{ ssi bornée} \quad \boxed{\Leftarrow} \quad \text{ Par l'absurde avec (25.5). } \\
        &\text{ ssi } \text{Pour toute suite $(u_n) \in X^{\mathbb{N}}$ avec $u_n\to a$, } f(u_n) = O(g(u_n)).
    \end{align*}
    \item On utilise la caractérisation séquentielle de la limite (nulle). 
\end{enumerate}

\setsection{13}
\section{Existence, unicité et expression du développement de Taylor de $f$}
\begin{tcolorbox}[title=Théorème 25.14, title filled=false, colframe=orange, colback=orange!10!white]
    Soit $f$ une fonction $n$ fois dérivable en $x_0$. Alors le développement de Taylor de $f$ en $x_0$ à l'ordre $n$ existe et est unique. Il est donné explicitement par :
    \begin{align*}
        \forall x\in \mathbb{R}, P(x) = \sum_{k=0}^{n} \frac{(x-x_0)^k}{k!} f^{(k)}(x_0)
    \end{align*}
\end{tcolorbox}

\noindent RAS, cf. (16.56)

\setsection{19}
\section{Formule de Taylor avec reste intégral de l'ordre $n$ au point $a$}
\begin{tcolorbox}[title=Théorème 25.20, title filled=false, colframe=orange, colback=orange!10!white]
    Soit $a < b$ et $f:[a, b] \to \mathbb{R}$ une fonction de classe $\mathcal{C}^{n+1}([a, b])$ Alors : 
    \begin{align*}
        \forall x\in [a, b], f(x) = \sum_{k=0}^{n} \frac{(x-a)^k}{k!} f^{(k)}(a) + \int_{a}^{x} \frac{(x-t)^n}{n!} f^{(n+1)}(t) \, dt
    \end{align*}
\end{tcolorbox}

\noindent On raisonne par récurrence sur $n\in \mathbb{N}$. \\
\begin{itemize}
    \item On suppose $f\in \mathcal{C}^1([a, b], \mathbb{R})$. On a :
    \begin{align*}
        \forall x \in [a, b], \sum_{k=0}^{0} \frac{(x-a)^k}{k!} f^{(k)}(a) + \int_{a}^{x} \frac{(x-t)^0}{0!} f'(t) \, dt &= f(a) + \int_{a}^{x} f'(t) \, dt \\
        &= f(x)
    \end{align*}

    \item On suppose le résultat vrai pour $n\in \mathbb{N}$. \\
    Soit $f\in \mathcal{C}^{n+2}([a, b], \mathbb{R})$. En particulier, $f\in \mathcal{C}^{n+1}([a, b], \mathbb{R})$. On a :
    \begin{align*}
        \forall x\in [a, b], f(x) &= \sum_{k=0}^{n} \frac{(x-a)^k}{k!} f^{(k)}(a) + \int_{a}^{x} \frac{(x-t)^n}{n!} f^{(n+1)}(t) \, dt \\
        &= \sum_{k=0}^{n} \frac{(x-a)^k}{k!} f^{(k)}(a) + \left[ -\frac{(x-t)^{n+1}}{(n+1)!} f^{(n+1)}(t) \right]_a^x + \int_{a}^{x} \frac{(x-t)^{n+1}}{(n+1)!} f^{(n+2)}(t) \, dt \\
        \text{(IPP) } &= \sum_{k=0}^{n} \frac{(x-a)^k}{k!} f^{(k)}(a) + \int_{a}^{x} \frac{(x-t)^{n+1}}{(n+1)!} f^{(n+2)}(t) \, dt
    \end{align*}
\end{itemize}


\end{document}