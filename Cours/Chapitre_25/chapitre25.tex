\documentclass[../main.tex]{subfiles}

\begin{document}
\setcounter{chapter}{24}
\chapter{Comparaison locale des fonctions}
\tableofcontents
\clearpage

\setsection{5}
\section{Caractérisation séquentielle}
\begin{tcolorbox}[title=Théorème 25.6, title filled=false, colframe=orange, colback=orange!10!white]
    Soit $f$ et $g$ deux fonctions sur $X$ et $a\in \overline{X}$. Alors : 
    \begin{enumerate}
        \item $f =_a O(g)$ si et seulement si pour toute suite $(u_n) \underset{n \to +\infty}{\longrightarrow} a$ à valeurs dans $X$, alors $f(u_n) = O(g(u_n))$. 
        \item $f =_a o(g)$ si et seulement si pour toute suite $(u_n) \underset{n \to +\infty}{\longrightarrow} a$ à valeurs dans $X$, alors $f(u_n) = o(g(u_n))$.
    \end{enumerate}
\end{tcolorbox}

\begin{enumerate}
    \item \begin{align*}
        f =_a O(g) &\text{ ssi } \text{il existe $h$ bornée au voisinage de $a$ tel que } f = g \cdot h \\
        &\text{ ssi } \text{Pour toute suite $(u_n) \in X^{\mathbb{N}}$ avec $u_n\to a$, } f(u_n) = g(u_n) \times w_n \text{ où $(w_n)$ est une suite bornée. } \\
        &\text{ } \boxed{\Rightarrow} \quad w_n = h(u_n) \text{ ssi bornée} \quad \boxed{\Leftarrow} \quad \text{ Par l'absurde avec (25.5). } \\
        &\text{ ssi } \text{Pour toute suite $(u_n) \in X^{\mathbb{N}}$ avec $u_n\to a$, } f(u_n) = O(g(u_n)).
    \end{align*}
    \item On utilise la caractérisation séquentielle de la limite (nulle). 
\end{enumerate}


\end{document}