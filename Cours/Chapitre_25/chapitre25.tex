\documentclass[../main.tex]{subfiles}

\begin{document}
\setcounter{chapter}{24}
\chapter{Comparaison locale des fonctions}
\tableofcontents
\clearpage

\setsection{5}
\section{Caractérisation séquentielle}
\begin{tcolorbox}[title=Théorème 25.6, title filled=false, colframe=orange, colback=orange!10!white]
    Soit $f$ et $g$ deux fonctions sur $X$ et $a\in \overline{X}$. Alors : 
    \begin{enumerate}
        \item $f =_a O(g)$ si et seulement si pour toute suite $(u_n) \underset{n \to +\infty}{\longrightarrow} a$ à valeurs dans $X$, alors $f(u_n) = O(g(u_n))$. 
        \item $f =_a o(g)$ si et seulement si pour toute suite $(u_n) \underset{n \to +\infty}{\longrightarrow} a$ à valeurs dans $X$, alors $f(u_n) = o(g(u_n))$.
    \end{enumerate}
\end{tcolorbox}

\begin{enumerate}
    \item \begin{align*}
        f =_a O(g) &\text{ ssi } \text{il existe $h$ bornée au voisinage de $a$ tel que } f = g \cdot h \\
        &\text{ ssi } \text{Pour toute suite $(u_n) \in X^{\mathbb{N}}$ avec $u_n\to a$, } f(u_n) = g(u_n) \times w_n \text{ où $(w_n)$ est une suite bornée. } \\
        &\text{ } \boxed{\Rightarrow} \quad w_n = h(u_n) \text{ ssi bornée} \quad \boxed{\Leftarrow} \quad \text{ Par l'absurde avec (25.5). } \\
        &\text{ ssi } \text{Pour toute suite $(u_n) \in X^{\mathbb{N}}$ avec $u_n\to a$, } f(u_n) = O(g(u_n)).
    \end{align*}
    \item On utilise la caractérisation séquentielle de la limite (nulle). 
\end{enumerate}

\setsection{13}
\section{Existence, unicité et expression du développement de Taylor de $f$}
\begin{tcolorbox}[title=Théorème 25.14, title filled=false, colframe=orange, colback=orange!10!white]
    Soit $f$ une fonction $n$ fois dérivable en $x_0$. Alors le développement de Taylor de $f$ en $x_0$ à l'ordre $n$ existe et est unique. Il est donné explicitement par :
    \begin{align*}
        \forall x\in \mathbb{R}, P(x) = \sum_{k=0}^{n} \frac{(x-x_0)^k}{k!} f^{(k)}(x_0)
    \end{align*}
\end{tcolorbox}

\noindent RAS, cf. (16.56)

\setsection{19}
\section{Formule de Taylor avec reste intégral de l'ordre $n$ au point $a$}
\begin{tcolorbox}[title=Théorème 25.20, title filled=false, colframe=orange, colback=orange!10!white]
    Soit $a < b$ et $f:[a, b] \to \mathbb{R}$ une fonction de classe $\mathcal{C}^{n+1}([a, b])$ Alors : 
    \begin{align*}
        \forall x\in [a, b], f(x) = \sum_{k=0}^{n} \frac{(x-a)^k}{k!} f^{(k)}(a) + \int_{a}^{x} \frac{(x-t)^n}{n!} f^{(n+1)}(t) \, dt
    \end{align*}
\end{tcolorbox}

\noindent On raisonne par récurrence sur $n\in \mathbb{N}$. \\
\begin{itemize}
    \item On suppose $f\in \mathcal{C}^1([a, b], \mathbb{R})$. On a :
    \begin{align*}
        \forall x \in [a, b], \sum_{k=0}^{0} \frac{(x-a)^k}{k!} f^{(k)}(a) + \int_{a}^{x} \frac{(x-t)^0}{0!} f'(t) \, dt &= f(a) + \int_{a}^{x} f'(t) \, dt \\
        &= f(x)
    \end{align*}

    \item On suppose le résultat vrai pour $n\in \mathbb{N}$. \\
    Soit $f\in \mathcal{C}^{n+2}([a, b], \mathbb{R})$. En particulier, $f\in \mathcal{C}^{n+1}([a, b], \mathbb{R})$. On a :
    \begin{align*}
        \forall x\in [a, b], f(x) &= \sum_{k=0}^{n} \frac{(x-a)^k}{k!} f^{(k)}(a) + \int_{a}^{x} \frac{(x-t)^n}{n!} f^{(n+1)}(t) \, dt \\
        &= \sum_{k=0}^{n} \frac{(x-a)^k}{k!} f^{(k)}(a) + \left[ -\frac{(x-t)^{n+1}}{(n+1)!} f^{(n+1)}(t) \right]_a^x + \int_{a}^{x} \frac{(x-t)^{n+1}}{(n+1)!} f^{(n+2)}(t) \, dt \\
        \text{(IPP) } &= \sum_{k=0}^{n} \frac{(x-a)^k}{k!} f^{(k)}(a) + \int_{a}^{x} \frac{(x-t)^{n+1}}{(n+1)!} f^{(n+2)}(t) \, dt
    \end{align*}
\end{itemize}

\setsection{21}
\section{Formule de Taylor-Lagrange à l'ordre $n$ au point $a$ évaluée en $b$ - Hors Programme}
\begin{tcolorbox}[title=Théorème 25.22, title filled=false, colframe=orange, colback=orange!10!white]
    Soit $a < b$ deux réels et $f:[a, b] \to \mathbb{R}$ une fonction de classe $\mathcal{C}^n$ sur $[a, b]$ et $n+1$ dérivable sur $]a, b[$. Alors :
    \begin{align*}
        \exists c \in ]a, b[, f(b) = \sum_{k=0}^{n} \frac{f^{(k)}(a)}{k!} (b-a)^k + \frac{(b-a)^{n+1}}{(n+1)!} f^{(n+1)}(c)
    \end{align*}
\end{tcolorbox}

\noindent On introduit : 
\begin{align*}
    g: [a, b] &\to \mathbb{R}; x \mapsto \sum_{k=0}^{n} \frac{f^{(k)}(x)}{k!} (b-x)^k + \frac{(b-x)^{n+1}}{(n+1)!} f^{(n+1)}(x) \text{ avec } A \in \mathbb{R}
\end{align*}
On remarque que $g(b) = f(b)$. \\
On choisit $A$ de telle sorte que $g(a) = f(b)$. \\
On pose : 
\begin{align*}
    A = \frac{-(n+1!)}{(b-a)^{n+1}} \left[ -\sum_{k=0}^{n} \frac{f^{(k)}(a)}{k!} (b-a)^k + f(b) \right]
\end{align*}
Par hypothèse, $g\in \mathcal{C}^0([a, b], \mathbb{R}) \cap \mathcal{D}^1(]a, b[, \mathbb{R})$. \\
D'après le théorème de Rolle, on choisit $c\in ]a, b[$ tel que $g'(c) = 0$. \\
Or : 
\begin{align*}
    \forall x \in ]a, b[, g'(x) &= \sum_{k=0}^{n} \frac{f^{(k+1)}(x)}{k!} (b-x)^k - \sum_{k=1}^{n} \frac{f^{(k)}(x)}{(k-1)!} (b-x)^{k-1} - A \frac{(b-x)^n}{n!} \\
    &= \frac{f^{(n+1)}(x)}{n!} (b-x)^n - A \frac{(b-x)^n}{n!}
\end{align*}
En particulier : 
\begin{align*}
    \frac{A(b-c)^n}{n!} &= \frac{f^{(n+1)}(c)}{n!} (b-c)^n
\end{align*}
Or $c \neq b$ donc $A = f^{(n+1)}(c)$. \\
On conclut avec $f(b) = g(a)$.

\setsection{26}
\section{Formule de Taylor-Young à l'ordre $n$ au point $x_0$}
\begin{tcolorbox}[title=Théorème 25.27, title filled=false, colframe=orange, colback=orange!10!white]
    Soit $I$ un intervalle ouvert de $\mathbb{R}$, $x_0\in I$ et $f:I \to \mathbb{R}$ une fonction de classe $\mathcal{C}^n$ au voisinage de $x_0$. Alors au voisinage de $x_0$, on a : 
    \begin{align*}
        f(x) =_{x\to x_0} \sum_{k=0}^{n} \frac{f^{(k)}(x_0)}{k!} (x - x_0)^k + o((x-x_0)^n)
    \end{align*}
\end{tcolorbox}

\noindent On a $f \in \mathcal{C}^n(I, \mathbb{R}) = \mathcal{C}^{(n-n+1)}(I, \mathbb{R})$. \\
D'après la formule de Taylor : 
\begin{align*}
    f(x) &= \sum_{k=0}^{n-1} \frac{f^{(k)}(x_0)}{k!} (x - x_0)^k + \int_{x_0}^{x} \frac{(x-t)^{n-1}}{(n-1)!} f^{(n)}(t) \, dt
\end{align*}
Montrons que : 
\begin{align*}
    \int_{x_0}^{x} \frac{(x-t)^{n-1}}{(n-1)!} f^{(n)}(t) \, dt =_{x\to x_0} \frac{f^{(n)}(x_0)}{n!} (x-x_0)^n + o((x-x_0)^n)
\end{align*}
On a : 
\begin{align*}
    \int_{x_0}^{x} \frac{(x - t)^{n-1}}{(n-1)!} f^{(n)}(t) \, dt - \frac{f^{(n)}(x_0)(x - x_0)^n}{n!} &= \int_{x_0}^{x} \frac{(x - t)^{n-1}}{(n-1)!} f^{(n)}(t) \, dt - \int_{x_0}^{x} \frac{(x - t)^{n-1}}{(n-1)!} f^{(n)}(x_0) \, dt \\
    &= \int_{x_0}^{x} \frac{(x - t)^{n-1}}{(n-1)!} [f^{(n)}(t) - f^{(n)}(x_0)] \, dt
\end{align*}
Soit $\varepsilon > 0$, on choisit $v\in \mathcal{V}(x_0)$ tel que :
\begin{align*}
    \forall x \in v, |f^{(n)}(x) - f^{(n)}(x_0)| \leq \varepsilon
\end{align*}
car $f^{(n)} \in \mathcal{C}^0(I, \mathbb{R})$. \\
Soit $x\in \mathcal{V}, x > x_0$. On a : 
\begin{align*}
    \left| \int_{x_0}^{x} \frac{(x - t)^{n-1}}{(n-1)!} [f^{(n)}(t) - f^{(n)}(x_0)] \, dt \right| &\leq \int_{x_0}^{x} \frac{(x - t)^{n-1}}{(n-1)!} |f^{(n)}(t) - f^{(n)}(x_0)| \, dt \\
    &\leq \varepsilon \int_{x_0}^{x} \frac{(x - t)^{n-1}}{(n-1)!} \, dt \\
    &\leq \frac{\varepsilon}{(n-1)!} \int_{x_0}^{x} (x - t)^{n-1} \, dt \\
    &= \frac{\varepsilon(x-x_0)^n}{n!}
\end{align*}
Le résultat reste vrai (au signe près) pour $x \leq x_0$. \\
Par définition (avec les $\varepsilon$), on a le résultat souhaité. 

\setsection{27}
\section{Développement limité de l'exponentielle}
\begin{tcolorbox}[title=Propostion 25.28, title filled=false, colframe=lightblue, colback=lightblue!10!white]
    La formule de Taylor-Young à l'ordre $n$ en $0$ de l'exponentielle donne l'égalité suivante au voisinage de $0$ :
    \begin{align*}
        e^x=_{x\to 0} \sum_{k=0}^{n} \frac{x^k}{k!} + o(x^n)
    \end{align*}
\end{tcolorbox}

\begin{align*}
    f = \exp \in \mathcal{C}^n(\mathbb{R}, \mathbb{R}) \text{ et } \forall x \in \mathbb{N}, f^{(k)}(0) = e^0 = 1 \\
\end{align*}

\section{Développement limité du logarithme}
\begin{tcolorbox}[title=Propostion 25.29, title filled=false, colframe=lightblue, colback=lightblue!10!white]
    La formule de Taylor-Young à l'ordre $n$ en $0$ de $x\mapsto \ln (1 + x)$ donne l'égalité suivante au voisinage de $1$ :
    \begin{align*}
        \ln (1 + x) =_{x\to 0} \sum_{k=1}^{n} \frac{(-1)^{k-1}x^k}{k} + o(x^n)
    \end{align*}
\end{tcolorbox}

\noindent $f:x\mapsto \ln(1+x) \in \mathcal{C}^n(]-1, \infty[, \mathbb{R})$. \\
\begin{align*}
    \forall x > -1, f'(x) &= \frac{1}{1+x} \\
    \forall k \in \mathbb{N}, \forall x > -1, f^{(k+1)}(x) &= \frac{(-1)^k k!}{(1+x)^{k+1}} \\
    f^{(k+1)}(0) &= (-1)^k k!
\end{align*}
Donc, d'après Taylor-Young :
\begin{align*}
    f(x) &=_{x\to 0} \sum_{k=1}^{n} \frac{f^{(k)}(0)}{k!}x^k + o(x^n) \\
    &=_{x\to 0} \sum_{k=1}^{n} \frac{(-1)^{k-1}(k-1)!}{k!}x^k + o(x^n) \\
    &=_{x\to 0} \sum_{k=1}^{n} \frac{(-1)^{k-1}}{k}x^k + o(x^n)
\end{align*}

\section{Développement limité de cosinus et sinus}
\begin{tcolorbox}[title=Propostion 25.30, title filled=false, colframe=lightblue, colback=lightblue!10!white]
    La formule de Taylor-Young à l'ordre $2n + 2$ pour le sinus et à l'ordre $2n + 1$ pour le cosinus en $0$ donne les égalités suivantes au voisinage de $0$ : 
    \begin{align*}
        \sin x &=_{x\to 0} \sum_{k=0}^{n} \frac{(-1)^k x^{2k+1}}{(2k+1)!} + o(x^{2n+2}) \quad \text{ et } \quad \cos x =_{x\to 0} \sum_{k=0}^{n} \frac{(-1)^k x^{2k}}{(2k)!} + o(x^{2n+1}) \\
    \end{align*}
\end{tcolorbox}

\noindent $\sin \in \mathcal{C}^{\infty}(\mathbb{R}, \mathbb{R})$
\begin{align*}
    \begin{cases}
        \sin^{(2k)}(0) = 0 \\
        \sin^{(2k+1)}(0) = 1 \\
        \sin^{(4k+3)}(0) = -1
    \end{cases}
\end{align*}
Donc : 
\begin{align*}
    \sin x &=_{x\to 0} \sum_{k=0}^{2n+2} \frac{\sin^{(k)}(0)}{k!}x^k + o(x^{2n+2}) \\
    &= \sum_{i=0}^{n} \frac{(-1)^i}{(2i+1)!}x^{2i+1} + o(x^{2n+2})
\end{align*}
Idem pour $\cos$.

\setsection{39}
\section{Unicité du DL}
\begin{tcolorbox}[title=Théorème 25.40, title filled=false, colframe=orange, colback=orange!10!white]
    Si $f$ admet un développement limité à l'ordre $n$ au voisinage de $x_0$, alors ce développement est unique. 
\end{tcolorbox}

\noindent On suppose que : 
\begin{align*}
    f(x) &=_{x\to x_0} \sum_{k=0}^{n} a_k (x - x_0)^k + o((x - x_0)^n) \\
    &=_{x\to x_0} \sum_{k=0}^{n} b_k (x - x_0)^k + o((x - x_0)^n) 
\end{align*}
On suppose par l'absurde que les développements sont différents. \\
On note $p = \min (k \mid a_k \neq b_k)$. \\
Or : 
\begin{align*}
    \sum_{k=0}^{n} a_k (x - x_0)^k &=_{x\to x_0} \sum_{k=0}^{n} b_k (x - x_0)^k + o((x - x_0)^n)
\end{align*}
Donc : 
\begin{align*}
    \sum_{k=p}^{n} a_k(x - x_0)^k &=_{x\to x_0} + o((x - x_0)^n) \\
    \text{donc } a_p (x - x_0)^p + \sum_{k=p+1}^{n} a_k(x - x_0)^k &=_{x\to x_0} b_p(x - x_0)^p + \sum_{k=p+1}^{n} b_k(x - x_0)^k + o((x - x_0)^n) \\
    \text{donc } a_p (x - x_0)^p &=_{x\to x_0} b_p(x - x_0)^p + o((x - x_0)^n) \\
    \text{donc } a_p &= b_p + o(1)
\end{align*}
Absurde car $a_p \neq b_p$.

\section{DL de fonctions paires ou impaires}
\begin{tcolorbox}[title=Propostion 25.41, title filled=false, colframe=lightblue, colback=lightblue!10!white]
    Soit $f$ une fonction admettant un DL à l'ordre $n$ au voisinage de $0$. Alors : 
    \begin{itemize}
        \item si $f$ est paire, son DL n'est constitué que de monômes de degré pair. 
        \item si $f$ est impaire, son DL n'est constitué que de monômes de degré impair. 
    \end{itemize}
\end{tcolorbox}

\begin{itemize}
    \item On suppose $f$ paire et : 
    \begin{align*}
        f(x) &=_{x\to 0} \sum_{k=0}^{n} a_k x^k + o(x^n) \\
    \end{align*}
    Donc : 
    \begin{align*}
        f(-x) &=_{x\to 0} \sum_{k=0}^{n} a_k (-1)^kx^k + o(x^n) \\
    \end{align*}
    Par unicité du DL : 
    \begin{align*}
        \forall k \in \llbracket 0, n \rrbracket, a_k = (-1)^k a_k
    \end{align*}
    Donc pour $k$ impair : 
    \begin{align*}
        a_k = 0
    \end{align*}
    \item Même raisonnement pour $f$ impaire.
\end{itemize}

\section{Remarque}
\begin{tcolorbox}[title=Remarque 25.42, title filled=false, colframe=lightblue, colback=lightblue!10!white]
    \begin{enumerate}
        \setcounter{enumi}{2}
        \item L'existence d'un DL à l'ordre $n$ en $x_0$ n'implique pas l'existence de la dérivée $n$-ième de $f$ en $x_0$. Ainsi, tous les DL ne sont pas obtenus par la formule de Taylor-Young. 
    \end{enumerate}
\end{tcolorbox}

\begin{enumerate}
    \setcounter{enumi}{2}
    \item Si $f$ admet un $\text{DL}_0$ en $x_0$, on a : 
    \begin{align*}
        f(x) &=_{x\to x_0} a + o(1)
    \end{align*}
    Donc : 
    \begin{align*}
        f(x) - a \underset{x \to x_0}{\longrightarrow} 0
    \end{align*}
    Donc : 
    \begin{align*}
        f(x) \underset{x \to x_0}{\longrightarrow} a
    \end{align*}
    Néecssairement, $a = f(x_0)$ et $f$ est continue en $x_0$. \\
    Si $f$ admet un $\text{DL}_1$ en $x_0$, on a : 
    \begin{align*}
        f(x) &=_{x\to x_0} f(x_0) + a (x - x_0) + o(x - x_0)
    \end{align*}
    Donc : 
    \begin{align*}
        \frac{f(x) - f(x_0)}{x - x_0} =_{x\to x_0} a + o(1) \underset{x \to x_0}{\longrightarrow} a
    \end{align*}
\end{enumerate}

\section{Exemple}
\begin{tcolorbox}[title=Exemple 25.43.2, title filled=false, colframe=darkgreen, colback=darkgreen!10!white]
    \begin{enumerate}
        \setcounter{enumi}{1}
        \item La fonction $f:t\mapsto \cos t + t^3 \sin \frac{1}{t}$ prolongée en $0$ par $f(0) = 1$ admet un DL d'ordre 2 en $0$, mais n'est pas deux fois dérivable en $0$. 
    \end{enumerate}
\end{tcolorbox}

\begin{enumerate}
    \setcounter{enumi}{1}
    \item \begin{align*}
        f(t) - \left( 1 - \frac{t^2}{2} \right) &= \cos t - 1 + \frac{t^2}{2} + t^3 \sin \frac{1}{t} \\
        &=_{t\to 0} o(t^2) + t^2 \times t \sin \frac{1}{t} \\
        &=_{t\to 0} o(t^2)
    \end{align*}
    Donc $f$ admet bien un $\text{DL}_2$ en $0$, donc un $\text{DL}_1$ en $0$, donc est dérivable en $0$ (et donc sur $\mathbb{R}$ par théorème d'opérations). 
    \begin{align*}
        \forall x \in \mathbb{R}, f'(x) &= -\sin x + 3x^2 \sin \frac{1}{x} - x\cos \frac{1}{x} \\
        \frac{f'(x)}{x} &= -\frac{\sin x}{x} + 3x \sin \frac{1}{x} - \cos \frac{1}{x}
    \end{align*}
\end{enumerate}

\setsection{49}
\section{Forme normalisée d'un DL au voisinage de $0$}
\begin{tcolorbox}[title=Propostion 25.50, title filled=false, colframe=lightblue, colback=lightblue!10!white]
    Soit $f$ une fonction définie au voisinage de $x_0$, admettant à l'ordre $n$ un DL non nul. Alors il existe un unique entier $m \leq n$ tel que pour $h$ au voisinage de $0$ on ait : 
    \begin{align*}
        f(x_0 + h) =_{x\to x_0} h^m(a_0 + a_1 h + \ldots + a_{n-m} h^{n-m}) + o(h^{n-m})
    \end{align*}
    avec $a_0 \neq 0$. Il s'agit de la \textbf{forme normalisée} du DL à l'ordre $n$ de $f$ au voisinage de $x_0$. 
\end{tcolorbox}

\begin{align*}
    f(x) &=_{x\to x_0} \sum_{k=0}^{n} a_k (x - x_0)^k + o((x - x_0)^n) \\
    &=_{x\to x_0} \sum_{k=m}^{n} a_k (x - x_0)^k + o((x - x_0)^k) \\
    &=_{x\to x_0} (x - x_0)^m \left( \sum_{k=0}^{n-m} a_{k+m} (x - x_0)^k + o((x - x_0)^{n-m}) \right)
\end{align*}
Puis on effectue un changment de variable : $x = x_0 + h$.

\setsection{55}
\section{Produit de DL}
\begin{tcolorbox}[title=Propostion 25.56, title filled=false, colframe=lightblue, colback=lightblue!10!white]
    Soit $f$ et $g$ deux fonctions définies sur un voisinage de $0$ et $P$ et $Q$ deux polynômes de degré au plus $n$. Si au voisinage de $0$ : 
    \begin{align*}
        f(x) &=_{x\to 0} P(x) + o(x^n) \quad \text{ et } \quad g(x) =_{x\to 0} Q(x) + o(x^n)
    \end{align*}
    Alors : 
    \begin{align*}
        (fg)(x) =_{x\to 0} T_n(PQ)(x) + o(x^n)
    \end{align*}
\end{tcolorbox}

\begin{align*}
    f(x)g(x) &=_{x\to 0} (P(x) + o(x^n))(Q(x) + o(x^n)) \\
    &=_{x\to 0} P(x)Q(x) + P(x)o(x^n) + Q(x)o(x^n) + o(x^n)o(x^n) \\
    &=_{x\to 0} P(x)Q(x) + o(x^n) \\
    &=_{x\to 0} T_n(PQ)(x) + o(x^n)
\end{align*}

\section{Exemple}
\begin{tcolorbox}[title=Exemple 25.57, title filled=false, colframe=darkgreen, colback=darkgreen!10!white]
    \begin{enumerate}
        \item $\frac{\cos x}{1 + x} =_{x\to 0} 1 - x + \frac{x^2}{2} - \frac{x^3}{2} + o(x^3)$
        \item $(e^x)^2 =_{x\to 0} 1 + 2x + 2x^2 + \frac{4x^3}{3} + o(x^3)$
    \end{enumerate}
\end{tcolorbox}

\begin{enumerate}
    \item \begin{align*}
        \frac{\cos x}{1 + x} &= \cos x \times (1 + x)^{-1} \\
        &=_{x\to 0} (1 - \frac{x^2}{2} + o(x^2))(1 - x + x^2 - x^3 + o(x^3)) \\
        &=_{x\to 0} 1 - x + \frac{x^2}{2} - \frac{x^3}{2} + o(x^3)
    \end{align*}

    \item \begin{align*}
        (e^x)^2 &=_{x\to 0} (1 + x + \frac{x^2}{2} + \frac{x^3}{6} + o(x^3))^2 \\
        &=_{x\to 0} 1 + 2x + 2x^2 + \frac{4x^3}{3} + o(x^3)
    \end{align*}
\end{enumerate}

\section{Exemple}
\begin{tcolorbox}[title=Exemple 25.58, title filled=false, colframe=darkgreen, colback=darkgreen!10!white]
    \begin{enumerate}
        \item $(\sin x - x)(\cos x - 1) =_{x\to 0} \frac{x^5}{12} - \frac{x^7}{90} + o(x^8)$
    \end{enumerate}
\end{tcolorbox}

\begin{enumerate}
    \item \begin{align*}
        (\sin x - x)(\cos x - 1) &=_{x\to 0} \left( -\frac{x^3}{6} + \frac{x^5}{120} + o(x^6) \right) \left( -\frac{x^2}{2} + \frac{x^4}{24} + o(x^5) \right) \\
        &=_{x\to 0} \frac{x^5}{12} + \left( \frac{-1}{2 \times 5!} - \frac{1}{3!4!} \right) x^7 + o(x^8) \\
        &=_{x\to 0} \frac{x^5}{12} - \frac{1}{5 \times 3 \times 3!} x^7 + o(x^8) \\
        &=_{x\to 0} \frac{x^5}{12} - \frac{x^7}{90} + o(x^8)
    \end{align*}
\end{enumerate}

\section{Composition de DL}
\begin{tcolorbox}[title=Propostion 25.59, title filled=false, colframe=lightblue, colback=lightblue!10!white]
    Soit $f$ et $g$ deux fonctions définies au voisinage de $0$ avec $f(0) = 0$. Si $P$ et $Q$ sont des développements limités de $f$ et $g$ en $0$ à l'ordre $n$, alors $T_n(Q\circ P)$ est un DL en $0$ de $g\circ f$ à l'ordre $n$ : 
    \begin{align*}
        g \circ f(x) =_{x\to 0} T_n(Q\circ P)(x) + o(x^n)
    \end{align*}
\end{tcolorbox}

\noindent On suppose que : 
\begin{align*}
    f(x) &=_{x\to 0} P(x) + o(x^n) \\
    g(x) &=_{x\to 0} Q(x) + o(x^n)
\end{align*}
Comme $f(0) = 0$, on a $P(0) = 0$. \\
\begin{align*}
    g \circ f(x) &=_{x\to 0} Q(f(x)) + o(x^n) \\
\end{align*}
Avec la notation $Q = \sum\limits_{k=0}^{n} b_k X^k$, on a :
\begin{align*}
    g\circ f(x) &=_{x\to 0} \sum_{k=0}^{n} b_k f(x)^k + o(f(x)^n) \\
    &=_{x\to 0} \sum_{k=0}^{n} b_k (P(x) + o(x^n))^k + o((P(x) + o(x^n))^n) \\
    &=_{x\to 0} \sum_{k=0}^{n} [b_k (P(x))^k + \underbrace{o(x^n)}_{P(x) =_{x\to 0} O(1)}] + o(\underbrace{P(x)^k}_{P(x) =_{x\to 0 = O(x)}}) \\
    &=_{x\to 0} \sum_{k=0}^{n} b_k P(x)^k + o(x^n) \\
    &=_{x\to 0} Q\circ P(x) + o(x^n) \\
    &=_{x\to 0} T_n(Q\circ P)(x) + o(x^n)
\end{align*}

\section{Exemple}
\begin{tcolorbox}[title=Exemple 25.59, title filled=false, colframe=darkgreen, colback=darkgreen!10!white]
    \begin{enumerate}
        \item $e^{\sin x} \underset{x\to 0}{=} 1 + x + \frac{x^2}{2} + o(x^3)$
        \item $e^{\cos x - 1} \underset{x\to 0}{=} 1 - \frac{x^2}{2} + \frac{x^4}{6} + o(x^4)$
    \end{enumerate}
\end{tcolorbox}

\begin{enumerate}
    \item \begin{align*}
        e^{\sin x} &\underset{x\to 0}{=} e^{x - \frac{x^3}{6} + o(x^3)} \\
        &\underset{x\to 0}{=} 1 + \left( x - \frac{x^3}{6} + o(x^3) \right) + \frac{1}{2} \left( x - \frac{x^3}{6} + o(x^3) \right)^2 + \frac{1}{6} \left( x - \frac{x^3}{6} + o(x^3)^3 \right)^3 + o(x^3) \\
        &\underset{x\to 0}{=} 1 + \left( x - \frac{x^3}{6} + o(x^3) \right) + \frac{1}{2} (x + O(x^3))^2 + \frac{1}{6} (x + O(x^3))^3 + o(x^3) \\
        &\underset{x\to 0}{=} 1 + x + \frac{1}{2} x^2 + o(x^3)
    \end{align*}

    \item \begin{align*}
        e^{\cos x - 1} &\underset{x\to 0}{=} 1 + \left( -\frac{X^2}{2} + \frac{x^4}{4!} + o(x^4) \right) + \frac{1}{2} \left( -\frac{x^2}{2} + O(x^4) \right)^2 + o(x^4) \\
        &\underset{x\to 0}{=} 1 - \frac{1}{2} x^2 + \frac{1}{6} x^4 + o(x^4)
    \end{align*}
\end{enumerate}


\end{document}