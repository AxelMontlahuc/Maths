\documentclass[../main.tex]{subfiles}

\begin{document}
\setcounter{chapter}{18}
\chapter{Convexité}
\tableofcontents
\clearpage

\setsection{6}
\section{Position du graphe d'une fonction convexe par rapport à ses sécantes}
\begin{tcolorbox}[title=Propostion 19.7, title filled=false, colframe=lightblue, colback=lightblue!10!white]
    Soit $f:I\to \mathbb{R}$ une fonction convexe et $(x, y) \in I^2$ avec $x < y$. \\
    Le graphe de $f$ est situé en-dessous de sa sécante sur l'intervalle $[x, y]$ et au-dessus à l'extérieur, soit sur $I \cap ]-\infty, x] \cup [y, +\infty[$.
\end{tcolorbox}

\noindent On pose $g:\mathbb{R}\to \mathbb{R}; t\mapsto \frac{f(y) - f(x)}{y - x}(t - x) + f(x)$. \\
$g$ paramètre la sécante passant par les points $(x, f(x))$ et $(y, f(y))$. \\
\begin{itemize}
    \item Sur $[x, y]$, RAF car $f$ est convexe. 
    \item Soit $t > y$. On pose $\lambda = \frac{y-x}{t-x} \neq 0 \in [0, 1]$. On a : 
    \begin{align*}
        \lambda t + (1-\lambda)x &= \frac{y-x}{t-x}t + \left( 1 - \frac{y-x}{t-x}\right)x \\
        &= \frac{t(y-x) + x(t-y))}{t-x} \\
        &= y
    \end{align*}
    Par convexité de $f$ : 
    \begin{align*}
        f(y) &= f(\lambda t + (1-\lambda)x) \\
        &\leq \lambda f(t) + (1-\lambda)f(x) \\
        \text{donc } f(t) &\geq \frac{1}{y} f(y) - \left( \frac{1}{y} - 1 \right)f(x) \\
        &= \frac{t-x}{y-x} f(y) - \left( \frac{t-x}{y-x} - 1 \right)f(x) \\
        &= \frac{t-x}{y-x} \times (f(y) - f(x)) + f(x) \\
        &= g(t)
    \end{align*}

    \item On raisonne de la même manière si $t \leq x < y$. 
\end{itemize}

\section{Inégalités des pentes}
\begin{tcolorbox}[title=Propostion 19.8, title filled=false, colframe=lightblue, colback=lightblue!10!white]
    Soit $f:I\to \mathbb{R}$ une fonction définie sur un intervalle $I$. 
    \begin{enumerate}
        \item $f$ est convexe si et seulement si pour tout $a \in I$, la fonction $x\mapsto \frac{f(x) - f(a)}{x - a}$ est croissante sur $I\backslash\{a\}$. 
        \item Si $f$ est convexe, alors pour tout $(a, b, c) \in I^3$ avec $a < b < c$, 
        \begin{align*}
            \frac{f(b) - f(a)}{b-a} \leq \frac{f(c) - f(a)}{c-a} \leq \frac{f(c) - f(b)}{c-b}
        \end{align*}
    \end{enumerate}
\end{tcolorbox}

\begin{enumerate}
    \item $\boxed{\Rightarrow}$ \\
    On suppose $f$ convexe. Soit $a \in I$ et $x < y$ dans $I\backslash\{a\}$.
    \begin{itemize}
        \item On suppose $x < a < y$. D'après (19.7) : 
        \begin{align*}
            f(y) \leq \frac{f(a) - f(x)}{a-x} \times (y-a)) + f(a)
        \end{align*}
        Donc :
        \begin{align*}
            \frac{f(y) - f(a)}{y-a} \geq \frac{f(a) - f(x)}{a-x}
        \end{align*}
        \item Si $x < a < y$, d'après (19.7) :
        \begin{align*}
            f(y) \geq \frac{f(a) - f(x)}{a-x} \times (y-a) + f(a)
        \end{align*}
        Donc :
        \begin{align*}
            \frac{f(y) - f(a)}{y-a} \geq \frac{f(a) - f(x)}{a-x}
        \end{align*}
        \item Les autres cas s'y ramènent. 
    \end{itemize} 

    \noindent$\boxed{\Leftarrow}$ \\
    On suppose que pour tout $a \in I$, $g_a:I\backslash\{a\}\to \mathbb{R}; x\mapsto \frac{f(x) - f(a)}{x - a}$ est croissante. \\
    Soit $x < y$ et $\lambda \in ]0, 1[$. On pose $a = \lambda y + (1-\lambda)x$. \\
    $g_a$ est croissante sur $I\backslash\{a\}$, donc :
    \begin{align*}
        g_a(x) \leq g_a(y)
    \end{align*}
    Donc :
    \begin{align*}
        \frac{f(x) - f(a)}{x-a} \leq \frac{f(y) - f(a)}{y-a}
    \end{align*}
    Donc : 
    \begin{align*}
        x-a &< 0 \text{ et } y-a > 0 \\
        (f(x) - f(a))(y-a) &\leq (f(y) - f(a))(x-a) \\
        \text{donc } f(a)(y-x) &\leq f(x)(y-a) - f(y)(x-a) \\
        \text{soit } f(a) &\leq f(x) \frac{y-a}{y-x} + f(y) \frac{a-x}{y-x} \\
        &= (1 - \lambda)f(x) + \lambda f(y) \\
    \end{align*}

    \item Soit $a < b < c$. \\
    \begin{align*}
        g_a(b) \leq g_a(c) = g_c(a) \leq g_c(b)
    \end{align*}
\end{enumerate}

\section{Continuité et dérivabilité des fonctions convexes}
\begin{tcolorbox}[title=Théorème 19.9, title filled=false, colframe=orange, colback=orange!10!white]
    Soit $f$ une fonction convexe sur un intervalle $I$ ouvert. La fonction $f$ est alors continue et possède des dérivées à gauche et à droite en tout point (où les limites osnt envisageables). Pour tout $a \in I$, on a
    \begin{align*}
        f'_g(a) \leq f'_d(a)
    \end{align*}
\end{tcolorbox}

\noindent Pour $a \in I$, on note encore $g_a:I\backslash\{a\}\to \mathbb{R}; x\mapsto \frac{f(x) - f(a)}{x - a}$. \\
Comme $g$ est définie à gauche et à droite de $a$ ($I$ est ouvert) et que $g$ est croissante sur $I\backslash\{a\}$, d'après le TLM $g$ admet des limites finies à gauche et à droite de $a$ et : 
\begin{align*}
    \lim_{a^+} g &= f'_d(a) \geq f'_g(a) = \lim_{a^-} g \\
    \forall x \neq a, f(x) &= \frac{f(x) - f(a)}{x-a}(x-a) + f(a) \\
    &\underset{x \to a^+}{\longrightarrow} f(a) \\
    &\underset{x \to a^-}{\longrightarrow} f(a)
\end{align*}

\setsection{10}
\section{Caractérisation des fonctions convexes par les variations de la dérivée}
\begin{tcolorbox}[title=Théorème 19.11, title filled=false, colframe=orange, colback=orange!10!white]
    Soit $f:I\to \mathbb{R}$ une fonction dérivable sur $I$. Alors $f$ est convexe si et seulement si $f'$ est croissante. 
\end{tcolorbox}

$\boxed{\Rightarrow}$ \\
On suppose $f$ convexe. Soit $x < y$. Soit $a$ tel que $x < a < y$. \\
D'après l'inégalité des pentes ($f$ est convexe), on a : 
\begin{align*}
    \frac{f(a) - f(x)}{a-x} \leq \frac{f(y) - f(x)}{y-x} \leq \frac{f(y) - f(a)}{y-a}
\end{align*}
En considérant les limiets $a\to x^+$ et $a\to y^-$ et par TCILPPL : 
\begin{align*}
    f'(x) \leq \frac{f(y) - f(x)}{y-x} \leq f'(y)
\end{align*}
Donc $f'$ est croissante. \\

$\boxed{\Leftarrow}$ \\
On suppose $f'$ croissante sur $I$. Soit $x < y$. Soit $a\in ]x, y[$. \\
On applique deux fois le TAF : on choisit $\alpha \in ]x, a[ \text{ et } \beta \in ]a, y[$ tels que :
\begin{align*}
    \frac{f(a) - f(x)}{-x+a} &= f'(\alpha) \text{ et } \frac{f(y) - f(a)}{y-a} = f'(\beta) \\
\end{align*}
Comme $f'$ est croissante, on a $f'(\alpha) \leq f'(\beta)$, soit :
\begin{align*}
    \frac{f(a) - f(x)}{a-x} &\leq \frac{f(y) - f(a)}{y-a} \\
    \text{donc } f(a) &\leq \frac{a-x}{y-x}f(y) + \frac{y-a}{y-x}f(x) \\
\end{align*}
Comme $a \in ]x, y[$, $a = \lambda y + (1-\lambda)x$ et aussi : 
\begin{align*}
    f(a) = f(\lambda y + (1-\lambda)x) \leq \lambda f(y) + (1-\lambda)f(x)
\end{align*}
Donc $f$ est convexe (sur $I$). 

\setsection{12}
\section{Caractérisation des fonctions convexes par les tangentes}
\begin{tcolorbox}[title=Propostion 19.13, title filled=false, colframe=lightblue, colback=lightblue!10!white]
    Soit $f:I\to \mathbb{R}$ une fonction dérivable. Alors $f$ est convexe sur $I$ si et seulement si le graphe de $f$ est situé au-dessus de toutes ses tangentes. 
\end{tcolorbox}

$\boxed{\Rightarrow}$ \\
On suppose $f$ convexe. Soit $a \in I$ et soit $\varphi:\mathbb{R}\to \mathbb{R}; t\mapsto f'(a)(t-a) + f(a)$. \\
On pose $h = f - \varphi \in \mathcal{D}^1(I, \mathbb{R})$ et $h' = f' - f'(a)$. \\
Or $f$ est convexe donc $f'$ est croissante sur $I$. Donc : 
\begin{align*}
    \begin{array}{|c|ccc|}
        \hline
        a &  &  &  \\
        \hline
        h' & - & 0 & + \\
        \hline
        h & \searrow & 0 & \nearrow \\
        \hline
        h & \multicolumn{3}{c|}{+} \\
        \hline
    \end{array}
\end{align*}

$\boxed{\Leftarrow}$ \\
Soit $x < y$ et $a = \lambda y + (1-\lambda)x \in ]x, y[$. \\
Par hypothèse, le graphe de $f$ est situé au-dessus de sa tangente en $a$. 
\begin{align*}
    \forall t \in I, f(t) \geq f'(a)(t-a) + f(a)
\end{align*}
En particulier :
\begin{align*}
    f(x) &\geq f'(a)(x-a) + f(a) \\
    f(y) &\geq f'(a)(y-a) + f(a)
\end{align*}
Donc : 
\begin{align*}
    (y-a)f(x) + (a-x)f(y) &\geq (y-a)f(a) \\
    \text{donc } f(a) &\leq \frac{y-a}{y-x}f(x) + \frac{a-x}{y-x}f(y) \\
    &= (1-\lambda)f(x) + \lambda f(y)
\end{align*}

\setsection{16}
\section{Somme de fonctions convexes}
\begin{tcolorbox}[title=Propostion 19.17, title filled=false, colframe=lightblue, colback=lightblue!10!white]
    La somme de deux fonctions convexes et convexe. 
\end{tcolorbox}

\noindent Soit $f$ et $g$ convexes. Soit $x < y$ et $a = \lambda x + (1-\lambda)y \in ]x, y[$. \\
On a :
\begin{align*}
    f(a) &\leq \lambda f(x) + (1-\lambda)f(y) \\
    g(a) &\leq \lambda g(x) + (1-\lambda)g(y)
\end{align*}
Donc :
\begin{align*}
    (f+g)(a) \leq \lambda (f+g)(x) + (1-\lambda)(f+g)(y)
\end{align*}
Donc $f+g$ est convexe.

\section{Composition de fonctions convexes}
\begin{tcolorbox}[title=Propostion 19.18, title filled=false, colframe=lightblue, colback=lightblue!10!white]
    Soit $f:I\to J$ et $g:J\to \mathbb{R}$ deux fonctions convexes avec $g$ croissante. Alors $g\circ f$ est convexe sur $I$. 
\end{tcolorbox}

\noindent Soit $x < y$ et $a = \lambda x + (1-\lambda)y \in ]x, y[$. \\
On a :
\begin{align*}
    f(a) &\leq \lambda f(x) + (1-\lambda)f(y) \\
    \text{donc } g\circ f(a) &\leq g(\lambda f(x) + (1-\lambda)f(y)) \\
    &\leq \lambda (g\circ f(x)) + (1-\lambda)(g\circ f(y))
\end{align*}
Donc $g\circ f$ est convexe. 

\setsection{18}
\section{Réciproque de fonctions convexes}
\begin{tcolorbox}[title=Propostion 19.19, title filled=false, colframe=lightblue, colback=lightblue!10!white]
    Soit $f:I\to J$ une fonction convexe bijective avec $I$ ouvert. Alors $g=f^{-1}$ est soit concave, soit convexe sur $J$. 
\end{tcolorbox}

\noindent Comme $f$ est convexe sur $I$ ouvert, $f$ est continue sur $I$ (19.9). \\
Or $f$ est bijective, donc $f$ est strictement monotone sur $I$ (15.72). \\
\begin{itemize}
    \item Supposons $f$ strictement croissante sur $I$. Soit $x < y$ dans $J = f(I)$. Soit $\lambda \in ]0, 1[$. Alors $g$ est strictement croissante. \\
    On pose $x = f(a)$ et $y = f(b)$. On a :
    \begin{align*}
        f(\lambda a + (1-\lambda)b) &\leq \lambda f(a) + (1-\lambda)f(b) \\
        &\leq \lambda x + (1-\lambda)y
    \end{align*}
    Or $g$ est strictement croissante, donc : 
    \begin{align*}
        \lambda g(x) + (1-y) g(y) &= \lambda a + (1-\lambda)b \\
        &\leq g(\lambda x + (1-\lambda)y)
    \end{align*}
    Donc $g$ est concave sur $J$. 

    \item Si $f$ est strictement décroissante (et donc $g$ strictement décroissante), alors $g$ est concave sur $J$. 
\end{itemize}

\section{Extrema des fonctions convexes}
\begin{tcolorbox}[title=Propostion 19.20, title filled=false, colframe=lightblue, colback=lightblue!10!white]
    Soit $f$ une fonction convexe définie par un intervalle $I$ ouvert. Alors $f$ admet un minimum global en un point $a$ si et seulement si $a$ est un point critique. 
\end{tcolorbox}

$\boxed{\Rightarrow}$ \\
RAF \\

$\boxed{\Leftarrow}$ \\
On suppose que $a$ est un point critique. Donc $f'(a) = 0$. \\
Or le graphe de $f$ est situé au-dessus de sa tangente en $a$, soit : 
\begin{align*}
    \forall x \in I, f(x) \geq \underbrace{f'(a)}_{0}(x-a) + f(a) = f(a)
\end{align*}
Donc $f(a)$ est un minimum global de $f$.

\setsection{23}
\section{Inégalité de Jensen}
\begin{tcolorbox}[title=Théorème 19.24, title filled=false, colframe=orange, colback=orange!10!white]
    Soit $f:I\to \mathbb{R}$ une fonction convexe. Soit $n\geq 2$. Pour tout $(x_1, \ldots, x_n) \in I^n$ et $(\lambda_1, \ldots, \lambda_n) \in [0; 1]^n$ avec $\sum\limits_{k=1}^{n} \lambda_k = 1$, alors
    $$f \left( \sum_{k=1}^{n} \lambda_k x_k \right) \leq \sum_{k=1}^{n} \lambda_k f(x_k)$$
\end{tcolorbox}

\noindent Par récurrence. 
\begin{itemize}
    \item Pour $n = 2$, RAF (cf. définition)
    \item On suppose la propriété vraie au rang $n$. \\
    Soit $(x_1, \ldots, x_{n+1}) \in I^{n+1}, (\lambda_1, \ldots, \lambda_{n+1}) \in [0, 1]^{n+1}$ avec $\sum\limits_{i=1}^{n+1} \lambda_i = 1$. \\
    Si $\lambda_{n+1} = 0$, on applique directement l'hypothèse au rang $n$ (RAF). \\
    On suppose $\lambda_{n+1} \neq 0$. On a : 
    \begin{align*}
        f \left( \sum_{i=1}^{n+1} \lambda_i x_i \right) &= f \left( \sum_{i=1}^{n-1} \lambda_i x_i + \lambda_n x_n + \lambda_{n+1}x_{n+1} \right) \\
        &= f \left( \sum_{i=1}^{n-1} \lambda_i x_i + (\lambda_n + \lambda_{n+1}) \times \left( \frac{\lambda_n}{\lambda_n + \lambda_{n+1}}x_n + \frac{\lambda_{n+1}}{\lambda_n + \lambda_{n+1}}x_{n+1} \right) \right) \\
        &\leq \sum_{i=1}^{n-1} \lambda_i f(x_i) + (\lambda_n + \lambda_{n+1}) \times f\left( \frac{\lambda_n}{\lambda_n + \lambda_{n+1}}x_n + \frac{\lambda_{n+1}}{\lambda_n + \lambda_{n+1}}x_{n+1} \right) \\
        &\leq \sum_{i=1}^{n-1} \lambda_i f(x_i) + (\lambda_n + \lambda_{n+1}) \times \left( \frac{\lambda_n}{\lambda_n + \lambda_{n+1}}f(x_n) + \frac{\lambda_{n+1}}{\lambda_n + \lambda_{n+1}}f(x_{n+1}) \right) \\
        &= \sum_{i=1}^{n} \lambda_i f(x_i)
    \end{align*}
\end{itemize}

\section{Exemple - Inégalité arithmético-géométrique}
\begin{tcolorbox}[title=Exemple 19.25, title filled=false, colframe=darkgreen, colback=darkgreen!10!white]
    Soit $n \geq 1$. Pour tout $(x_1, \ldots, x_n) \in (\mathbb{R}_+^*)^n$
    $$\frac{n}{\sum\limits_{k=1}^{n} \frac{1}{x_k}} \leq \sqrt[n]{\prod_{k=1}^{n} x_k} \leq \frac{1}{n}\sum_{k=1}^{n} x_k$$
\end{tcolorbox}

\noindent La fonction logarithme est concave sur $\mathbb{R}_+^*$. Soit $(x_1, \ldots, x_n) \in (\mathbb{R}_+^*)^n$. \\
On remarque que $\sum\limits_{k=1}^{n} \frac{1}{n} = 1$. D'après l'inégalité de Jensen :
\begin{align*}
    \ln \left( \frac{1}{n} \sum_{k=1}^{n} x_k \right) &\geq \frac{1}{n} \sum_{k=1}^{n} \ln(x_k) \\
    &= \frac{1}{n} \ln \left( \prod_{k=1}^{n} x_k \right) \\
    &= \ln \left( \sqrt[n]{\prod_{k=1}^{n} x_k} \right) \\
\end{align*} 
On compose alors par $\exp$ (strictement croissante). \\
D'après le résultat précédent appliqué à $\left( \frac{1}{x_1}, \ldots, \frac{1}{x_n} \right)$ : 
\begin{align*}
    0 < \frac{1}{\sqrt[n]{\prod\limits_{k=1}^{n} x_k}} = \sqrt[n]{\prod_{k=1}^{n} \frac{1}{x_k}} \leq \frac{1}{n} \sum_{k=1}^{n} \frac{1}{x_k} \\
\end{align*}
Donc ($x\mapsto \frac{1}{x}$ est strictement décroissante sur $\mathbb{R}_+^*$) :
\begin{align*}
    \frac{n}{\sum\limits_{k=1}^{n} \frac{1}{x_k}} \leq \sqrt[n]{\prod_{k=1}^{n} x_k}
\end{align*}


\end{document}