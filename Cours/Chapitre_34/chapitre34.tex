\documentclass[../main.tex]{subfiles}

\begin{document}
\setcounter{chapter}{33}
\chapter{Espaces préhilbertiens réels}
\tableofcontents
\clearpage

\setsection{3}
\section{Produit scalaire canonique sur $\mathbb{R}^n$}
\begin{tcolorbox}[title=Théorème 34.4, title filled=false, colframe=orange, colback=orange!10!white]
    L'application
    $$\mathbb{R}^n \times \mathbb{R}^n \rightarrow \mathbb{R} ;(X, Y) \mapsto{ }^{\mathrm{t}} X Y=\sum_{k=1}^n x_k y_k$$
    est un produit scalaire sur $\mathbb{R}^n$, appelé produit scalaire canonique.
\end{tcolorbox}

\noindent Pour $X, Y \in \mathbb{R}^n$ :
\begin{itemize}
    \item $^tXY\in \mathbb{R}$ donc ${^tY}X = ^t({^tX}Y) = {^tX}Y$
    \item bilinéarité : RAF
    \item $^tXX = \sum\limits_{k=1}^{n} x_k^2 \geq 0$ et $\sum\limits_{k=1}^{n} x_k^2 = 0 \Leftrightarrow \forall k\in \llbracket 1, n \rrbracket, x_k = 0 \Leftrightarrow x = 0$
\end{itemize}


\end{document}