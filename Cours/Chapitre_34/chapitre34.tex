\documentclass[../main.tex]{subfiles}

\begin{document}
\setcounter{chapter}{33}
\chapter{Espaces préhilbertiens réels}
\tableofcontents
\clearpage

\setsection{3}
\section{Produit scalaire canonique sur $\mathbb{R}^n$}
\begin{tcolorbox}[title=Théorème 34.4, title filled=false, colframe=orange, colback=orange!10!white]
    L'application
    $$\mathbb{R}^n \times \mathbb{R}^n \rightarrow \mathbb{R} ;(X, Y) \mapsto{ }^{\mathrm{t}} X Y=\sum_{k=1}^n x_k y_k$$
    est un produit scalaire sur $\mathbb{R}^n$, appelé produit scalaire canonique.
\end{tcolorbox}

\noindent Pour $X, Y \in \mathbb{R}^n$ :
\begin{itemize}
    \item $^tXY\in \mathbb{R}$ donc ${^tY}X = ^t({^tX}Y) = {^tX}Y$
    \item bilinéarité : RAF
    \item $^tXX = \sum\limits_{k=1}^{n} x_k^2 \geq 0$ et $\sum\limits_{k=1}^{n} x_k^2 = 0 \Leftrightarrow \forall k\in \llbracket 1, n \rrbracket, x_k = 0 \Leftrightarrow x = 0$
\end{itemize}

\section{Exemple}
\begin{tcolorbox}[title=Exemple , title filled=false, colframe=darkgreen, colback=darkgreen!10!white]
    Montrer que 
    \begin{align*}
        (X, Y) \mapsto {^tX} \begin{pmatrix}
            2 & 1 \\
            1 & 2
        \end{pmatrix} Y
    \end{align*}
    est un exemple de produit scalaire sur $\mathbb{R}^2$ distinct du produit scalaire usuel. 
\end{tcolorbox}

\begin{itemize}
    \item bilinéarité : RAF
    \item Pour $X, Y\in \mathbb{R}^2, {^tX}\begin{pmatrix}
        2 & 1 \\
        1 & 2
    \end{pmatrix} Y\in \mathbb{R}$, donc : 
    \begin{align*}
        ^tX \begin{pmatrix}
            2 & 1 \\
            1 & 2
        \end{pmatrix} Y &= {^t \left( {^t}X \begin{pmatrix}
            2 & 1 \\
            1 & 2
        \end{pmatrix} Y \right)} \\
        &= ^tY {^t \begin{pmatrix}
            2 & 1 \\
            1 & 2
        \end{pmatrix}} X \\
        &= {^tY} \begin{pmatrix}
            2 & 1 \\
            1 & 2
        \end{pmatrix} X
    \end{align*}
    On a : 
    \begin{align*}
        {^tX} \begin{pmatrix}
            2 & 1 \\
            1 & 2
        \end{pmatrix} X &= \begin{pmatrix}
            x & y
        \end{pmatrix} \begin{pmatrix}
            2x + y \\
            x + 2y
        \end{pmatrix} \\
        &= 2x^2 + 2xy + 2y^2 \\
        &= \underbrace{2(x^2 + xy + y^2)}_{\geq 0 \text{ car } x^2 + xy + y^2 \geq |xy|}
    \end{align*}
    En particulier, si $^tX \begin{pmatrix}
        2 & 1 \\
        1 & 2
    \end{pmatrix} X = 0$ alors $|xy| = 0$, puis $x = y = 0$. \\
    La forme est définie positive. 
\end{itemize}

\setsection{13}
\section{Identités remarquables}
\begin{tcolorbox}[title=Propostion 34.14, title filled=false, colframe=lightblue, colback=lightblue!10!white]
    Pour tout $(x, y) \in E^2$, on a :
    $$\|x+y\|^2=\|x\|^2+2\langle x, y\rangle+\|y\|^2$$
    et
    $$\langle x+y, x-y\rangle=\|x\|^2-\|y\|^2$$
\end{tcolorbox}

\begin{align*}
    \|x+y\|^2 &= \langle x+y, x+y\rangle \\
    &= \langle x, x\rangle + \langle x, y\rangle + \langle y, x\rangle + \langle y, y\rangle \text{ (bilinéarité)} \\
    &= \|x\|^2 + 2\langle x, y\rangle + \|y\|^2 \text{ (symétrie)}
\end{align*}
Idem pour la seconde identité. 

\section{Proposition 34.15 bis}
\begin{tcolorbox}[title=Propostion 34.15 bis, title filled=false, colframe=lightblue, colback=lightblue!10!white]
    Soit $\|.\|$ une norme euclidienne. Soit $x\in E, \lambda\in \mathbb{R}$. 
    \begin{itemize}
        \item $\|\lambda x\| = |\lambda|\|x\|$
        \item $\|x\| = 0 \Leftrightarrow x = 0$
    \end{itemize}
\end{tcolorbox}

\begin{align*}
    \|\lambda x\|^2 &= \langle \lambda x, \lambda x\rangle \\
    &= \lambda^2 \|x\|^2
\end{align*}

\section{Inégalité de Cauchy-Schwarz, inégalité triangulaire}
\begin{tcolorbox}[title=Théorème 34.16, title filled=false, colframe=orange, colback=orange!10!white]
    Soit $E$ un espace préhilbertien réel et $x$ et $y$ dans $E$.
    \begin{itemize}
        \item Inégalité de Cauchy-Schwarz :
        $$|\langle x, y\rangle| \leq\|x\| \times\|y\|$$
        avec égalité si et seulement si $x$ et $y$ sont colinéaires.
        \item Inégalité triangulaire :
        $$\|\|x\|-\| y\|\|\leq\| x+y\| \leq\|x\|+\|y\|$$
        l'inégalité de droite est une égalité si et seulement si $x$ et $y$ sont positivement colinéaires.
        \item Inégalité triangulaire, version distance :
        $$|d(x, y)-d(y, z)| \leq d(y, z) \leq d(x, y)+d(y, z)$$
    \end{itemize}
\end{tcolorbox}

\begin{itemize}
    \item Si $x = 0$, l'inégalité est vérifiée pour tout $y\in E$. \\
    On suppose $x\neq 0$. On considère, pour $y\in E$ fixé : 
    \begin{align*}
        \varphi:\mathbb{R}\to \mathbb{R}; t&\mapsto \|tx + y\|^2 \\
        &= \langle tx+y, tx+y\rangle \\
        &= t^2 \|x\|^2 + 2t\langle x, y\rangle + \|y\|^2
    \end{align*}
    $f$ est une fonction polynomiale de degré $2$ ($\|x|\neq 0$) positive donc de discriminant $\Delta \leq 0$. \\
    Or $\Delta = 4\langle x, y\rangle^2 - 4\|x\|^2\|y\|^2$. D'où le résultat. \\
    Si $\Delta = 0$, alors $f$ s'annule une unique fois en $t_0$. On a alors $\|t_0 x + y\|^2 = 0$. \\
    Donc $t_0 x + y = 0$. \\
    Donc $(x, y)$ est liée. \\
    Réciproquement, si $(x, y)$ est liée, alors $y = t_0 x$ ($x\neq 0$) et on a encore $f(t_0) = 0$. 

    \item Pour $(x, y)\in E^2$ : 
    \begin{align*}
        \|x+y\| \leq \|x\| + \|y\| &\Leftrightarrow \|x+y\|^2\leq (\|x\| + \|y\|)^2 \\
        &\Leftrightarrow \|x\|^2 + 2\langle x, y\rangle + \|y\|^2 \leq \|x\|^2 + 2\|x\|\|y\| + \|y\|^2 \\
        &\Leftrightarrow \langle x, y\rangle \leq \|x\|\|y\|
    \end{align*}
    La dernière assertion est vraie d'après l'inégalité de Cauchy-Schwarz, la première l'est tout autant. \\
    RAS pour l'inégalité généralisée. \\
    Si $\|x+y\| = \|x\| + \|y\|$, le cas d'égalité de Cauchy-Schwarz affirme que (par ex) : 
    \begin{align*}
        y = \alpha x, \alpha\in \mathbb{R}
    \end{align*}
    Mais alors (en supposant $x \neq 0$) : 
    \begin{align*}
        \| 1 + \alpha\| \|x\| = \|x + y\| = (1 + |\alpha|) \|x\|
    \end{align*}
    Donc $|1 + \alpha| = 1 + |\alpha|$. \\
    Nécessairement, $\alpha\geq 0$
\end{itemize}

\section{Exemple}
\begin{tcolorbox}[title=Exemple 34.17, title filled=false, colframe=darkgreen, colback=darkgreen!10!white]
    Pour tout $\left(x_1, \ldots, x_n\right) \in \mathbb{R}^n$,
    $$\left(\sum_{k=1}^n x_k\right)^2 \leq n \sum_{k=1}^n x_k^2$$
    avec égalité si et seulement si $x_1 = \cdots = x_n$.
\end{tcolorbox}

\noindent On munit $E = \mathbb{R}^n$ de son produit scalaire canonique. \\
On applique l'inégalité de Cauchy-Schwarz aux vecteurs $\begin{pmatrix}
    x_1 \\ \vdots \\ x_n
\end{pmatrix}$ et $\begin{pmatrix}
    1 \\ \vdots \\ 1
\end{pmatrix}$. 
\begin{align*}
    \left| \left\langle \begin{pmatrix}
        x_1 \\ \vdots \\ x_n
    \end{pmatrix}, \begin{pmatrix}
        1 \\ \vdots \\ 1
    \end{pmatrix} \right\rangle \right| &\leq \left\| \begin{pmatrix}
        x_1 \\ \vdots \\ x_n
    \end{pmatrix} \right\| \left\| \begin{pmatrix}
        1 \\ \vdots \\ 1
    \end{pmatrix} \right\| \\
    \left| \sum_{k=1}^{n} 1\times x_k \right| &\leq \sqrt{\sum_{k=1}^{n} x_k^2} \sqrt{\sum_{k=1}^{n} 1^2} \\
    &= \sqrt{n} \sqrt{\sum_{k=1}^{n} x_k^2} \\
\end{align*}

\section{Exemple}
\begin{tcolorbox}[title=Exemple 34.18, title filled=false, colframe=darkgreen, colback=darkgreen!10!white]
    Soit $a$ et $b$ deux réels tels que $a<b$. Pour tout $f \in \mathcal{C}^1([a ; b], \mathbb{R})$, on a
    $$f(b)^2-f(a)^2 \leq 2 \sqrt{\int_a^b f(t)^2 \,dt} \sqrt{\int_a^b f^{\prime}(t)^2 \,dt}$$
\end{tcolorbox}

\noindent On munit $\mathcal{C}^0([a, b], \mathbb{R})$ du produit scalaire usuel : 
\begin{align*}
    \forall (f, g)\in \mathcal{C}^0([a, b], \mathbb{R})^, \langle f, g \rangle &= \int_{a}^{b} f(t) g(t) \,dt
\end{align*}
On applique l'inégalité de Cauchy-Schwarz aux vecteurs $f$ et $f'$ : 
\begin{align*}
    \|f\|\times \|f'\| &\geq |\langle f, f'\rangle| \\
    &= \left| \int_{a}^{b} f(t) f'(t) \,dt \right| \\
    &= \left| \left[ \frac{f^2(t)}{2} \right]_a^b \right| \\
    &= \left|  \frac{f(b)^2 - f(a)^2}{2} \right|
\end{align*}

\setsection{19}
\section{Vecteur orthogonal à tout vecteur}
\begin{tcolorbox}[title=Théorème 34.20, title filled=false, colframe=orange, colback=orange!10!white]
    Dans un espace préhilbertien réel, le vecteur nul est le seul vecteur orthogonal à tout vecteur. 
\end{tcolorbox}

\indent \boxed{\Rightarrow} RAF \\
\indent \boxed{\Leftarrow} Si $x$ est orthogonal à tout vecteur de $E$, alors $x\bot x$, donc $\|x\|^2 = 0$, donc $x = 0$. 

\section{Exemple}
\begin{tcolorbox}[title=Exemple 34.21, title filled=false, colframe=darkgreen, colback=darkgreen!10!white]
    Exemple 34.21 Montrer que pour le produit scalaire
    $$
(X, Y) \mapsto^t X\begin{pmatrix}
    2 & 1 \\
    1 & 2
    \end{pmatrix} Y
    $$
    sur $\mathbb{R}^2$, la base canonique n'est pas orthormale, mais la famille $\left(\frac{1}{\sqrt{2}}(1,0), \frac{1}{\sqrt{6}}(1,-2)\right)$ l'est.
\end{tcolorbox}

\noindent $\| (1, 0)\| = \sqrt{2} = \| (0, 1)\|$ \\
$\langle (1, 0), (0, 1)\rangle = 1$ \\
Donc $\\| \frac{1}{\sqrt{2}}(1, 0)\| = 1$ \\
On a $\| (1, -2)\| = \sqrt{6}$ \\
Et $\langle (1, 0), (1, -2)\rangle = 0$. 

\setsection{22}
\section{Exemple}
\begin{tcolorbox}[title=Exemple 34.23, title filled=false, colframe=darkgreen, colback=darkgreen!10!white]
    La famille des fonctions $t \mapsto \sin (n t)$, où $n \in \mathbb{N}$ est orthonormale dans $\mathcal{C}([0 ; 2 \pi], \mathbb{R})$ pour le produit scalaire
    $$(f, g) \mapsto \frac{1}{\pi} \int_0^{2 \pi} f(t) g(t) \,dt$$
\end{tcolorbox}

\noindent On note pour $n\in \mathbb{N}$, $f_n: [0 ; 2\pi] \to \mathbb{R};x\mapsto \sin(nx)$. \\
Soit $p\neq n$. \\
\begin{align*}
    \langle f_p, f_n \rangle &= \frac{1}{\pi} \int_{0}^{2\pi} f_n(t) f_p(t) \,dt \\
    &= \frac{1}{\pi} \int_{0}^{2\pi} \sin(nt) \sin(pt) \,dt \\
    &= \frac{1}{2\pi} \int_{0}^{2\pi} (\cos((n-m)t) - \cos((n+m)t) \,dt \\
    &= \frac{1}{2\pi} \left[ \frac{1}{n-p} \sin((n-p)t) - \frac{1}{n+p}\sin((n+p)t) \right]^{2\pi}_0 \text{ ($n\neq p$)} \\
    &= 0
\end{align*}
Si $n = p$ alors : 
\begin{align*}
    \|f_n\| &= \frac{1}{2\pi} \int_{0}^{2\pi} (1 - \cos(2nt)) \,dt \\
    &= 1
\end{align*}
Donc $(f_n)$ est bien une famille orthonormée. 

\section{Exemple}
\begin{tcolorbox}[title=Exemple 34.24, title filled=false, colframe=darkgreen, colback=darkgreen!10!white]
    Dans $\mathcal{C}([-1 ; 1], \mathbb{R})$, l'ensemble des fonctions paires et l'ensemble des fonctions impaires sont deux sous-espaces vectoriels orthogonaux pour le produit scalaire
    $$(f, g) \mapsto \int_{-1}^1 f(t) g(t) \,dt$$
\end{tcolorbox}

\noindent D'après le chapitre 4, si $f$ est impaire : 
\begin{align*}
    \int_{-1}^{1} f(t) \,dt = 0
\end{align*}
Si $f$ est paire et $g$ impaire, alors $fg$ est impaire et ainsi $\langle f, g \rangle = 0$. 

\section{Propriétés des familles orthogonales}
\begin{tcolorbox}[title=Théorème 34.25, title filled=false, colframe=orange, colback=orange!10!white]
    Soit $E$ un espace préhilbertien réel.
    \begin{enumerate}
        \item Théorème de Pythagore : pour tout $(x, y) \in E^2, x$ et $y$ sont orthogonaux ssi $\|x+y\|^2=\|x\|^2+\|y\|^2$. De surcro ${ }^{\wedge}$ it, si $\left(x_1, \ldots, x_n\right)$ est une famille orthogonale de vecteurs de $E$, alors
        $$\left\|\sum_{k=1}^n x_k\right\|^2=\sum_{k=1}^n\left\|x_k\right\|^2$$
        \item Toute famille orthogonale de vecteurs non nuls de $E$ est libre. En particulier, si $E$ est de dimension finie $n$ non nulle, toute famille orthogonale de $n$ vecteurs est une base orthogonale.
    \end{enumerate}
\end{tcolorbox}

\begin{enumerate}
    \item RAF
    \item Soit $(e_i)_{i\in I}$ une famille orthogonale. Soit $(\lambda_i)_{i\in I}$ une famille de scalaires à support fini. On suppose en outre que : 
    \begin{align*}
        \forall i\in I, e_i\neq 0
    \end{align*}
    On suppose que $\sum\limits_{i\in I} \lambda_i e_i = 0$. \\
    Soit $j\in I$. 
    \begin{align*}
        \langle \sum_{i\in I} \lambda_i e_i, e_j \rangle &= 0 \\
        &= \sum_{i\in I} \lambda_i \underbrace{\langle e_i, e_j \rangle}_{= 0 \text{ pour } i\neq j} \\
        \text{donc } \lambda_j \| e_j \|^2 &= 0 \\
        \text{donc } \lambda_j &= 0
    \end{align*}
    Donc la famille est libre. 
\end{enumerate}


\end{document}