\documentclass[../main.tex]{subfiles}

\begin{document}
\setcounter{chapter}{33}
\chapter{Espaces préhilbertiens réels}
\tableofcontents
\clearpage

\setsection{3}
\section{Produit scalaire canonique sur $\mathbb{R}^n$}
\begin{tcolorbox}[title=Théorème 34.4, title filled=false, colframe=orange, colback=orange!10!white]
    L'application
    $$\mathbb{R}^n \times \mathbb{R}^n \rightarrow \mathbb{R} ;(X, Y) \mapsto{ }^{\mathrm{t}} X Y=\sum_{k=1}^n x_k y_k$$
    est un produit scalaire sur $\mathbb{R}^n$, appelé produit scalaire canonique.
\end{tcolorbox}

\noindent Pour $X, Y \in \mathbb{R}^n$ :
\begin{itemize}
    \item $^tXY\in \mathbb{R}$ donc ${^tY}X = ^t({^tX}Y) = {^tX}Y$
    \item bilinéarité : RAF
    \item $^tXX = \sum\limits_{k=1}^{n} x_k^2 \geq 0$ et $\sum\limits_{k=1}^{n} x_k^2 = 0 \Leftrightarrow \forall k\in \llbracket 1, n \rrbracket, x_k = 0 \Leftrightarrow x = 0$
\end{itemize}

\section{Exemple}
\begin{tcolorbox}[title=Exemple , title filled=false, colframe=darkgreen, colback=darkgreen!10!white]
    Montrer que 
    \begin{align*}
        (X, Y) \mapsto {^tX} \begin{pmatrix}
            2 & 1 \\
            1 & 2
        \end{pmatrix} Y
    \end{align*}
    est un exemple de produit scalaire sur $\mathbb{R}^2$ distinct du produit scalaire usuel. 
\end{tcolorbox}

\begin{itemize}
    \item bilinéarité : RAF
    \item Pour $X, Y\in \mathbb{R}^2, {^tX}\begin{pmatrix}
        2 & 1 \\
        1 & 2
    \end{pmatrix} Y\in \mathbb{R}$, donc : 
    \begin{align*}
        ^tX \begin{pmatrix}
            2 & 1 \\
            1 & 2
        \end{pmatrix} Y &= {^t \left( {^t}X \begin{pmatrix}
            2 & 1 \\
            1 & 2
        \end{pmatrix} Y \right)} \\
        &= ^tY {^t \begin{pmatrix}
            2 & 1 \\
            1 & 2
        \end{pmatrix}} X \\
        &= {^tY} \begin{pmatrix}
            2 & 1 \\
            1 & 2
        \end{pmatrix} X
    \end{align*}
    On a : 
    \begin{align*}
        {^tX} \begin{pmatrix}
            2 & 1 \\
            1 & 2
        \end{pmatrix} X &= \begin{pmatrix}
            x & y
        \end{pmatrix} \begin{pmatrix}
            2x + y \\
            x + 2y
        \end{pmatrix} \\
        &= 2x^2 + 2xy + 2y^2 \\
        &= \underbrace{2(x^2 + xy + y^2)}_{\geq 0 \text{ car } x^2 + xy + y^2 \geq |xy|}
    \end{align*}
    En particulier, si $^tX \begin{pmatrix}
        2 & 1 \\
        1 & 2
    \end{pmatrix} X = 0$ alors $|xy| = 0$, puis $x = y = 0$. \\
    La forme est définie positive. 
\end{itemize}


\end{document}