\documentclass[../main.tex]{subfiles}

\begin{document}
\setcounter{chapter}{1}
\chapter{Logique}
\tableofcontents
\clearpage

\setcounter{section}{16}
\section{Equivalence logiques}
\subsection{Double négation}
\begin{displaymath}
    \begin{array}{|c|c|c|}
    \hline
    p & \lnot p & \lnot(\lnot p)\\ % Use & to separate the columns
    \hline % Put a horizontal line between the table header and the rest.
    V & F & V\\
    F & V & F\\
    \hline
    \end{array}
\end{displaymath}
On remarque que la première et la deuxième colonne sont identiques, on a donc :
$$p \iff \lnot(\lnot p)$$

\subsection{Commutativité}
\begin{displaymath}
    \begin{array}{|c|c|c|c|}
    \hline
    p & q & p \land q & q \land p\\ % Use & to separate the columns
    \hline % Put a horizontal line between the table header and the rest.
    V & V & V & V\\
    V & F & F & F\\
    F & V & F & F\\
    F & F & F & F\\
    \hline
    \end{array}
\end{displaymath}
On remarque que la troisième et la quatrième colonne sont identiques, on a donc :
$$p \land q \iff q \land p$$
Raisonnement analogue pour la disjonction $\lor$.

\subsection{Associativité}
\begin{displaymath}
    \begin{array}{|c|c|c|c|c|c|c|}
        \hline
        p & q & r & p \land q & (p \land q) \land r & q \land r & p \land (q \land r)\\ % Use & to separate the columns
        \hline % Put a horizontal line between the table header and the rest.
        V & V & V & V & V & V & V\\
        V & V & F & V & F & F & F\\
        V & F & V & F & F & F & F\\
        V & F & F & F & F & F & F\\
        F & V & V & F & F & V & F\\
        F & V & F & F & F & F & F\\
        F & F & V & F & F & F & F\\
        F & F & F & F & F & F & F\\
        \hline
    \end{array}
\end{displaymath}
On remarque que la cinquième et la septième colonne sont identiques, on a donc :
$$(p\land q)\land r\iff p \land (q\land r)$$
Raisonnement analogue pour la disjonction $\lor$.

\subsection{Loi de Morgan}
\begin{displaymath}
    \begin{array}{|c|c|c|c|c|c|c|}
    \hline
    p & q & p \land q & \lnot(p \land q) & \lnot p & \lnot q & (\lnot p)\lor(\lnot q)\\ % Use & to separate the columns
    \hline % Put a horizontal line between the table header and the rest.
    V & V & V & F & F & F & F\\
    V & F & F & V & F & V & V\\
    F & V & F & V & V & F & V\\
    F & F & F & V & V & V & V\\
    \hline
    \end{array}
\end{displaymath}
On remarque que la quatrième et la septième colonne sont identiques, on a donc :
$$\lnot (p\land q) \iff (\lnot p)\lor(\lnot q)$$
Raisonnement analogue pour $\lnot (p\lor q) \iff (\lnot p)\land(\lnot q)$

\subsection{Double implication}
\begin{displaymath}
    \begin{array}{|c|c|c|c|c|c|}
    \hline
    p & q & p \Leftrightarrow q & p \Rightarrow q & q \Rightarrow p & (p\Rightarrow q)\land(q\Rightarrow p)\\ % Use & to separate the columns
    \hline % Put a horizontal line between the table header and the rest.
    V & V & V & V & V & V\\
    V & F & F & F & V & F\\
    F & V & F & V & F & F\\
    F & F & V & V & V & V\\
    \hline
    \end{array}
\end{displaymath}
On remarque que la troisième et la sixième colonne sont identiques, on a donc :
$$(p\Leftrightarrow q) \iff ((p\Rightarrow q)\land (q \Rightarrow p))$$

\subsection{Distributivité}
\begin{displaymath}
    \begin{array}{|c|c|c|c|c|c|c|c|}
    \hline
    p & q & r & p \land q & r \lor (p \land q) & r\lor p & r \lor q & (r\lor p) \land (r \lor q)\\ % Use & to separate the columns
    \hline % Put a horizontal line between the table header and the rest.
    V & V & V & V & V & V & V & V\\
    V & V & F & V & V & V & V & V\\
    V & F & V & F & V & V & V & V\\
    V & F & F & F & F & V & F & F\\
    F & V & V & F & V & V & V & V\\
    F & V & F & F & F & F & V & F\\
    F & F & V & F & V & V & V & V\\
    F & F & F & F & F & F & F & F\\
    \hline
    \end{array}
\end{displaymath}
On remarque que la cinquième et la huitième colonne sont identiques, on a donc :
$$r\lor (p \land q) \iff (r\lor p)\land (r\lor q)$$

\end{document}