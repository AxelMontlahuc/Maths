\documentclass[../main.tex]{subfiles}

\begin{document}
\setcounter{chapter}{14}
\chapter{Limites et continuité}
\tableofcontents
\clearpage

\setcounter{section}{5}
\section{Limite en un point du domaine}
\begin{tcolorbox}[title=Propostion 15.6, title filled=false, colframe=lightblue, colback=lightblue!10!white]
    Si $a \in X$ et si $f(x)$ admet une limite finie en $a$, alors cette limite est nécessairement égale à $f(a)$. 
\end{tcolorbox}

Comme $f(x)$ admet une limite finie $b$ quand $x \to a$ : 
\begin{align*}
    \forall \epsilon, \exists \nu > 0, \forall x \in X, |x - a| \leq \nu \Rightarrow |f(x) - b| \leq \epsilon
\end{align*}
Or pour tout $\epsilon > 0$ : 
\begin{align*}
    |a - a| \leq \nu \text{ (quelque soit $\nu$)}
\end{align*}
Donc : 
\begin{align*}
    \forall \epsilon, |f(a) - b| \leq \epsilon
\end{align*}
Donc $\boxed{f(a) = b}$. 

\setcounter{section}{14}
\section{Comparaison des limites de deux fonctions coincidant au voisinage de $a$}
\begin{tcolorbox}[title=Propostion 15.15, title filled=false, colframe=lightblue, colback=lightblue!10!white]
    Soit $f$ et $g$ deux fonctions coincidant au voisinage d'un point $a$. Alors, si $f$ admet une limite (finie ou infinie) en $a$, alors $g$ aussi et
    $$\lim_{x\to a} f(x) = \lim_{x\to a} g(x)$$
\end{tcolorbox}

On choisit $W \in \mathcal{V}(a)$ tel que $W \cap X = W \cap Y$ et $\left. f \right|_{W \cap X} = \left. g \right|_{W \cap Y}$. \\
Soit $b \in \overline{\mathbb{R}}$ tel que $f(x)$ tend vers $b$ quand $x \to a$. \\
Soit $V \in \mathcal{V}(b)$. On choisit $U \in \mathcal{V}(a)$ tel que : 
\begin{align*}
    f(U \cap X) \subset V
\end{align*}
Or $$W \cap U \in \mathcal{V}(a) \text{ et } \subset{f(W \cap U \cap X)}_{g(W \cap U \cap Y)} \subset V$$
Donc $g$ admet une limite en $a$ égale à $b$

\setcounter{section}{16}
\section{Unicité de la limite, cas réel}
\begin{tcolorbox}[title=Théorème 15.17, title filled=false, colframe=orange, colback=orange!10!white]
    Soit $a \in \overline{X}$ et $f$ une fonction réelle. Sous réserve d'existence, la limite de $f(x)$, lorsque $x$ tend vers $a$ est unique. 
\end{tcolorbox}

Par l'absurde. On suppose que $f$ possède deux limites $l \neq l'$ en $a$. \\
On choisit $u \in \mathcal{V}(l)$ et $u' \in \mathcal{V}(l')$ tels que $u \cap u' = \emptyset$. \\
Par définition, on choisit $(W, W') \in \mathcal{V}(a)^2$ tels que $f(W \cap X) \subset U$ et $f(W' \cap X) \subset U'$. \\
Or $\underbrace{W \cap W'}_{\neq \emptyset} \not \in \mathcal{V}(a)$ et $f(\underbrace{W \cap W' \cap X}_{\neq \emptyset}) \subset U \cap U' = \emptyset$. \\
Absurde.

\setcounter{section}{22}
\section{Propostion}
\begin{tcolorbox}[title=Propostion 15.23, title filled=false, colframe=lightblue, colback=lightblue!10!white]
    Soit $a \in \overline{X}$. Soit $(Z_i)_{i\in I}$ une famille \textbf{finie} de sous-ensembles de $\mathbb{R}$ tels que $X \in \bigcup\limits_{i \in I} Z_i$ (on dit que $(Z_i)$ est un \textbf{recouvrement} de $X$). La fonction $f$ admet au point $a$ une limite $\ell$ (finie ou infinie) si et seulement si pour tout $i$ tel que la limite de $f$ en $a$ sur $Z_i$ est envisageable, cette limite existe et vaut $\ell$. 
\end{tcolorbox}

$\boxed{\Rightarrow}$ \\
On suppose que $\lim\limits_a f = \ell$. \\
Soit $i \in I$ tel que $a \in \overline{X \cap Z}$. \\
Soit $V \in \mathcal{V}(\ell)$. On choisit $U \in \mathcal{V}(a)$ tel que $f(U \cap X) \subset V$. \\
EN particulier $f(\underbrace{U\cap X \cap Z_i}_{\subset U \cap X}) \subset V = \left. f \right|_{X \cap Z_i} (U \cap X \cap Z_i)$. \\ \\

$\boxed{\Leftarrow}$ \\
Notons $J \subset I$ l'ensemble des indices pour lesquels la limite est envisageable en $Z_i$. \\
Soit $V \in \mathcal{V}(\ell)$. Pour tout $i \in J$, comme $\lim\limits_{x \to a \\ x \in Z_i} = \ell$ on choisit $U_i \in \mathcal{V}(a)$ tel que $\left. f \right|_{Z_i \cap X} (U_i \cap Z_i \cap X) \subset V$. \\
On pose $U = \bigcap\limits_{i \in J} U_i \in \mathcal{V}(a)$ car $J$ est fini. \\
On choisit $U' \in \mathcal{V}(a)$ tel que $U' \cap \left(\bigcup\limits_{i \in I \backslash J} Z_i \right) = \emptyset$. \\
$f(U \cap U' \cap X) \subset V$ \\
Donc $\boxed{\lim\limits_a f = \ell}$. 

\setcounter{section}{29}
\section{Composition de limites}
\begin{tcolorbox}[title=Propostion 15.30, title filled=false, colframe=lightblue, colback=lightblue!10!white]
    Soit $f:X\to \mathbb{R}$, $g:Y\to \mathbb{R}$ deux fonctions avec $f(X) \subset Y$. Soit $a \in \overline{X}$, $b \in \overline{Y}$ et $c \in \overline{\mathbb{R}}$. Si $\lim\limits_a f = b$ et si $\lim\limits_b g = c$, alors $\lim\limits_a g \circ f = c$. 
\end{tcolorbox}

Soit $W \in \mathcal{V}(c)$. On choisit $V \in \mathcal{V}(b)$ tel que : 
\begin{align*}
    g(V \cap Y) \subset W
\end{align*}
On choisit $U \in \mathcal{V}(a)$ tel que : 
\begin{align*}
    f(U \cap X) \subset V \cap Y \text{ ($\lim\limits_a f = b$)}
\end{align*}
On a alors : 
\begin{align*}
    \boxed{g \circ f(U \cap X) \subset W}
\end{align*}

\setcounter{section}{31}
\section{Limites et inégalités strictes}
\begin{tcolorbox}[title=Propostion 15.32, title filled=false, colframe=lightblue, colback=lightblue!10!white]
    Soit $f:X \to \mathbb{R}$, $a \in \overline{X}$, $m \in \mathbb{R}$ et $M \in \mathbb{R}$. 
    \begin{enumerate}
        \item Si $\lim\limits_a f < M$ alors $f(x) < M$ au voisinage de $a$ 
        \item Si $\lim\limits_a f > m$ alors $f(x) > m$ au voisinage de $a$. 
    \end{enumerate}
\end{tcolorbox}

\begin{enumerate}
    \item Notons $b = \lim\limits_a f \in \overline{\mathbb{R}}$. Si $b < M$, on choisit $U \in \mathcal{V}(b)$ et $U' \in \mathcal{V}(M)$ avec $U < U'$. \\
    Comme $\lim\limits_a f = b$, on choisit $W \in \mathcal{V}(a)$ tel que : 
    \begin{align*}
        f(W \cap X) \subset U
    \end{align*}
\end{enumerate}

\section{Limite et inégalités larges}
\begin{tcolorbox}[title=Propostion 15.33, title filled=false, colframe=lightblue, colback=lightblue!10!white]
    Soit $f:X\to \mathbb{R}$ et $g:X\to \mathbb{R}$ deux fonctions et $a \in \overline{X}$. On suppose que $f$ et $g$ possède des limites finies en $a$. \\
    Si $f(x) \leq g(x)$ au voisinage de $a$, alors $\lim\limits_a f \leq \lim\limits_a g$. \\
    Ce résultat est le plus souvent utilisé lorsqu'une des deux fonctions est constante. 
\end{tcolorbox}

RAF : absurde + (15.32)

\section{Caractérisations séquentielle de la limite d'une fonction}
\begin{tcolorbox}[title=Théorème 15.34, title filled=false, colframe=orange, colback=orange!10!white]
    Soit $f:X\to \mathbb{R}$ une fonction et $a \in \overline{X}$ et $\ell \in \overline{\mathbb{R}}$. Sont équivalentes : 
    \begin{enumerate}
        \item $\lim\limits_a f = \ell \Leftrightarrow \forall u_n \to a, \lim f(u_n) = \ell \text{ ($= f(\lim u_n)$)}$
        \item Pour toute suite $(u_n)$ de limite $a$ à valeurs dans $X$, la suite $(f(u_n))$ a pour limite $\ell$. 
    \end{enumerate}
\end{tcolorbox}

$\boxed{1 \Rightarrow 2}$ \\
On suppose que $\lim\limits_a f = \ell$. \\
Soit $(u_n) \in X^{\mathbb{N}}$ avec $u_n \underset{n \to +\infty}{\longrightarrow} a$. \\
Soit $V \in \mathcal{V}(\ell)$. On choisit $U \in \mathcal{V}(a)$ tel que : 
\begin{align*}
    f(U \cap X) \subset V \text{ ($\lim\limits_a f = \ell$)}
\end{align*}
Comme $u_n \underset{n \to +\infty}{\longrightarrow} a$, on choisit $N \in \mathbb{N}$ tel que : 
\begin{align*}
    \forall n \geq N, u_n \in U \cap X
\end{align*}
Donc : 
\begin{align*}
    \forall n \geq N, f(u_n) \in V
\end{align*}
Donc : 
\begin{align*}
    f(u_n) \underset{n \to +\infty}{\longrightarrow} \ell
\end{align*} \\

$\boxed{1 \Leftarrow 2}$ \\
Par contraposée. On suppose que $f$ n'admet pas $\ell$ comme limite en $a$. Pour tout $n \in \mathbb{N}$, on note : 
\begin{align*}
    V_n = \begin{cases}
        ]a - \frac{1}{n+1}, a + \frac{1}{n+1}[ \text{ si $a \in \mathbb{R}$} \\
        [n, +\infty[ \text{ si $a = +\infty$} \\
        ]-\infty, -n] \text{ si $a = -\infty$}
    \end{cases}
\end{align*}
Par définition, il existe $W \in \mathcal{V}(\ell)$ tel que pour tout $V \in \mathcal{V}(a)$, il existe $x \in V \cap X$ et $f(x) \neq W$. \\
Pour tout $n \in \mathbb{N}$, on choisit $x_n \in V_n \cap X$ tel que $f(x_n) \neq W$. \\
Par construction : 
\begin{align*}
    (x_n) \in X^\mathbb{N}, x_n \underset{n \to +\infty}{\longrightarrow} a \text{ et } f(x_n) \cancel{\underset{n \to +\infty}{\longrightarrow}} \ell
\end{align*}

\setcounter{section}{38}
\section{Théorème de la limite monotone}
\begin{tcolorbox}[title=Théorème 15.39, title filled=false, colframe=orange, colback=orange!10!white]
    Soit $a \in \mathbb{R}$ et $b\in \mathbb{R} \cup \{+\infty\}$ avec $a < b$ et $f:[a, b[ \to \mathbb{R}$ une fonction croissante.
    \begin{enumerate}
        \item La limite $\lim\limits_{a^+} f$ existe et est finie. Plus précisément, on a $f(a) \leq \lim\limits_{a^+} f$. 
        \item Pour tout $c \in ]a, b[$, $\lim\limits_{c^-} f$ et $\lim\limits_{c^+} f$ existent et sont finies. Plus précisément : $\lim\limits_{c^-} f \leq f(c) \leq \lim\limits_{c^+} f$. 
        \item La limite $\lim\limits_{b} f$ existe et est soit finie, soit égale à $+\infty$. 
    \end{enumerate}
\end{tcolorbox}

\begin{enumerate}
    \item On note $F = f(]a, b[)$. Comme $f$ est définie au voisinage de $a$, $]a, b[ \neq \emptyset$ et $F \neq \emptyset$. \\
    Par ailleurs, comme $f$ est croissante sur $]a, b[$, $F$ est minorée par $f(a)$. \\
    D'après la propriété fondamentale de $\mathbb{R}$, $F$ possède une borne inférieure notée $\alpha$, avec $f(a) \leq \alpha$. \\
    Montrons par définition que $\lim\limits_{a^+} f = \alpha$. \\
    Soit $\epsilon > 0$, $\alpha + \epsilon$ n'est pas un minorant de $F$ par définition de $\alpha$. On choisit : 
    \begin{align*}
        \alpha \leq f(x_0) < \alpha + \epsilon
    \end{align*}
    Par croissance de $f$ sur $]a, b[$ : 
    \begin{align*}
        \forall x \in ]a, x_0[, \alpha \leq f(x) \leq f(x_0) < \alpha + \epsilon
    \end{align*}
    On pose $\eta = x_0 - a > 0$, on a montré que : 
    \begin{align*}
        \boxed{\forall x \in ]a - \eta[ \cap ]a, b[, |f(x) - \alpha| < \epsilon}
    \end{align*}

    \item Pour $c \in ]a, b[$, en appliquant (15.39.1) à $\left. f \right|_{[a, b[}$, on montre que $\lim\limits_{c^+} f$ existe et $f(x) \leq \lim\limits_{x^+} f$. \\
    On adapte ensuite la preuve de (15.39.1) : 
    \begin{align*}
        F = f(]a, c[), \alpha = \sup (F)
    \end{align*}
    pour montrer que $\lim\limits_{c^-} f$ existe et 

    \item Par disjonction de cas. \\
    \begin{itemize}
        \item Si $f$ est majorée : on adapte la 2ème partie de (15.39.2). 
        \item Si $f$ n'est pas majorée. Soit $A \in \mathbb{R}$. Comme $f$ n'est pas majorée, on choisit $x_0 \in ]a, b[$ tel que $f(x_0) > A$. \\
        Comme $f$ est croissante : 
        \begin{align*}
            \forall x \geq x_0, f(x) > A
        \end{align*}
        Donc $\lim\limits_{b} f = +\infty$. 
    \end{itemize}
\end{enumerate}

\setcounter{section}{58}
\section{Théorème des valeurs intermédiaires : version 1}
\begin{tcolorbox}[title=Théorème 15.59, title filled=false, colframe=orange, colback=orange!10!white]
    Soit $f$ une fonction continue sur un intervalle $I$ d'extrémité $a$ et $b$ dans $\overline{\mathbb{R}}$ (avec existence des limites dans le cas des bornes infinies). Alors si $f(a) > 0$ et $f(b) < 0$ (ou l'inverse), il exsite $c \in ]a, b[$, tel que $f(c) = 0$. 
\end{tcolorbox}

On note $A = \{ x\in I, f(x) > 0 \}$. \\
\begin{itemize}
    \item $A \neq \emptyset$ car $f$ est définie et strictement positive au voisinage de $a$ (15.32). 
    \item $A$ est majoré car $f$ est strictement négative au voisinage de $b$ (et tout élément dans ce voisinage est un majorant). 
\end{itemize}
D'après la propriété fondamentale de $\mathbb{R}$, $A$ possède une borne supérieure notée $c \in ]a, b[$. 
\begin{itemize}
    \item On a $c \not \in A$. En effet, si $f(x) > 0$, alors $f$ est strictement postivie sur un voisinage de $c$, et comme $f$ est définie à droite de $c$, cela contredirait que c'est un majorant de $A$. \\
    Donc $f(c) \leq 0$. 
    \item Si $f(c) < 0$, alors $f$ est strictement négative au voisinage à gauche de $c$. \\
    Absurde car $c$ est le plus petit des majorants. \\
    Conclusion, $\boxed{f(c) = 0}$. 
\end{itemize}

\section{Théorème des valeurs intermédiaires : version 2}
\begin{tcolorbox}[title=Théorème 15.60, title filled=false, colframe=orange, colback=orange!10!white]
    Soit $f$ une fonction continue sur un intervalle $I$ et soit $M = \sup_I f(x)$ et $m = \inf_I f(x)$ (éventuellement infinies). \\
    Alors $f$ prend toutes les valeurs de l'intervalle $]m; M[$ :
    \begin{align*}
        \forall x_0\in ]m ; M[, \exists c \in I, f(c) = x_0. 
    \end{align*}
\end{tcolorbox}

RAF : (15.59) à $f - x_0$. 

\section{Théorème des valeurs intermédiaires : version 3}
\begin{tcolorbox}[title=Théorème 15.61, title filled=false, colframe=orange, colback=orange!10!white]
    L'image d'un intervalle quelconque par une fonction continue est un intervalle. 
\end{tcolorbox}

Définition d'un intervalle par connexité. 

\setcounter{section}{64}
\section{Théorème de Heine}
\begin{tcolorbox}[title=Théorème 15.65, title filled=false, colframe=orange, colback=orange!10!white]
    Une fonction continue sur un segment est uniformément continue sur ce segment. 
\end{tcolorbox}

\noindent \underline{Rappel :}
\begin{align*}
    C^0(I) &: \forall x \in I, \forall \epsilon > 0, \exists \eta > 0, \forall y \in I, |x-y| < \eta \Rightarrow |f(x) - f(y)| < \epsilon \\
    Cu(I) &: \forall \epsilon > 0, \exists \eta > 0, \forall (x,y) \in I^2, |x-y| < \eta \Rightarrow |f(x) - f(y)| < \epsilon
\end{align*}
On raisonne par l'absurde. Soit $f$ continue sur $[a,b]$ mais non uniformément continue sur $[a,b]$. \\
On choisit $\epsilon$ tel que : 
\begin{align*}
    \forall \eta > 0, \exists (x, y) \in [a,b]^2, |x - y| < \eta \text{ et } |f(x) - f(y)| \geq \epsilon
\end{align*}
Ainsi, pour tout $b \in \mathbb{N}^*$, on choisit un couple $(x_n, y_n) \in [a,b]^2$ tel que : 
\begin{align*}
    |x_n - y_n| < \frac{1}{n} \text{ et } \underbrace{|f(x_n) - f(y_n)|}_{(*)} \geq \epsilon
\end{align*}
En particulier $(x_n)$ est bornée donc d'après le théorème de Bolzano-Weierstrass, on en extrait $(x_{\varphi(n)})$ suite convergente vers $\ell$. \\
D'après le TCILPPL, $\ell \in [a,b]$. \\
Comme : 
\begin{align*}
    \forall n \in \mathbb{N}, |x_{\varphi(n)} - y_{\varphi(n)}| < \frac{1}{\varphi(n)} \underset{n \to +\infty}{\longrightarrow} 0
\end{align*}
Alors : 
\begin{align*}
    y_{\varphi(n)} \underset{n \to +\infty}{\longrightarrow} \ell
\end{align*}
Par continuité : 
\begin{align*}
    f(x_{\varphi(n)}) \underset{n \to +\infty}{\longrightarrow} f(\ell) \text{ et } f(y_{\varphi(n)}) \underset{n \to +\infty}{\longrightarrow} f(\ell)
\end{align*}
Donc par opération : 
\begin{align*}
    |f(x_{\varphi(n)}) - f(y_{\varphi(n)})| \underset{n \to +\infty}{\longrightarrow} 0
\end{align*}
Absurde d'après $(*)$. 

\setcounter{section}{66}
\section{Caractérisation des intervalles compacts}
\begin{tcolorbox}[title=Lemme 15.67, title filled=false, colframe=orange, colback=orange!10!white]
    Les intervalles compacts de $\mathbb{R}$ sont exactement les segments, c'est-à-dire les intervalles fermés bornés $[a,b]$. 
\end{tcolorbox}

Les segments sont bien compacts (BW et TCILPPL). \\
\begin{itemize}
    \item Si $I = ]-\infty, a[$, 
    \begin{align*}
        u_n &= a-n-1 \underset{n \to +\infty}{\longrightarrow} -\infty \not\in I \\
        u_n &= a - \frac{1}{n+1} \underset{n \to +\infty}{\longrightarrow} a \not\in I
    \end{align*} 
\end{itemize}

\section{Image d'un compact par une fonction continue}
\begin{tcolorbox}[title=Lemme 15.68, title filled=false, colframe=orange, colback=orange!10!white]
    L'image continue d'un compact est compact. 
\end{tcolorbox}

Soit $I$ un segment, donc un intervalle. \\
Comme $f$ est continue sur $I$, $f(I)$ est un intervalle (TVI v3). \\
Montrons que $f(I)$ est compact. \\
Soit $(y_n) \in f(I)^{\mathbb{N}}$. Pour tout $n \in \mathbb{N}$, soit $x_n \in I$ tel que : 
\begin{align*}
    y_n = f(x_n)
\end{align*}
Or $I$ est compact (15.67), on choisit : 
\begin{align*}
    x_{\varphi(n)} \underset{n \to +\infty}{\longrightarrow} \ell \in I
\end{align*}
$y_{\varphi(n)} \underset{n \to +\infty}{\longrightarrow} f(\ell)$ car $f$ est continue sur $I$. 

\setcounter{section}{71}
\section{Théorème 15.72}
\begin{tcolorbox}[title=Théorème 15.72, title filled=false, colframe=orange, colback=orange!10!white]
    Soit $I$ un intervalle et $f$ une fonction continue sur $I$. Alors $f$ est injective si et seulement si $f$ est strictement monotone. 
\end{tcolorbox}

$\boxed{\Leftarrow}$ \\
RAS \\ \\

$\boxed{\Rightarrow}$ \\
Supposons $f$ non strictement monotone. \\
On peut supposer qu'il existe alors : 
\begin{align*}
    x < y < z
\end{align*}
tels que $f(x) < f(y)$ et $f(z) < f(y)$. \\
Soit :
\begin{align*}
    \lambda = \frac{f(y) + \max(f(y), f(z))}{2} &\in ]f(x), f(y)[ \\
    &\in ]f(z), f(y)[
\end{align*}
Par continuité de $f$ sur les intervalles $]x, y[$ et $]y, z[$, il existe $\alpha \in ]x, y[$ et $\beta \in ]y, z[$ tels que : 
\begin{align*}
    f(\alpha) = \lambda = f(\beta)
\end{align*}
Donc $f$ n'est pas injective. 


\end{document}