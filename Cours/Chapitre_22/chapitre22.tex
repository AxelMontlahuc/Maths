\documentclass[../main.tex]{subfiles}

\begin{document}
\setcounter{chapter}{21}
\chapter{Espaces de dimension finie}
\tableofcontents
\clearpage

\setsection{2}
\section{Nombre maximal de vecteurs linéairement indépendants}
\begin{tcolorbox}[title=Propostion 22.3, title filled=false, colframe=lightblue, colback=lightblue!10!white]
    Soit $E$ un $\mathbb{K}$-ev de dimension finie engendré par $n$ éléments. Alors toute partie libre de $E$ possède au plus $n$ éléments. 
\end{tcolorbox}

\noindent Soit $G$ une famille génératrice de $E$ avec $G = (g_1, \ldots, g_n)$. Soit $\mathcal{L}$ une famille libre de $E$. \\
Supposons par l'absurde que $|\mathcal{L}| > n$. Pour $k\in \llbracket 1, n \rrbracket$, on note : 
\begin{align*}
    P(k): \text{"$E$ est engendré par $n-k$ vecteurs de $G$ et $k$ vecteurs de $\mathcal{L}$"}
\end{align*}
Pour $k=0$, la famille convient. \\
On suppose que pour $k \in \llbracket 0, n-1 \rrbracket$, $E = Vect(\underbrace{g_1, \ldots, g_{n-k}}_{\in G}, \underbrace{l_1, \ldots, l_k}_{\in L})$ \\
Comme $l_{k+1} \in E$, on écrit $l_{k+1} = \sum\limits_{i=1}^{n-k} \alpha_i g_i + \sum\limits_{i=1}^k \beta_i l_j$. \\
Comme $\mathcal{L}$ est libre, $l_{k+1} \not\in Vect(l1, \ldots, l_k)$. \\
Donc il existe $i \in \llbracket 1, n-k \rrbracket, \alpha_i \neq 0$ et quitte à renommer les $g_i$, on peut supposer $\alpha_{n-k} \neq 0$ et ainsi : 
\begin{align*}
    g_{n-k} \in Vect(g_1, \ldots, g_{n-k}, l_1, \ldots, l_k, l_n+1)
\end{align*}
Ainsi : 
\begin{align*}
    E = Vect(g_1, \ldots, g_{n-k}, l_1, \ldots, l_k, l_{k+1})
\end{align*}
Par récurrence, $P(k)$ est vraie pour $k\in \llbracket 0, n \rrbracket$, en particulier, $P(n)$ est vraie. \\
$(l1, \ldots, l_n)$ est une base de $E$. Or $l_{n+1} \in E$ et $(l_1, \ldots, l_{n+1})$ libre. 
Absurde. 

\setsection{4}
\section{Algroithme de la base incomplète}
\begin{tcolorbox}[title=Théorème 22.5, title filled=false, colframe=orange, colback=orange!10!white]
    Soit $E \neq \{0\}$ un $\mathbb{K}$-ev de dimension finie et $\{x_i\}_{1\leq i\leq n}$ une partie génératrice de $E$ dont les $p$ premiers vecteurs sont linéairement indépendants. Dans ces conditions, $E$ possède une base constituée des vecteurs $x_1, \ldots, x_p$ et de certains vecteurs $x_{p+1}, \ldots, x_n$. 
\end{tcolorbox}

\noindent On utilise l'algorithme suivant : \\
On initialise $\mathcal{F} = (x_1, \ldots, x_p)$. Pour tout $k \in \llbracket p+1, n \rrbracket$ : 
\begin{itemize}
    \item Si $x_k \in Vect(\mathcal{F})$, on laisse $\mathcal{F}$ invariant. 
    \item Si $x_k \not\in Vect(\mathcal{F})$, on remplace $\mathcal{F}$ par $\mathcal{F} \cup \{x_k\}$.
\end{itemize}
L'algorithme s'arrête en temps fini. \\
La famille $\mathcal{F}$ obtenue est libre, elle est également génératrice car : 
\begin{align*}
    \forall i \in \llbracket 1, n \rrbracket, x_i \in \mathcal{F} \text{ ou } x_i \in Vect(\mathcal{F})
\end{align*}
Donc $E = Vect(x_i)_{i\in \llbracket 1, n \rrbracket} \subset Vect(\mathcal{F}) \subset E$.
Donc $\mathcal{F}$ est une base. 

\setsection{7}
\section{Théorème de la base incomplète}
\begin{tcolorbox}[title=Théorème 22.8, title filled=false, colframe=orange, colback=orange!10!white]
    Soit $E\neq \{0\}$ un $\mathbb{K}$-ev de dimension finie. 
    \begin{enumerate}
        \item Toute famille libre de $E$ peut être complétée en une base finie de $E$. 
        \item De toute famille génératrice de $E$ on peut extraire une base finie de $E$. 
    \end{enumerate}
    En particulier, $E$ possède une base finie. 
\end{tcolorbox}

\noindent Soit $\mathcal{G}$ une famille génératrice finie. 
\begin{enumerate}
    \item Soit $\mathcal{L}$ une famille libre. On applique l'algorithme de la base incomplète à $\mathcal{L} \cup \mathcal{G}$ qui fournit une base $B$ de $E$ contenant $\mathcal{L}$. 
    \item Comme $\mathcal{G}$ est génératrice, on fixe $x \neq 0 \in \mathcal{G}$ comme premier vecteur de $\mathcal{G}$ et on lui applique l'algorithme de la base incomplète. \\
    La base obtenue est bien constituée de vecteurs de $\mathcal{G}$. 
\end{enumerate}

\section*{Remarque}
\begin{tcolorbox}[title=Remarque, title filled=false, colframe=lightblue, colback=lightblue!10!white]
    Si $\mathcal{G}$ est une famille génératrice, elle contient nécessairement une famille génératrice finie. 
\end{tcolorbox}

\setsection{10}
\section{Caractérisation de la dimension finie par le cardinal des familles libres}
\begin{tcolorbox}[title=Corollaire 22.11, title filled=false, colframe=orange, colback=orange!10!white]
    Soit $E$ un espace vectoriel. Alors $E$ est de dimension finie si et seulement si toute famille libre de $E$ est de cardinal fini. 
\end{tcolorbox}

$\boxed{\Rightarrow}$ \\
On suppose $E$ de dimension finie. Donc $E$ possède une famille génératrice à $n$ vecteurs. \\
Donc les familles libres de $E$ ont un cardinal inférieur à $n$. \\
Elles sont finies. \\ \\

$\boxed{\Leftarrow}$ \\
Par contraposée, on suppose $E$ de dimension infinie. \\
Soit $x\in E$ avec $x\neq 0$. \\
On pose $x_1 = x$. Comme $E$ est de dimension infinie, on choisit $x_2 \in E \backslash Vect(x_1)$. \\
On poursuit les raisonnement par récurrence pour obtenir une famille libre $(x_n)_{n\in \mathbb{N}^*}$. 

\section{Théorème de la dimension}
\begin{tcolorbox}[title=Théorème 22.12, title filled=false, colframe=orange, colback=orange!10!white]
    Soit $E \neq \{0\}$ un espace vectoriel de dimension finie. Toutes les bases de $E$ sont finies et sont de même cardinal. 
\end{tcolorbox}

\noindent Soit $B$ et $B'$ deux bases. On a : 
\begin{align*}
    |B| \leq |B'| \text{ et } |B'| \leq |B|
\end{align*}
Donc : 
\begin{align*}
    |B| = |B'|
\end{align*}

\setsection{17}
\section{Caractérisation des bases en dimension finie}
\begin{tcolorbox}[title=Théorème 22.18, title filled=false, colframe=orange, colback=orange!10!white]
    Soit $E$ un $\mathbb{K}$-ev de dimension finie $n\neq 0$. Une famille de $n$ vecteurs est une base si, et seulement si, elle est libre, si, et seulement si, elle est génératrice.
\end{tcolorbox}

\noindent Soit $\mathcal{F}$ une famille avec $|\mathcal{F}| = \dim E = n$. \\
\begin{itemize}
    \item On suppose que $\mathcal{F}$ est libre. \\
    On applique sur $\mathcal{F}$ le théorème de la base incomplète. \\
    On obtient alors une base $B$ de $E$ avec : 
    \begin{align*}
        \mathcal{F} \subset E
    \end{align*}
    Or $|B| = \dim E = |\mathcal{F}|$. \\
    Donc $\mathcal{F} = B$. 

    \item On suppose $\mathcal{F}$ génératrice. On procède de la même manière en utilisant le théorème de la base extraite. 
\end{itemize}

\setsection{19}
\section{Majoration du rang et cas d'égalité}
\begin{tcolorbox}[title=Propostion 22.20, title filled=false, colframe=lightblue, colback=lightblue!10!white]
    On a
    \begin{align*}
        rg (x_1, \ldots, x_k) \leq k
    \end{align*}
    avec égalité si et seulement si la famille est libre. 
\end{tcolorbox}

\noindent Soit $Vect((x_i)_{i\leq k})$ possède un système fini de $k$ vecteurs générateurs.  $$\dim (Vect(x_1, \ldots, x_k)) \leq k$$
\begin{itemize}
    \item Si $\dim(Vect(x_1, \ldots, x_k)) = k$, alors (22.18), $(x_1, \ldots, x_k)$ est une base, donc est libre. 
    \item Si la famille est libre, c'est une base de $Vect(x_1, \ldots, x_k)$, donc $\dim(Vect(x_1, \ldots, x_k)) = k$.
\end{itemize}

\setsection{21}
\section{Dimension d'un sous-espace vectoriel}
\begin{tcolorbox}[title=Propostion 22.22, title filled=false, colframe=lightblue, colback=lightblue!10!white]
    Soit $E$ un $\mathbb{K}$-ev de dimension finie et $F$ un sous-espace vectoriel de $E$. Alors $F$ est de dimension finie et $\dim F \leq \dim E$, avec égalité si et seulement si $F = E$.
\end{tcolorbox}

\noindent Soit $F$ un sous-espace vectoriel de $E$, avec $E$ de dimension finie. \\
Ainsi, $F$ est lui-même de dimension finie (22.11). \\
Si $\mathcal{L}$ est une famille libre de $F$ :
\begin{align*}
    |\mathcal{L}| \leq \dim E
\end{align*}
Donc (il suffit de prendre pour $\mathcal{L}$ une base de $F$) : 
\begin{align*}
    \dim F \leq \dim E
\end{align*}
Si $\dim F = \dim E$, alors une base de $F$ est aussi une base de $E$ (22.18). \\
Ainsi : 
\begin{align*}
    F = Vect(B) = E
\end{align*}

\section{Formule de Grassmann}
\begin{tcolorbox}[title=Théorème 22.23, title filled=false, colframe=orange, colback=orange!10!white]
    Soit $E$ un espace vectoriel, $F$ et $G$ deux sous-espaces vectoriels de dimensions finies. Alors $F + G$ est de dimension finie et : 
    \begin{align*}
        \dim(F+G) = \dim F + \dim G - \dim F \cap G
    \end{align*}
\end{tcolorbox}

\noindent $F \cap G \subset F$, donc $F \cap G$ est de dimension finie. \\
On note $n = \dim F \cap G$. \\
On choisit une base $(e_1, \ldots, e_n)$ de $F \cap G$. \\
On complète cette famille libre en :
\begin{itemize}
    \item une base $(e_1, \ldots, e_n, f_1, \ldots, f_p)$ de $F$
    \item une base $(e_1, \ldots, e_n, g_1, \ldots, g_q)$ de $G$
\end{itemize}
Montrons que $(E_1, \ldots, e_n, f_1, \ldots, f_p, g_1, \ldots, g_q)$ est une base de $F+G$. \\
\begin{align*}
    F + G &= Vect(e_1, \ldots, e_n, f_1, \ldots, f_p) + Vect(e_1, \ldots, e_n, g_1, \ldots, g_q) \\
    &= Vect(e_1, \ldots, e_n, f_1, \ldots, f_p, g_1, \ldots, g_q)
\end{align*}
La famille génératrice. 
On suppose : 
\begin{align*}
    \sum_{i=1}^{n} \alpha_i e_i + \sum_{i=1}^{p} \beta_i f_i + \sum_{i=1}^{q} \gamma_i g_i = 0
\end{align*}
Donc : 
\begin{align*}
    \sum_{i=1}^{q} \gamma_i g_i = -\sum_{i=1}^{n} \alpha_i e_i - \sum_{i=1}^{p} \beta_i f_i \in F \cap G
\end{align*}
Donc (liberté de $(e_1, \ldots, e_n, g_1, \ldots, g_q)$) : 
\begin{align*}
    (\gamma_1, \ldots, \gamma_q) = (0, \ldots, 0)
\end{align*}
Puis : 
\begin{align*}
    \sum_{i=1}^{n} \alpha_i e_i + \sum_{i=1}^{p} \beta_i f_i = 0
\end{align*}
Donc (liberté de $(e_1, \ldots, e_n, f_1, \ldots, f_p)$) : 
\begin{align*}
    (\alpha_1, \ldots, \alpha_n) &= (0, \ldots, 0) \\
    (\beta_1, \ldots, \beta_p) &= (0, \ldots, 0)
\end{align*}
Donc : 
\begin{align*}
    \dim (F + G) &= n + p + q \\
    &= n + p + n + q - n \\
    &= \dim F + \dim G - \dim F \cap G
\end{align*}

\setsection{26}
\section{Caractérisation des couples de sous-espaces vectoriels supplémentaires}
\begin{tcolorbox}[title=Propostion 22.27, title filled=false, colframe=lightblue, colback=lightblue!10!white]
    Soit $E$ un espace de dimension finie, $F$ et $G$ deux sous-espaces vectoriels de $F$. Alors $F$ et $G$ sont supplémentaires si et seulement si : 
    \begin{align*}
        F \cap G = \{0\} \text{ et } \dim F + \dim G = \dim E
    \end{align*}
    si et seulement si : 
    \begin{align*}
        F + G = E \text{ et } \dim F + \dim G = \dim E
    \end{align*}
\end{tcolorbox}

\begin{align*}
    F \text{ et } G \text{ sont supplémentaires} &\text{ ssi } F \oplus G = E \\
    &\text{ ssi } F \cap G = \{0\} \text{ et } F + G = E \\
    \text{(} \boxed{\Rightarrow} \text{ 22.26 } \boxed{\Leftarrow} \text{ 22.26, 22.22} \text{)} &\text{ ssi } F \cap G = \{0\} \text{ et } \dim F + \dim G = \dim E \\
    \text{(} \boxed{\Rightarrow} \text{ 22.26 } \boxed{\Leftarrow} \text{ 22.23} \text{)} &\text{ ssi } F + G = E \text{ et } \dim F + \dim G = \dim E
\end{align*}

\section{Existence et dimension d'un supplémentaire en dimension finie}
\begin{tcolorbox}[title=Théorème 22.28, title filled=false, colframe=orange, colback=orange!10!white]
    Soit $E$ un espace vectoriel de dimension finie et $F$ un sous espace vectoriel de $E$. Alors il existe un supplémentaire $S$ de $F$ et : 
    \begin{align*}
        \dim S = \dim E - \dim F
    \end{align*}
\end{tcolorbox}

\begin{itemize}
    \item Si $F = \{0\}$, $E$ convient. 
    \item Si $F \neq \{0\}$, on choisit une base de $F$ $(f_1, \ldots, f_p)$ que l'on complète en une base $(f_1, \ldots, f_p, s_1, \ldots, s_q)$ de $E$ ($\dim E = p+q$). \\
    $S = Vect(s_1, \ldots, s_q)$ convient. 
\end{itemize}

\setsection{29}
\section{Base de $\mathcal{L}(E, F)$}
\begin{tcolorbox}[title=Propostion 22.30, title filled=false, colframe=lightblue, colback=lightblue!10!white]
    Si $E$ et $F$ sont de dimension finie, la famille $(u_{i, j})_{(i, j)\in I\times J}$ décrite dans l'exemple précédent est une base de $\mathcal{L}(E, F)$. 
\end{tcolorbox}

\begin{itemize}
    \item Montrons que $(u_{i, j})$ est libre. \\
    On suppose $\sum\limits_{(i, j)\in I\times J} \lambda_{i, j} u_{i, j} = 0$. \\
    \begin{align*}
        \forall k \in I, \sum_{(i, j)\in I\times J} \lambda_{i, j} u_{i, j}(b_k) &= 0 \\
        \text{donc } \sum_{(i, j)\in I\times J} \lambda_{i, j} \delta_{i, k} c_j &= 0 \\
        \text{donc } \sum_{j\in J} \lambda_{k, j} c_j &= 0
    \end{align*}
    Par liberté des $(c_j)$, on a :
    \begin{align*}
        \forall k \in I, \forall j \in J, \lambda_{k, j} = 0
    \end{align*}

    \item Montrons que $(u_{i, j})$ est génératrice. \\
    Soit $f\in \mathcal{L}(E, F)$. \\
    Pour tout $k \in I$, $f(b_k) = \sum\limits_{j\in J} \lambda_{k, j} c_j$ ($(c_j)$ est une base de $F$). \\
    Alors : 
    \begin{align*}
        f = \sum_{(i, j) \in I \times J} \lambda_{i, j} u_{i, j} \text{ (théorème de rigidité)}
    \end{align*}
\end{itemize}

\setsection{31}
\section{Dimension d'espaces isomorphes}
\begin{tcolorbox}[title=Propostion 22.32, title filled=false, colframe=lightblue, colback=lightblue!10!white]
    Soit $E$ et $F$ deux espaces isomorphes. Si l'un des deux est de dimension finie, alors les deux le sont et : 
    \begin{align*}
        \dim E = \dim F
    \end{align*}
    Réciproquement, si $E$ et $F$ sont de dimension finie avec $\dim E = \dim F$, alors $E$ et $F$ sont isomorphes.
\end{tcolorbox}

\begin{itemize}
    \item Si $\dim E = n$, on choisit $B$ une base de $E$. \\
    Si $f:E\to F$ est un isomorphisme, alors $f(B)$ est une base de $F$. \\
    Donc $F$ est de dimension finie et $\dim F = |f(B)| = |B| = n = \dim E$. 

    \item On suppose que $\dim E = n = \dim F$. \\
    Soit $(e_1, \ldots, e_n)$ une base de $E$ et $(f_1, \ldots, f_n)$ une base de $F$. \\
    On définit (théorème de rigidité) $u \in \mathcal{L}(E, F)$ par : 
    \begin{align*}
        \forall i \in \llbracket 1, n \rrbracket, u(e_i) = f_i
    \end{align*}
    D'après (21.70), $u$ est un isomorphisme. 
\end{itemize}

\setsection{34}
\section{Rang d'une famille génératrice}
\begin{tcolorbox}[title=Propostion 22.35, title filled=false, colframe=lightblue, colback=lightblue!10!white]
    Soit $(x_i)_{i\in I}$ une famille génératrice de $E$. Le rang de $u$, s'il existe est égal au rang de la famille $(u(x_i))_{i\in I}$.
\end{tcolorbox}

\begin{align*}
    rg(u) &= \dim(Im(u)) \\
    &= \dim(Vect(u(x_i))_{i\in I}) \text{ (21.21)} \\
    &= rg(u(x_i))_{i\in I}
\end{align*}

\section{Existence et majoration du rang en dimension finie}
\begin{tcolorbox}[title=Propostion 22.36, title filled=false, colframe=lightblue, colback=lightblue!10!white]
    \begin{itemize}
        \item Soit $u\in \mathcal{L}(E, F)$. Si $E$ ou $F$ sont de dimension finie, alors $Im(u)$ est également de dimension finie et (avec les conditions adéquates) : 
        \begin{align*}
            rg(u) \leq \dim E \text{ ou } rg(u) \leq \dim F
        \end{align*}
        \item Avec les conditions appropriées : 
        \begin{itemize}
            \item $rg(u) = \dim E$ si et seulement si $u$ est injective
            \item $rg(u) = \dim F$ si et seulement si $u$ est surjective
        \end{itemize}
    \end{itemize}
\end{tcolorbox}

\noindent On suppose $E$ et $F$ de dimension finie.
\begin{itemize}
    \item $Im(u) \subset F$ et $\dim(Im(u)) \leq \dim F$ et $rg(u) = \dim F$ si et seulement si (22.22) $Im(u) = F$ si et seulement si $u$ est surjective.
    \item Soit $(e_1, \ldots, e_n)$ une base de $E$. \\
    Comme $(e_1, \ldots, e_n)$ engendre $E$ : 
    \begin{align*}
        rg(u) &= rg(u(e_1), \ldots, u(e_n)) \text{ (22.35)} \\
        &\leq n = \dim E \text{ (22.20)}
    \end{align*}
    \begin{align*}
        rg(u(e_1), \ldots, u(e_n)) = n &\text{ ssi } (u(e_1), \ldots, u(e_n)) \text{ est libre} \\
        \text{(21.68)} &\text{ ssi } u \text{ est injective} \\
    \end{align*}
\end{itemize}


\end{document}