\documentclass[../main.tex]{subfiles}

\begin{document}
\setcounter{chapter}{21}
\chapter{Espaces de dimension finie}
\tableofcontents
\clearpage

\setsection{2}
\section{Nombre maximal de vecteurs linéairement indépendants}
\begin{tcolorbox}[title=Propostion 22.3, title filled=false, colframe=lightblue, colback=lightblue!10!white]
    Soit $E$ un $\mathbb{K}$-ev de dimension finie engendré par $n$ éléments. Alors toute partie libre de $E$ possède au plus $n$ éléments. 
\end{tcolorbox}

\noindent Soit $G$ une famille génératrice de $E$ avec $G = (g_1, \ldots, g_n)$. Soit $\mathcal{L}$ une famille libre de $E$. \\
Supposons par l'absurde que $|\mathcal{L}| > n$. Pour $k\in \llbracket 1, n \rrbracket$, on note : 
\begin{align*}
    P(k): \text{"$E$ est engendré par $n-k$ vecteurs de $G$ et $k$ vecteurs de $\mathcal{L}$"}
\end{align*}
Pour $k=0$, la famille convient. \\
On suppose que pour $k \in \llbracket 0, n-1 \rrbracket$, $E = Vect(\underbrace{g_1, \ldots, g_{n-k}}_{\in G}, \underbrace{l_1, \ldots, l_k}_{\in L})$ \\
Comme $l_{k+1} \in E$, on écrit $l_{k+1} = \sum\limits_{i=1}^{n-k} \alpha_i g_i + \sum\limits_{i=1}^k \beta_i l_j$. \\
Comme $\mathcal{L}$ est libre, $l_{k+1} \not\in Vect(l1, \ldots, l_k)$. \\
Donc il existe $i \in \llbracket 1, n-k \rrbracket, \alpha_i \neq 0$ et quitte à renommer les $g_i$, on peut supposer $\alpha_{n-k} \neq 0$ et ainsi : 
\begin{align*}
    g_{n-k} \in Vect(g_1, \ldots, g_{n-k}, l_1, \ldots, l_k, l_n+1)
\end{align*}
Ainsi : 
\begin{align*}
    E = Vect(g_1, \ldots, g_{n-k}, l_1, \ldots, l_k, l_{k+1})
\end{align*}
Par récurrence, $P(k)$ est vraie pour $k\in \llbracket 0, n \rrbracket$, en particulier, $P(n)$ est vraie. \\
$(l1, \ldots, l_n)$ est une base de $E$. Or $l_{n+1} \in E$ et $(l_1, \ldots, l_{n+1})$ libre. 
Absurde. 

\setsection{4}
\section{Algroithme de la base incomplète}
\begin{tcolorbox}[title=Théorème 22.5, title filled=false, colframe=orange, colback=orange!10!white]
    Soit $E \neq \{0\}$ un $\mathbb{K}$-ev de dimension finie et $\{x_i\}_{1\leq i\leq n}$ une partie génératrice de $E$ dont les $p$ premiers vecteurs sont linéairement indépendants. Dans ces conditions, $E$ possède une base constituée des vecteurs $x_1, \ldots, x_p$ et de certains vecteurs $x_{p+1}, \ldots, x_n$. 
\end{tcolorbox}

\noindent On utilise l'algorithme suivant : \\
On initialise $\mathcal{F} = (x_1, \ldots, x_p)$. Pour tout $k \in \llbracket p+1, n \rrbracket$ : 
\begin{itemize}
    \item Si $x_k \in Vect(\mathcal{F})$, on laisse $\mathcal{F}$ invariant. 
    \item Si $x_k \not\in Vect(\mathcal{F})$, on remplace $\mathcal{F}$ par $\mathcal{F} \cup \{x_k\}$.
\end{itemize}
L'algorithme s'arrête en temps fini. \\
La famille $\mathcal{F}$ obtenue est libre, elle est également génératrice car : 
\begin{align*}
    \forall i \in \llbracket 1, n \rrbracket, x_i \in \mathcal{F} \text{ ou } x_i \in Vect(\mathcal{F})
\end{align*}
Donc $E = Vect(x_i)_{i\in \llbracket 1, n \rrbracket} \subset Vect(\mathcal{F}) \subset E$.
Donc $\mathcal{F}$ est une base. 

\setsection{7}
\section{Théorème de la base incomplète}
\begin{tcolorbox}[title=Théorème 22.8, title filled=false, colframe=orange, colback=orange!10!white]
    Soit $E\neq \{0\}$ un $\mathbb{K}$-ev de dimension finie. 
    \begin{enumerate}
        \item Toute famille libre de $E$ peut être complétée en une base finie de $E$. 
        \item De toute famille génératrice de $E$ on peut extraire une base finie de $E$. 
    \end{enumerate}
    En particulier, $E$ possède une base finie. 
\end{tcolorbox}

\noindent Soit $\mathcal{G}$ une famille génératrice finie. 
\begin{enumerate}
    \item Soit $\mathcal{L}$ une famille libre. On applique l'algorithme de la base incomplète à $\mathcal{L} \cup \mathcal{G}$ qui fournit une base $B$ de $E$ contenant $\mathcal{L}$. 
    \item Comme $\mathcal{G}$ est génératrice, on fixe $x \neq 0 \in \mathcal{G}$ comme premier vecteur de $\mathcal{G}$ et on lui applique l'algorithme de la base incomplète. \\
    La base obtenue est bien constituée de vecteurs de $\mathcal{G}$. 
\end{enumerate}

\section*{Remarque}
\begin{tcolorbox}[title=Remarque, title filled=false, colframe=lightblue, colback=lightblue!10!white]
    Si $\mathcal{G}$ est une famille génératrice, elle contient nécessairement une famille génératrice finie. 
\end{tcolorbox}

\setsection{10}
\section{Caractérisation de la dimension finie par le cardinal des familles libres}
\begin{tcolorbox}[title=Corollaire 22.11, title filled=false, colframe=orange, colback=orange!10!white]
    Soit $E$ un espace vectoriel. Alors $E$ est de dimension finie si et seulement si toute famille libre de $E$ est de cardinal fini. 
\end{tcolorbox}

$\boxed{\Rightarrow}$ \\
On suppose $E$ de dimension finie. Donc $E$ possède une famille génératrice à $n$ vecteurs. \\
Donc les familles libres de $E$ ont un cardinal inférieur à $n$. \\
Elles sont finies. \\ \\

$\boxed{\Leftarrow}$ \\
Par contraposée, on suppose $E$ de dimension infinie. \\
Soit $x\in E$ avec $x\neq 0$. \\
On pose $x_1 = x$. Comme $E$ est de dimension infinie, on choisit $x_2 \in E \backslash Vect(x_1)$. \\
On poursuit les raisonnement par récurrence pour obtenir une famille libre $(x_n)_{n\in \mathbb{N}^*}$. \\


\end{document}