\documentclass[../main.tex]{subfiles}

\begin{document}
\setcounter{chapter}{34}
\chapter{Familles sommables}
\tableofcontents
\clearpage

\setsection{1}
\section{Reformulation}
\begin{tcolorbox}[title=Propostion 35.2, title filled=false, colframe=lightblue, colback=lightblue!10!white]
    Soit $\sum a_n$ une séries à termes positifs. Alors $\sum_{n \geq 0} a_n$ est bien définie dans $\overline{\mathbb{R}}_{+}$et
    $$\sum_{n \geq 0} a_n=\sup \left\{\sum_{k \in J} a_k, J \in \mathcal{P}_f(\mathbb{N})\right\}$$
\end{tcolorbox}

\noindent En notant, pour $n\in \mathbb{N}, S_n = \sum\limits_{k=0}^{n} a_k$, on a : 
\begin{align*}
    S_n \underset{n \to +\infty}{\longrightarrow} \sum_{k\geq 0}^{a_k} 
\end{align*}
Or pour tout $n\in \mathbb{N}, S_n\in \left\{ \sum_{k\in J} a_k\mid J\in \mathcal{P}_f(\mathbb{N}) \right\}$. \\
Donc $\sum_{k\geq 0} a_k \leq \sup \left\{ \sum_{k\in J} a_k\mid J\in \mathcal{P}_f(\mathbb{N}) \right\} = S$. \\
Par ailleurs, pour $J\in \mathcal{P}_f(\mathbb{N})$, on pose $N = \max J$ et $J\subset \llbracket 0, N \rrbracket$ et : 
\begin{align*}
    \sum_{k\in J} a_k  \leq \sum_{k=0}^{N} a_k \leq \sum_{k>0} a_k 
\end{align*}
Par définition de la borne supérieure : 
\begin{align*}
    S \leq \sum_{k\geq 0} a_k
\end{align*}
Donc : 
\begin{align*}
    \sum_{k\geq 0} a_k = S
\end{align*}

\setsection{4}
\section{Croissance de la somme}
\begin{tcolorbox}[title=Propostion 35.5, title filled=false, colframe=lightblue, colback=lightblue!10!white]
    Soit $\left(a_i\right)_{i \in I}$ et $\left(b_i\right)_{i \in I}$ deux familles à valeurs dans $\overline{\mathbb{R}}_{+}$. Si pour tout $i \in I, a_i \leq b_i$, alors
    $$\sum_{i \in I} a_i \leq \sum_{i \in I} b_i$$
\end{tcolorbox}

\noindent Soit $J\in \mathcal{P}_f(I)$. Comme : 
\begin{align*}
    \forall i\in J, a_i\leq b_i
\end{align*}
Alors : 
\begin{align*}
    \sum_{i\in j} a_i \leq \sumç{i\in J} b_i \leq \sum_{i\in I} b_i
\end{align*}
$\sum\limits_{i\in I} b_i$ est un majorant de $\left\{ \sum_{i\in J} a_i\mid J\in \mathcal{P}_f(I) \right\}$. \\
Par définition : 
\begin{align*}
    \sum_{i\in I} a_i \leq \sum_{i\in I} b_i
\end{align*}

\setsection{7}
\section{Lien avec les séries à termes positifs}
\begin{tcolorbox}[title=Propostion 35.8, title filled=false, colframe=lightblue, colback=lightblue!10!white]
    Soit $\sum a_n$ une séries à termes positifs.
    \begin{enumerate}
        \item On a $\sum_{n=0}^{+\infty} a_n=\sum_{n \in \mathbb{N}} a_n$ (le terme de gauche correspond à la somme de la série tandis que le terme de droite à la somme de la famille sommable). 
        \item En particulier, $\sum a_n$ converge si et seulement si la famille $\left(a_n\right)_{n \in \mathbb{N}}$ est sommable. 
    \end{enumerate}
\end{tcolorbox}

\begin{enumerate}
    \item (35.2)
    \item Théorème de la Limite Monotone
\end{enumerate}

\setsection{9}
\section{Invariance de la somme d'une série à termes positifs par permutation des termes}
\begin{tcolorbox}[title=Corollaire 35.10, title filled=false, colframe=orange, colback=orange!10!white]
    Soit $\sum a_n$ une série à termes positifs et $\sigma \in S_{\mathbb{N}}$. Alors
    $$\sum_{n=0}^{+\infty} a_n=\sum_{n=0}^{+\infty} a_{\sigma(n)}$$
    Cette égalité reste vraie dans $\overline{\mathbb{R}}_{+}$.
\end{tcolorbox}

\noindent Soit $\sigma\in \mathcal{S}_\mathbb{N}$. \\
$\sigma$ induit une bijection $\mathcal{P}_f(\mathbb{N})\to \mathcal{P}_f(\mathbb{N});J\mapsto \sigma(J)$ de $\left\{ \sum_{i\in J}| J\in \mathcal{P}_f(\mathbb{N}) \right\}$ sur $\left\{ \sum_{i\in I} a_{\sigma(i)}\mid J\in \mathcal{P}_f(\mathbb{N}) \right\}$. \\
Ces deux ensembles ont donc la même borne supérieure. Donc : 
\begin{align*}
    \sum_{i\in \mathbb{N}} a_i = \sum_{i\in \mathbb{N}} a_{\sigma(i)} 
\end{align*}
Soit : 
\begin{align*}
    \sum_{i=0}^{+\infty} a_i = \sum_{i=0}^{+\infty} a_{\sigma(i)} \text{ (35.8)}
\end{align*}

\setsection{11}
\section{Restriction}
\begin{tcolorbox}[title=Propostion 35.12, title filled=false, colframe=lightblue, colback=lightblue!10!white]
    Soit $(a_i)_{i \in I}$ une famille d'éléments de $\overline{\mathbb{R}}_{+}$. Soit $J\subset I$, alors : 
    \begin{align*}
        \sum_{i\in J} a_i \leq \sum_{i\in I} a_i
    \end{align*}
\end{tcolorbox}

\noindent Comme $\mathcal{P}_f(J)\subset \mathcal{P}_f(I)$ :
\begin{align*}
    \left\{ \sum_{j\in K} a_j\mid K\in \mathcal{P}_f(J) \right\} \subset \left\{ \sum_{j\in K} a_j\mid K\in \mathcal{P}_f(I) \right\}
\end{align*}
Donc (Chapitre 9) : 
\begin{align*}
    \sum_{i\in J} a_i \leq \sum_{i\in I} a_i
\end{align*}

\section{Preque linéarité}
\begin{tcolorbox}[title=Propostion 35.13, title filled=false, colframe=lightblue, colback=lightblue!10!white]
    Soit $\left(a_i\right)$ et $\left(b_i\right)$ deux familles d'éléments de $\overline{\mathbb{R}}_{+}$et $(\lambda, \mu) \in \overline{\mathbb{R}}_{+}^2$. On a
    $$\sum_{i \in I}\left(\lambda a_i+\mu b_i\right)=\lambda \sum_{i \in I} a_i+\mu \sum_{i \in I} b_i$$
\end{tcolorbox}

\noindent On rappelle (Chapitre 9) que : 
\begin{align*}
    \sup(aA + bB) = a\sup A + b\sup B
\end{align*}
($a\geq 0, b\geq 0$ et $\sup A, \sup B$ existent). 

\setsection{13}
\section{Sommation par paquets}
\begin{tcolorbox}[title=Théorème 35.14, title filled=false, colframe=orange, colback=orange!10!white]
    Soit $I$ un ensemble quelconque et $I=\bigsqcup\limits_{k \in K} I_k$ un recouvrement disjoint de $I$. Soit $a=\left(a_i\right)_{i \in I}$ une famille d'éléments de $\overline{\mathbb{R}}_{+}$. Alors
    $$\sum_{i \in I} a_i=\sum_{k \in K}\left(\sum_{i \in I_k} a_i\right)$$
\end{tcolorbox}

\begin{itemize}
    \item Montrons que $\sum\limits_{i\in I} a_i\leq \sum\limits_{k\in K}\left(\sum\limits_{i\in I_k} a_i\right)$ : \\
    Soit $J\in \mathcal{P}_f(I)$. \\
    Pour tout $k\in K$, on note $J_k = J\cap I_k$. \\
    Ainsi, $J_k\in \mathcal{P}_f(I_k)$ et :
    \begin{align*}
        \bigsqcup_{k\in K} J_k &= \bigsqcup_{k\in K} (J\cap I_k) \\
        &= J\cap \bigsqcup_{k\in K} I_k \\
        &= J\cap I \\
        &= J
    \end{align*}
    On pose également $L = \{k\in K, J_k\neq\emptyset \}$. \\
    Alors $L\in \mathcal{P}_f(K)$ et :
    \begin{align*}
        \bigsqcup_{k\in L} J_k &= \bigsqcup_{k\in K} J_k = J
    \end{align*}
    Les ensembles étant finis :
    \begin{align*}
        \sum_{i\in J} a_i = \sum_{k\in L} \sum_{i\in J_k} a_i\leq \sum_{k\in L}\sum_{i\in I_k} a_i \leq \sum_{k\in K} \sum_{i\in I_k} a_i
    \end{align*}
    Par définition de la borne supérieure : 
    \begin{align*}
        \sum_{i\in I} a_i \leq \sum_{k\in K}\sum_{i\in I_k} a_i
    \end{align*}

    \item On prouve $\sum\limits_{k\in K}\sum\limits_{i\in I_k} a_i\leq \sum\limits_{i\in I} a_i$. \\
    Soit $L\in \mathcal{P}_f(K)$. \\
    Soit, pour $k\in L, J_k\in \mathcal{P}_f(I_k)$. 
    $\bigsqcup_{k\in L} J_k$ est donc une partie finie de $I$. \\
    \begin{align*}
        \underbrace{\sum_{k\in L}\sum_{i\in J_k} a_i}_{\text{car } \bigsqcup J_k \text{ est fini}} \leq \sum_{i\in I} a_i
    \end{align*}
\end{itemize}
En utilisant un nombre fini de fois la définition de la borne supérieure : 
\begin{align*}
    \sum_{k\in L}\sum_{i\in I_k} a_i \leq \sum_{i\in I} a_i
\end{align*}
Toujours par définition de la borne supérieure : 
\begin{align*}
    \sum_{k\in K}\sum_{i\in I_k} a_i \leq \sum_{i\in I} a_i 
\end{align*}

\setsection{15}
\section{Théorème de Fubini positif}
\begin{tcolorbox}[title=Corollaire 35.16, title filled=false, colframe=orange, colback=orange!10!white]
    Soit $I$ et $J$ deux ensembles quelconques et $\left(a_{i, j}\right)_{(i, j) \in I \times J}$ une famille de réels positifs. Alors
    $$\sum_{i \in I} \sum_{j \in J} a_{i, j}=\sum_{(i, j) \in I \times J} a_{i, j}=\sum_{j \in J} \sum_{i \in I} a_{i, j}$$
\end{tcolorbox}

\begin{align*}
    \bigsqcup_{j\in J} I\times \{j\} = I\times J = \bigsqcup_{i\in I} \{i\} \times J
\end{align*}
On conclut avec le théorème de sommation par paquets. 

\section{Exemple}
\begin{tcolorbox}[title=Exemple 35.17, title filled=false, colframe=darkgreen, colback=darkgreen!10!white]
    Montrer que $\sum_{n\geq 2} (\zeta(n)-1) = 1$. 
\end{tcolorbox}

\noindent $\left( \frac{1}{k^n} \right)_{n\geq 2, k\geq 2}$ est une famille de $\mathbb{R}_+$. 
\begin{align*}
    \sum_{n\geq 2} \left( \sum_{k\geq 1} \frac{1}{k^n} - 1 \right) &= \sum_{n\geq 2} \left( \sum_{k\geq 2} \frac{1}{k^n} \right) \\
    &= \sum_{k\geq 2} \left( \sum_{n\geq 2} \frac{1}{k^n} \right) \text{ (Fubini positif)} \\
    &= \sum_{k\geq 2} \left( \frac{1}{k^2} \frac{1}{1 - \frac{1}{k}} \right) \text{ (progresion géométrique)} \\
    &= \sum_{k\geq 2} \frac{1}{k^2 - k} \\
    &= \sum_{k\geq 2} \left( \frac{1}{k-1} - \frac{1}{k} \right) \text{ (DES)} \\
    &= 1 \text{ (télescopage)}
\end{align*}


\end{document}