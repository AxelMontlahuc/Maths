\documentclass[../main.tex]{subfiles}

\begin{document}
\setcounter{chapter}{20}
\chapter{Applications linéaires}
\tableofcontents
\clearpage

\setsection{3}
\section{Exemple}
\begin{tcolorbox}[title=Exemple 21.4.1, title filled=false, colframe=darkgreen, colback=darkgreen!10!white]
    L'application de $\mathbb{R}^2$ dans $\mathbb{R}$ définie par $f(x, y) = 2x + 3y$. 
\end{tcolorbox}

\noindent Soit $((x, y), (x', y'), \lambda) \in (\mathbb{R}^2)^2 \times \mathbb{R}$. On a
\begin{align*}
    f((x, y) + \lambda (x', y')) &= f(x + \lambda x', y + \lambda y') \\
    &= 2(x + \lambda x') + 3(y + \lambda y') \\
    &= 2x + 3y + \lambda(2x' + 3y') \\
    &= f(x, y) + \lambda f(x', y').
\end{align*}

\setsection{7}
\section{Structure de $\mathcal{L}(E, F)$}
\begin{tcolorbox}[title=Propostion 21.8, title filled=false, colframe=lightblue, colback=lightblue!10!white]
    $\mathcal{L}(E, F)$ est un estpace vectoriel sur $\mathbb{K}$.
\end{tcolorbox}

\noindent \begin{itemize}
    \item $\mathcal{L}(E, F) \subset F^E$
    \item $\overline{0} \mathcal{L}(E, F)$
    \item Soit $(f, g) \in \mathcal{L}(E, F)^2$ et $\alpha \in \mathbb{K}$. Soit $(x, y) \in E^2, \lambda \in \mathbb{K}$. On a : 
    \begin{align*}
        (f + \alpha g)(x + \lambda y) &= f(x + \lambda y) + \alpha g(x + \lambda y) \\
        &= f(x) + \lambda f(y) + \alpha g(x) + \alpha \lambda g(y) \\
        &= f(x) + \alpha g(x) + \lambda (f(y) + \alpha g(y)) \\
        &= (f + \alpha g)(x) + \lambda (f + \alpha g)(y).
    \end{align*}
    Donc $f + \alpha g \in \mathcal{L}(E, F)$.
\end{itemize}

\setsection{9}
\section{Composition de deux AL}
\begin{tcolorbox}[title=Propostion 21.10, title filled=false, colframe=lightblue, colback=lightblue!10!white]
    Soit $f \in \mathcal{L}(E, F)$ et $g \in \mathcal{L}(F, G)$, alors $g \circ f \in \mathcal{L}(E, G)$.
\end{tcolorbox}

\noindent Soit (x, y) $\in E^2$ et $\lambda \in \mathbb{K}$ : 
\begin{align*}
    g \circ f(x + \lambda y) &= g(f(x + \lambda y)) \\
    &= g(f(x) + \lambda f(y)) \\
    &= g(f(x)) + \lambda g(f(y)) \\
    &= g \circ f(x) + \lambda g \circ f(y).
\end{align*}
Donc $g \circ f \in \mathcal{L}(E, G)$.

\setsection{12}
\section{Bilinéarité de la composition}
\begin{tcolorbox}[title=Propostion 21.13, title filled=false, colframe=lightblue, colback=lightblue!10!white]
    La composition d'application linéaire est bilinéaire. En termes plus précis, $E$, $F$ et $G$ étant des $\mathbb{K}$-ev, l'application
    \begin{align*}
        \Psi : \mathcal{L}(E, F) \times \mathcal{L}(F, G) &\longrightarrow \mathcal{L}(E, G); (u, v) \mapsto v \circ u
    \end{align*}
    est une application bilinéaire. 
\end{tcolorbox}

\noindent D'après la remarque (21.11), $\Psi$ est linéaire à droite. \\
\begin{align*}
    \forall u \in \mathcal{L}(E, F), \forall (v, v') \in \mathcal{L}(F, G)^2, \forall \lambda \in \mathbb{K}, \Psi(u, v + \lambda v') &= \Psi(u, v) + \lambda \Psi(u, v')
\end{align*}
Soit $(u, u') \in \mathcal{L}(E, F)^2, v \in \mathcal{L}(F, G), \lambda \in \mathbb{K}$. On a :
\begin{align*}
    \forall x \in \mathbb{E}, \Psi(u + \lambda u', v)(x) &= v \circ (u + \lambda u')(x) \\
    &= v(u(x) + \lambda u'(x)) \\
    &= v(u(x)) + \lambda v(u'(x)) \\
    &= \Psi(u, v)(x) + \lambda \Psi(u', v)(x)
\end{align*}
Donc $\Psi(u + \lambda u', v) = \Psi(u, v) + \lambda \Psi(u', v)$. \\

\setsection{15}
\section{Structure des images directes et réciproques}
\begin{tcolorbox}[title=Propostion 21.16, title filled=false, colframe=lightblue, colback=lightblue!10!white]
    \begin{enumerate}
        \item Soit $E'$ un sev de $E$. Alors $f(E')$ est un sev de $F$.
        \item Soit $F'$ un sev de $F$. Alors $f^{-1}(F')$ est un sev de $E$.
    \end{enumerate}
\end{tcolorbox}

\begin{enumerate}
    \item \begin{itemize}
        \item $f(E') \subset F$
        \item $0 = f(0) \in f(E')$
        \item Soit $(x, y) \in f(E')^2, \lambda \in \mathbb{K}$. On écrit $x = f(\alpha), y = f(\beta)$ avec $(\alpha, \beta) \in E'^2$. \\
        \begin{align*}
            x + \lambda y &= f(\alpha) + \lambda f(\beta) \\
            &= f(\alpha + \lambda \beta) \\
            &\in f(E')
        \end{align*}
    \end{itemize}

    \item \begin{itemize}
        \item $f^{-1}(F') \subset E$
        \item $0 = f(0) \in f^{-1}(F')$
        \item Soit $(x, y) \in f^{-1}(F')^2, \lambda \in \mathbb{K}$. 
        \begin{align*}
            f(x + \lambda y) &= f(x) + \lambda f(y) \in F' \\
            \text{donc } x + \lambda y &\in f^{-1}(F')
        \end{align*}
    \end{itemize}
\end{enumerate}

\setsection{20}
\section{Famille génératrice de $Im(f)$}
\begin{tcolorbox}[title=Propostion 21.21, title filled=false, colframe=lightblue, colback=lightblue!10!white]
    Soit $f\in \mathcal{L}(E, F)$ et $(e_i)_{i \in I}$ une famille génératrice de $E$. Alors $(f(e_i)_{i \in I})$ est une famille génératrice de $Im(f)$. \\
    Soit 
    \begin{align*}
        Im(f) &= Vect(f(e_i)_{i \in I})
    \end{align*}
\end{tcolorbox}

\begin{itemize}
    \item Pour tout $i \in I, f(e_i) \in Im(f)$. \\
    Comme $Im(f)$ est un sev : 
    \begin{align*}
        Vect(f(e_i)_{i \in I}) \subset Im(f)
    \end{align*}

    \item Soit $a\in Im(f)$. On choisit $x\in E$ tel que $a = f(x)$. \\
    Comme $(e_i)_{i \in I}$ est une famille génératrice de $E$, on peut écrit $x = \sum\limits_{i\in I} \lambda_i e_i$ où $(\lambda_i)_{i\in I}$ est à spport fini. 
    \begin{align*}
        a &= f\left(\sum_{i\in I} \lambda_i e_i\right) \\
        &= \sum_{i\in I} \lambda_i f(e_i) \\
        &\in Vect(f(e_i)_{i \in I})
    \end{align*}
\end{itemize}

\setsection{22}
\section{Réciproque d'un isomophisme}
\begin{tcolorbox}[title=Théorème 12.23, title filled=false, colframe=orange, colback=orange!10!white]
    Soit $f$ un isomorphisme de $E$ vers $F$. Alors $f^{-1}$ est une application linéaire, donc un isomophisme de $F$ vers $E$. 
\end{tcolorbox}

\noindent On pose $g = f^{-1}$. Soit $(x, y) \in F^2, \lambda \in \mathbb{K}$. 
\begin{align*}
    g(x + \lambda y) &= g(f(g(x)) + \lambda f(g(y))) \\
    &= g(f(g(x)) + \lambda f(g(y))) \\
    &= g(x) + \lambda g(y)
\end{align*}
Donc $g \in \mathcal{L}(F, E)$.

\setsection{40}
\section{Structure de l'ensemble des polynômes annulateurs - Hors Programme}
\begin{tcolorbox}[title=Propostion 21.41 - HP, title filled=false, colframe=lightblue, colback=lightblue!10!white]
    L'ensemble des polynômes annulateurs de $f$ est un idéal de $\mathbb{K}[X]$.
\end{tcolorbox}

\noindent Si $P$ et $Q$ annulent $u$, alors : 
\begin{align*}
    (P-Q)(u) &= P(u) - Q(u) = 0_{\mathcal{L}(E)}
\end{align*}
Si $B \in \mathbb{K}[X]$ : 
\begin{align*}
    (PB)(u) &= P(u) \circ B(u) = B(u) \circ 0_{\mathcal{L}(E)} = 0_{\mathcal{L}(E)}
\end{align*}

\setsection{51}
\section{Caractérisation de l'image d'un projecteur}
\begin{tcolorbox}[title=Propostion 21.52, title filled=false, colframe=lightblue, colback=lightblue!10!white]
    Soit $p$ un projecteur de $E$. Alors $x\in Im(p)$ si et seulement si $p(x) = x$. Soit : 
    \begin{align*}
        Im(p) = \ker(p - id_E)
    \end{align*}
\end{tcolorbox}

\noindent Soit $p$ un projecteur. Soit $x\in E$. 
\begin{itemize}
    \item Si $x\in Im(p)$, on choisit $y\in E$ tel que $x = p(y)$. \\
    Donc $p(x) = p^2(y) = p(y) = x$.
    \item Si $p(x) = x$, alors $x\in Im(p)$. 
    \item \begin{align*}
        x\in Im(p) &\Leftrightarrow p(x) = x \\
        &\Leftrightarrow p(x) - x = 0 \\
        &\Leftrightarrow (p - id)(x) = 0 \\
        &\Leftrightarrow x\in \ker(p - id)
    \end{align*}
\end{itemize}


\end{document}