\documentclass[../main.tex]{subfiles}

\begin{document}
\setcounter{chapter}{20}
\chapter{Applications linéaires}
\tableofcontents
\clearpage

\setsection{3}
\section{Exemple}
\begin{tcolorbox}[title=Exemple 21.4.1, title filled=false, colframe=darkgreen, colback=darkgreen!10!white]
    L'application de $\mathbb{R}^2$ dans $\mathbb{R}$ définie par $f(x, y) = 2x + 3y$. 
\end{tcolorbox}

\noindent Soit $((x, y), (x', y'), \lambda) \in (\mathbb{R}^2)^2 \times \mathbb{R}$. On a
\begin{align*}
    f((x, y) + \lambda (x', y')) &= f(x + \lambda x', y + \lambda y') \\
    &= 2(x + \lambda x') + 3(y + \lambda y') \\
    &= 2x + 3y + \lambda(2x' + 3y') \\
    &= f(x, y) + \lambda f(x', y').
\end{align*}


\end{document}