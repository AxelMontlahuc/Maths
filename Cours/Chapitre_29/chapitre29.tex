\documentclass[../main.tex]{subfiles}

\begin{document}
\setcounter{chapter}{28}
\chapter{Groupe symétrique}
\tableofcontents
\clearpage

\setsection{25}
\section{Lemme 26}
\begin{tcolorbox}[title=Lemme 29.26, title filled=false, colframe=orange, colback=orange!10!white]
    Soit $\sigma\in \mathcal{S}_n$. On a : 
    \begin{align*}
        \left| \prod_{1\leq i < j \leq n} (\sigma(i) - \sigma(j)) \right| = \prod_{X\in \mathcal{P}_2(\llbracket 1, n \rrbracket)} \delta_{\sigma} (X) = \prod_{1\leq i < j \leq n} (j - i)
    \end{align*}
\end{tcolorbox}

\begin{itemize}
    \item La première égalité est justifiée car on a une bijection entre $\{ (i, j) \mid 1 \leq i < j \leq n \}$ et $\mathcal{P}_2(\llbracket 1, n \rrbracket)$. 
    \item La seconde égalité est justifiée d'après (28.23). 
\end{itemize}

\setsection{28}
\section{Propriété fondamentale de la signature}
\begin{tcolorbox}[title=Théorème 29.29, title filled=false, colframe=orange, colback=orange!10!white]
    La signature est un morphisme de groupe de $(\mathcal{S}_n, \circ)$ dans $(\{-1, 1\}, \times)$. 
\end{tcolorbox}

Montrons que $\epsilon(\sigma \circ \xi) = \epsilon(\sigma) \times \epsilon(\xi)$. \\
Pour $\sigma, \xi \in \mathcal{S}_n$ : 
\begin{align*}
    \epsilon(\sigma\circ\xi) &= \frac{\prod\limits_{1\leq i < j\leq n} (\sigma\circ\xi(j) - \sigma\circ\xi(i))}{\prod\limits_{1\leq i < j\leq n} (j - i)} \times \frac{\prod\limits_{1\leq i < j\leq n} (\xi(j) - \xi(i))}{\prod\limits_{1\leq i < j\leq n} (\xi(j) - \xi(i))} \\
    &= \epsilon(\xi) \times \prod_{X\in \mathcal{P}(\llbracket 1, n \rrbracket)} \tau_\sigma (\xi(X)) \\
    &= \epsilon(\xi) \times \prod_{X\in \mathcal{P}(\llbracket 1, n \rrbracket)} \tau_\sigma (X) \\
    &= \epsilon(\xi) \times \epsilon(\sigma)
\end{align*}

\setsection{34}
\section{Décomposition d'une transposition à l'aide des $\tau_i$}
\begin{tcolorbox}[title=Propostion 29.35, title filled=false, colframe=lightblue, colback=lightblue!10!white]
    soit $1\leq i < j\leq n$ et $\tau = (i, j)$. Alors : 
    \begin{align*}
        \tau = \tau_{j-1} \circ \cdots \circ \tau_{i+1} \circ \tau_i \circ \tau_{i+1} \circ \cdots \circ \tau_{j-1}
    \end{align*}
\end{tcolorbox}

\begin{itemize}
    \item Si $k>j$, alors pour tout $p\in \llbracket i, j-1 \rrbracket$, $\tau_p(k) = k$. \\
    Donc $\sigma(k) = k$. \\
    Cela reste vrai si $k < i$. 
    \item On a :
    \begin{align*} 
        \sigma(i) &= \tau_{j-1}\circ \tau_{j-2} \circ \cdots \circ \tau_{i+1} \circ \tau_i \\
        &= \tau_{j-1}\circ \cdots \circ \tau_{i+1}(i+1) \\
        &= \tau_{j-1}(j-1) \\
        &= j \\
        \sigma(j) &= \tau_{j-1}\circ\cdots\tau_i\circ\cdots\circ\tau_{j-1}(j) \\
        &= \tau_{j-1}\circ\cdots\tau_i\circ\cdots\tau_{j-2}(j-1) \\
        &= \tau_{j-1}\circ\cdots\tau_i(i+1) \\
        &= \tau_{j-1}\circ\cdots\tau_{i+1}(i) \\
        &= i
    \end{align*}
    \item Si $i < k < j$, alors : 
    \begin{align*}
        \sigma(k) &= \tau_{j-1}\circ\cdots\circ\tau_i\circ\cdots\tau_k(k) \\
        &= \tau_{j-1}\circ\cdots\circ\tau_i\circ\cdots\tau_{k-1}(k+1) \\
        &= \tau_{j-1}\circ\cdots\circ\tau_k(k+1) \\
        &= \tau_{j-1}\circ\cdots\tau_{k+1}(k) \\
        &= k
    \end{align*}
\end{itemize}

\setsection{36}
\section{Caractère générateur des transpositions}
\begin{tcolorbox}[title=Théorème 29.37, title filled=false, colframe=orange, colback=orange!10!white]
    Toute permutation $\sigma\in \mathcal{S}_n$ est un produit de transposition. 
\end{tcolorbox}

\noindent On prouve le résultat par récurrence sur $\mathbb{N}\setminus \{0, 1\}$. 
\begin{itemize}
    \item pour $n=2$, $\mathcal{S}_2 = \{ id, \begin{pmatrix}
        1 & 2
    \end{pmatrix} \}$ et $id = \begin{pmatrix}
        1 & 2
    \end{pmatrix}^2$. 

    \item On suppose le résultat vrai pour $n\geq 2$. \\
    Soit $\sigma\in \mathcal{S}_{n+1}$. \\
    \begin{itemize}
        \item Si $\sigma(n+1) = n+1$, $\sigma$ induit naturellement une permutation $\tilde \sigma$ sur $S_n$, donc $\tilde \sigma$ est un produit de transpositions $\tilde\tau$, et chaque $\tilde\tau$ se relève en une transposition $\tau$ de $\mathcal{S}_{n+1}$. 
        \item Si $\sigma(n+1) = i\in \llbracket 1, n \rrbracket$, alors : 
        \begin{align*}
            \varphi = \begin{pmatrix}
                i & n+1
            \end{pmatrix}
            \circ \sigma \in \mathcal{S}_{n+1}
        \end{align*}
        et $\varphi(n+1) = n+1$. \\
        D'après le point précédent, $\varphi$ est un produit de transposition. \\
        Donc $\sigma = \begin{pmatrix}
            i & n+1
        \end{pmatrix} \circ \varphi$ est aussi un produit de transposition. 
    \end{itemize}
\end{itemize}

\setsection{39}
\section{Effet de la conjugaison sur un cycle}
\begin{tcolorbox}[title=Théorème 28.40, title filled=false, colframe=orange, colback=orange!10!white]
    Soit $\sigma\in \mathcal{S}_n$ et $\begin{pmatrix}
        a_1 & \cdots & a_k
    \end{pmatrix}$ un cycle. Alors : 
    \begin{align*}
        \sigma\circ \begin{pmatrix}
            a_1 & \cdots & a_k
        \end{pmatrix} \circ \sigma^{-1} = \begin{pmatrix}
            \sigma(a_1) & \cdots & \sigma(a_k)
        \end{pmatrix}
    \end{align*}
\end{tcolorbox}

\begin{itemize}
    \item Si $\sigma^{-1}(i) \not\in \{a_1, \ldots, a_n\}$ alors $\sigma\circ \begin{pmatrix}
        a_1 & \cdots & a_k
    \end{pmatrix} \circ \sigma^{-1}(i) = \sigma\circ \sigma^{-1}(i) = i$. 
    \item Si $\sigma^{-1}(i) = a_j$, alors $\sigma\circ \begin{pmatrix}
        a_1 & \cdots & a_k
    \end{pmatrix} \circ \sigma^{-1}(i) =  \sigma(a_{j+1})$
\end{itemize}


\end{document}