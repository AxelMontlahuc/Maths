\documentclass[../main.tex]{subfiles}

\begin{document}
\setcounter{chapter}{28}
\chapter{Groupe symétrique}
\tableofcontents
\clearpage

\setsection{25}
\section{Lemme 26}
\begin{tcolorbox}[title=Lemme 29.26, title filled=false, colframe=orange, colback=orange!10!white]
    Soit $\sigma\in \mathcal{S}_n$. On a : 
    \begin{align*}
        \left| \prod_{1\leq i < j \leq n} (\sigma(i) - \sigma(j)) \right| = \prod_{X\in \mathcal{P}_2(\llbracket 1, n \rrbracket)} \delta_{\sigma} (X) = \prod_{1\leq i < j \leq n} (j - i)
    \end{align*}
\end{tcolorbox}

\begin{itemize}
    \item La première égalité est justifiée car on a une bijection entre $\{ (i, j) \mid 1 \leq i < j \leq n \}$ et $\mathcal{P}_2(\llbracket 1, n \rrbracket)$. 
    \item La seconde égalité est justifiée d'après (28.23). 
\end{itemize}

\setsection{28}
\section{Propriété fondamentale de la signature}
\begin{tcolorbox}[title=Théorème 29.29, title filled=false, colframe=orange, colback=orange!10!white]
    La signature est un morphisme de groupe de $(\mathcal{S}_n, \circ)$ dans $(\{-1, 1\}, \times)$. 
\end{tcolorbox}

Montrons que $\epsilon(\sigma \circ \xi) = \epsilon(\sigma) \times \epsilon(\xi)$. \\
Pour $\sigma, \xi \in \mathcal{S}_n$ : 
\begin{align*}
    \epsilon(\sigma\circ\xi) &= \frac{\prod\limits_{1\leq i < j\leq n} (\sigma\circ\xi(j) - \sigma\circ\xi(i))}{\prod\limits_{1\leq i < j\leq n} (j - i)} \times \frac{\prod\limits_{1\leq i < j\leq n} (\xi(j) - \xi(i))}{\prod\limits_{1\leq i < j\leq n} (\xi(j) - \xi(i))} \\
    &= \epsilon(\xi) \times \prod_{X\in \mathcal{P}(\llbracket 1, n \rrbracket)} \tau_\sigma (\xi(X)) \\
    &= \epsilon(\xi) \times \prod_{X\in \mathcal{P}(\llbracket 1, n \rrbracket)} \tau_\sigma (X) \\
    &= \epsilon(\xi) \times \epsilon(\sigma)
\end{align*}

\setsection{34}
\section{Décomposition d'une transposition à l'aide des $\tau_i$}
\begin{tcolorbox}[title=Propostion 29.35, title filled=false, colframe=lightblue, colback=lightblue!10!white]
    soit $1\leq i < j\leq n$ et $\tau = (i, j)$. Alors : 
    \begin{align*}
        \tau = \tau_{j-1} \circ \cdots \circ \tau_{i+1} \circ \tau_i \circ \tau_{i+1} \circ \cdots \circ \tau_{j-1}
    \end{align*}
\end{tcolorbox}

\begin{itemize}
    \item Si $k>j$, alors pour tout $p\in \llbracket i, j-1 \rrbracket$, $\tau_p(k) = k$. \\
    Donc $\sigma(k) = k$. \\
    Cela reste vrai si $k < i$. 
    \item On a :
    \begin{align*} 
        \sigma(i) &= \tau_{j-1}\circ \tau_{j-2} \circ \cdots \circ \tau_{i+1} \circ \tau_i \\
        &= \tau_{j-1}\circ \cdots \circ \tau_{i+1}(i+1) \\
        &= \tau_{j-1}(j-1) \\
        &= j \\
        \sigma(j) &= \tau_{j-1}\circ\cdots\tau_i\circ\cdots\circ\tau_{j-1}(j) \\
        &= \tau_{j-1}\circ\cdots\tau_i\circ\cdots\tau_{j-2}(j-1) \\
        &= \tau_{j-1}\circ\cdots\tau_i(i+1) \\
        &= \tau_{j-1}\circ\cdots\tau_{i+1}(i) \\
        &= i
    \end{align*}
    \item Si $i < k < j$, alors : 
    \begin{align*}
        \sigma(k) &= \tau_{j-1}\circ\cdots\circ\tau_i\circ\cdots\tau_k(k) \\
        &= \tau_{j-1}\circ\cdots\circ\tau_i\circ\cdots\tau_{k-1}(k+1) \\
        &= \tau_{j-1}\circ\cdots\circ\tau_k(k+1) \\
        &= \tau_{j-1}\circ\cdots\tau_{k+1}(k) \\
        &= k
    \end{align*}
\end{itemize}

\setsection{36}
\section{Caractère générateur des transpositions}
\begin{tcolorbox}[title=Théorème 29.37, title filled=false, colframe=orange, colback=orange!10!white]
    Toute permutation $\sigma\in \mathcal{S}_n$ est un produit de transposition. 
\end{tcolorbox}

\noindent On prouve le résultat par récurrence sur $\mathbb{N}\setminus \{0, 1\}$. 
\begin{itemize}
    \item pour $n=2$, $\mathcal{S}_2 = \{ id, \begin{pmatrix}
        1 & 2
    \end{pmatrix} \}$ et $id = \begin{pmatrix}
        1 & 2
    \end{pmatrix}^2$. 

    \item On suppose le résultat vrai pour $n\geq 2$. \\
    Soit $\sigma\in \mathcal{S}_{n+1}$. \\
    \begin{itemize}
        \item Si $\sigma(n+1) = n+1$, $\sigma$ induit naturellement une permutation $\tilde \sigma$ sur $S_n$, donc $\tilde \sigma$ est un produit de transpositions $\tilde\tau$, et chaque $\tilde\tau$ se relève en une transposition $\tau$ de $\mathcal{S}_{n+1}$. 
        \item Si $\sigma(n+1) = i\in \llbracket 1, n \rrbracket$, alors : 
        \begin{align*}
            \varphi = \begin{pmatrix}
                i & n+1
            \end{pmatrix}
            \circ \sigma \in \mathcal{S}_{n+1}
        \end{align*}
        et $\varphi(n+1) = n+1$. \\
        D'après le point précédent, $\varphi$ est un produit de transposition. \\
        Donc $\sigma = \begin{pmatrix}
            i & n+1
        \end{pmatrix} \circ \varphi$ est aussi un produit de transposition. 
    \end{itemize}
\end{itemize}

\setsection{39}
\section{Effet de la conjugaison sur un cycle}
\begin{tcolorbox}[title=Théorème 29.40, title filled=false, colframe=orange, colback=orange!10!white]
    Soit $\sigma\in \mathcal{S}_n$ et $\begin{pmatrix}
        a_1 & \cdots & a_k
    \end{pmatrix}$ un cycle. Alors : 
    \begin{align*}
        \sigma\circ \begin{pmatrix}
            a_1 & \cdots & a_k
        \end{pmatrix} \circ \sigma^{-1} = \begin{pmatrix}
            \sigma(a_1) & \cdots & \sigma(a_k)
        \end{pmatrix}
    \end{align*}
\end{tcolorbox}

\begin{itemize}
    \item Si $\sigma^{-1}(i) \not\in \{a_1, \ldots, a_n\}$ alors $\sigma\circ \begin{pmatrix}
        a_1 & \cdots & a_k
    \end{pmatrix} \circ \sigma^{-1}(i) = \sigma\circ \sigma^{-1}(i) = i$. 
    \item Si $\sigma^{-1}(i) = a_j$, alors $\sigma\circ \begin{pmatrix}
        a_1 & \cdots & a_k
    \end{pmatrix} \circ \sigma^{-1}(i) =  \sigma(a_{j+1})$. 
\end{itemize}

\section{Corollaire 29.41}
\begin{tcolorbox}[title=Corollaire 29.41, title filled=false, colframe=orange, colback=orange!10!white]
    Soit $\varphi:\mathcal{S}_n\to \{ -1, 1 \}$ un morphisme. Soit $\alpha\in \{ 1-, 1 \}$. S'il existe une transposition $\tau_0$ telle que $\varphi(\tau_0) = \alpha$, alors pour toute transposition $\tau$, on a $\varphi(\tau) = \alpha$. \\
    Ainsi, $\varphi$ prend une valeur constante sur les transpositions.
\end{tcolorbox}

Par conjugaison. Soit $\tau_0 = \begin{pmatrix}
    i & j
\end{pmatrix}$ et $\tau = \begin{pmatrix}
    k & l
\end{pmatrix}$. \\
On a : 
\begin{align*}
    \tau = \sigma \circ \tau_0 \circ \sigma^{-1}
\end{align*}
avec $\sigma = \begin{pmatrix}
    i & k & j & l
\end{pmatrix}$. Alors : 
\begin{align*}
    \varphi(\tau) &= \varphi(\sigma \circ \tau_0 \circ \sigma^{-1}) \\
    &= \varphi(\sigma) \times \varphi(\tau_0) \times \varphi(\sigma^{-1}) \\
    &= \varphi(\tau_0) \times \varphi(\sigma)^2 \\
    &= \varphi(\tau_0)
\end{align*}

\section{Unicité de la signature}
\begin{tcolorbox}[title=Théorème 29.42, title filled=false, colframe=orange, colback=orange!10!white]
    La signature est l'unique morphisme de groupe non trivial de $\mathcal{S}_n$ dans $\{-1, 1\}$. 
\end{tcolorbox}

Soit $\varphi$ un morphismede groupes de $\mathcal{S}_n$ dans $\{\pm 1\}$. \\
Soit $\sigma\in \mathcal{S}_n$. D'après (29.37), $\sigma = \tau_1\circ \cdots \circ \tau_k$. 
\begin{itemize}
    \item Si la valeur prise par $\varphi$ sur les transpositions et $1$ (29.40), alors : 
    \begin{align*}
        \varphi(\sigma) = \prod_{i=1}^{k} \varphi(\tau_i) = 1
    \end{align*}
    Donc $\varphi$ est triviale. 

    \item Si la valeur prise par $\varphi$ sur les transpositions est $-1$ (29.41), alors :
    \begin{align*}
        \varphi(0) = \prod_{i=1}^{k} \varphi(\tau_i) = (-1)^k = \epsilon(\sigma)
    \end{align*}
    Donc $\varphi = \epsilon$.
\end{itemize}

\setsection{51}
\section{Décomposition en cycle d'une permutation}
\begin{tcolorbox}[title=Théorème 29.52, title filled=false, colframe=orange, colback=orange!10!white]
    Soit $\sigma$ une permutation de $\mathcal{S}_n$. A permutation près des facteurs, il existe une unique décomposition de $\sigma$ en produit de cycle à supports disjoints. 
    \begin{align*}
        \sigma = C_1\circ \cdots\circ C_k
    \end{align*}
    telles que les supports des cycles forment un partition de $\llbracket 1, n \rrbracket$. De plus, l'unique cycle de cette décomposition contenant $x$ est égale à $C_x$. 
\end{tcolorbox}

\begin{itemize}
    \item \underline{Existence :} On note $\{ \overline{C_1}, \ldots, \overline{C_k} \} = \llbracket 1, n \rrbracket / \equiv_\sigma$. \\
    On note (29.49) $c_i$ la permutation induite par $\sigma$ sur $\overline{C_i}$ ($C_i = \begin{pmatrix}
        p & \sigma(p) & \cdots & \sigma^{j}(p) \\
    \end{pmatrix}$). \\
    On pose $\varphi = C_1 \circ \ldots \circ C_k$. \\
    Soit $i\in \llbracket 1, n \rrbracket$, alors $i\in \overline{C_q}$ avec $q\in \llbracket 1, k \rrbracket$. \\
    D'après (29.51), $\varphi(i) = C_q(i) = \sigma(i)$. \\
    Donc $\varphi = \sigma$. 
    \item \underline{Unicité :} On suppose que $\sigma = C_1 \circ \ldots \circ C_k = U_1 \circ \ldots \circ U_q$. \\
    Soit $i\in \llbracket 1, n \rrbracket$. $i\in supp(C_1) \in supp(U_1)$ (quitte à permuter les rôles). \\
    On a donc $\sigma(i) = C_1(i) = U_1(i)$ et $\sigma^2(i) = C_1^2(i) = U_1^2(i)$ et .... \\
    Donc $C_1 = U_1$. 
\end{itemize}

\setsection{61}
\section{Décomposition d'un cycle en transpositions}
\begin{tcolorbox}[title=Lemme 29.62, title filled=false, colframe=orange, colback=orange!10!white]
    Soit $(i_1, \ldots, i_k)$ des entiers deux à deux distincts de $\llbracket 1, n \rrbracket$. Alors : 
    \begin{align*}
        \begin{pmatrix}
            i_1 & i_2 & \cdots & i_k
        \end{pmatrix} = \begin{pmatrix}
            i_1 & i_k
        \end{pmatrix} \circ \begin{pmatrix}
            i_1 & i_{k-1}
        \end{pmatrix} \circ \cdots \circ \begin{pmatrix}
            i_1 & i_2
        \end{pmatrix}
    \end{align*}
\end{tcolorbox}

\noindent On note $\sigma = \begin{pmatrix}
    i_1 & i_k
\end{pmatrix} \cdots \begin{pmatrix}
    i_1 & i_2
\end{pmatrix}$. \\
Soit $p\not\in \{ i_1, \ldots, i_k \}$. On a bien $\sigma(p) = p$. \\
Soit $i_j\in \{ i_1, \ldots, i_k \}$. ($j\neq k$) \\
\begin{itemize}
    \item \begin{align*}
        \sigma(i_1) &= \begin{pmatrix}
            i_1 & i_k
        \end{pmatrix} \cdots \begin{pmatrix}
            i_1 & i_2
        \end{pmatrix}(i_1) \\
        &= \begin{pmatrix}
            i_1 & i_k
        \end{pmatrix} \cdots \begin{pmatrix}
            i_1 & i_3
        \end{pmatrix}(i_2) \\
        &= i_2
    \end{align*}

    \item \begin{align*}
        \sigma(i_j) &= \begin{pmatrix}
            i_1 & i_k
        \end{pmatrix} \cdots \begin{pmatrix}
            i_1 & i_2
        \end{pmatrix}(i_j) \\
        &= \begin{pmatrix}
            i_1 & i_k
        \end{pmatrix} \cdots \begin{pmatrix}
            i_1 & i_{j}
        \end{pmatrix}(i_j) \\
        &= \begin{pmatrix}
            i_1 & i_k
        \end{pmatrix} \cdots \begin{pmatrix}
            i_1 & i_{j+1}
        \end{pmatrix}(i_1) \\
        &= i_{j+1}
    \end{align*}
    
    \item \begin{align*}
        \sigma(i_k) &= \begin{pmatrix}
            i_1 & i_k
        \end{pmatrix} (i_k) \\
        &= i_1
    \end{align*}
\end{itemize}

\section{Signature d'un cycle}
\begin{tcolorbox}[title=Propostion 29.63, title filled=false, colframe=lightblue, colback=lightblue!10!white]
    Soit $C$ un cucle et $\ell(C)$ sa longueur. Alors : 
    \begin{align*}
        \epsilon(C) = (-1)^{\ell(C) - 1}
    \end{align*}
\end{tcolorbox}

\noindent Avec ce qui précède : 
\begin{align*}
    \epsilon(\sigma) &= \prod_{j=2}^{k} \epsilon(\begin{pmatrix}
        i_1 & i_j
    \end{pmatrix}) \\
    &= (-1)^{k-1} \\
    &= (-1)^{\ell(C) - 1}
\end{align*}

\section{Détermination de $\epsilon$ par le type cyclique}
\begin{tcolorbox}[title=Théorème 29.64, title filled=false, colframe=orange, colback=orange!10!white]
    Soit $\sigma$ une permutation de $\mathcal{S}_n$ et $c(\sigma)$ le nombre de parts dans son support cyclique (ou de façon équivalente dans son type cyclique). Alors : 
    \begin{align*}
        \epsilon(\sigma) = (-1)^{n - c(\sigma)}
    \end{align*}
\end{tcolorbox}

\noindent Soit $\sigma = C_1\circ \cdots\circ C_{c(\sigma)}$. On a : 
\begin{align*}
    \epsilon(0) &= \prod_{i=1}^{c(\sigma)} \epsilon(C_i) \text{ ($\epsilon$ morphisme)} \\
    &= \prod_{i=1}^{c(\sigma)} (-1)^{\ell(C_i) - 1} \text{ (29.63)} \\
    &= (-1)^{\sum\limits_{i=1}^{c(\sigma)} [\ell(C_i) - 1]} \\
    &= (-1)^{\sum\limits_{i=1}^{c(\sigma)} \ell(C_i) - c(\sigma)} \\
    &= (-1)^{n - c(\sigma)}
\end{align*}

\setsection{68}
\section{Exemple}
\begin{tcolorbox}[title=Exemple 29.69, title filled=false, colframe=darkgreen, colback=darkgreen!10!white]
    Calculer la signature de la permutation suivante : 
    \begin{align*}
        \sigma = \begin{pmatrix}
            1 & 2 & 3 & \cdots & n & n+1 & n+2 & \cdots & 2n \\
            2 & 4 & 6 & \cdots & 2n & 1 & 3 & \cdots & 2n-1
        \end{pmatrix}
    \end{align*}
\end{tcolorbox}

\noindent Pour chaque $i\in \llbracket 1, n \rrbracket$, le couple $(i, k+n)$ donne une inversion avec $k\in \llbracket 1, i \rrbracket$. \\
On dénombre donc : 
\begin{align*}
    \sum_{i=1}^{n} i = \frac{n(n+1)}{2}
\end{align*}
Donc : 
\begin{align*}
    \epsilon(\sigma) &= (-1)^{\frac{n(n+1)}{2}} \\
\end{align*}

\section*{Exercice 4}
\noindent Soit $g\in G$. On note $\varphi_g:G\to G; h\mapsto gh$. \\
On a $\varphi_g\in S(G)$ car $\varphi_g$ est bijective de réciproque $\varphi_g^{-1}$. \\
On note $\varphi:G\to S(G); g\mapsto \varphi_g$. \\
On a évidemment pour tout $(g, k)\in G^2$ : 
\begin{align*}
    \varphi(g \times k) = \varphi_g \circ \varphi_k
\end{align*}
Donc $\varphi$ est un morphisme de groupe. \\
Si $g\in \ker \varphi$, $\varphi(g) = \operatorname{id}$, donc : 
\begin{align*}
    \forall h\in G, \varphi_g(h) &= h \\
    \text{donc } g &= 1_G
\end{align*}
Donc $G$ est isomorphe à $\operatorname{Im} \varphi$, qui est un sous-groupe de $S(G)$. 


\end{document}