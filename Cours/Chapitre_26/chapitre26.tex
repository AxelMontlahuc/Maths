\documentclass[../main.tex]{subfiles}

\begin{document}
\setcounter{chapter}{25}
\chapter{Intégration sur un segment}
\tableofcontents
\clearpage

\setsection{11}
\section{Image d'une fonction en escalier}
\begin{tcolorbox}[title=Propostion 26.12, title filled=false, colframe=lightblue, colback=lightblue!10!white]
    L'image d'une fonction en escalier est un ensemble fini. En particulier, une fonction en escalier est bornée. 
\end{tcolorbox}

\noindent Si $v = \{ \sigma_0, \ldots, \sigma_n \}$ est une subdivision associée à $f$, alors : 
\begin{align*}
    |Im(f)| &\leq \underbrace{n}_{\text{valeurs sur chaque intervalle ouvert}} + \underbrace{n + 1}_{\text{valeurs de } f(v_i)} = 2n + 1
\end{align*}

\setsection{13}
\section{Subdivision commune}
\begin{tcolorbox}[title=Lemme 26.14, title filled=false, colframe=orange, colback=orange!10!white]
    Soit $f$ et $g$ deux fonctions en escalier. Il existe une subdivision commune associée à $f$ et $g$. 
\end{tcolorbox}

\noindent Si $\sigma$ est une subdivision associée à $f$ et $\tau$ est une subdivision associée à $g$ : 
\begin{align*}
    \sigma \cup \tau &\leq \sigma \\
    &\leq \tau
\end{align*}
Donc $\sigma \cup \tau$ est une subdivision commune associée à $f$ et $g$.

\section{Structure de l'ensemble des fonctions en escalier}
\begin{tcolorbox}[title=Théorème 26.15, title filled=false, colframe=orange, colback=orange!10!white]
    L'ensemble $Esc([a,b])$ des fonctions en escalier sur $[a,b]$ est un sous-espace vectoriel de $\mathbb{R}^[a, b]$ (c'est même une sous-algèbre). 
\end{tcolorbox}

\noindent PRAS (26.14)

\setsection{16}
\section{Théorème}
\begin{tcolorbox}[title=Théorème 26.17, title filled=false, colframe=orange, colback=orange!10!white]
    Pour toutes subdivisions $\sigma$ et $\tau$ associées à $f$, on a : 
    \begin{align*}
        I(f, \sigma) &= I(f, \tau)
    \end{align*}
    Autrement dit, la quantité $I(f, \sigma)$ est indépendante du choix de la subdivision associée. 
\end{tcolorbox}

\noindent Dans un premier temps, on suppose $\tau \subset \sigma$. \\
Notons : 
\begin{align*}
    \tau &= \{ \tau_0, \ldots, \tau_n \} \\
    &= \{ v_{i_0}, \ldots, v_{i_n} \}
\end{align*}
On note $f_k$ la valeur constante de $f$ sur $]\tau_k, \tau_{k+1}[$ et ainsi : 
\begin{align*}
    I(f, \tau) &= \sum_{k=0}^{n-1} (\sigma_{i_{k+1}} - \sigma_{i_k}) f_k \\
    &= \sum_{k=0}^{n-1} \left[ \sum_{p=i_k}^{i_{k+1} - 1} (\sigma_{p+1} - \sigma_p) \right] f_k \\
    &= \sum_{k=0}^{n-1} \sum_{p=i_k}^{i_{k+1} - 1} (\sigma_{p+1} - \sigma_p) f_p \\
    &= \sum_{p=0}^{i_n - 1} (\sigma_{p+1} - \sigma_p) f_p \\
    &= I(f, \sigma)
\end{align*}
Dans le cas général : 
\begin{align*}
    I(f, \tau) = I(f, \tau \cup \sigma) = I(f, \sigma)
\end{align*}


\end{document}