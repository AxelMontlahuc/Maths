\documentclass[../main.tex]{subfiles}

\begin{document}
\setcounter{chapter}{25}
\chapter{Intégration sur un segment}
\tableofcontents
\clearpage

\setsection{11}
\section{Image d'une fonction en escalier}
\begin{tcolorbox}[title=Propostion 26.12, title filled=false, colframe=lightblue, colback=lightblue!10!white]
    L'image d'une fonction en escalier est un ensemble fini. En particulier, une fonction en escalier est bornée. 
\end{tcolorbox}

\noindent Si $v = \{ \sigma_0, \ldots, \sigma_n \}$ est une subdivision associée à $f$, alors : 
\begin{align*}
    |Im(f)| &\leq \underbrace{n}_{\text{valeurs sur chaque intervalle ouvert}} + \underbrace{n + 1}_{\text{valeurs de } f(v_i)} = 2n + 1
\end{align*}

\setsection{13}
\section{Subdivision commune}
\begin{tcolorbox}[title=Lemme 26.14, title filled=false, colframe=orange, colback=orange!10!white]
    Soit $f$ et $g$ deux fonctions en escalier. Il existe une subdivision commune associée à $f$ et $g$. 
\end{tcolorbox}

\noindent Si $\sigma$ est une subdivision associée à $f$ et $\tau$ est une subdivision associée à $g$ : 
\begin{align*}
    \sigma \cup \tau &\leq \sigma \\
    &\leq \tau
\end{align*}
Donc $\sigma \cup \tau$ est une subdivision commune associée à $f$ et $g$.

\section{Structure de l'ensemble des fonctions en escalier}
\begin{tcolorbox}[title=Théorème 26.15, title filled=false, colframe=orange, colback=orange!10!white]
    L'ensemble $Esc([a,b])$ des fonctions en escalier sur $[a,b]$ est un sous-espace vectoriel de $\mathbb{R}^[a, b]$ (c'est même une sous-algèbre). 
\end{tcolorbox}

\noindent PRAS (26.14)

\setsection{16}
\section{Théorème}
\begin{tcolorbox}[title=Théorème 26.17, title filled=false, colframe=orange, colback=orange!10!white]
    Pour toutes subdivisions $\sigma$ et $\tau$ associées à $f$, on a : 
    \begin{align*}
        I(f, \sigma) &= I(f, \tau)
    \end{align*}
    Autrement dit, la quantité $I(f, \sigma)$ est indépendante du choix de la subdivision associée. 
\end{tcolorbox}

\noindent Dans un premier temps, on suppose $\tau \subset \sigma$. \\
Notons : 
\begin{align*}
    \tau &= \{ \tau_0, \ldots, \tau_n \} \\
    &= \{ v_{i_0}, \ldots, v_{i_n} \}
\end{align*}
On note $f_k$ la valeur constante de $f$ sur $]\tau_k, \tau_{k+1}[$ et ainsi : 
\begin{align*}
    I(f, \tau) &= \sum_{k=0}^{n-1} (\sigma_{i_{k+1}} - \sigma_{i_k}) f_k \\
    &= \sum_{k=0}^{n-1} \left[ \sum_{p=i_k}^{i_{k+1} - 1} (\sigma_{p+1} - \sigma_p) \right] f_k \\
    &= \sum_{k=0}^{n-1} \sum_{p=i_k}^{i_{k+1} - 1} (\sigma_{p+1} - \sigma_p) f_p \\
    &= \sum_{p=0}^{i_n - 1} (\sigma_{p+1} - \sigma_p) f_p \\
    &= I(f, \sigma)
\end{align*}
Dans le cas général : 
\begin{align*}
    I(f, \tau) = I(f, \tau \cup \sigma) = I(f, \sigma)
\end{align*}

\setsection{20}
\begin{tcolorbox}[title=Propostion 26.21, title filled=false, colframe=lightblue, colback=lightblue!10!white]
    Soit $f$ une fonction en escalier sur $[a,b]$ et soit $c\in ]a, b[$, alors $f$ est en escalier sur $[a,c]$ et $[c,b]$ et : 
    \begin{align*}
        \int_{a}^{b} f(x) \,dx = \int_{a}^{c} f(x) \,dx + \int_{c}^{b} f(x) \,dx
    \end{align*}
\end{tcolorbox}

\noindent Soit $\sigma$ associée à $f$, $\sigma \cup \{c\}$ est toujours associée à $f$, alors $\sigma \cup \{c\} \cap [a, c]$ est associée à $f_{[a,c]}$. \\
RAS pour la suite. 

\setsection{22}
\section{Intégrale de deux fonctions en escalier égales presque partout}
\begin{tcolorbox}[title=Propostion 26.23, title filled=false, colframe=lightblue, colback=lightblue!10!white]
    Si deux fonctions en escalier ne différent qu'en un nombre fini de points, alors leurs intégrales sont égales.
\end{tcolorbox}

\noindent Dans ce cas, $f-g$ est nulle presque partout et on utilise la linéarité et (26.20). 

\section{Positivité ou croissance de l'intégrale}
\begin{tcolorbox}[title=Propostion 26.24, title filled=false, colframe=lightblue, colback=lightblue!10!white]
    Soit $f$ et $g$ deux fonctions en escalier sur $[a,b]$ (avec $a\leq b$) telles que pour tout $x\in [a, b], f(x) \leq g(x)$, alors : 
    \begin{align*}
        \int_{a}^{b} f(x) \,dx &\leq \int_{a}^{b} g(x) \,dx
    \end{align*}
    En particulier, si $f$ est en escalier sur $[a,b]$ et positive, alors :
    \begin{align*}
        \int_{a}^{b} f(x) \,dx &\geq 0
    \end{align*}
\end{tcolorbox}

\noindent En reprenant la notation du (20.18), pour tout $i$, $f_i \geq 0$. Donc : 
\begin{align*}
    \int_{a}^{b} f(x) \,dx \geq 0
\end{align*}
On obtient la croissance par linéarité. 

\setsection{25}
\section{Inéglité triangulaire intégrale}
\begin{tcolorbox}[title=Propostion 26.26, title filled=false, colframe=lightblue, colback=lightblue!10!white]
    Soit $f$ une fonction en escalier sur $[a,b]$ (avec toujours $a\leq b$) à valeurs réelles. Alors $|f|$ est aussi en escalier sur $[a,b]$ et :
    \begin{align*}
        \left|\int_{a}^{b} f(x) \,dx \right| &\leq \int_{a}^{b} |f(x)| \,dx
    \end{align*}
\end{tcolorbox}

\noindent Si $\sigma$ est associée à $f$, elle reste associée à $|f|$ et ensuite on utilise l'inégalité triangulaire classique avec (26.20). 

\setsection{35}
\section{Théorème}
\begin{tcolorbox}[title=Théorème 26.36, title filled=false, colframe=orange, colback=orange!10!white]
    $f$ est intégrable si et seulement si $I_-(f)$ et $I_+(f)$ existent et si $I_-(f) = I_+(f)$.
\end{tcolorbox}

$\boxed{\Rightarrow}$ \\
On suppose $f$ intégrable. Donc $Esc_+(f)$ et $Esc_-(f)$ ne sont pas vides. \\
En particulier $A_+(f) \neq \emptyset$ est minoré et $A_-(f) \neq \emptyset$ est majoré. \\
D'après la propriété fondamentale de $\mathbb{R}$, $I_-(f)$ et $I_+(f)$ sont bien définis. \\
Soit $\epsilon > 0$, on choisit $(h, g) \in Esc_-(f) \times Esc_+(f)$ tel que : 
\begin{align*}
    \int_{a}^{b} (g - h)(x) \,dx < \epsilon
\end{align*}
Donc : 
\begin{align*}
   I_+ \leq \int_{a}^{b} g(x) \,dx < \int_{a}^{b} h(x) \,dx + \epsilon \leq I_- + \epsilon \\
\end{align*}
Donc : 
\begin{align*}
    I_+ \leq I_- + \epsilon
\end{align*}
Donc :
\begin{align*}
    I_+ \leq I_-
\end{align*}
Donc : 
\begin{align*}
    I_+ = I_-
\end{align*} \\

$\boxed{\Leftarrow}$ \\
On suppose $I_+ = I_-$. \\
Soit $\epsilon > 0$. \\
$I_+ + \frac{\epsilon}{2}$ ne minore pas $A_+$. \\
$I_- - \frac{\epsilon}{2}$ ne majore pas $A_-$. \\
On choisit donc $h\in Esc_-$ et $g\in Esc_+$ telles que : 
\begin{align*}
    \int_{a}^{b} g(x) \,dx &< I_+ + \frac{\epsilon}{2} \\
    \int_{a}^{b} h(x) \,dx &> I_- - \frac{\epsilon}{2}
\end{align*}
Donc :
\begin{align*}
    \int_{a}^{b} (g(x) - h(x)) \,dx < I_+ - I_- + \epsilon = \epsilon 
\end{align*}

\setsection{41}
\section{Intégrabilité des fonctions monotones}
\begin{tcolorbox}[title=Théorème 26.42, title filled=false, colframe=orange, colback=orange!10!white]
    Soit $f$ une fonction monotone sur $[a, b]$. Alors $f$ est intégrable sur $[a, b]$. 
\end{tcolorbox}

\noindent On suppose $f$ croissante. Alors $f$ est bornée (minorée par $f(a)$, majorée par $f(b)$). \\
Pour tout $n\in \mathbb{N}^*$, on note $\sigma_n$ la subdivision régulière de $[a, b]$ à $n$ pas. 
\begin{align*}
    \forall k\in \llbracket 0, n \rrbracket, \sigma_k^{(n)} = a + \frac{(b-a)}{n}k
\end{align*}
On définit $h_n \in Esc_-(f)$ et $g_n \in Esc_+(f)$ par :
\begin{align*}
    \begin{cases}
        \forall x\in ]\sigma_k^{(n)}, \sigma_{k+1}^{(n)}], g_n(x) &= f(\sigma_{k+1}^{(n)}) \\
        g_n(a) = f(a) \\
    \end{cases}
     \\
     \begin{cases}
        \forall x\in [\sigma_k^{(n)}, \sigma_{k+1}^{(n)}[, h_n(x) &= f(\sigma_k^{(n)}) \\
        h_n(b) = f(b) \\
     \end{cases}
\end{align*}
\begin{align*}
    \int_{a}^{b} (g_n - h_n) &= \sum_{k=0}^{n-1} \frac{b-a}{n} \times (f(\sigma_{k+1}^{(n)}) - f(\sigma_k^{(n)})) \\
    &= \frac{b-a}{n} (f(b) - f(a)) \\
    &\underset{n \to +\infty}{\longrightarrow} 0
\end{align*}
D'après (26.41), $f$ est intégrable. 

\section{Intégrabilité des fonctions continues}
\begin{tcolorbox}[title=Théorème 26.43, title filled=false, colframe=orange, colback=orange!10!white]
    Soit $f$ une fonction continue sur $[a, b]$. Alors $f$ est intégrable sur $[a, b]$.
\end{tcolorbox}

\noindent Soit $f\in \mathcal{C}^0([a, b], \mathbb{R})$. \\
Comme $[a, b]$ est un segment, $f$ est uniformément continue sur $[a, b]$ d'après le theorème de Heine. \\
Soit $\epsilon > 0$. On choisit $\eta > 0$ tel que : 
\begin{align*}
    \forall (x, y) \in [a, b]^2, |x - y| < \eta \Rightarrow |f(x) - f(y)| < \epsilon
\end{align*}
Soit $\sigma^{(n)}$ la subsdivision régulière de $[a, b]$ à $n$ pas ($n \geq 1$). \\
On choisit $n$ tel que $\frac{b-a}{n} < \eta$. \\
Pour $k\in \llbracket 0, n-1 \rrbracket$, $f$ est continue sur $[\sigma_k^{(n)}, \sigma_{k+1}^{(n)}]$ donc $y$ atteint ses bornes ($[\sigma_k^{(n)}, \sigma_{k+1}^{(n)}]$ est compact/théorème des bornes atteintes). \\
On note alors $m_k$ et $M_k$ respesctivement les minimum et maximum sur $[\sigma_k^{(n)}, \sigma_{k+1}^{(n)}]$. \\
On pose alors $h_n$ et $g_n$. \\
\begin{itemize}
    \item Pour $x\in [\sigma_k^{(n)}, \sigma_{k+1}^{(n)}[, h_n(x) = m_k$ et $g_n(x) = M_k$. \\
    \item $h_n(b) = g_n(b) = f(b)$
\end{itemize}
Par construction, $h_n \in Esc_-(f)$ et $g_n \in Esc_+(f)$, et : 
\begin{align*}
    \int_{a}^{b} (g_n - h_n) = \sum_{k=0}^{n-1} (\sigma_{k+1}^{(n)} - \sigma_k^{(n)}) (M_k - m_k) < \sum_{k=0}^{n-1} (\sigma_{k+1}^{(n)} - \sigma_k^{(n)}) \times \epsilon = \epsilon \times (b - a) \\
\end{align*}
Par définition : 
\begin{align*}
    \int_{a}^{b} (g_n - h_n)  \underset{n \to +\infty}{\longrightarrow} 0
\end{align*}

\setsection{45}
\section{Relation de Chasles}
\begin{tcolorbox}[title=Propostion 26.46, title filled=false, colframe=lightblue, colback=lightblue!10!white]
    Soit une fonction $f$ définie sur $[a, b]$ et $c\in ]a, b[$. Alors $f$ est intégrable sur $[a, b]$ si et suelement si $f$ est intégrable sur $[a, c]$ et $[c, b]$ et dans ce cas : 
    \begin{align*}
        \int_{a}^{b} f(x) \,dx &= \int_{a}^{c} f(x) \,dx + \int_{c}^{b} f(x) \,dx
    \end{align*}
\end{tcolorbox}

\noindent cf. annexe

\setsection{48}
\section{Croissance et positivité de l'intégrale}
\begin{tcolorbox}[title=Propostion 20.49, title filled=false, colframe=lightblue, colback=lightblue!10!white]
    Soit $f$ et $g$ deux fonction intégrables sur $[a, b]$ (avec $a \leq b$) telles que pour tout $x\in [a, b], f(x) \leq g(x)$. Alors : 
    \begin{align*}
        \int_{a}^{b} f(x) \,dx &\leq \int_{a}^{b} g(x) \,dx
    \end{align*}
    En particulier, si $f$ est intégrable sur $[a, b]$ et positive, alors :
    \begin{align*}
        \int_{a}^{b} f(x) \,dx &\geq 0
    \end{align*}
\end{tcolorbox}

\noindent Si $f\geq 0$, alors $0\in Esc_-(f)$. \\
\begin{align*}
    \int_{a}^{b} 0 = 0 \in A_-(f)
\end{align*}
Donc :
\begin{align*}
    I_-(f) = \int_{a}^{b} f \geq 0
\end{align*}

\setsection{50}
\section{Inégalité triangulaire intégrale}
\begin{tcolorbox}[title=Propostion 26.51, title filled=false, colframe=lightblue, colback=lightblue!10!white]
    Soit $f$ une fonction intégrable sur $[a, b]$, alors $|f|$ est intégrable sur $[a, b]$ et : 
    \begin{align*}
        \left| \int_{a}^{b} f(x) \,dx \right| \leq \int_{a}^{b} |f(x)| \,dx
    \end{align*}
\end{tcolorbox}

\noindent On suppose $f$ intégrable sur $[a, b]$. \\
On choisit $(\varphi_n, \theta_n)$ associé à $f$ (26.39). \\
Comme : 
\begin{align*}
    \forall x\in [a, b], ||f(x)| - |\varphi_n(x)|| \leq |f(x) - \varphi_n(x)| \leq \theta_n(x)
\end{align*}
Alors ($|\varphi_n|, \theta_n)$ est associée à $|f|$. Par conséquent, $|f|$ est intégrable sur $[a, b]$. On a : 
\begin{align*}
    \int_{a}^{b} f(x) \,dx = \lim_{n\to +\infty} \int_{a}^{b} |\varphi_n(x)| \,dx
\end{align*}
Or, d'après (26.26) :
\begin{align*}
    \left| \int_{a}^{b} \varphi_n(x) \,dx \right| \leq \int_{a}^{b} |\varphi_n(x)| \,dx
\end{align*}
Donc, d'arpès le TCILPPL : 
\begin{align*}
    \left| \int_{a}^{b} f(x) \,dx \right| \leq \int_{a}^{b} |f(x)| \,dx
\end{align*}

\setsection{55}
\section{Bornitude des fonctions continues par morceaux}
\begin{tcolorbox}[title=Propostion 26.56, title filled=false, colframe=lightblue, colback=lightblue!10!white]
    Les fonctions continues par morceaux sur un segment $[a, b]$ sont bornées.
\end{tcolorbox}

\noindent Soit $f$ continue par morceaux sur $[a, b]$. \\
Soit $\sigma$ une subdivision associée. \\
Comme $f$ est continue sur $]\sigma_i, \sigma_{i+1}[$ et que $f$ possède des limites finies en $\sigma_i^+$ et $\sigma_{i+1}^-$, $f$ se prolonge par continuité en $f_i$ sur $[\sigma_i, \sigma_{i+1}]$. \\
D'après le théorème des bornes atteintes, $f_i$ est bornée. \\
Donc $\left. f \right|_{]\sigma_i, \sigma_{i+1}[}$ est également bornée. \\
Donc $\left. f \right|_{[a, b] \backslash \{\sigma_0, \ldots, \sigma_n\}}$ est bornée. \\
Donc $f$ est bornée sur $[a, b]$ car $f$ est définie sur chaque $\sigma_i$. 

\setsection{57}
\section{Intégrabilité des fonctions continues par morceaux}
\begin{tcolorbox}[title=Théorème 26.58, title filled=false, colframe=orange, colback=orange!10!white]
    Toute fonction continue par morceaux sur le segment $[a, b]$ est intégrable. 
\end{tcolorbox}

\noindent Soit $f\in \mathcal{CM}([a, b], \mathbb{R})$. \\
Soit $\sigma$ une subdivision associée à $f$. \\
Sur chaque intervalle $]\sigma_i, \sigma_{i+1}[$, $f$ se prolonge par continuité en $f_i$ sur $[\sigma_i, \sigma_{i+1}]$. \\
Donc $f_i$ est intégrable sur $[\sigma_i, \sigma_{i+1}]$ et $f_i$ et $\left. f\right|_{[\sigma_i, \sigma_{i+1}]}$ sont égales presque partout, donc $\left. f\right|_{[\sigma_i, \sigma_{i+1}]}$ est intégrable sur $[\sigma_i, \sigma_{i+1}]$. \\
D'après la relation de Chasles, $f$ est intégrable sur $[a, b]$. 

\setsection{60}
\section{Norme}
\begin{tcolorbox}[title=Propostion 26.61, title filled=false, colframe=lightblue, colback=lightblue!10!white]
    Pour toute fonction $f$ et $g$ bornées sur un même segment $[a, b]$, on a : 
    \begin{align*}
        ||f + g||_{\infty} &\leq ||f||_{\infty} + ||g||_{\infty} \\
    \end{align*}
    et si $\lambda \in \mathbb{R}$, alors : 
    \begin{align*}
        ||\lambda f||_{\infty} &= |\lambda| \times ||f||_{\infty} \\
    \end{align*}
    Enfin : 
    \begin{align*}
        ||f||_{\infty} = 0 \Leftrightarrow f = 0
    \end{align*}
\end{tcolorbox}

\begin{itemize}
    \item D'après l'inégalité triangulaire : 
    \begin{align*}
        \forall x [a, b], |f(x) + g(x)| &\leq |f(x)| + |g(x)| \\
        &\leq ||f||_{\infty} + ||g||_{\infty}
    \end{align*}
    Par définition : 
    \begin{align*}
        ||f + g||_{\infty} \leq ||f||_{\infty} + ||g||_{\infty}
    \end{align*}
    \item RAF
    \item Si $f = 0$, $||f||_{\infty} = 0$. \\
    Si $||f||_{\infty} = 0$, alors $\forall x\in [a, b], |f(x)| = 0$. \\
    Donc $f = 0$. 
\end{itemize}

\setsection{62}
\section{Densité}
\begin{tcolorbox}[title=Théorème 26.63, title filled=false, colframe=orange, colback=orange!10!white]
    \begin{itemize}
        \item Soit $f$ une fonction continue sur $[a, b]$. Alors il existe une suite de fonctions en escalier $(\varphi_n)_{n\in \mathbb{N}}$ telle que :
        \begin{align*}
            ||f - \varphi_n||_{\infty} &\underset{n \to +\infty}{\longrightarrow} 0
        \end{align*}
        \item Soit $f$ une fonction continue par morceaux sur $[a, b]$ alors il existe une suite de fonctions en escalier $(\varphi_n)_{n\in \mathbb{N}}$ telle que :
        \begin{align*}
            ||f - \varphi_n||_{\infty} &\underset{n \to +\infty}{\longrightarrow} 0
        \end{align*}
    \end{itemize}
\end{tcolorbox}

\begin{itemize}
    \item Soit $f\in \mathcal{C}^0([a, b], \mathbb{R})$, donc $f$ est uniformément continue sur $[a, b]$. \\
    Soit $\epsilon > 0$, on choisit $\eta > 0$ module de continuité uniforme associé à $\epsilon$. \\
    Soit $n\in \mathbb{N}^*$, on introduit la subdivision régulière $\sigma^{(n)}$ de $[a, b]$. \\
    On choisit $n$ tel que $\frac{b-a}{n} < \eta$. \\
    Pour tout $k\in \llbracket 0, n-1 \rrbracket$, $f$ est continue sur $[\sigma_k^{(n)}, \sigma_{k+1}^{(n)}]$ donc y atteint ses bornes (max) $M_k$. On définit $\varphi_n \in Esc([a, b], \mathbb{R})$ par :
    \begin{itemize}
        \item si $x\in [\sigma_k^{(n)}, \sigma_{k+1}^{(n)}[$, alors $\varphi_n(x) = M_k$ \\
        \item $\varphi_n(b) = f(b)$
    \end{itemize}
    Par construction, pour tout $x\in [a, b]$ : 
    \begin{align*}
        |f(x) - \varphi_n(x)| \leq \epsilon
    \end{align*}
    Donc :
    $$||f - \varphi_n||_{\infty} \leq \epsilon$$
    Par définition :
    $$||f - \varphi_n||_{\infty} \underset{n \to +\infty}{\longrightarrow} 0$$
    \item Si $f\in \mathcal{CM}([a, b], \mathbb{R})$, et $\sigma$ une subdivision associée à $f$, on applique le résultat précédent sur chaque intervalle $[\sigma_i, \sigma_{i+1}]$. \\
\end{itemize}

\section{Théorème fondamental du calcul intégral}
\begin{tcolorbox}[title=Théorème 26.64, title filled=false, colframe=orange, colback=orange!10!white]
    Soit $f$ une fonction continue sur un intervalle $I$. Soit $x_0\in I$. Alors l'application : 
    \begin{align*}
        x\mapsto \int_{x_0}^{x} f(t) \,dt
    \end{align*}
    est l'unique primitive de $f$ sur $I$ qui s'annule en $x_0$. \\
\end{tcolorbox}

\noindent Notons $F:x\mapsto \int_{x_0}^{x} f(t) \,dt$, bien définie car $f$ est continue sur $I$. \\
$F(x_0) = 0$. \\
Montrons que $F$ est une primitive de $f$ sur $I$. \\
Soit $a\in I$ et soit $x\neq a$. 
\begin{align*}
    \frac{F(x) - F(a)}{x - a} = \frac{1}{x - a} \int_{a}^{x} f(t) \,dt
\end{align*}
Donc : 
\begin{align*}
    \frac{F(x) - F(a)}{x - a} - f(a) &= \frac{1}{x - a} \int_{a}^{x} f(t) \,dt - \frac{1}{x - a} \int_{a}^{x} f(a) \,dt \\
    &= \frac{1}{x - a} \int_{a}^{x} (f(t) - f(a)) \,dt
\end{align*}
Soit $\epsilon > 0$, par continuité de $f$ en $a$, on choisit $\eta > 0$ tel que : 
\begin{align*}
    \forall x\in I, |x - a| < \eta \Rightarrow |f(x) - f(a)| < \epsilon
\end{align*}
On suppose $x > a$ et $x - a < \eta$, d'après l'inégalité triangulaire, on a : \\
\begin{align*}
    \left| \frac{F(x) - F(a)}{x - a} - f(a) \right| &\leq \frac{1}{x - a} \int_{a}^{x} |f(t) - f(a)| \,dt \\
    &\leq \frac{1}{x - a} \int_{a}^{x} \epsilon \,dt \\
    &= \epsilon
\end{align*}
Cela reste valable si $x < a$ et $|x - a| < \eta$. \\
Donc : 
\begin{align*}
    \frac{F(x) - F(a)}{x - a} \underset{x \to a}{\longrightarrow} f(a)
\end{align*}

\setsection{65}
\section{Limite}
\begin{tcolorbox}[title=Propostion 26.66, title filled=false, colframe=lightblue, colback=lightblue!10!white]
    Pour toute fonction $f\in \mathcal{C}^0([a, b], \mathbb{R})$ : 
    \begin{align*}
        \int_{a}^{b} f(t) \,dt &= \lim_{x\to b^-} \int_{a}^{x} f(t) \,dt \quad\text{ et }\quad \int_{a}^{b} f(t) \,dt = \lim_{x\to a^+} \int_{x}^{b} f(t) \,dt
    \end{align*}
\end{tcolorbox}

\noindent On fixe $a$ et on pose $F:x\mapsto \int_{a}^{x} f(t) \,dt$. \\
Donc $F\in \mathcal{C}^0([a, b], \mathbb{R})$. \\
Donc $F(b) = \lim\limits_{x\to b} F(x)$. 

\setsection{67}
\section{Exemple}
\begin{tcolorbox}[title=Exemple 26.68, title filled=false, colframe=darkgreen, colback=darkgreen!10!white]
    La fonction $\varphi:x\mapsto \int_{0}^{x} \exp (xt^2) \,dt$ est définie et dérivable sur $\mathbb{R}^*$, de dérivée : 
    \begin{align*}
        x\mapsto \frac{3e{x^3}}{2} - \frac{1}{2x} \int_{0}^{x} \exp (xt^2) \,dt
    \end{align*}
\end{tcolorbox}

\noindent Pour $x > 0$ : 
\begin{align*}
    \varphi(x) = \int_{0}^{x} \exp (xt^2) \,dt = \int_{0}^{1} \exp (xt^2) \,dt + \int_{1}^{x} e^{xt^2} \,dt
\end{align*}
On effectue le changement de variable $u^2 = xt^2$, soit $u = \sqrt{x} t$ donc $du = \sqrt{x} \,dt$. \\
Si $t = 0$, $u = 0$. \\
Si $t = x$, $u = x^{\frac{3}{2}}$. 
\begin{align*}
    \varphi(x) &= \frac{1}{\sqrt{x}} \int_{0}^{x^{\frac{3}{2}}} e^{u^2} \,du \\
    &= \frac{1}{\sqrt{x}} F(x^{\frac{3}{2}}) \\
\end{align*}
avec d'après le TFCI $F:x\mapsto \int_{0}^{x} e^{u^2} \,du \in \mathcal{C}^1(\mathbb{R}, \mathbb{R})$. \\
Par opération, $\varphi$ est dérivable sur $\mathbb{R}_+^*$ et : 
\begin{align*}
    \varphi'(x) &= -\frac{1}{2x \sqrt{x}} F(x^{\frac{3}{2}}) + \frac{3}{2} F'(x^{\frac{3}{2}}) \\
    &= -\frac{1}{2x} \int_{0}^{x} e^{xt^2} \,dt + \frac{3}{2} e^{x^{3}}
\end{align*}
Pour $x < 0$, on effectue le changement de variable $u^2 = -xt^2$, soit $u = \sqrt{-x} t$ et on suit la méthode principale. 

\section{Intégrale nulle d'une fonction positive et continue}
\begin{tcolorbox}[title=Propostion 26.69, title filled=false, colframe=lightblue, colback=lightblue!10!white]
    Soit $f:[a, b]\to \mathbb{R}$ une fonction continue et positive, avec $a < b$. Alors : 
    \begin{align*}
        \int_{a}^{b} f(t) \,dt = 0 \Leftrightarrow f = 0
    \end{align*}
\end{tcolorbox}

$\boxed{\Rightarrow}$ \\
$f$ est continue et positive, donc d'après le TFCI : \\
$F:x\mapsto \int_{a}^{x} f(t) \,dt$ est dérivable sur $[a, b]$ avec $F' = f \geq 0$ donc $F$ est croissante sur $[a, b]$. \\
Or $F(a) = 0 = F(b)$. \\
Donc $F = 0$, puis $f = F' = 0$. 

\section{Somme de Riemann}
\begin{tcolorbox}[title=Théorème 26.70, title filled=false, colframe=orange, colback=orange!10!white]
    Soit $f$ une fonction continue sur $[a, b]$. Alors : 
    \begin{align*}
        \int_{a}^{b} f(x) \,dx = \lim_{n\to +\infty} \frac{b - a}{n} \sum_{k=0}^{n-1} f\left(a + k\frac{(b-a)}{n}\right) = \lim_{n\to +\infty} \sum_{k=1}^{n} \frac{b-a}{n} f\left(a + k\frac{(b-a)}{n}\right) 
    \end{align*}
    Plus généralement, soit pour tout $n\in \mathbb{N}, \sigma^{(n)} = (\sigma^{(n)}_k)_{k\in \llbracket 0, n \rrbracket}$ une subdivision et supposons que la suite des pas vérifie : 
    \begin{align*}
        p(\sigma^{(n)}) \underset{n \to +\infty}{\longrightarrow} 0
    \end{align*}
    et soit pour tout $n\in \mathbb{N}$ et tout $k\in \llbracket 0, \ell_n - 1 \rrbracket$, $x_{n,k}$ un élément de $[\sigma_k^{(n)}, \sigma_{k+1}^{(n)}]$. Alors :
    \begin{align*}
        \int_{a}^{b} f(x) \,dx = \lim_{n\to +\infty} \sum_{k=0}^{n-1} (\sigma_{k+1}^{(n)} - \sigma_k^{(n)}) f(x_{n,k}) 
    \end{align*}
\end{tcolorbox}

\noindent Soit $\epsilon > 0$, on choisit $\eta$ un module de continuité uniforme pour $f$ d'après le théorème de Heine. \\
On définit, pour tout $n\in \mathbb{N}^*, \varphi_n \in Esc([a, b], \mathbb{R})$ par : 
\begin{itemize}
    \item pour $x\in [\sigma_k^{(n)}, \sigma_{k+1}^{(n)}[$, $\varphi_n(x) = f(x_{n,k})$
    \item $\varphi_n(b) = f(b)$
\end{itemize}
Or $p(\sigma^{(n)}) \underset{n \to +\infty}{\longrightarrow} 0$. On choisit $N\in \mathbb{N}$ tel que : 
\begin{align*}
    \forall n\geq N, p(\sigma^{(n)}) < \eta
\end{align*}
Pour $n\geq N$ : 
\begin{align*}
    |f(x) - \varphi_n(x)| \leq \epsilon
\end{align*}
Par définition : 
\begin{align*}
    ||f - \varphi_n||_{\infty} \longrightarrow 0
\end{align*}
Donc : 
\begin{align*}
    \int_{a}^{b} f(x) \,dx = \lim_{n\to +\infty} \int_{a}^{b} \varphi_n(x) \,dx
\end{align*}
Puis (26.18). 

\setsection{71}
\section{Exemple}
\begin{tcolorbox}[title=Exemple 26.72, title filled=false, colframe=darkgreen, colback=darkgreen!10!white]
    On montre que : 
    \begin{align*}
        \sum_{k=1}^{n} \frac{1}{n+k} \underset{n \to +\infty}{\longrightarrow} \ln 2
    \end{align*}
\end{tcolorbox}

\begin{align*}
    \sum_{k=1}^{n} \frac{1}{n+k} = \frac{1}{n} \sum_{k=1}^{n} \frac{1}{1 + \frac{4}{n}} = \frac{1}{n} \sum_{k=1}^{n} f(\frac{k}{n})
\end{align*}
avec $f:x\mapsto \frac{1}{1 + x}\in \mathcal{C}^0([0, 1], \mathbb{R})$.
Donc TSR : 
\begin{align*}
    \sum_{k=1}^{n} \frac{1}{n+k} \underset{n \to +\infty}{\longrightarrow} \int_{0}^{1} f(x) \,dx = \int_{0}^{1} \frac{1}{1+x} \,dx = \ln(2) - \ln(1) = \ln(2)
\end{align*}

\setsection{74}
\section{Inégalité triangulaire intégrale dans $\mathbb{C}$}
\begin{tcolorbox}[title=Théorème 26.75, title filled=false, colframe=orange, colback=orange!10!white]
    Soit $f:[a, b]\to \mathbb{C}$ intégrable, avec $a < b$. Alors $|f|$ est aussi intégrable et : 
    \begin{align*}
        \left| \int_{a}^{b} f(t) \,dt \right| \leq \int_{a}^{b} |f(t)| \,dt
    \end{align*}
\end{tcolorbox}

On décompose $\int_{a}^{b} f(t) \,dt = re^{i\theta}$ avec $r\geq 0$ et $\theta\in \mathbb{R}$. \\
Par opération, $|f|$ est intégrable. \\
On pose $g = e^{-i\theta} \times f$. \\
Par linéarité : $$\int_{a}^{b} g(t) \,dt = e^{-i\theta} \int_{a}^{b} f(t) \,dt = r$$
On décompose $g = g_r + ig_i$. \\
Par définition : $$\int_{a}^{b} g(t) \,dt = \int_{a}^{b} g_r(t) \,dt + i \int_{a}^{b} g_i(t) \,dt$$
Donc : $$\int_{a}^{b} g_r(t) \,dt = r \quad \text{et} \quad \int_{a}^{b} g_i(t) \,dt = 0$$
\begin{align*}
    \left| \int_{a}^{b} f(t) \,dt \right| = r = \int_{a}^{b} g_r(t) \,dt = \left| \int_{a}^{b} g_r(t) \right| \underbrace{\leq}_{\text{IT sur $\mathbb{R}$}} \int_{a}^{b} |g_r(t)| \,dt \underbrace{\leq}_{\text{croissance de l'I}} \int_{a}^{b} |g(t)| = \int_{a}^{b} |f(t)| \,dt \\ 
\end{align*}

\section*{Exercice 17}
\begin{tcolorbox}[title=Exercice 26.17, title filled=false, colframe=darkgreen, colback=darkgreen!10!white]
    Soit $f$ et $g$ deux fonctions continues sur $\mathbb{R}$ telles que pour tout $x\in \mathbb{R}$, on ait : 
    \begin{align*}
        f(x) = \int_{0}^{x} g(t) \,dt \quad \text{et} \quad g(x) = \int_{0}^{x} f(t) \,dt
    \end{align*}
    Montrer que $f = g = 0$.
\end{tcolorbox}

\noindent D'après le TFCI, $(f, g)\in \mathcal{C}^{\infty}(\mathbb{R}, \mathbb{R})$
\begin{align*}
    f'' = f \quad \text{et} \quad g'' = g
\end{align*}
D'après le chapitre 7, on choisit $(a, b, c, d)\in \mathbb{R}^4$ tel que : 
\begin{align*}
    \begin{cases}
        f:x\mapsto a e^x + b e^{-x} \\
        g:x\mapsto c e^x + d e^{-x}
    \end{cases}
\end{align*}
Or $f' = g$ donc $a = c$ et $b = d$ par liberté de $(x\mapsto e^{x}, x\mapsto e^{-x})$. \\
\begin{align*}
    f(0) &= 0 = g(0) \\
    \text{donc } a &= b = c = d = 0
\end{align*}
Donc : 
\begin{align*}
    f = g = 0
\end{align*}

\section{Lemme de Riemann-Lesbegue}
\begin{tcolorbox}[title=Lemme 26.7, title filled=false, colframe=orange, colback=orange!10!white]
    Soit $f:[a, b] \to \mathcal{C}$. On suppose que $f\in \mathcal{C}^1([a, b], \mathbb{R})$, alors :
    \begin{align*}
        \int_{a}^{b} f(x)e^{int} \,dt \underset{n \to +\infty}{\longrightarrow} 0
    \end{align*}
\end{tcolorbox}

\noindent $f\in \mathcal{C}^1([a, b], \mathbb{R})$. \\
Par IPP ($f\in \mathcal{C}^1([a, b], \mathbb{R})$, $t\mapsto \frac{e^{int}}{in}\in \mathcal{C}^1([a, b], \mathbb{C})$) :
\begin{align*}
    \int_{a}^{b} f(x) e^{int} \,dt &= \left[ \frac{f(x) e^{int}}{in} \right]_a^b - \frac{1}{in} \int_{a}^{b} f'(x) e^{int} \,dt \\
    &= \frac{f(b) e^{inb} - f(a) e^{ina}}{in} - \frac{1}{in} \int_{a}^{b} f'(x) e^{int} \,dt
\end{align*}
D'après l'inégalité triangulaire :
\begin{align*}
    \frac{1}{in} \left| \int_{a}^{b} f'(x) e^{int} \,dt \right| &\leq \frac{1}{in} \int_{a}^{b} |f'(x)| \,dt
\end{align*}


\end{document}