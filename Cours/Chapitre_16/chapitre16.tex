\documentclass[../main.tex]{subfiles}

\begin{document}
\setcounter{chapter}{15}
\chapter{Arithmétique des polynômes}
\tableofcontents
\clearpage

\section{Division euclidienne}
\begin{tcolorbox}[title=Théorème 16.1, title filled=false, colframe=orange, colback=orange!10!white]
    Soit $A \in \mathbb{K}[X]$ et $B \in \mathbb{K}[X]$ non nul, il existe un unique couple de polynômes $(Q,R)$ tel que $A = BQ + R$ avec $\deg R < \deg B$. Le polynôme $Q$ est appelé \textbf{quotient} et $R$ le \textbf{reste}. 
\end{tcolorbox}

\noindent \underline{Existence :} \\
On raisonne par récurrence sur le degré de $A$. \\
\begin{itemize}
    \item Pour $n = \deg A = 0$. Soit $A \in \mathbb{K}[X]$.
    \begin{itemize}
        \item Si $\deg B > 0$, alors $(0, A)$ convient. \\
        \item Si $\deg B = 0$, le couple $(B^{-1} \times A, 0)$ convient (comme $B$ est constant et non nul), alors $B \in \mathbb{K}^*$ donc inversible). \\
    \end{itemize}

    \item On suppose le résultat vrai pour tout $A \in \mathbb{K}_n[X]$. \\
    Soit $A \in \mathbb{K}_{n+1}[X]$ avec $\deg A = n+1$. \\
    On écrit $A = \underbrace{a}_{\neq 0} X^{n+1} + A_1$ avec $A_1 \in \mathbb{K}_n[X]$. 
    \begin{itemize}
        \item Si $\deg A < \deg B$, le couple $(0, A)$ convient. 
        \item Si $\deg A \geq \deg B$ et on note $b$ le coefficient dominant de $B$ : 
        \begin{align*}
            A - ab^{-1} B \times X^{n+1 - \deg B} \in \mathbb{K}_n[X]
        \end{align*}
        D'après l'hypothèse de récurrence, on choisit $(Q, R) \in \mathbb{K}[X]^2$ tel que $\deg R < \deg B$ et $A - ab^{-1} B \times X^{n+1 - \deg B} = QB + R$. \\
        Donc : 
        \begin{align*}
            A = \left[ Q + ab^{-1}X^{n+1 - \deg A} \right] \times B + R
        \end{align*}
    \end{itemize}
\end{itemize}

\noindent\underline{Unicité :} \\
On suppose que $A = BQ + R = BQ_1 + R_1$. \\
Donc : 
\begin{align*}
    B(Q - Q_1) &= R_1 - R \\
    \text{donc } \underbrace{\deg{(B(Q - Q_1))}}_{\deg{B} + \deg{Q - Q_1}} &= \deg{(R_1 - R)} \\
    &\leq \max(\deg{R_1}, \deg{R}) \\
    &< \deg{B} \\
    \text{donc } \deg{(Q - Q_1)} &< 0 \\
    \text{donc } Q - Q_1 &= 0 \\
    \text{puis } R_1 - R = 0
\end{align*}

\setcounter{section}{6}
\section{Proposition 16.7}
\begin{tcolorbox}[title=Propostion 16.7, title filled=false, colframe=lightblue, colback=lightblue!10!white]
    On a :
    \begin{enumerate}
        \item Soit $A$ et $P$ deux polynômes non nuls. Si $A|P$ et si $P|A$, alors il existe $\alpha \in \mathbb{K}^*$ tel que $P = \alpha A$. (La relation de divisibilité n'est pas antisymétrique)
        \item Si $A|B$ et si $B|C$, alors $A|C$. La relation de divisibilité est transitive. 
        \item Pour tout $A \in \mathbb{K}[X]$ non nul, $A|A$. La relation de divisibilité est réflexive. 
    \end{enumerate}
\end{tcolorbox}

\begin{enumerate}
    \item $P \neq 0$, $A \neq 0$. Si $A | P$ et $P | A$, alors (16.6.2) :
    \begin{align*}
        \deg A \leq \deg P \text{ et } \deg P \leq \deg A
    \end{align*}
    Donc : 
    \begin{align*}
        \deg P = \deg A
    \end{align*}
    Or $A|P$, alors : 
    \begin{align*}
        P = A \times Q
    \end{align*}
    Puis : 
    \begin{align*}
        \deg P = \deg (AQ) = \deg A + \deg Q \text{ ($\mathbb{K}$ est intègre)}
    \end{align*}
    Donc :
    \begin{align*}
        \deg Q = 0
    \end{align*}
    Donc : 
    \begin{align*}
        Q = \alpha \in \mathbb{K}^*
    \end{align*}

    \item RAS
    \item RAS
\end{enumerate}

\setcounter{section}{14}
\section{Principalité de $\mathbb{K}[X]$}
\begin{tcolorbox}[title=Théorème 16.15, title filled=false, colframe=orange, colback=orange!10!white]
    Soit $I$ un idéal de $\mathbb{K}[X]$ non réduit à $\{0\}$. Il existe un unique polynôme unitaire $D$ tel que
    $$I = D \mathbb{K}[X]$$
\end{tcolorbox}

\noindent \underline{Existence :} \\
Soit $I \neq \{0\}$ un idéal. \\
On note $A = \{ \deg P, P \in I\backslash \{0\} \} \subset \mathbb{N}$. \\
$A \neq \emptyset$ ($I \neq \{0\}$), d'après la propriété fondamentale de $\mathbb{N}$, $A$ possède un plus petit élément noté $n \geq 0$. \\
Comme $n \in A$, on choisit $D \in I$ tel que $\deg D = n$. \\
Comme $I$ est un idéal de $\mathbb{K}[X]$ et que $\mathbb{K} = \mathbb{K}_0[X] \subset \mathbb{K}[X]$, on a : 
\begin{align*}
    \forall \alpha \in \mathbb{K}, \alpha D \in I
\end{align*}
On peut donc supposer $D$ unitaire. 
Comme $I$ est un idéal de $\mathbb{K}[X]$, on a : 
\begin{align*}
    D \times \mathbb{K}[X] \subset I
\end{align*}
Soit $P \in I$. On effectue la division euclidienne de $P$ par $D$ ($\neq 0$) : 
\begin{align*}
    P = BD + R
\end{align*}
avec $\deg R \subset \deg D$. \\
Or : 
\begin{align*}
    R &= \underbrace{P}_{\in I} - \underbrace{BD}_{\in I} \\
    &\in I
\end{align*}
Par définition de $\deg D = n$, $R = 0$. \\ \\

\noindent \underline{Unicité :} \\
\begin{align*}
    I = D \mathbb{K}[X] = J \mathbb{K}[X] \\
\end{align*}
avec $D$ et $J$ unitaires. \\
Or ils sont associés, donc égaux. 

\setcounter{section}{16}
\section{Existence de $pgcd$}
\begin{tcolorbox}[title=Propostion 16.17, title filled=false, colframe=lightblue, colback=lightblue!10!white]
    Si $A$ et $B$ sont deux polynômes non nuls, de tels PGCD existent. 
\end{tcolorbox}

\noindent Soit $A, B$ dans $\mathbb{K}[X]$, $(A, B) \neq (0, 0)$.  \\
On note $\mathcal{C} = \{ \deg P, P|A \text{ et } P|B \text{ et } P\neq 0 \} \subset \mathbb{N}$. \\
$\mathcal{C} \neq \emptyset$ car $0 \in \mathcal{C}$ et $\mathcal{C}$ est majoré par $\deg B$ ($\max (\deg A, \deg B)$). \\
L'existence est assurée par la propriété fondamentale de $\mathbb{N}$. 

\section{Principalité de $\mathbb{K}[X]$}
\begin{tcolorbox}[title=Propostion 16.18, title filled=false, colframe=lightblue, colback=lightblue!10!white]
    Soit $A$ et $B$ deux polynômes non tous deux nuls. Soit $D \in \mathbb{K}[X]$. Alors $\Delta$ est un PGCD de $A$ et $B$ si et seulement si 
    $$A \mathbb{K}[X] + B \mathbb{K}[X] = D \mathbb{K}[X].$$
\end{tcolorbox}

\noindent D'après (16.15), on choisit $F \in \mathbb{K}[X]$ tel que :
\begin{align*}
    A \mathbb{K}[X] + B \mathbb{K}[X] = F \mathbb{K}[X]
\end{align*}
Soit $D \in \mathbb{K}[X]$. \\ \\

$\boxed{\Rightarrow}$ \\
On suppose que $D$ est un PGCD. \\
Donc $D|A$ et $D|B$. \\
Donc $D|F \text{ (combinaison $F \in A \mathbb{K}[X] + B \mathbb{K}[X]$)}$. \\
Or $F|A$ et $F|B$ ($A \in F \mathbb{K}[X]$, $B \in F \mathbb{K}[X]$). \\
Par maximalité de $\deg D$, on a $F$ et $D$ associés. \\ \\

$\boxed{\Leftarrow}$ \\
\begin{align*}
    D \mathbb{K}[X] = A \mathbb{K}[X] + B \mathbb{K}[X] = F \mathbb{K}[X]
\end{align*}
Donc $D|A$ et $D|B$. \\
Pour tout diviseur commun $P$ de $A$ et $B$, $P|A$ et $P|B$. \\
Donc $P|D$ ($D \in A \mathbb{K}[X] + B \mathbb{K}[X])$. \\
Donc $\deg D$ est maximal pour la divisibilité. 

\setcounter{section}{23}
\section{Lemme de préparation au calcul pratique du PGCD unitaire}
\begin{tcolorbox}[title=Lemme 16.24, title filled=false, colframe=orange, colback=orange!10!white]
    Soit $A$ et $B$ deux polynômes tels que $B \neq 0$. Pour tout $Q \in \mathbb{K}[X]$, on a $A \wedge B = (A - BQ) \wedge B$. \\
    En particulier, si $Q$ et $R$ sont le quotient et le reste de la division euclidienne de $A$ par $B$ Alors $A \wedge B = B \wedge R$.
\end{tcolorbox}

\begin{align*}
    (A \wedge B) \mathbb{K}[X] &= A \mathbb{K}[X] + B \mathbb{K}[X] \\
    &= (A - BQ) \mathbb{K}[X] + B \mathbb{K}[X] \\
    &= ((A - BQ) \wedge B) \mathbb{K}[X]
\end{align*}
Donc $A \wedge B$ et $(A - BQ) \wedge B$ sont associés, unitaires par définition, donc égaux. 

\setsection{25}
\section{Exemple}
\begin{tcolorbox}[title=Exemple alternatif 16.26, title filled=false, colframe=darkgreen, colback=darkgreen!10!white]
    Trouver les PGCD de $A = X^5 + 2X$ et de $B = X^4 + 2X^3 + 4$ et une relation de Bézout. 
\end{tcolorbox}

\begin{align*}
    X^5 + 2X &= (X^4 + 2X^3 + 4)(X - 2) + 4X^3 - 2X + 8 \\
    X^4 + 2X^3 + 4 &= (4X^3 - 2X + 8)(\frac{1}{4}X + \frac{1}{2}) + \frac{1}{2}X^2 - X \\
    4X^3 - 2X + 8 &= (\frac{1}{2}X^2 - X)(8X + 16) + 14X + 8 \\
    \frac{1}{2}X^2 - X &= (14X + 8)(\frac{1}{28}X - \frac{9}{14 \times 7}) + \frac{9 \times 4}{7^2} \\
    A \wedge B = 1
\end{align*}
\begin{align*}
    \frac{9 \times 4}{7^2} &= \frac{1}{2}X^2 - X - (14X + 8)(\frac{1}{28}X - \frac{9}{2 \times 7^2}) \\
    &= \frac{1}{2}X^2 - X - (4X^3 - 2X + 8 - (\frac{1}{2}X^2 - X)(8X + 16))(\frac{1}{28}X - \frac{9}{2 \times 7^2}) \\
\end{align*}

\section{Propriétés du PGCD}
\begin{tcolorbox}[title=Propostion 16.27, title filled=false, colframe=lightblue, colback=lightblue!10!white]
    L'opération $\wedge$ est commutative et associative. Par ailleurs, si $C$ est unitaire, alors $(A \wedge B) C = (AC) \wedge (BC)$.
\end{tcolorbox}

\noindent Soit $(A, B, C) \in \mathbb{K}[X]^3$ non tous nuls. 
\begin{align*}
    (A \wedge B) \mathbb{K}[X] &= A \mathbb{K}[X] + B \mathbb{K}[X] \\
    &= B \mathbb{K}[X] + A \mathbb{K}[X] \\
    &= (B \wedge A) \mathbb{K}[X]
\end{align*}
Donc $A \wedge B$ et $B \wedge A$ sont associés et unitaires donc égaux. 
\begin{align*}
    ((A \wedge B) \wedge C) \mathbb{K}[X] &= (A \wedge B) \mathbb{K}[X] + C \mathbb{K}[X] \\
    &= A \mathbb{K}[X] + B \mathbb{K}[X] + C \mathbb{K}[X] \\
    &= (A \wedge (B \wedge C)) \mathbb{K}[X]
\end{align*}
Donc $A \wedge (B \wedge C)$ et $(A \wedge B) \wedge C$ sont associés et unitaires donc égaux. \\
On suppose $C$ unitaire. \\
On a : 
\begin{align*}
    (A \wedge B) \mathbb{K}[X] &= A \mathbb{K}[X] + B \mathbb{K}[X] \\
    \text{donc } (A \wedge B)C \mathbb{K}[X] &= AC \mathbb{K}[X] + BC \mathbb{K}[X] \\
    &= ((AC) \wedge (BC)) \mathbb{K}[X]
\end{align*}
Ainsi $C(A \wedge B)$ et $(AC) \wedge (BC)$ sont associés et unitaires donc égaux.

\setsection{28}
\section{Existence de PPCM}
\begin{tcolorbox}[title=Propostion 16.29, title filled=false, colframe=lightblue, colback=lightblue!10!white]
    Soit $\mathbb{K}$ un corps. Soit $A$ et $B$ deux polynômes non nuls de $\mathbb{K}[X]$. Alors $A$ et $B$ admettent des PPCM. 
\end{tcolorbox}

\noindent On note $\mathcal{D} = \{ \deg P, A|P, B|P, P\neq 0 \} \subset \mathbb{N}$. \\
\begin{align*}
    \deg AB \in \mathcal{D} \neq \emptyset
\end{align*}
On conclut avec la propriété fondamentale de $\mathbb{N}$. 

\section{Caractérisation des PPCM par les idéaux}
\begin{tcolorbox}[title=Propostion 16.30, title filled=false, colframe=lightblue, colback=lightblue!10!white]
    Soit $A$ et $B$ deux polynômes non nuls de $\mathbb{K}[X]$ et soit $P \in \mathbb{K}[X]$. Alors $P$ est un PPCM de $A$ et $B$ si et seulement si 
    $$A \mathbb{K}[X] \cap B \mathbb{K}[X] = P \mathbb{K}[X].$$
\end{tcolorbox}

\noindent $A \mathbb{K}[X] \cap B \mathbb{K}[X]$ est un idéal de $\mathbb{K}[X]$, donc de la forme $M \mathbb{K}[X]$ (16.15). \\
Montrons que $P$ est un PPCM de $A$ et $B$ si et seulement si $P$ et $M$ sont associés. \\ \\

$\boxed{\Rightarrow}$ \\
On a donc : 
\begin{align*}
    P &\in A \mathbb{K}[X] \cap B \mathbb{K}[X] \\
    &\in M \mathbb{K}[X] \\
\end{align*}
Donc $M|P$. \\
Or $M$ est un multiple commun à $A$ et $B$, donc par définition de $P$, on a : 
\begin{align*}
    \boxed{\deg P \leq \deg M}
\end{align*}
Donc $P$ et $M$ sont associés. \\ \\

$\boxed{\Leftarrow}$ \\
On suppose $P$ et $M$ associés, donc : 
\begin{align*}
    P \mathbb{K}[X] &= M \mathbb{K}[X] \\
    &= A \mathbb{K}[X] \cap B \mathbb{K}[X]
\end{align*}
En particulier, $P$ est un multiple commun à $A$ et $B$ et pour tout $Q \in A \mathbb{K}[X] \cap B \mathbb{K}[X]$, donc $P|Q$. \\
Donc : 
\begin{align*}
    \boxed{deg P \leq \deg Q}
\end{align*}

\setsection{41}
\section{Cas d'unicité d'une relation de Bézout}
\begin{tcolorbox}[title=Propostion 16.42, title filled=false, colframe=lightblue, colback=lightblue!10!white]
    Soit $A$ et $B$ non constants et premiers entre eux. Il existe un unique couple $(U, V) \in \mathbb{K}[X]^2$ tel que
    $$AU + BV = 1 \text{ et } \deg U < \deg B \text{ et } \deg V < \deg A.$$
\end{tcolorbox}

\noindent \underline{Existence :} \\
Soit $(C, D) \in \mathbb{K}[X]^2$ tel que (16.37 - Bézout) :
\begin{align*}
    AC + BD = 1
\end{align*}
On effectue la dviision euclidienne de $C$ par $B$ :
\begin{align*}
    C &= BE + U \text{ avec } \deg U < \deg B \\
    \text{donc } AU + B(\underbrace{D + AE}_{V}) &= 1 \\
    \text{donc } \deg (AU + BV) &= 0 \\
\end{align*}
Si $\deg V \geq \deg A$, alors :
\begin{align*}
    \deg B + \deg V &\geq \deg B + \deg A \\
    &> \deg U + \deg B \\
    &= \deg AU
\end{align*}
Donc $\deg (AU + BV) = \deg BV > 0$. \\
Absurde. \\
L'exsitence est prouvée. \\ \\

\noindent \underline{Unicité :} \\
Avec es hypothèses correspondantes : 
\begin{align*}
    AU_1 + BV_1 &= 1 = AU_2 + BV_2 \\
    \text{donc } A(U_1 - U_2) &= B(V_2 - V_1) \\
    \text{donc } A | B(V_2 - V_1)
\end{align*}
Or $A \wedge B = 1$, donc $A | (V_2 - V_1)$. \\
Or $\deg (V_2 - V_1) < \deg A$. \\
Donc $V_2 - V_1 = 0$. \\
Puis $A(U_1 - U_2) = 0$, donc $U_1 - U_2 = 0$ car $\mathbb{K}[X]$ est intègre avec $A \neq 0$.

\section{Corollaire}
\begin{tcolorbox}[title=Corollaire 16.43, title filled=false, colframe=orange, colback=orange!10!white]
    Soit $A$, $B$ et $C$ trois polynômes avec $A$ et $B$ premiers entre eux. Alors $A \wedge (BC) = A \wedge C$. 
\end{tcolorbox}

\begin{itemize}
    \item $A \wedge C | A$ donc $A \wedge C | A \wedge (BC)$. Donc $A \wedge C | BC$. 
    \item $A \wedge (BC) | A$. Or $A \wedge B = 1$ donc on peut écrire $AU + BV = 1$. Donc $ACU + BCV = C$. \\
    Or $A \wedge (BC) | ACU + BCV$ soit $A \wedge (BC) | C$. Donc $A \wedge (BC) | A \wedge C$.
\end{itemize}
Ainsi, $A \wedge C$ et $A \wedge (BC)$ sont associés et unitaires donc égaux.

\section{Caractérisation des PGCD et PPCM}
\begin{tcolorbox}[title=Propostion 16.44, title filled=false, colframe=lightblue, colback=lightblue!10!white]
    Soit $A$ et $B$ deux polynômes non nuls, $M$ et $D$ deux polynômes. Alors
    $$ M = A \vee B \Leftrightarrow (M \text{ unitaire et } \exists (U, V) \in \mathbb{K}[X]^2, M = AU = BV \text{ et } U \wedge V = 1). $$
    $$ D = A \wedge B \Leftrightarrow (D \text{ unitaire et } \exists (U, V) \in \mathbb{K}[X]^2, A = DU \text{ et } B = DV \text{ et } U \wedge V = 1). $$
\end{tcolorbox}

\begin{itemize}
    \item \indent \indent $\boxed{\Rightarrow}$ \\
    $M = A \vee B$. On écrit $M = AU + BV$ avec $(U, V) \in \mathbb{K}[X]^2$. \\
    On note $R = U \wedge V$. On écrit $U = RU_1$ et $V = RV_1$. \\
    Ainsi :
    \begin{align*}
        M = RAU_1 &= RBV_1 \\
        \text{donc } R(AU_1 - BV_1) &= 0 \\
        \text{donc } AU_1 &= BV_1 \text{ ($\mathbb{K}[X]$ est intègre)} \\
    \end{align*}
    Donc $M_1 = AU_1 = BV_1$ est un multiple commun et par minimalité des degrés : 
    \begin{align*}
        RM_1 = M | M_1 \text{ donc } R = 1
    \end{align*}

    $\boxed{\Leftarrow}$ \\
    Par hypothèse, $M$ est un multiple commun, donc : 
    \begin{align*}
        M \in A \mathbb{K}[X] \cap B \mathbb{K}[X] = (A \vee B) \mathbb{K}[X]
    \end{align*}
    Donc $A \vee B | M$. \\
    Donc $M = D \times A \vee B$. \\
    Or $A \vee B = AU_1 = BV_1$. \\
    Donc $M = DAU_1 = DBV_1 = AU = BV$. \\
    Donc : 
    \begin{align*}
        A(DU_1 - U) &= 0 \\
        B(DV_1 - V) &= 0
    \end{align*}
    Or $\mathbb{K}[X]$ est intègre donc $DU_1 = U$ et $DV_1 = V$. \\
    Donc $D | U \wedge V = 1$. \\
    
    \item \indent \indent $\boxed{\Rightarrow}$ \\
    $D = A \wedge B$. On écrit $A = DU$ et $B = DV$. \\
    Or pour $R = U \wedge V$, on écrit $U = RU_1$ et $V = RV_1$. \\
    Donc $A = DRU_1$ et $B = DRV_1$. \\
    Donc $DR | A$ et $DR | B$. \\
    Donc $DR | D$. \\
    Nécessairement, $R = 1$. \\

    $\boxed{\Leftarrow}$ \\
    Par hypothèse, $D | A$ et $D | B$, donc $D | A \wedge B$. \\
    Comme $U \wedge V = 1$, d'après le théorème de Bézout : 
    \begin{align*}
        UU_1 + VV_1 &= 1 \\
        \text{donc } DUU_1 + DVV_1 &= D \\
        \text{soit } AU_1 + BV_1 &= D \\
        \text{donc } A \wedge B &| D
    \end{align*}
    Ainsi, $A \wedge B$ et $D$ sont associés. Or ils sont unitaires, donc égaux. 
\end{itemize}

\setsection{52}
\section{Caractérisation des racines par la divisibilité}
\begin{tcolorbox}[title=Théorème 16.53, title filled=false, colframe=orange, colback=orange!10!white]
    Soit $\mathbb{K}$ un corps, $P \in \mathbb{K}[X]$ et $r \in \mathbb{K}$. Alors $r$ est racine de $P$ si et seulement si $X - r$ divise $P$. Donc s'il existe $Q \in \mathbb{K}[X]$ tel que $P = (X - r)Q$.
\end{tcolorbox}

$\boxed{\Leftarrow}$ \\
Si $P = (X - r)Q$, alors : 
\begin{align*}
    \tilde{P}(r) &= (X - r)\tilde{Q}(r)  \\
    &= 0 \times \tilde Q (r) \\
    &= 0
\end{align*}

$\boxed{\Rightarrow}$ \\
On suppose $r$ racine de $P$. \\
On effectue la division euclidienne de $P$ par $X - r$ :
\begin{align*}
    P &= (X - r)Q + R, R \in \mathbb{K}_0[X]
\end{align*}
Donc $0 = \tilde P(r) = \tilde R(r)$. \\
Donc $R = 0$. \\
Donc $X - r | P$.

\setsection{55}
\section{Formule de Taylor pour les polynômes}
\begin{tcolorbox}[title=Théorème 16.56, title filled=false, colframe=orange, colback=orange!10!white]
    Soit $\mathbb{K}$ un corps de caractéristique nulle, $P$ un polynôme de $\mathbb{K}[X]$ de degré $d$ et $a \in \mathbb{K}$, alors
    $$ P = \sum_{k=0}^{d} \frac{P^{(k)}(a)}{k!} (X - a)^k. $$
\end{tcolorbox}

\noindent On note $E_k = X^k$, pour $k \in \mathbb{N}$. \\
On a, pour $i \in \mathbb{N}$ : 
\begin{align*}
    E_k^{(i)} = \begin{cases}
        \frac{k!}{(k-i)!}X^{k-i} & \text{ si } i \leq k \\
        0 & \text{ si } i > k
    \end{cases}
\end{align*}
Ainsi : 
\begin{align*}
    E_k(X + a) &= (X + a)^k \\
    &= \sum_{i=0}^{k} \binom{k}{i} a^{k-i} X^i \\
    &= \sum_{i=0}^{k} \frac{k!}{i!(k-i)!} a^{k-i} X^i \\
    &= \sum_{i=0}^{k} \frac{E_k^{(i)}(a)}{i!} X^i
\end{align*}
Soit $P = \sum\limits_{k=0}^{d} a_k X^k = \sum_{k=0}^{d} a_k E_k$. \\
Ainsi :
\begin{align*}
    P(x + a) &= \sum_{k=0}^{d} a_k E_k(X + a) \\
    &= \sum_{k=0}^{d} a_k \sum_{i=0}^{k} \frac{E_k^{(i)}(a)}{i!} X^i \\
    &= \sum_{i=0}^{d} \frac{1}{i!} \left( \sum_{k=i}^{d} a_k E_k^{(i)}(a) \right) X_i \\
    &= \sum_{i=0}^{d} \frac{1}{i!} \left( \sum_{k=0}^{d} a_k E_k^{(i)}(a) \right) X_i \\
    &= \sum_{i=0}^{d} \frac{1}{i!} P^{(i)}(a) X^i
\end{align*}

\section{Caractérisation de la multiplicité par les dérivées}
\begin{tcolorbox}[title=Théorème 16.57, title filled=false, colframe=orange, colback=orange!10!white]
    Soit $\mathbb{K}$ un corps de caractéristique nulle, $P \in \mathbb{K}[X]$ et $a \in \mathbb{K}$. Le réel $a$ est racine d'ordre multiplicité $k$ de $P$ si et seulement si 
    $$P(a) = P'(a) = \ldots = P^{(k-1)}(a) = 0 \text{ et } P^{(k)}(a) \neq 0.$$
\end{tcolorbox}

$\boxed{\Leftarrow}$ \\
D'après la formule de Taylor : 
\begin{align*}
    P &= \sum_{i=0}^{d} \frac{P^{(i)}(a)}{i!} (X - a)^i \\
    &= \sum_{i=k}^{d} \frac{P^{(i)}(a)}{i!} (X - a)^i \\
    &= (X - a)^k \underbrace{\sum_{i=k}^{d} \frac{P^{(i)}(a)}{i!} (X - a)^{i-k}}_{= Q} \\
    Q(a) &= \frac{P^{(k)}(a)}{k!} \neq 0
\end{align*}

$\boxed{\Rightarrow}$ \\
$P = (\underbrace{X - a}_B)^k Q$ avec $Q(a) \neq 0$. \\
Pour tout $i \in \llbracket 0, k-1 \rrbracket$ : 
\begin{align*}
    P^{(i)} &= (BQ)^{(i)} \\
    &= \sum_{l=0}^{i} \binom{i}{l} B^{(l)} Q^{(i-l)} \\
    P^{(i)}(a) &= 0 \\
    P^{(k)} &= \binom{k}{k} B^{(k)}(a) \times Q^{(k-k)}(a) \\
    &= k! \times Q(a) \neq 0
\end{align*}

\setsection{58}
\section{Caractérisation de la multiplicité des racines par la divisibilité}
\begin{tcolorbox}[title=Théorème 16.59, title filled=false, colframe=orange, colback=orange!10!white]
    Soit $\mathbb{K}$ un corps. Soit $P \in \mathbb{K}[X]$ et $r_1, \ldots, r_k$ des racines deux à deux distinctes de $P$, de multiplicités respectives $a_1, \ldots, a_k$. Alors $(X - r_1)^{a_1} \ldots (X - r_k)^{a_k}$ divise $P$ et $r_1, \ldots, r_k$ ne sont pas racines du quotient.
\end{tcolorbox}

\noindent RAF : \\
$$(X - r_i)^{\alpha_1} \wedge (X - r_k)^{\alpha_k} = 1 \text{ si } i \neq k$$

\setsection{62}
\section{Polynômes formels et fonctions polynomiales}
\begin{tcolorbox}[title=Théorème 16.63, title filled=false, colframe=orange, colback=orange!10!white]
    Soit $\mathbb{K}$ un corps infini. Alors l'application de $\mathbb{K}[X]$ dans $\mathbb{K}[x]$ qui à un polynôme formel associe sa fonction polynomiale est un isomorphisme d'anneaux. 
\end{tcolorbox}

\noindent RAF : $\varphi(P) = \varphi(Q)$ donc $\varphi(P - Q) = 0$ \\
$\tilde{P} - \tilde{Q}$ s'annule sur $\mathbb{K}$ infini et on applique (16.62). 

\setsection{65}
\section{Caractérisation des polynômes interpolateurs}
\begin{tcolorbox}[title=Lemme 16.66, title filled=false, colframe=orange, colback=orange!10!white]
    Le polynôme $L_i$ est l'unique polynôme de degré au plus $n$ tel que pour tout $j \in \llbracket 0, n \rrbracket, L_i(x_j) = \delta_{ij}$.
\end{tcolorbox}

\noindent\underline{Existence :} RAF \\
\noindent\underline{Unicité :} (16.61.3) \\

\setsection{68}
\section{Corollaire}
\begin{tcolorbox}[title=Corollaire 16.69, title filled=false, colframe=orange, colback=orange!10!white]
    Soit $P$ le polynôme d'interpolation de Lagrange associé à la famille $(x_i)_{0 \leq i \leq n}$ et aux valeurs $(y_i)_{0 \leq i \leq n}$ Soit $P_0 = (X - x_0) \ldots (X - x_n)$. L'ensemble $E$ des polynômes $Q$ (sans restriction de degré) tel que pour tout $i \in \llbracket 0, n \rrbracket, Q(x_i) = y_i$ est décrit par
    $$E = P + (P_0) = \{ P + (X - x_0) \ldots (X - x_n)R, R \in \mathbb{K}[X] \}$$
\end{tcolorbox}

$\boxed{\supset}$ \\
Si $Q = P + (X - x_0) \ldots (X - x_n)R$, alors :
\begin{align*}
    \forall i \in \llbracket 0, n \rrbracket, Q(x_i) &= P(x_i) = y_i
\end{align*}
Donc $Q\in E$. \\ \\

$\boxed{\subset}$ \\
Soit $Q \in E$, alors $x_0, \ldots, x_n$ sont racines de $Q - P$. \\
Donc $(X - x_0) \ldots (X - x_n) | Q - P$. \\

\setsection{73}
\section{Proposition}
\begin{tcolorbox}[title=Propostion 16.74 (HP), title filled=false, colframe=lightblue, colback=lightblue!10!white]
    Soit $P$ un polynôme scindé non constant de $\mathbb{R}[X]$ à racines simples. Alors $P'$ est scindé, et ses racines séparent celles de $P$. 
\end{tcolorbox}

\noindent Soit $P = \prod\limits_{k=1}^{n} (x - x_k)$ avec $x_1 < \ldots < x_n$. \\
D'après le théroème de Rolle, comme $P(x_1) = P(x_2) = \ldots = P(x_n)$ pour tout $k \in \llbracket 1, n-1 \rrbracket$, on choisit $y_k \in ]x_k, x_{k+1}[$ tel que $P'(y_k) = 0$. \\
On a donc : 
\begin{align*}
    x_1 < y_1 < x_2 < y_2 < \ldots < y_{n-1} < x_n
\end{align*}
et $y_1, \ldots, y_{n-1}$ sont $n-1$ racines distinctes de $P'$ de degré $n-1$ ($\mathbb{R}$ de caractéristique nulle). \\
Donc $P'$ est scindé (à racines simples). 

\setsection{75}
\section{Relation de Viète}
\begin{tcolorbox}[title=Théorème 16.76, title filled=false, colframe=orange, colback=orange!10!white]
    Soit $P = \sum\limits_{k=0}^{n} a_k X^k$ un polynôme de degré $n$, scindé, de racines (éventuellement non distinctes, apparaissant dans la liste autant de fois que sa multiplicité) $r_1, \ldots, r_n$ alors pour tout $k \in \llbracket 0, n \rrbracket$ :
    $$\sum_{1 \leq i_1 < \ldots < i_k \leq n} r_{i_1} \ldots r_{i_k} = (-1)^k \frac{a_{n-k}}{a_n}$$
\end{tcolorbox}

\begin{align*}
    P &= \sum_{k=0}^{n} a_k X^k \\
    &= a_n \prod_{k=1}^{n} (X - r_k) \\
\end{align*}
Les relations de Viète consistent simplement à développer l'expression de droite et à identifier les mnômes de degré $n-k$.
\begin{align*}
    a_{n-k} = (-1)^k a_n \sum_{1 \leq i_1 < \ldots < i_k \leq n} r_{i_1} \ldots r_{i_k}
\end{align*}

\setsection{87}
\section{Lemme}
\begin{tcolorbox}[title=Lemme 16.88, title filled=false, colframe=orange, colback=orange!10!white]
    Soit $P$ un polynôme irréductible de $\mathbb{K}[X]$ et $A$ un polynôme non multiple de $P$. Alors $A$ et $P$ ont premiers entre eux. 
\end{tcolorbox}

\noindent Soit $D \text{ unitaire } \in \mathcal{D}_{A, P}$.\\
Si $P \not | A$, alors $D \neq U(P)$. \\
Donc $D = 1$. \\
Donc $P \wedge A = 1$.

\setsection{97}
\section{Caractérisation de la divisibilité dans $\mathbb{C}[X]$ par les racines}
\begin{tcolorbox}[title=Théorème 16.98, title filled=false, colframe=orange, colback=orange!10!white]
    Soit $P$ et $Q$ deux polynômes de $\mathbb{C}[X]$. Alors $P$ divise $Q$ si et seulement si toute racine de $P$ est aussi une racine de $Q$, et que sa multiplicité dans $Q$ est supérieure ou  égale à sa multiplicité dans $P$.
\end{tcolorbox}

$\boxed{\Rightarrow}$ \\
Supposons $P|Q$. \\
Soit $r$ une racine de $P$ de multiplicité $\alpha$. Donc : 
\begin{align*}
    (X - r)^{\alpha} &| P \\
    \text{donc } (X - r)^{\alpha} &| Q \\
\end{align*}
Donc $r$ est racine de $Q$ de multiplicité supérieure à $\alpha$. \\ \\

$\boxed{\Leftarrow}$ \\
On décompose  $P = \lambda \prod\limits_{i=1}^{n} (X - r_i)^{\alpha_i}$ ($P$ est scindé sur $\mathbb{C}$). \\
Par hypothèse, $\prod\limits_{i=1}^{n} (X - r_i)^{\alpha_i} | Q$. \\
Donc $P | Q$. 


\end{document}