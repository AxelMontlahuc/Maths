\documentclass[../main.tex]{subfiles}

\begin{document}
\setcounter{chapter}{29}
\chapter{Déterminant}
\tableofcontents
\clearpage

\setsection{3}
\section{Exemple}
\begin{tcolorbox}[title=Exemple 30.4, title filled=false, colframe=darkgreen, colback=darkgreen!10!white]
    On considrèe l'application : 
    \begin{align*}
        \delta:\mathbb{K}^2 \times \mathbb{K}^2\to \mathbb{K}; ((a, b), (c, d)) \mapsto ad - bc
    \end{align*}
    Montrer que cette application est bien $2$-linéaire. 
\end{tcolorbox}

\begin{align*}
    \delta \left( \begin{pmatrix}
        a \\ b
    \end{pmatrix}, \begin{pmatrix}
        c \\ d
    \end{pmatrix} + \lambda \begin{pmatrix}
        c' \\ d'
    \end{pmatrix} \right) &= \delta \left( \begin{pmatrix}
        a \\ b
    \end{pmatrix}, \begin{pmatrix}
        c + \lambda c' \\ d + \lambda d'
    \end{pmatrix} \right) \\
    &= a(d + \lambda d') - b(c + \lambda c') \\
    &= ad - bc + \lambda(ad' - bc') \\
    &= \delta \left( \begin{pmatrix}
        a \\ b
    \end{pmatrix}, \begin{pmatrix}
        c \\ d
    \end{pmatrix} \right) + \lambda \delta \left( \begin{pmatrix}
        a \\ b
    \end{pmatrix}, \begin{pmatrix}
        c' \\ d'
    \end{pmatrix} \right)
\end{align*}

\setsection{10}
\section{Détermination d'une application n-linéaire sur une base}
\begin{tcolorbox}[title=Propostion 30.11, title filled=false, colframe=lightblue, colback=lightblue!10!white]
    Soit pour tout $i\in \llbracket 1, n \rrbracket$, $(e_{i,j})_{1\leq j\leq d}$ une base de $E_i$ et pour tout $(j_1, \ldots, j_n)\in \llbracket 1, d_1 \rrbracket \times \cdots \times \llbracket 1, d_n \rrbracket$, $f_{j_1, \ldots, j_n}\in F$. \\
    Alors il existe une unique application $n$-linéaire $f:E_1\times \cdots \times E_n\to F$ telle que : 
    \begin{align*}
        \forall (j_1, \ldots, j_n)\in \llbracket 1, d_1 \rrbracket \times \cdots \times \llbracket 1, d_n \rrbracket, \varphi(e_{1,j_1}, \ldots, e_{n,j_n}) = f_{j_1, \ldots, j_n}
    \end{align*}
\end{tcolorbox}

\noindent Si $(e_{i,j})_{1\leq j\leq d}$ est une base de $E_i$ alors $((e_{1, 2}, 0, \ldots, 0, \ldots, e_{1, d}, 0, \ldots, 0), \ldots, (0, \ldots, 0, e_{n, 1}, \ldots, (0, \ldots, 0, e_{n, d})))$ est une base de $E_1\times \cdots \times E_n$. (22.16), théorème de rigidité. 


\end{document}