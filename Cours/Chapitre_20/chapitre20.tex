\documentclass[../main.tex]{subfiles}

\begin{document}
\setcounter{chapter}{19}
\chapter{Espace Vectoriels}
\tableofcontents
\clearpage

\setsection{1}
\section{Propriétés du $0$, régularité}
\begin{tcolorbox}[title=Propostion 20.2, title filled=false, colframe=lightblue, colback=lightblue!10!white]
    Soit $E$ un $\mathbb{K}-ev$. Pour tout $x \in E$ :
    \begin{enumerate}
        \item $0_{\mathbb{K}}.x = 0_E$
        \item pour tout $\lambda \in \mathbb{K}$, $\lambda.0_E = 0_E$
        \item (-1).x = -x
        \item si $x\neq 0_E$, $$\lambda.x = 0_E \Rightarrow \lambda = 0_{\mathbb{K}}$$
        \item si $x\neq 0_{\mathbb{K}}$, $$\lambda.x = 0_E \Rightarrow x = 0_E$$
    \end{enumerate}
\end{tcolorbox}

\begin{enumerate}
    \item $0_{\mathbb{K}}.x = (0_{\mathbb{K}} + 0_{\mathbb{K}}).x = 0_{\mathbb{K}}.x + 0_{\mathbb{K}}.x$. Donc $0_E = 0_{\mathbb{K}}.x$.
    \item RAS. 
    \item $x + (-1).x = (1 - 1).x = 0_{\mathbb{K}}.x = 0_E$.
    \item Par l'absurde, si $\lambda \neq 0_{\mathbb{K}}$, de $\lambda x = 0_E$ on tire $\lambda^{-1} \lambda x = \lambda^{-1} x 0_E$, soit $x = 0_E$. Absurde. 
    \item Idem. 
\end{enumerate}

\setsection{9}
\section{Espace vectoriel de référence}
\begin{tcolorbox}[title=Propostion 20.10, title filled=false, colframe=lightblue, colback=lightblue!10!white]
    \begin{enumerate}
        \item $\mathbb{K}$ est un espace vectoriel sur lui-même. 
        \item Plus généralement, soit $E$ un espace vectoriel sur $\mathbb{K}$ et $F$ un ensemble quelconque. Alors l'ensemble des fonctions $E^F$ est un espace vectoriel sur $\mathbb{K}$.
    \end{enumerate}
\end{tcolorbox}

\begin{enumerate}
    \item RAF. 
    \item Soit $E$ un $\mathbb{K}-ev$ et $F$ un ensemble quelconque. \\
    $E^F$ est un groupe abélien (cf. chap 10). \\
    Le produit externe est défini par : 
    \begin{align*}
        \mathbb{K} \times E^F &\longrightarrow E^F \\
        (\lambda, f) &\longmapsto (\lambda.f, x \mapsto \lambda.f(x))
    \end{align*}
    Vérification facile. 
\end{enumerate}

\section{Transfert de structure}
\begin{tcolorbox}[title=Lemme 20.11, title filled=false, colframe=orange, colback=orange!10!white]
    Soit $E$ un espace vectoriel sur $\mathbb{K}$, $G$ un ensemble quelconque et $\varphi:E\to G$ une bijection. Alors en définissant sur $G$ une loi interne et un loi externe par
    $$\forall (x, y, \lambda) \in G \times G \times \mathbb{K}, x + y = \varphi(\varphi^{-1}(x) + \varphi^{-1}(y)) \text{et} \lambda.x = \varphi(\lambda \varphi^{-1}(x)),$$
    on munit $G$ d'une structure d'espace vectoriel. 
\end{tcolorbox}

\noindent Vérifions les axiomes. 
\begin{itemize}
    \item LCI : 
    \begin{align*}
        (x + y) +  &= \varphi(\varphi^{-1}(x+y) + \varphi(z)) \\
        &= \varphi(\underbrace{\varphi^{-1}(x) + \varphi^{-1}(y) + \varphi^{-1}(z)}_{\text{associativité dans $E$}}) \\
        &= x + (y + z) \\
        x + \varphi(0) &= \varphi(\varphi^{-1}(x) + 0) = x \text{ (} \varphi \text{ neutre)} \\
        x + \varphi(-\varphi^{-1}(x)) &= \varphi(\varphi^{-1}(x) - \varphi^{-1}(x)) = \varphi(0) \\
        x + y = y + x
    \end{align*}

    \item 
    \begin{align*}
        \lambda.(\mu.x) &= \varphi(\lambda \varphi^{-1}(\mu x)) \\
        &= \varphi(\lambda \mu \varphi^{-1}(x)) \\
        &= (\lambda \mu).x \\
        1.x &= \varphi(1.\varphi^{-1}(x)) \\
        &= \varphi\circ \varphi^{-1}(x) \\
        &= x \\
        (\mu + \lambda).x &= \varphi((\mu + \lambda).\varphi^{-1}(x)) \\
        &= \varphi(\mu \varphi^{-1}(x) + \lambda \varphi^{-1}(x)) \\
        &= \varphi(\mu \varphi^{-1}(x)) + \varphi(\lambda \varphi^{-1}(x)) \\
        &= \mu.x + \lambda.x
    \end{align*}
    De même pour la dernière. 
\end{itemize}


\end{document}