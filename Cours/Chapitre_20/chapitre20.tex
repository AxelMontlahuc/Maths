\documentclass[../main.tex]{subfiles}

\begin{document}
\setcounter{chapter}{19}
\chapter{Espace Vectoriels}
\tableofcontents
\clearpage

\setsection{1}
\section{Propriétés du $0$, régularité}
\begin{tcolorbox}[title=Propostion 20.2, title filled=false, colframe=lightblue, colback=lightblue!10!white]
    Soit $E$ un $\mathbb{K}-ev$. Pour tout $x \in E$ :
    \begin{enumerate}
        \item $0_{\mathbb{K}}.x = 0_E$
        \item pour tout $\lambda \in \mathbb{K}$, $\lambda.0_E = 0_E$
        \item (-1).x = -x
        \item si $x\neq 0_E$, $$\lambda.x = 0_E \Rightarrow \lambda = 0_{\mathbb{K}}$$
        \item si $x\neq 0_{\mathbb{K}}$, $$\lambda.x = 0_E \Rightarrow x = 0_E$$
    \end{enumerate}
\end{tcolorbox}

\begin{enumerate}
    \item $0_{\mathbb{K}}.x = (0_{\mathbb{K}} + 0_{\mathbb{K}}).x = 0_{\mathbb{K}}.x + 0_{\mathbb{K}}.x$. Donc $0_E = 0_{\mathbb{K}}.x$.
    \item RAS. 
    \item $x + (-1).x = (1 - 1).x = 0_{\mathbb{K}}.x = 0_E$.
    \item Par l'absurde, si $\lambda \neq 0_{\mathbb{K}}$, de $\lambda x = 0_E$ on tire $\lambda^{-1} \lambda x = \lambda^{-1} x 0_E$, soit $x = 0_E$. Absurde. 
    \item Idem. 
\end{enumerate}

\setsection{9}
\section{Espace vectoriel de référence}
\begin{tcolorbox}[title=Propostion 20.10, title filled=false, colframe=lightblue, colback=lightblue!10!white]
    \begin{enumerate}
        \item $\mathbb{K}$ est un espace vectoriel sur lui-même. 
        \item Plus généralement, soit $E$ un espace vectoriel sur $\mathbb{K}$ et $F$ un ensemble quelconque. Alors l'ensemble des fonctions $E^F$ est un espace vectoriel sur $\mathbb{K}$.
    \end{enumerate}
\end{tcolorbox}

\begin{enumerate}
    \item RAF. 
    \item Soit $E$ un $\mathbb{K}-ev$ et $F$ un ensemble quelconque. \\
    $E^F$ est un groupe abélien (cf. chap 10). \\
    Le produit externe est défini par : 
    \begin{align*}
        \mathbb{K} \times E^F &\longrightarrow E^F \\
        (\lambda, f) &\longmapsto (\lambda.f, x \mapsto \lambda.f(x))
    \end{align*}
    Vérification facile. 
\end{enumerate}

\section{Transfert de structure}
\begin{tcolorbox}[title=Lemme 20.11, title filled=false, colframe=orange, colback=orange!10!white]
    Soit $E$ un espace vectoriel sur $\mathbb{K}$, $G$ un ensemble quelconque et $\varphi:E\to G$ une bijection. Alors en définissant sur $G$ une loi interne et un loi externe par
    $$\forall (x, y, \lambda) \in G \times G \times \mathbb{K}, x + y = \varphi(\varphi^{-1}(x) + \varphi^{-1}(y)) \text{et} \lambda.x = \varphi(\lambda \varphi^{-1}(x)),$$
    on munit $G$ d'une structure d'espace vectoriel. 
\end{tcolorbox}

\noindent Vérifions les axiomes. 
\begin{itemize}
    \item LCI : 
    \begin{align*}
        (x + y) +  &= \varphi(\varphi^{-1}(x+y) + \varphi(z)) \\
        &= \varphi(\underbrace{\varphi^{-1}(x) + \varphi^{-1}(y) + \varphi^{-1}(z)}_{\text{associativité dans $E$}}) \\
        &= x + (y + z) \\
        x + \varphi(0) &= \varphi(\varphi^{-1}(x) + 0) = x \text{ (} \varphi \text{ neutre)} \\
        x + \varphi(-\varphi^{-1}(x)) &= \varphi(\varphi^{-1}(x) - \varphi^{-1}(x)) = \varphi(0) \\
        x + y = y + x
    \end{align*}

    \item 
    \begin{align*}
        \lambda.(\mu.x) &= \varphi(\lambda \varphi^{-1}(\mu x)) \\
        &= \varphi(\lambda \mu \varphi^{-1}(x)) \\
        &= (\lambda \mu).x \\
        1.x &= \varphi(1.\varphi^{-1}(x)) \\
        &= \varphi\circ \varphi^{-1}(x) \\
        &= x \\
        (\mu + \lambda).x &= \varphi((\mu + \lambda).\varphi^{-1}(x)) \\
        &= \varphi(\mu \varphi^{-1}(x) + \lambda \varphi^{-1}(x)) \\
        &= \varphi(\mu \varphi^{-1}(x)) + \varphi(\lambda \varphi^{-1}(x)) \\
        &= \mu.x + \lambda.x
    \end{align*}
    De même pour la dernière. 
\end{itemize}

\setsection{15}
\section{Caractérisation des sous-espaces vectoriels}
\begin{tcolorbox}[title=Théorème 20.16, title filled=false, colframe=orange, colback=orange!10!white]
    Soit $E$ un $\mathbb{K}$-espace vectoriel. Un ensemble $F$ est un sous-espace vectoriel de $E$ si et seulement si
    \begin{enumerate}
        \item $F \subset E$;
        \item $0 \in F$;
        \item $F$ est stable par combinaisons linéaire, ce qui équivaut à 
        $$\forall (x, y) \in F^2, \forall \lambda\in \mathbb{K}, \lambda x + y \in F.$$
    \end{enumerate}
\end{tcolorbox}

$\boxed{\Rightarrow}$
\begin{enumerate}
    \item Oui. 
    \item $F$ est un sous-groupe de $E$ donc $0_E \in F$. 
    \item Pour tout $(x, y) \in F^2$, $\lambda \in \mathbb{K}$, $\lambda .x \in F$ et $y \in F$. Donc $\lambda x + y \in F$.
\end{enumerate}

$\boxed{\Leftarrow}$ \\
D'après (3) avec : 
\begin{itemize}
    \item $y = 0$ : $\times$ est LCE. 
    \item $\lambda = 1$ : $+$ est LCI. 
\end{itemize}
$0\in F$ et $\lambda = -1$, $F$ est un sous-groupe, donc un groupe abélien.
RAF pour les 4 dernières propriétés.

\setsection{21}
\section{Propostion 20.22}
\begin{tcolorbox}[title=Propostion 20.22, title filled=false, colframe=lightblue, colback=lightblue!10!white]
    Soit $E$ un $\mathbb{K}$-espace vectoriel, $D_1$ et $D_2$ deux droites vectorielles. Alors soit $D_1 \cap D_2 = \{ 0_E \}$, soit $D_1 = D_2$. 
\end{tcolorbox}

\noindent Par définition, $0_E \in D_1 \cap D_2$. \\
Supposons $D_1 \cap D_2 \neq \{ 0_E \}$ et fixons $x \in D_1 \cap D_2$ avec $x \neq 0_E$. \\
Soit $v\in D_1$. Par définition, on écrit $D_1 = \mathbb{K} x_1$ et $D_2 = \mathbb{K} x_2$. \\
On a donc $v = \alpha x_1$, $x = \lambda_1 x_1 = \lambda_2 x_2$ avec $\lambda_1 \neq 0, \lambda_2 \neq 0$. \\
Ainsi : 
\begin{align*}
    v = \alpha \lambda_1^{-1} \lambda_1 x_1 = \alpha \lambda_1^{-1} x = \alpha \lambda_1^{-1} \lambda_2 x_2 \in D_2
\end{align*}
Donc $D_1 \subset D_2$ et par symétrie, $\boxed{D_1 = D_2}$.

\setsection{26}
\section{Intersection de sous-espaces vectoriels}
\begin{tcolorbox}[title=Propostion 20.27, title filled=false, colframe=lightblue, colback=lightblue!10!white]
    Soit $E$ une espace vectoriel et $(E_i)_{i\in I}$ une famille de sous-espaces vectoriels de $E$. Alors $\bigcap\limits_{i\in I} E_i$ est un sous-espace vectoriel de $E$.
\end{tcolorbox}

\begin{itemize}
    \item $\bigcap\limits_{i\in I} E_i \subset E$.
    \item $\forall i \in I, 0 \in E_i$ donc $0 \in \bigcap\limits_{i\in I} E_i$.
    \item Soit $(x, y) \in \left[\bigcap\limits_{i \in I} E_i \right]^2, \lambda \in \mathbb{K}$ : 
    \begin{align*}
        \forall x \in I, \lambda x + y \in E_i
    \end{align*}
    Donc $\lambda x + y \in \bigcap\limits_{i\in I} E_i$.
\end{itemize}

\setsection{33}
\section{Description de $Vect(X)$}
\begin{tcolorbox}[title=Propostion 20.34, title filled=false, colframe=lightblue, colback=lightblue!10!white]
    Soit $E$ un $\mathbb{K}$-ev et $X$ un sous-ensemble de $E$. Alors $Vect(X)$ est l'ensemble des combinaisons linéaires d'éléments de $X$. 
\end{tcolorbox}

\noindent On note $F$ l'ensemble des combinaisons linéaires de vecteurs de $X$. \\
$F$ est un sous-espace vectoriel de $E$ qui contient $X$. \\
Par définition, $Vect(X) \subset F$. \\
Or $Vect(X)$ est un sous-espace vectoriel qui contient $X$. Il doit donc contenir les combinaisons linéaiers de $X$ soit $F$. \\
Donc $\boxed{F = Vect(X)}$.

\setsection{35}
\section{Opérations sur les sous-espaces vectoriels engendrés}
\begin{tcolorbox}[title=Propostion 20.36, title filled=false, colframe=lightblue, colback=lightblue!10!white]
    Soit $A$ et $B$ deux ensembles. On a
    \begin{enumerate}
        \item $A \subset Vect(A)$
        \item Si $A \subset B$ alors $Vect(A) \subset Vect(B)$. 
        \item $A = Vect(A)$ si et seulement si $A$ est un espace vectoriel.
        \item $Vect(Vect(A)) = Vect(A)$.
        \item $Vect(A \cup \{x\}) = Vect(A)$ si et seulement si $x \in Vect(A)$.
    \end{enumerate}
\end{tcolorbox}

\begin{enumerate}
    \item RAF
    \item RAF (20.24)
    \item Si $A = \underbrace{Vect(A)}_{\text{sous-espace vectoriel}}$, alors $A$ est un sous-espace vectoriel. \\
    Si $A$ est un espace vectoriel, par minimalité, $A = Vect(A)$.
    \item RAF (20.36.3)
    \item On a toujours $Vect(A \cup \{x\}) \supset Vect(A)$(2 0.36.2) si $Vect(A \cup \{x\}) \subset Vect(A)$. \\
    Or $x \in Vect(A \cup \{x\})$. \\
    Donc $x \in Vect(A)$. \\
    Réciproquement, si $x \in Vect(A)$, d'après (20.34) : 
    \begin{align*}
        Vect(A \cup \{x\}) &\subset Vect(A) \\
    \end{align*}
    Si $u \in Vect(A \cup \{x\})$, alors : 
    \begin{align*}
        u &= \lambda_1 a_1 + \ldots + \lambda_n a_n + \lambda_{n+1} x \\
        &= \lambda_1 a_1 + \ldots + \lambda_n a_n + \lambda_{n+1} (\mu_1 a'_1 + \ldots + \mu_p a'_p) \\
        &\in Vect(A)
    \end{align*}
\end{enumerate}

\setsection{40}
\section{Somme de sous-espaces vectoriels engendrés}
\begin{tcolorbox}[title=Propostion 20.41, title filled=false, colframe=lightblue, colback=lightblue!10!white]
    Soit $X$ et $Y$ deux sous-ensembles de $E$. Alors
    $$Vect(X \cup Y) = Vect(X) + Vect(Y)$$
\end{tcolorbox}

\noindent On a : 
\begin{align*}
    Vect(X) &\subset Vect(X \cup Y) \\
    Vect(Y) &\subset Vect(X \cup Y) \\
    \text{donc } Vect(X) + Vect(Y) &\subset Vect(X \cup Y)
\end{align*}
Par minimalité : 
\begin{align*}
    \boxed{Vect(X \cup Y) = Vect(X) + Vect(Y)}
\end{align*}

\setsection{42}
\section{Description d'une somme d'un nombre fini de sous-espaces vectoriels}
\begin{tcolorbox}[title=Propostion 20.43, title filled=false, colframe=lightblue, colback=lightblue!10!white]
    Soit $E_1, \ldots, E_n$ et $F$ des sous-espaces vectoriels de $E$. Sont équivalentes : 
    \begin{enumerate}
        \item $F = E_1 + \ldots + E_n$;
        \item $F = (\ldots((E_1 + E_2) + E_3) + \ldots + E_{n-1}) + E_n$;
        \item $F = \{ x_1 + x_2 + \ldots + x_n | (x_1, \ldots, x_n) \in E_1 \times \ldots \times E_n \}$. 
    \end{enumerate}
\end{tcolorbox}

\begin{enumerate}
    \setcounter{enumi}{1}
    \item Associativité fournie par la définition. 
    \item (20.39) + (20.43.2)
\end{enumerate}

\section*{Exemple}
\noindent Dans $\mathbb{R}^3$, $E = Vect((1, 0, 0))$ et $F = Vect((0, 1, 0), (0, 0, 1))$. \\
Soit $u \in E \cap F$. \\
$u = \alpha(1, 0, 0) = \beta(0, 1, 0) + \gamma(0, 0, 1)$. \\
Donc $(-\alpha, \beta, \gamma) = (0, 0, 0)$. \\
Donc $\alpha = \beta = \gamma = 0$. \\ \\

\noindent Dans $\mathbb{R}^4$ avec $e_1 = (1, 0, 0, 0)$, $e_2 = (0, 1, 0, 0)$, $e_3 = (0, 0, 1, 0)$ et $e_4 = (0, 0, 0, 1)$. \\
$E = Vect(e_1 + e_2 + e_3, e_1 + e_2 + e_3 + e_4)$ \\
$F = Vect(e_1 + e_3, 2e_2 + e_1 - e_4)$ \\
Soit $u \in E \cap F$. \\
\begin{align*}
    u &= \alpha(e_1 + e_2 + e_3) + \beta(e_1 + e_2 + e_3 + e_4) = (\alpha + \beta, \alpha + \beta, \beta) \\
    &= \gamma(e_1 + e_3) + \delta(2e_2 + e_1 - e_4) = (\gamma + \delta, 2\delta, \gamma, -\delta)
\end{align*}
Donc : 
\begin{align*}
    &\begin{cases}
        \alpha + \beta - \gamma - \delta = 0 \\
        \alpha + \beta - 2\delta = 0 \\
        \alpha + \beta - \gamma = 0 \\
        \beta + \delta = 0
    \end{cases} \\
    \text{donc } 
    &\begin{cases}
        \delta = 0 \text{ ($L_1 - L_3$)} \\
        \beta = 0 \text{ ($L_4$)} \\
        \alpha = 0 \text{ ($L_2$)} \\
        \gamma = 0 \text{ ($L_2$)}
    \end{cases}
\end{align*}
Donc : 
\begin{align*}
    \boxed{E \cap F = \{0\}}
\end{align*}

\setsection{46}
\section{Unicité de l'écriture de la somme directe}
\begin{tcolorbox}[title=Remarque 20.47, title filled=false, colframe=lightblue, colback=lightblue!10!white]
    En d'autres termes, la somme est directe si et seulement si tout élément $x$ de $E_1 \oplus \ldots \oplus E_n$ s'écrit de façon unique sous la forme $x = x_1 + \ldots + x_n$. 
\end{tcolorbox}

$\boxed{\Rightarrow}$ \\
On suppose que la somme est directe. \\
Soit $x \in E_1 \oplus \ldots \oplus E_n$. \\
On écrit : 
\begin{align*}
    x &= x_1 + \ldots + x_n \\
    &= x'_1 + \ldots + x'_n \\
    \text{donc } \underbrace{x'_n - x_n}_{\in E_n} &= (\underbrace{x_1 - x'_1}_{\in E_1}) + \ldots + (\underbrace{x_{n-1} - x'_{n-1}}_{\in E_{n-1}}) \in E_n \cap (E_1 + \ldots + E_{n-1}) = \{0\} \\
    \text{donc } x'_n &= x_n
\end{align*}
On poursuit par récurrence. \\ \\

$\boxed{\Leftarrow}$ \\
On remarque que $0 = 0 + \ldots 0$. \\
Soit $u \in E_n \cap (E_1 + \ldots + E_{n-1})$. \\
Donc : 
\begin{align*}
    u &= e_n = e_1 + \ldots + e_{n-1} \\
    \text{donc } e_1 + \ldots + e_{n-1} &= 0
\end{align*}
Par unicité : 
\begin{align*}
    \forall i \in \llbracket 1, n-1 \rrbracket, e_i &= 0 \\
    \text{donc } u &= 0
\end{align*}
On termine le travail par récurrence. 

\setsection{50}
\section{Famille libre}
\begin{tcolorbox}[title=Propostion 20.51, title filled=false, colframe=lightblue, colback=lightblue!10!white]
    Une famille $(x_i)_{i\in I}$ de vecteurs de $E$ est \textbf{libre} si une des propriétés équivalentes suivantes est vérifiée :
    \begin{enumerate}
        \item Pour toute famille $(\lambda_i)_{i\in I}$ de scalaires de $\mathbb{K}$, à support fini, $\sum\limits_{i\in I} \lambda_i x_i = 0 \Rightarrow \forall i \in I, \lambda_i = 0$.
        \item Pour tout $x \in Vect((x_i)_{i\in I})$ il existe une \textbf{unique} famille $(\lambda_i)_{i\in I}$ de scalaires de $\mathbb{K}$, à support fini, telle que $x = \sum\limits_{i\in I} \lambda_i x_i$. \\
        Si de plus, $I = \llbracket 1, n \rrbracket$, les points précédents sont équivalents aux points suivants :
        \item Les $x_i$ sont non nuls et la somme $\mathbb{K} x_1 \oplus \ldots \oplus \mathbb{K} x_n$ est directe. 
        \item La fonction $\varphi:\mathbb{K}^n\to E; (\lambda_1, \ldots, \lambda_n) \mapsto \lambda_1 x_1 + \ldots + \lambda_n x_n$ est injective.
    \end{enumerate}
\end{tcolorbox}

$\boxed{1 \Rightarrow 2}$ \\
On écrit, pour tout $x \in Vect((x_i)_{i\in I})$ :
\begin{align*}
    x = \sum_{i\in I} \lambda_i x_i &= \sum_{i\in I} \mu_i x_i \\
    \text{donc } \sum_{i\in I} (\lambda_i - \mu_i) x_i &= 0
\end{align*}
Comme $(\lambda_i)$ et ($\mu_i$) sont des familles de sclaires à support fini, $(\lambda_i - \mu_i)$ aussi et d'après (20.51.1) : 
\begin{align*}
    \forall i \in I, \lambda_i - \mu_i = 0
\end{align*}

$\boxed{2 \Rightarrow 1}$ \\
Soit $\sum\limits_{i\in I} = 0$ avec $(\lambda_i)$ une famille de scalaires à support fini. \\
Comme : 
\begin{align*}
    0 = \sum_{i\in I} 0 x_i
\end{align*}
Par unicité :
\begin{align*}
    \forall i \in I, \lambda_i = 0
\end{align*}

$\boxed{1, 2 \Rightarrow 3}$ \\
Nécessairement, les $x_i$ sont tous non nuls (sinon, on écrit $1 \times x_1 = 0$). \\
Soit $x \in (\mathbb{K} + \ldots + \mathbb{K} x_{n-1}) \cap \mathbb{K} x_n$. \\
On écrit :
\begin{align*}
    x = \alpha_1 x_1 + \ldots + \alpha_{n-1} x_{n-1} &= \alpha_n x_n \\
    \text{donc } \alpha_1 x_1 + \ldots + \alpha_{n-1} x_{n-1} - \alpha_n x_n &= 0
\end{align*}
Par hypothèse : 
\begin{align*}
    \forall i \in \llbracket 1, n \rrbracket, \alpha_i = 0
\end{align*}
On poursuit le travail par récurrence pour montrer que la somme est directe. \\ \\

$\boxed{3 \Rightarrow 4}$ \\ \\
RAF : (20.47)

$\boxed{4 \Rightarrow 1, 2}$ \\
RAF : définition de l'injectivité pour $2$. 

\section{Exemple}
\begin{tcolorbox}[title=Exemple 20.54, title filled=false, colframe=darkgreen, colback=darkgreen!10!white]
    \begin{enumerate}
        \item Montrer que la famille $((1, 1), (0, 1))$ est libre. 
        \item Montrer que la famille $((1, 2, 1), (1, 0, 1), (0, 1, -1))$ est libre. 
        \item Montrer que la famille $((1, 2, 1), (1, 0, 1), (1, 6, 1))$ est liée. 
    \end{enumerate}
\end{tcolorbox}

\begin{enumerate}
    \item On suppose $\alpha(1, 1) + \beta(0, 1) = 0$. \\
    Donc : 
    \begin{align*}
        &\begin{cases}
            \alpha = 0 \\
            \alpha + \beta = 0
        \end{cases} \\
        \text{donc } &\alpha = \beta = 0
    \end{align*}
    La famille est libre. 

    \item Par équivalence. Soit $(a, b, c) \in \mathbb{R}^3$. On a :\\
    \begin{align*}
        &a(1, 2, 1) + b(1, 0, 1) + c(0, 1, -1) = (0, 0, 0) \\
        &\Leftrightarrow \begin{cases}
            a + b &= 0 \\
            2a + c &= 0 \\
            a + b - c &= 0
        \end{cases} \\
        &\Leftrightarrow \begin{cases}
            b + a &= 0 \\
            2a + c &= 0 \\
            c &= 0
        \end{cases} \\
        &\Leftrightarrow a = b = c = 0
    \end{align*}
    La famille est libre. 

    \item Avec les mêmes notations : 
    \begin{align*}
        &a(1, 2, 1) + b(1, 0, 1) + c(1, 6, 1) = (0, 0, 0) \\
        &\Leftrightarrow \begin{cases}
            a + b + c &= 0 \\
            2a + 6c &= 0 \\
            a + b + c &= 0
        \end{cases} \\
        &\Leftrightarrow \begin{cases}
            a + b + c &= 0 \\
            a + 3c &= 0 \\
        \end{cases}
    \end{align*}
    Le système admet des solutions non nulles (par exemple $(-3, 2, 1)$), donc la famille est liée. 
\end{enumerate}

\setsection{57}
\section{Caractérisation de la liberté pour des familles infinies}
\begin{tcolorbox}[title=Propostion 20.58, title filled=false, colframe=lightblue, colback=lightblue!10!white]
    Une famille $(x_i)_{i\in I}$ est libre si et seulement si toutes ses sous-familles finies sont libres.
\end{tcolorbox}

$\boxed{\Rightarrow}$ \\
RAF : (20.57) \\ \\

$\boxed{\Leftarrow}$ \\
Soit $(\lambda_i)_{i\in I}$ une famille à support fini telle que : \\
\begin{align}
    \sum_{i\in I} \lambda_i x_i = 0
\end{align}
On choisit $J \subset I$, fini, tel que : 
\begin{align*}
    \forall i \in I \backslash J, \lambda_i &= 0 \\
    \text{donc } \sum_{i\in J} \lambda_i x_i &= 0
\end{align*}
Or $(x_i)_{i\in J}$ est libre (finie), donc : 
\begin{align*}
    \forall i \in J, \lambda_i &= 0 \\
    \text{donc } \forall i \in I, \lambda_i &= 0
\end{align*}

\setsection{59}
\section{Caractérisation de la liberté pour les familles infinies indexées par $\mathbb{N}$}
\begin{tcolorbox}[title=Propostion 20.60, title filled=false, colframe=lightblue, colback=lightblue!10!white]
    Une famille $(x_i)_{i\in \mathbb{N}}$ est libre si et seulement si pour tout $n\in \mathbb{N}$, la famille $(x_0, \ldots, x_n)$ est libre. 
\end{tcolorbox}

$\boxed{\Rightarrow}$ \\
Si $(x_i)_{i\in \mathbb{N}}$ est libre, alors (20.58) ses sous-familles finies sont libres, en particulier celles sous la forme $(x_0, \ldots, x_n)$. \\ \\

$\boxed{\Leftarrow}$ \\
Soit $(x_i)_{i\in J}$ avec $J$ un sous-ensemble fini de $\mathbb{N}$. \\
Or pose $n = \max J$, donc $J \subset \llbracket 0, n \rrbracket$. \\
Par hypothèse, $(x_0, \ldots, x_n)$ est libre. \\
Donc (20.57), $(x_i)_{i\in J}$ est libre. \\
D'après (20.58), $(x_i)_{i\in \mathbb{N}}$ est libre. 

\section{Ajout d'un élément à une famille libre}
\begin{tcolorbox}[title=Propostion 20.61, title filled=false, colframe=lightblue, colback=lightblue!10!white]
    Soit $(x_i)_{i\in I}$ une famille libre de $E$ et $x_j \in E$ avec $j \not\in I$. La famille $(x_i)_{i\in I \cup \{j\}}$ est libre si et seulement si $x_j \not\in Vect((x_i)_{i\in I})$. 
\end{tcolorbox}

$\boxed{\Rightarrow}$ \\
Si $x_j \in Vect(x_i)_{i\in I}$, alors $(x_i)_{i\in I \cup \{j\}}$ est liée. \\
En effet, $x_j = \sum\limits_{i\in J} \lambda_i x_i$ avec $J$ fini. \\
Donc $\sum\limits_{i \in J\cup \{j\}} \lambda_i x_i = 0$ avec $\lambda_j = -1$. \\
La famille $(x_i)_{i\in J \cup \{j\}}$. \\ \\

$\boxed{\Leftarrow}$ \\
On suppose que $(x_i)_{i\in J \cup \{j\}}$ est liée. \\
On choisit une famille de scalaires à support fini $(\lambda_i)_{i\in I \cup \{j\}}$ telle que : 
\begin{align*}
    \sum_{i \in I \cup \{j\}} \lambda_i x_i = 0 \text{ et } (\lambda_i) \neq (0) 
\end{align*}
Donc : 
\begin{align*}
    \lambda_j + x_j + \sum_{i\in I} = 0 
\end{align*}
Comme $(x_i)_{i\in I}$ est libre, $\lambda_j \neq 0$ et $x_j = -\sum\limits_{i\in I} \lambda_i \lambda_j^{-1} x_i \in Vect((x_i)_{i\in I})$. 

\setsection{62}
\section{Généricité d'une famille libre maximale}
\begin{tcolorbox}[title=Propostion 20.63, title filled=false, colframe=lightblue, colback=lightblue!10!white]
    Une famille libre maximale est génératrice dans le sens de la définition ci après : tout élément de $E$ est combinaison linéaire de vecteurs de la famille. 
\end{tcolorbox}

\noindent Soit $\mathcal{F}$ une famille libre maximale. \\
Soit $x\in E$. Alors $\mathcal{F} \cup \{x\}$ est liée. \\
Donc (20.61) : 
\begin{align*}
    x \in Vect(\mathcal{F})
\end{align*}

\section{Caractérisation des sommes directes par la liberté}
\begin{tcolorbox}[title=Propostion 20.64, title filled=false, colframe=lightblue, colback=lightblue!10!white]
    Soit $E_1, \ldots, E_n$ des espaces sous-espaces vectoriels non triviaux de $E$. Alors la somme $E_1 \oplus \ldots \oplus E_n$ est directe si et seulement si tout $n$-uplet $(x_1, \ldots, x_n)$ d'éléments tous non nuls de $E_1 \times \ldots \times E_n$ est une famille libre dans $E$. 
\end{tcolorbox}

$\boxed{\Rightarrow}$ \\
On suppose $\bigoplus\limits_{i=1}^n E_i$. Soit $(x_1, \ldots, x_n) \in E_1 \times \ldots \times E_n, x_i \neq 0$. \\
Soit $(\lambda_1, \ldots, \lambda_n) \in \mathbb{K}^n$ telle que : 
\begin{align*}
    \sum_{i=1}^{n} \lambda_i x_i = 0
\end{align*}
En particulier, $\lambda_i x_n = -\sum\limits_{i=1}^{n-1} \lambda_i x_i \in E_n \cap \sum\limits_{i=1}^{n-1} E_i = \{0\}$. 
Donc $\lambda_n = 0$. On réitère le procédé pour trouver $\lambda_n = \ldots = \lambda_1 = 0$. \\
Donc $(x_1, \ldots, x_n)$ est libre. \\ \\

$\boxed{\Leftarrow}$ \\
Soit $x\in E_n \cap \sum E_i$. On écrit $x = x_n = \sum\limits_{i=1}^{n-1} x_i$. \\
Donc : 
\begin{align*}
    x_1 + \ldots + x_{n-1} - x_n = 0
\end{align*}
Par hypotèse, on doit avoir : 
\begin{align*}
    x_n = x_{n-1} = \ldots = x_1 = 0
\end{align*}
Donc $x=0$ et $E_n \cap \left( \sum\limits_{i=1}^{n-1}E_i \right) = \{0\}$. \\
On réitère le procédé pour montrer que $\bigoplus\limits_{i=1}^n E_i$.

\section{Somme directes et caractérisation de familles libres}
\begin{tcolorbox}[title=Propostion 20.65, title filled=false, colframe=lightblue, colback=lightblue!10!white]
    \begin{enumerate}
        \item Soit $F$ et $G$ deux sous-espaces vectoriels de $E$ tel que $F+G$ soit directe. Alors la concaténation d'une famille libre de $F$ et d'une famille libre d $G$ est une famille libre de $E$. 
        \item Réciproquement, si $(b_1, \ldots, b_n)$ est une famille libre de $E$, alors $Vect(b_1, \ldots, b_k) \oplus Vect(b_{k+1}, \ldots, b_n)$ est directe. 
    \end{enumerate}
\end{tcolorbox}

\begin{enumerate}
    \item $(x_1, \ldots, x_k)$ famille libre de $F$. \\
    $(x_{k+1}, \ldots, x_n)$ famille libre de $G$. \\
    Soit $(\lambda_i)_{i\in \llbracket 1, n \rrbracket} \in \mathbb{K}^n$ telle que : 
    \begin{align*}
        \sum_{i=1}^{n} \lambda_i x_i &= 0 \\
        \text{donc } \sum_{i=1}^{k} \lambda_i x_i &= -\sum_{i=k+1}^{n} \lambda_i x_i \in F \cap G = \{0\} \\
        \text{donc } \sum_{i=1}^{k} \lambda_i x_i &= 0 = \sum_{i = k+1}^{n} \lambda_i x_i \\
        \text{donc } \lambda_i &= 0 \text{ pour } i\in \llbracket 1, k \rrbracket \cup \llbracket k+1, n \rrbracket
    \end{align*}

    \item RAS
\end{enumerate}

\setsection{65}
\section{Familles génératrices}
\begin{tcolorbox}[title=Propostion 20.66, title filled=false, colframe=lightblue, colback=lightblue!10!white]
    Une famille $(x_i)_{i\in I}$ de vecteurs de $E$ est une famille \textbf{génératrice de $E$} si l'une des propriétés équivalentes est satisfaite : 
    \begin{enumerate}
        \item Tout $x\in E$ est une combinaison linéaire des $x_i, i\in I$. 
        \item $Vect((x_i)_{i\in I}) = E$. \\
        Si de plus $I = \llbracket 1, n \rrbracket$, les points précédents sont équivalents à : 
        \item $E = \sum\limits_{i=1}^{n} \mathbb{K} x_i$. 
        \item La fonction $\varphi: \mathbb{K}^n \to E; (\lambda_1, \ldots, \lambda_n) \mapsto \lambda_1 x_1 + \ldots + \lambda_n x_n$ est surjective.
    \end{enumerate}
\end{tcolorbox}

$\boxed{1 \Leftrightarrow 2}$ \\
RAF, il s'agit des définitions. \\ \\

$\boxed{2 \Leftrightarrow 3}$ \\
Lorsque $I = \llbracket 1, n \rrbracket$ : 
\begin{align*}
    Vect((x_i)_{i\in I}) &= Vect(x_1, \ldots, x_n) \\
    &= \mathbb{K} x_1 + \ldots + \mathbb{K} x_n \text{ (20.44)}
\end{align*}
Donc $2 \Leftrightarrow 3$. \\ \\

$\boxed{3 \Leftrightarrow 4}$ \\
RAF, il s'agit des définitions.

\setsection{67}
\section{Stabilité des familles génératrices par ajout}
\begin{tcolorbox}[title=Propostion 20.68, title filled=false, colframe=lightblue, colback=lightblue!10!white]
    Toute famille contenant une famille génératrice de $E$ est une famille génératrice de $E$.
\end{tcolorbox}

\noindent Soit $(x_i)_{i\in I}$ une famille quelconque et on suppose qu'il existe $J \subset I$ tel que $(x_i)_{i\in J}$ est génératrice. \\
\begin{align*}
    E \supset Vect((x_i)_{i\in I}) \supset Vect((x_i)_{i\in J}) = E
\end{align*}

\section{Restriction d'une famille génératrice}
\begin{tcolorbox}[title=Propostion 20.69, title filled=false, colframe=lightblue, colback=lightblue!10!white]
    La famille obtenue en retirant un élément $x$ d'une famille génératrice de $E$ est encore génératrice si et seulement si $x$ est une combinaison linéaire des autres vecteurs de la famille. 
\end{tcolorbox}

\noindent RAF : (20.36.5)

\setsection{70}
\section{Liberté d'une famille génératrice minimale}
\begin{tcolorbox}[title=Propostion 20.71, title filled=false, colframe=lightblue, colback=lightblue!10!white]
    Une famille génératrice minimale est libre. 
\end{tcolorbox}

\noindent Soit $(x_i)_{i\in I}$ une famille génératrice minimale. \\
On suppose $\sum\limits_{i\in I} \lambda_i x_i = 0$ avec $(\lambda_i)_{i\in I}$ une famille de scalaires à support fini. \\
Soit $k\in I$, on a : 
\begin{align*}
    \lambda_k x_k = -\sum_{i\in i \neq k} \lambda_i x_i \in Vect((x_i)_{i\in I \backslash \{k\}})
\end{align*}
Or $x_k \not \in Vect((x_i)_{i \neq k})$ car la famille est minimale (20.69). \\
Donc $\lambda_k = 0$. 


\end{document}