\documentclass[../main.tex]{subfiles}

\begin{document}
\setcounter{chapter}{19}
\chapter{Espace Vectoriels}
\tableofcontents
\clearpage

\setsection{1}
\section{Propriétés du $0$, régularité}
\begin{tcolorbox}[title=Propostion 20.2, title filled=false, colframe=lightblue, colback=lightblue!10!white]
    Soit $E$ un $\mathbb{K}-ev$. Pour tout $x \in E$ :
    \begin{enumerate}
        \item $0_{\mathbb{K}}.x = 0_E$
        \item pour tout $\lambda \in \mathbb{K}$, $\lambda.0_E = 0_E$
        \item (-1).x = -x
        \item si $x\neq 0_E$, $$\lambda.x = 0_E \Rightarrow \lambda = 0_{\mathbb{K}}$$
        \item si $x\neq 0_{\mathbb{K}}$, $$\lambda.x = 0_E \Rightarrow x = 0_E$$
    \end{enumerate}
\end{tcolorbox}

\begin{enumerate}
    \item $0_{\mathbb{K}}.x = (0_{\mathbb{K}} + 0_{\mathbb{K}}).x = 0_{\mathbb{K}}.x + 0_{\mathbb{K}}.x$. Donc $0_E = 0_{\mathbb{K}}.x$.
    \item RAS. 
    \item $x + (-1).x = (1 - 1).x = 0_{\mathbb{K}}.x = 0_E$.
    \item Par l'absurde, si $\lambda \neq 0_{\mathbb{K}}$, de $\lambda x = 0_E$ on tire $\lambda^{-1} \lambda x = \lambda^{-1} x 0_E$, soit $x = 0_E$. Absurde. 
    \item Idem. 
\end{enumerate}

\setsection{9}
\section{Espace vectoriel de référence}
\begin{tcolorbox}[title=Propostion 20.10, title filled=false, colframe=lightblue, colback=lightblue!10!white]
    \begin{enumerate}
        \item $\mathbb{K}$ est un espace vectoriel sur lui-même. 
        \item Plus généralement, soit $E$ un espace vectoriel sur $\mathbb{K}$ et $F$ un ensemble quelconque. Alors l'ensemble des fonctions $E^F$ est un espace vectoriel sur $\mathbb{K}$.
    \end{enumerate}
\end{tcolorbox}

\begin{enumerate}
    \item RAF. 
    \item Soit $E$ un $\mathbb{K}-ev$ et $F$ un ensemble quelconque. \\
    $E^F$ est un groupe abélien (cf. chap 10). \\
    Le produit externe est défini par : 
    \begin{align*}
        \mathbb{K} \times E^F &\longrightarrow E^F \\
        (\lambda, f) &\longmapsto (\lambda.f, x \mapsto \lambda.f(x))
    \end{align*}
    Vérification facile. 
\end{enumerate}

\section{Transfert de structure}
\begin{tcolorbox}[title=Lemme 20.11, title filled=false, colframe=orange, colback=orange!10!white]
    Soit $E$ un espace vectoriel sur $\mathbb{K}$, $G$ un ensemble quelconque et $\varphi:E\to G$ une bijection. Alors en définissant sur $G$ une loi interne et un loi externe par
    $$\forall (x, y, \lambda) \in G \times G \times \mathbb{K}, x + y = \varphi(\varphi^{-1}(x) + \varphi^{-1}(y)) \text{et} \lambda.x = \varphi(\lambda \varphi^{-1}(x)),$$
    on munit $G$ d'une structure d'espace vectoriel. 
\end{tcolorbox}

\noindent Vérifions les axiomes. 
\begin{itemize}
    \item LCI : 
    \begin{align*}
        (x + y) +  &= \varphi(\varphi^{-1}(x+y) + \varphi(z)) \\
        &= \varphi(\underbrace{\varphi^{-1}(x) + \varphi^{-1}(y) + \varphi^{-1}(z)}_{\text{associativité dans $E$}}) \\
        &= x + (y + z) \\
        x + \varphi(0) &= \varphi(\varphi^{-1}(x) + 0) = x \text{ (} \varphi \text{ neutre)} \\
        x + \varphi(-\varphi^{-1}(x)) &= \varphi(\varphi^{-1}(x) - \varphi^{-1}(x)) = \varphi(0) \\
        x + y = y + x
    \end{align*}

    \item 
    \begin{align*}
        \lambda.(\mu.x) &= \varphi(\lambda \varphi^{-1}(\mu x)) \\
        &= \varphi(\lambda \mu \varphi^{-1}(x)) \\
        &= (\lambda \mu).x \\
        1.x &= \varphi(1.\varphi^{-1}(x)) \\
        &= \varphi\circ \varphi^{-1}(x) \\
        &= x \\
        (\mu + \lambda).x &= \varphi((\mu + \lambda).\varphi^{-1}(x)) \\
        &= \varphi(\mu \varphi^{-1}(x) + \lambda \varphi^{-1}(x)) \\
        &= \varphi(\mu \varphi^{-1}(x)) + \varphi(\lambda \varphi^{-1}(x)) \\
        &= \mu.x + \lambda.x
    \end{align*}
    De même pour la dernière. 
\end{itemize}

\setsection{15}
\section{Caractérisation des sous-espaces vectoriels}
\begin{tcolorbox}[title=Théorème 20.16, title filled=false, colframe=orange, colback=orange!10!white]
    Soit $E$ un $\mathbb{K}$-espace vectoriel. Un ensemble $F$ est un sous-espace vectoriel de $E$ si et seulement si
    \begin{enumerate}
        \item $F \subset E$;
        \item $0 \in F$;
        \item $F$ est stable par combinaisons linéaire, ce qui équivaut à 
        $$\forall (x, y) \in F^2, \forall \lambda\in \mathbb{K}, \lambda x + y \in F.$$
    \end{enumerate}
\end{tcolorbox}

$\boxed{\Rightarrow}$
\begin{enumerate}
    \item Oui. 
    \item $F$ est un sous-groupe de $E$ donc $0_E \in F$. 
    \item Pour tout $(x, y) \in F^2$, $\lambda \in \mathbb{K}$, $\lambda .x \in F$ et $y \in F$. Donc $\lambda x + y \in F$.
\end{enumerate}

$\boxed{\Leftarrow}$ \\
D'après (3) avec : 
\begin{itemize}
    \item $y = 0$ : $\times$ est LCE. 
    \item $\lambda = 1$ : $+$ est LCI. 
\end{itemize}
$0\in F$ et $\lambda = -1$, $F$ est un sous-groupe, donc un groupe abélien.
RAF pour les 4 dernières propriétés.

\setsection{21}
\section{Propostion 20.22}
\begin{tcolorbox}[title=Propostion 20.22, title filled=false, colframe=lightblue, colback=lightblue!10!white]
    Soit $E$ un $\mathbb{K}$-espace vectoriel, $D_1$ et $D_2$ deux droites vectorielles. Alors soit $D_1 \cap D_2 = \{ 0_E \}$, soit $D_1 = D_2$. 
\end{tcolorbox}

\noindent Par définition, $0_E \in D_1 \cap D_2$. \\
Supposons $D_1 \cap D_2 \neq \{ 0_E \}$ et fixons $x \in D_1 \cap D_2$ avec $x \neq 0_E$. \\
Soit $v\in D_1$. Par définition, on écrit $D_1 = \mathbb{K} x_1$ et $D_2 = \mathbb{K} x_2$. \\
On a donc $v = \alpha x_1$, $x = \lambda_1 x_1 = \lambda_2 x_2$ avec $\lambda_1 \neq 0, \lambda_2 \neq 0$. \\
Ainsi : 
\begin{align*}
    v = \alpha \lambda_1^{-1} \lambda_1 x_1 = \alpha \lambda_1^{-1} x = \alpha \lambda_1^{-1} \lambda_2 x_2 \in D_2
\end{align*}
Donc $D_1 \subset D_2$ et par symétrie, $\boxed{D_1 = D_2}$.

\setsection{26}
\section{Intersection de sous-espaces vectoriels}
\begin{tcolorbox}[title=Propostion 20.27, title filled=false, colframe=lightblue, colback=lightblue!10!white]
    Soit $E$ une espace vectoriel et $(E_i)_{i\in I}$ une famille de sous-espaces vectoriels de $E$. Alors $\bigcap\limits_{i\in I} E_i$ est un sous-espace vectoriel de $E$.
\end{tcolorbox}

\begin{itemize}
    \item $\bigcap\limits_{i\in I} E_i \subset E$.
    \item $\forall i \in I, 0 \in E_i$ donc $0 \in \bigcap\limits_{i\in I} E_i$.
    \item Soit $(x, y) \in \left[\bigcap\limits_{i \in I} E_i \right]^2, \lambda \in \mathbb{K}$ : 
    \begin{align*}
        \forall x \in I, \lambda x + y \in E_i
    \end{align*}
    Donc $\lambda x + y \in \bigcap\limits_{i\in I} E_i$.
\end{itemize}

\setsection{33}
\section{Description de $Vect(X)$}
\begin{tcolorbox}[title=Propostion 20.34, title filled=false, colframe=lightblue, colback=lightblue!10!white]
    Soit $E$ un $\mathbb{K}$-ev et $X$ un sous-ensemble de $E$. Alors $Vect(X)$ est l'ensemble des combinaisons linéaires d'éléments de $X$. 
\end{tcolorbox}

\noindent On note $F$ l'ensemble des combinaisons linéaires de vecteurs de $X$. \\
$F$ est un sous-espace vectoriel de $E$ qui contient $X$. \\
Par définition, $Vect(X) \subset F$. \\
Or $Vect(X)$ est un sous-espace vectoriel qui contient $X$. Il doit donc contenir les combinaisons linéaiers de $X$ soit $F$. \\
Donc $\boxed{F = Vect(X)}$.

\setsection{35}
\section{Opérations sur les sous-espaces vectoriels engendrés}
\begin{tcolorbox}[title=Propostion 20.36, title filled=false, colframe=lightblue, colback=lightblue!10!white]
    Soit $A$ et $B$ deux ensembles. On a
    \begin{enumerate}
        \item $A \subset Vect(A)$
        \item Si $A \subset B$ alors $Vect(A) \subset Vect(B)$. 
        \item $A = Vect(A)$ si et seulement si $A$ est un espace vectoriel.
        \item $Vect(Vect(A)) = Vect(A)$.
        \item $Vect(A \cup \{x\}) = Vect(A)$ si et seulement si $x \in Vect(A)$.
    \end{enumerate}
\end{tcolorbox}

\begin{enumerate}
    \item RAF
    \item RAF (20.24)
    \item Si $A = \underbrace{Vect(A)}_{\text{sous-espace vectoriel}}$, alors $A$ est un sous-espace vectoriel. \\
    Si $A$ est un espace vectoriel, par minimalité, $A = Vect(A)$.
    \item RAF (20.36.3)
    \item On a toujours $Vect(A \cup \{x\}) \supset Vect(A)$(2 0.36.2) si $Vect(A \cup \{x\}) \subset Vect(A)$. \\
    Or $x \in Vect(A \cup \{x\})$. \\
    Donc $x \in Vect(A)$. \\
    Réciproquement, si $x \in Vect(A)$, d'après (20.34) : 
    \begin{align*}
        Vect(A \cup \{x\}) &\subset Vect(A) \\
    \end{align*}
    Si $u \in Vect(A \cup \{x\})$, alors : 
    \begin{align*}
        u &= \lambda_1 a_1 + \ldots + \lambda_n a_n + \lambda_{n+1} x \\
        &= \lambda_1 a_1 + \ldots + \lambda_n a_n + \lambda_{n+1} (\mu_1 a'_1 + \ldots + \mu_p a'_p) \\
        &\in Vect(A)
    \end{align*}
\end{enumerate}


\end{document}