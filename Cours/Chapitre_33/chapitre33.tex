\documentclass[../main.tex]{subfiles}

\begin{document}
\setcounter{chapter}{32}
\chapter{Variables aléatoires réelles finies}
\tableofcontents
\clearpage

\setsection{2}
\section{Exemple}
\begin{tcolorbox}[title=Exemple 33.3, title filled=false, colframe=darkgreen, colback=darkgreen!10!white]
    Un dé à 6 faces numérotées de 1 à 6 a été truqué de telle sorte que les faces 1,2 et 3 tombent avec une probabilité $\frac{1}{6}$, les faces 4 et 5 avec une probabilité $\frac{1}{12}$ et 6 avec une probabilité de $\frac{1}{3}$. Quelle numéro obtient-on en moyenne?
\end{tcolorbox}

\begin{align*}
    E(X) &= 1\times \frac{1}{6} + 2\times \frac{1}{6} + 3\times \frac{1}{6} + 4\times \frac{1}{12} + 5\times \frac{1}{12} + 6\times \frac{1}{3} \\
    &= \frac{45}{12} \\
    &= \frac{15}{4}
\end{align*}


\end{document}