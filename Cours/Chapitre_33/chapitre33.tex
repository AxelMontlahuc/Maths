\documentclass[../main.tex]{subfiles}

\begin{document}
\setcounter{chapter}{32}
\chapter{Variables aléatoires réelles finies}
\tableofcontents
\clearpage

\setsection{2}
\section{Exemple}
\begin{tcolorbox}[title=Exemple 33.3, title filled=false, colframe=darkgreen, colback=darkgreen!10!white]
    Un dé à 6 faces numérotées de 1 à 6 a été truqué de telle sorte que les faces 1,2 et 3 tombent avec une probabilité $\frac{1}{6}$, les faces 4 et 5 avec une probabilité $\frac{1}{12}$ et 6 avec une probabilité de $\frac{1}{3}$. Quelle numéro obtient-on en moyenne?
\end{tcolorbox}

\begin{align*}
    E(X) &= 1\times \frac{1}{6} + 2\times \frac{1}{6} + 3\times \frac{1}{6} + 4\times \frac{1}{12} + 5\times \frac{1}{12} + 6\times \frac{1}{3} \\
    &= \frac{45}{12} \\
    &= \frac{15}{4}
\end{align*}

\section{Espérance des lois usuelles}
\begin{tcolorbox}[title=Théorème 33.4, title filled=false, colframe=orange, colback=orange!10!white]
    Soit $X$ une variable aléatoire réelle sur $\Omega$.
    \begin{enumerate}
        \item Variable aléatoire constante : si $X$ est constante de valeur $m$, alors $\mathrm{E}(X)=m$.
        \item Loi uniforme : si $E=\left\{x_1, \ldots, x_n\right\}$ est une partie de $\mathbb{R}$ et si $X \hookrightarrow \mathcal{U}(E)$, alors $\mathrm{E}(X)$ est la moyenne naturelle des valeurs $x_1, \ldots, x_n$ de $X$ :
        $$\mathrm{E}(X)=\frac{1}{n} \sum_{k=1}^n x_k$$
        \item Loi de Bernoulli : soit $p \in[0 ; 1]$. Si $X \hookrightarrow \mathcal{B}(p)$, alors $\mathrm{E}(X)=p$.
        \item Exemple fondamental : pour tout événement $A \in \mathcal{P}(\Omega), \mathrm{E}\left(\mathbb{1}_A\right)=P(A)$.
        \item Loi binomiale : soit $n \in \mathbb{N}^*$ et $p \in[0 ; 1]$. Si $X \hookrightarrow \mathcal{B}(n, p)$, alors $\mathrm{E}(X)=n p$. 
    \end{enumerate}
\end{tcolorbox}

\begin{enumerate}
    \item Si $X(\Omega) = \{m\}, P(X=m)=1$ et $E(X) = 1\times m = m$. 
    \item Si $X\hookrightarrow \mathcal{U}(\{x_1, \ldots, x_n\}$ alors : 
    \begin{align*}
        \forall i\in \llbracket 1, n \rrbracket, P(X = x_i) = \frac{1}{n}
    \end{align*}
    Donc :
    \begin{align*}
        E(X) &= \sum_{k=1}^{n} P(X = x_k) x_k \\
        &= \frac{1}{n} \sum_{k=1}^{n} x_k
    \end{align*}
    \item Si $X\hookrightarrow \mathcal{B}(p)$ alors :
    \begin{align*}
        E(X) &= 1\times p + 0\times (1-p) \\
        &= p
    \end{align*}
    \item Si $A\subset \Omega$, alors : 
    \begin{align*}
        \mathbb{1}_A\hookrightarrow \mathcal{B}(P(A)) \text{ (32.21)}
    \end{align*}
    Donc (3) $E(\mathbb{1}_A) = P(A)$.
    \item Par définition : 
    \begin{align*}
        E(X) &= \sum_{k=0}^{n} P(X = k) k \\
        &= \sum_{k=0}^{n} k \binom{n}{k} p^k (1-p)^{n-k}
    \end{align*}
    \underline{Première méthode :} \\
    Soit $Q = (1-p + Y)^n\in \mathbb{R}[Y]$. 
    \begin{align*}
        Q &= \sum_{k=0}^{n} \binom{n}{k} (1-p)^{n-k} Y^k
        \text{donc } Q' &= \sum_{k=1}^{n} k \binom{n}{k} (1-p)^{n-k} Y^{k-1} \\
        \text{donc } YQ' &= \sum_{k=0}^{n} k \binom{n}{k} (1-p)^{n-k} Y^k
    \end{align*}
    Par ailleurs $YQ' = n(1-p + Y)^{n-1}$. \\
    En évaluant les deux expressions en $p$, on obtient l'expression voulue : 
    \begin{align*}
        E(X) &= \sum_{k=0}^{n} k \binom{n}{k} p^k (1-p)^{n-k} = np
    \end{align*}
    \underline{Deuxième méthode :} \\
    On poursuit le calcul de $E(X)$ en utilisant la formule du capitaine. \\
    \underline{Troisième méthode :} \\
    En utilisant la linéarité de l'espérance. 
\end{enumerate}

\section{Propriétés de l'espérance}
\begin{tcolorbox}[title=Propostion 33.5, title filled=false, colframe=lightblue, colback=lightblue!10!white]
    Soit $X$ et $Y$ deux variables aléatoires réelles sur $\Omega$.
    \begin{enumerate}
        \item Reformulation : $\mathrm{E}(X)=\sum_{\omega \in \Omega} P(\{\omega\}) X(\omega)$.
        \item Linéarité : pour tout $(\lambda, \mu) \in \mathbb{R}^2, \mathrm{E}(\lambda X+\mu Y)=\lambda \mathrm{E}(X)+\mu \mathrm{E}(Y)$.
        \item Positivité : si $X \geq 0$, alors $\mathrm{E}(X) \geq 0$.
        \item Croissance : si $X \leq Y$, alors $\mathrm{E}(X) \leq \mathrm{E}(Y)$.
        \item Inégalité triangulaire : $|\mathrm{E}(X)| \leq \mathrm{E}(|X|)$.
    \end{enumerate}
\end{tcolorbox}

\begin{enumerate}
    \item On rappelle qe $\{(X=x)\}_{x\in X(\Omega)}$ est un SCE. \\
    Ainsi : 
    \begin{align*}
        E(X) &= \sum_{x\in X(\Omega)} P(X=x) x \\
        &= \sum_{x\in X(\Omega)} \left[ \sum_{\omega\in (X=x)} P(X = \omega) \right] x \\
        &= \sum_{\omega\in \Omega} P(\{\omega\}) X(\omega)
    \end{align*}
    \item \begin{align*}
        E(\lambda X+\mu Y) &= \sum_{\omega\in \Omega} P(\{\omega\}) (\lambda X(\omega)+\mu Y(\omega)) \\
        &= \lambda \sum_{\omega\in \Omega} P(\{\omega\}) X(\omega) + \mu \sum_{\omega\in \Omega} P(\{\omega\}) Y(\omega) \\
        &= \lambda E(X) + \mu E(Y)
    \end{align*}
    \item Si $X\geq 0$, alors : 
    \begin{align*}
        E(X) &= \sum_{\omega\in \Omega} \underbrace{P(\{\omega\})}_{\geq 0} \underbrace{X(\omega)}_{\geq 0} \\
        &\geq 0
    \end{align*}
    \item RAS (2 + 3)
    \item \begin{align*}
        |E(X)| &= \left| \sum_{\omega\in \Omega} P(\{\omega\}) X(\omega) \right| \\
        &\leq \sum_{\omega\in \Omega} P(\{\omega\}) |X(\omega)| \\
        &= E(|X|)
    \end{align*}
\end{enumerate}

\section{Exemple}
\begin{tcolorbox}[title=Exemple 33.7, title filled=false, colframe=darkgreen, colback=darkgreen!10!white]
    Qu'obtient-on en moyenne quand on lance deux fois un dé à 6 faces et qu'on additionne les résultats obtenus ?
\end{tcolorbox}

\noindent $X_1, X_2 \hookrightarrow \mathcal{U}(P(\llbracket 1, 6 \rrbracket))$. 
\begin{align*}
    E(X_1 + X_2) &= E(X_1) + E(X_2) \\
    &= 2\times \frac{1}{6} \sum_{k=1}^{6} k \\
    &= 7
\end{align*}

\section{Formule de transfert}
\begin{tcolorbox}[title=Théorème 33.8, title filled=false, colframe=orange, colback=orange!10!white]
    Soit $X$ une variable aléatoire sur $\Omega$ et $f: X(\Omega) \rightarrow \mathbb{R}$ une fonction. L'espérance de $f(X)$ est entièrement déterminée par $f$ et la loi de $X$ :
    $$\mathrm{E}(f(X))=\sum_{x \in X(\Omega)} P(X=x) f(x)$$
\end{tcolorbox}

\noindent $\{(X = x)\}_{x\in X(\Omega)}$ est un SCE. 
\begin{align*}
    E(f(X)) &= \sum_{x\in X(\Omega)} P(\{\omega\}) f(X(\omega)) \\
    &= \sum_{x\in X(\Omega)} \left( \sum_{\omega\in (X=x)} P(\{w\}) f(X(\omega)) \right) \\
    &= \sum_{x\in X(\Omega)} \left( \sum_{\omega\in (X = x)} P(\{w\}) \right) f(x) \\
    &= \sum_{x\in X(\Omega)} P(X=x) f(x)
\end{align*}

\setsection{9}
\section{Exemple}
\begin{tcolorbox}[title=Exemple 33.10, title filled=false, colframe=darkgreen, colback=darkgreen!10!white]
    Soit $X$ une variable aléatoire suivant une loi unifore sur $\llbracket 1, n \rrbracket$. Donner un équivalent simple de $E(X)$ et de $E(X^2)$. 
\end{tcolorbox}

\noindent $X\hookrightarrow \mathcal{U}(\llbracket 1, n \rrbracket)$. 
\begin{itemize}
    \item \begin{align*}
        E(X) &= \frac{1}{n} \sum_{k=1}^{n} k \\
        &= \frac{n+1}{2} \\
        &\underset{n\to +\infty}{\sim} \frac{n}{2}
    \end{align*}
    \item \begin{align*}
        E(X^2) &= \sum_{k=1}^{n} \frac{1}{n} k^2 \\
        &= \frac{(n+1)(2n+1)}{6} \\
        &\underset{n\to +\infty}{\sim} \frac{n^2}{3}
    \end{align*}
\end{itemize}

\section{Espérance du produit de deux variables aléatoires réelles indépendantes}
\begin{tcolorbox}[title=Théorème 33.11, title filled=false, colframe=orange, colback=orange!10!white]
    Soit $X$ et $Y$ deux variables aléatoires réelles sur $\Omega$. Si $X$ et $Y$ sont indépendantes, alors
    $$\mathrm{E}(X Y)=\mathrm{E}(X) \mathrm{E}(Y)$$
    Ce résultat s'étend naturellement à un nombre fini quelconque de variables aléatoires réelles indépendantes.
\end{tcolorbox}

\begin{align*}
    E(x)E(Y) &= \left( \sum_{x\in X(\Omega)} P(X=x) x \right) \left( \sum_{y\in Y(\Omega)} P(Y=y) y \right) \\
    &= \sum_{(x, y)\in X(\Omega) \times Y(\Omega)} P(X=x) P(Y=y) x y \\
    &= \sum_{(x, y)\in X(\Omega) \times Y(\Omega)} P(X=x \text{ et } Y=y) x y \text{ (indépendance)} \\
    &= E(XY)
\end{align*}

\setsection{12}
\section{Propriétés de la variance}
\begin{tcolorbox}[title=Propostion 33.13, title filled=false, colframe=lightblue, colback=lightblue!10!white]
    Soit $X$ une variable aléatoire réelle.
    \begin{enumerate}
        \item $\mathrm{V}(X)=\mathrm{E}\left(X^2\right)-\mathrm{E}(X)^2$.
        \item $\mathrm{V}(X)=0 \Leftrightarrow P(X=\mathrm{E}(X))=1$. On dit dans ce cas que $X$ est presque sûrement constante.
        \item Pour tout $(a, b) \in \mathbb{R}^2$, on a
        $$\mathrm{V}(a x+b)=a^2 \mathrm{~V}(X)$$
    \end{enumerate}
    En particulier, si $\sigma(X)>0$, la variable $\frac{X-\mathrm{E}(X)}{\sigma(X)}$ est centrée réduite.
\end{tcolorbox}

\begin{enumerate}
    \item \begin{align*}
        V(X) &= E((X - E(X))^2) \\
        &= E(X^2 - 2X E(X) + E(X)^2) \\
        &= E(X^2) - 2E(X)E(X) + E(X)^2 \\
        &= E(X^2) - E(X)^2
    \end{align*}

    \item \begin{align*}
        V(X) = 0 &\Leftrightarrow E((X - E(X))^2) = 0 \\
        \text{(fonction de transfert)\quad}&\Leftrightarrow \sum_{x\in X(\Omega)} P(X=x) (x - E(X))^2 = 0 \\
        &\Leftrightarrow \forall x\in X(\Omega)\setminus \{E(X)\}, P(X=x) = 0 \\
        &= P(X = E(X)) = 1
    \end{align*}

    \item \begin{align*}
        V(aX + b) &= E((aX + b - E(aX + b))^2) \\
        &= E(a^2(X-E(X))) \text{ (linéarité)} \\
        &= a^2 V(X) \text{ (linéarité)}
    \end{align*}
\end{enumerate}

\setsection{14}
\section{Propriétés de la covariance}
\begin{tcolorbox}[title=Propostion 33.15, title filled=false, colframe=lightblue, colback=lightblue!10!white]
    On a :
    \begin{enumerate}
        \item $\mathrm{V}(X)=\operatorname{cov}(X, X)$ et $\operatorname{cov}(X, Y)=\operatorname{cov}(Y, X)$.
        \item $\operatorname{cov}(X, Y)=\mathrm{E}(X Y)-\mathrm{E}(X) \mathrm{E}(Y)$
        \item $\mathrm{V}(X+Y)=\mathrm{V}(X)+2 \cdot \operatorname{cov}(X, Y)+\mathrm{V}(Y)$.
        \item Si $X$ et $Y$ sont indépendantes, alors $\operatorname{cov}(X, Y)=0$ et $\mathrm{V}(X+Y)=\mathrm{V}(X)+\mathrm{V}(Y)$.
    \end{enumerate}
    Les assertions 3 et 4 se généralisent au cas de variables aléatoires réelles $X_1, \ldots, X_n$ sur $\Omega$. Dans ce cas :
    $$\mathrm{V}\left(\sum_{i=1}^n X_i\right)=\sum_{i=1}^n \mathrm{~V}\left(X_i\right)+2 \sum_{1 \leq i<j \leq n} \operatorname{cov}\left(X_i, X_j\right)$$
    Si $X_1, \ldots, X_n$ sont (seulement) deux à deux indépendantes, alors
    $$\mathrm{V}\left(\sum_{i=1}^n X_i\right)=\sum_{i=1}^n \mathrm{~V}\left(X_i\right)$$
\end{tcolorbox}

\begin{enumerate}
    \item RAF
    \item \begin{align*}
        \operatorname{cov}(X, Y) &= E((X - E(X))(Y - E(Y))) \\
        &= E(XY - E(X)Y - E(Y)X + E(X) E(Y)) \\
        &= E(XY) - E(X)E(Y)
    \end{align*}
    \item \begin{align*}
        V(X+Y) &= E((X + Y - E(X + Y))^2) \\
        &= E((X - E(X)) + (Y - E(Y)))^2 \\
        &= V(X) + 2 \operatorname{cov}(X, Y) + V(Y)
    \end{align*}
    \item On suppose que $X$ et $Y$ sont indépendantes, donc :
    \begin{align*}
        E(XY) &= E(X)E(Y)
    \end{align*}
    Puis, par (2) :
    \begin{align*}
        \operatorname{cov}(X, Y) &= 0
    \end{align*}
\end{enumerate}

\section{Variance des lois usuelles}
\begin{tcolorbox}[title=Théorème 33.16, title filled=false, colframe=orange, colback=orange!10!white]
    Soit $X$ une variable aléatoire réelle et $p \in[0 ; 1]$.
    \begin{enumerate}
        \item $\quad$ Si $X \hookrightarrow \mathcal{B}(p)$ alors $\mathrm{V}(X)=p(1-p)$.
        \item $\quad$ Si $X \hookrightarrow \mathcal{B}(n, p)$, alors $\mathrm{V}(X)=n p(1-p)$.
    \end{enumerate}
\end{tcolorbox}

\begin{enumerate}
    \item Si $X \hookrightarrow \mathcal{B}(p)$, alors $X^2\hookrightarrow B(p)$. 
    \begin{align*}
        V(X) &= E(X^2) - E(X)^2 = p - p^2 = p(1-p)
    \end{align*}
    \item Si $X_i\hookrightarrow \mathcal{B}(p)$ avec $X_1, \ldots, X_n$ indépendantes, alors : 
    \begin{align*}
        V(X_1 + \cdots + X_n) &= \sum_{k=1}^{n} V(X_k) \text{ (indépendance)} \\
        &= \sum_{k=1}^{n} p(1-p) \text{ (1)} \\
        &= np(1-p)
    \end{align*}
\end{enumerate}

\section{Inégalité de Markov}
\begin{tcolorbox}[title=Théorème 33.17, title filled=false, colframe=orange, colback=orange!10!white]
    Soit $X$ une variable aléatoire réelle. Pour tout $a>0$, on a
    $$P(|X| \geq a) \leq \frac{\mathrm{E}(|X|)}{a}$$
\end{tcolorbox}

\noindent On remarque que : 
\begin{align*}
    \mathbb{1}_{(|X|\geq a)} a \leq \mathbb{1}_{(|X|\geq a)}|X| \leq |X|
\end{align*}
Par croissance de l'espérance : 
\begin{align*}
    a P(|X|\geq a) = a E(\mathbb{1}_{(|X|\geq a)}) \leq E(|X|)
\end{align*}


\end{document}