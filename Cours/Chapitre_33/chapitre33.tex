\documentclass[../main.tex]{subfiles}

\begin{document}
\setcounter{chapter}{32}
\chapter{Variables aléatoires réelles finies}
\tableofcontents
\clearpage

\setsection{2}
\section{Exemple}
\begin{tcolorbox}[title=Exemple 33.3, title filled=false, colframe=darkgreen, colback=darkgreen!10!white]
    Un dé à 6 faces numérotées de 1 à 6 a été truqué de telle sorte que les faces 1,2 et 3 tombent avec une probabilité $\frac{1}{6}$, les faces 4 et 5 avec une probabilité $\frac{1}{12}$ et 6 avec une probabilité de $\frac{1}{3}$. Quelle numéro obtient-on en moyenne?
\end{tcolorbox}

\begin{align*}
    E(X) &= 1\times \frac{1}{6} + 2\times \frac{1}{6} + 3\times \frac{1}{6} + 4\times \frac{1}{12} + 5\times \frac{1}{12} + 6\times \frac{1}{3} \\
    &= \frac{45}{12} \\
    &= \frac{15}{4}
\end{align*}

\section{Espérance des lois usuelles}
\begin{tcolorbox}[title=Théorème 33.4, title filled=false, colframe=orange, colback=orange!10!white]
    Soit $X$ une variable aléatoire réelle sur $\Omega$.
    \begin{enumerate}
        \item Variable aléatoire constante : si $X$ est constante de valeur $m$, alors $\mathrm{E}(X)=m$.
        \item Loi uniforme : si $E=\left\{x_1, \ldots, x_n\right\}$ est une partie de $\mathbb{R}$ et si $X \hookrightarrow \mathcal{U}(E)$, alors $\mathrm{E}(X)$ est la moyenne naturelle des valeurs $x_1, \ldots, x_n$ de $X$ :
        $$\mathrm{E}(X)=\frac{1}{n} \sum_{k=1}^n x_k$$
        \item Loi de Bernoulli : soit $p \in[0 ; 1]$. Si $X \hookrightarrow \mathcal{B}(p)$, alors $\mathrm{E}(X)=p$.
        \item Exemple fondamental : pour tout événement $A \in \mathcal{P}(\Omega), \mathrm{E}\left(\mathbb{1}_A\right)=P(A)$.
        \item Loi binomiale : soit $n \in \mathbb{N}^*$ et $p \in[0 ; 1]$. Si $X \hookrightarrow \mathcal{B}(n, p)$, alors $\mathrm{E}(X)=n p$. 
    \end{enumerate}
\end{tcolorbox}

\begin{enumerate}
    \item Si $X(\Omega) = \{m\}, P(X=m)=1$ et $E(X) = 1\times m = m$. 
    \item Si $X\hookrightarrow \mathcal{U}(\{x_1, \ldots, x_n\}$ alors : 
    \begin{align*}
        \forall i\in \llbracket 1, n \rrbracket, P(X = x_i) = \frac{1}{n}
    \end{align*}
    Donc :
    \begin{align*}
        E(X) &= \sum_{k=1}^{n} P(X = x_k) x_k \\
        &= \frac{1}{n} \sum_{k=1}^{n} x_k
    \end{align*}
    \item Si $X\hookrightarrow \mathcal{B}(p)$ alors :
    \begin{align*}
        E(X) &= 1\times p + 0\times (1-p) \\
        &= p
    \end{align*}
    \item Si $A\subset \Omega$, alors : 
    \begin{align*}
        \mathbb{1}_A\hookrightarrow \mathcal{B}(P(A)) \text{ (32.21)}
    \end{align*}
    Donc (3) $E(\mathbb{1}_A) = P(A)$.
    \item Par définition : 
    \begin{align*}
        E(X) &= \sum_{k=0}^{n} P(X = k) k \\
        &= \sum_{k=0}^{n} k \binom{n}{k} p^k (1-p)^{n-k}
    \end{align*}
    \underline{Première méthode :} \\
    Soit $Q = (1-p + Y)^n\in \mathbb{R}[Y]$. 
    \begin{align*}
        Q &= \sum_{k=0}^{n} \binom{n}{k} (1-p)^{n-k} Y^k
        \text{donc } Q' &= \sum_{k=1}^{n} k \binom{n}{k} (1-p)^{n-k} Y^{k-1} \\
        \text{donc } YQ' &= \sum_{k=0}^{n} k \binom{n}{k} (1-p)^{n-k} Y^k
    \end{align*}
    Par ailleurs $YQ' = n(1-p + Y)^{n-1}$. \\
    En évaluant les deux expressions en $p$, on obtient l'expression voulue : 
    \begin{align*}
        E(X) &= \sum_{k=0}^{n} k \binom{n}{k} p^k (1-p)^{n-k} = np
    \end{align*}
    \underline{Deuxième méthode :} \\
    On poursuit le calcul de $E(X)$ en utilisant la formule du capitaine. \\
    \underline{Troisième méthode :} \\
    En utilisant la linéarité de l'espérance. 
\end{enumerate}

\section{Propriétés de l'espérance}
\begin{tcolorbox}[title=Propostion 33.5, title filled=false, colframe=lightblue, colback=lightblue!10!white]
    Soit $X$ et $Y$ deux variables aléatoires réelles sur $\Omega$.
    \begin{enumerate}
        \item Reformulation : $\mathrm{E}(X)=\sum_{\omega \in \Omega} P(\{\omega\}) X(\omega)$.
        \item Linéarité : pour tout $(\lambda, \mu) \in \mathbb{R}^2, \mathrm{E}(\lambda X+\mu Y)=\lambda \mathrm{E}(X)+\mu \mathrm{E}(Y)$.
        \item Positivité : si $X \geq 0$, alors $\mathrm{E}(X) \geq 0$.
        \item Croissance : si $X \leq Y$, alors $\mathrm{E}(X) \leq \mathrm{E}(Y)$.
        \item Inégalité triangulaire : $|\mathrm{E}(X)| \leq \mathrm{E}(|X|)$.
    \end{enumerate}
\end{tcolorbox}

\begin{enumerate}
    \item On rappelle qe $\{(X=x)\}_{x\in X(\Omega)}$ est un SCE. \\
    Ainsi : 
    \begin{align*}
        E(X) &= \sum_{x\in X(\Omega)} P(X=x) x \\
        &= \sum_{x\in X(\Omega)} \left[ \sum_{\omega\in (X=x)} P(X = \omega) \right] x \\
        &= \sum_{\omega\in \Omega} P(\{\omega\}) X(\omega)
    \end{align*}
    \item \begin{align*}
        E(\lambda X+\mu Y) &= \sum_{\omega\in \Omega} P(\{\omega\}) (\lambda X(\omega)+\mu Y(\omega)) \\
        &= \lambda \sum_{\omega\in \Omega} P(\{\omega\}) X(\omega) + \mu \sum_{\omega\in \Omega} P(\{\omega\}) Y(\omega) \\
        &= \lambda E(X) + \mu E(Y)
    \end{align*}
    \item Si $X\geq 0$, alors : 
    \begin{align*}
        E(X) &= \sum_{\omega\in \Omega} \underbrace{P(\{\omega\})}_{\geq 0} \underbrace{X(\omega)}_{\geq 0} \\
        &\geq 0
    \end{align*}
    \item RAS (2 + 3)
    \item \begin{align*}
        |E(X)| &= \left| \sum_{\omega\in \Omega} P(\{\omega\}) X(\omega) \right| \\
        &\leq \sum_{\omega\in \Omega} P(\{\omega\}) |X(\omega)| \\
        &= E(|X|)
    \end{align*}
\end{enumerate}

\section{Exemple}
\begin{tcolorbox}[title=Exemple 33.7, title filled=false, colframe=darkgreen, colback=darkgreen!10!white]
    Qu'obtient-on en moyenne quand on lance deux fois un dé à 6 faces et qu'on additionne les résultats obtenus ?
\end{tcolorbox}

\noindent $X_1, X_2 \hookrightarrow \mathcal{U}(P(\llbracket 1, 6 \rrbracket))$. 
\begin{align*}
    E(X_1 + X_2) &= E(X_1) + E(X_2) \\
    &= 2\times \frac{1}{6} \sum_{k=1}^{6} k \\
    &= 7
\end{align*}

\section{Formule de transfert}
\begin{tcolorbox}[title=Théorème 33.8, title filled=false, colframe=orange, colback=orange!10!white]
    Soit $X$ une variable aléatoire sur $\Omega$ et $f: X(\Omega) \rightarrow \mathbb{R}$ une fonction. L'espérance de $f(X)$ est entièrement déterminée par $f$ et la loi de $X$ :
    $$\mathrm{E}(f(X))=\sum_{x \in X(\Omega)} P(X=x) f(x)$$
\end{tcolorbox}

\noindent $\{(X = x)\}_{x\in X(\Omega)}$ est un SCE. 
\begin{align*}
    E(f(X)) &= \sum_{x\in X(\Omega)} P(\{\omega\}) f(X(\omega)) \\
    &= \sum_{x\in X(\Omega)} \left( \sum_{\omega\in (X=x)} P(\{w\}) f(X(\omega)) \right) \\
    &= \sum_{x\in X(\Omega)} \left( \sum_{\omega\in (X = x)} P(\{w\}) \right) f(x) \\
    &= \sum_{x\in X(\Omega)} P(X=x) f(x)
\end{align*}


\end{document}