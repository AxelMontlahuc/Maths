\documentclass[../main.tex]{subfiles}

\begin{document}
\setcounter{chapter}{26}
\chapter{Séries numériques}
\tableofcontents
\clearpage

\setsection{5}
\section{Série géométrique}
\begin{tcolorbox}[title=Théorème 27.6, title filled=false, colframe=orange, colback=orange!10!white]
    Soit $a\in \mathbb{C}$. La série $\sum a^n$ converge si et seulement si $|a|<1$. Dans ce cas : 
    \begin{align*}
        \sum_{n=0}^{+\infty} a^n = \frac{1}{1-a}
    \end{align*}
\end{tcolorbox}

\noindent Soit $n\in \mathbb{N}$. 
\begin{align*}
    S_n = \sum_{k=0}^{n} a^k &= \frac{1-a^{n+1}}{1-a} \text{ ($a\neq 1$)} \\
    &\underset{n \to +\infty}{\longrightarrow} \frac{1}{1-a} \text{ ($|a|<1$)}
\end{align*}
La série converge et $\sum\limits_{n\geq 0} a^n = \frac{1}{1-a}$. 

\setsection{10}
\section{Deux séries de termes généraux égaux presque partout}
\begin{tcolorbox}[title=Propostion 27.11, title filled=false, colframe=lightblue, colback=lightblue!10!white]
    Si $(u_n)$ et $(v_n)$ ne diffèrent que d'un nombre fini de termes, alors $\sum u_n$ et $\sum v_n$ sont de même nature. 
\end{tcolorbox}

\noindent On note $A = \{ n\in \mathbb{N}, u_n \neq v_n \}$. Supposons $A\neq \emptyset$. \\
D'après les hypothèses, $A$ est majoré donc possède un maximum $N$ d'après la propriété fondamentale de $\mathbb{N}$. \\
On note $(S_n)$ et $(S_n')$ les sommes partielles associée à $\sum u_n$ et $\sum v_n$. \\
Pour $n \geq N$ : 
\begin{align*}
    S_n = S_n' + K \text{ où } K = \sum_{k\in A} (u_k - v_k) \text{ (constant)}
\end{align*}
Ainsi $(S_n)$ converge si et seulement si $(S_n')$ converge. 


\end{document}