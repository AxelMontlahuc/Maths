\documentclass[../main.tex]{subfiles}

\begin{document}
\setcounter{chapter}{26}
\chapter{Séries numériques}
\tableofcontents
\clearpage

\setsection{5}
\section{Série géométrique}
\begin{tcolorbox}[title=Théorème 27.6, title filled=false, colframe=orange, colback=orange!10!white]
    Soit $a\in \mathbb{C}$. La série $\sum a^n$ converge si et seulement si $|a|<1$. Dans ce cas : 
    \begin{align*}
        \sum_{n=0}^{+\infty} a^n = \frac{1}{1-a}
    \end{align*}
\end{tcolorbox}

\noindent Soit $n\in \mathbb{N}$. 
\begin{align*}
    S_n = \sum_{k=0}^{n} a^k &= \frac{1-a^{n+1}}{1-a} \text{ ($a\neq 1$)} \\
    &\underset{n \to +\infty}{\longrightarrow} \frac{1}{1-a} \text{ ($|a|<1$)}
\end{align*}
La série converge et $\sum\limits_{n\geq 0} a^n = \frac{1}{1-a}$. 

\setsection{10}
\section{Deux séries de termes généraux égaux presque partout}
\begin{tcolorbox}[title=Propostion 27.11, title filled=false, colframe=lightblue, colback=lightblue!10!white]
    Si $(u_n)$ et $(v_n)$ ne diffèrent que d'un nombre fini de termes, alors $\sum u_n$ et $\sum v_n$ sont de même nature. 
\end{tcolorbox}

\noindent On note $A = \{ n\in \mathbb{N}, u_n \neq v_n \}$. Supposons $A\neq \emptyset$. \\
D'après les hypothèses, $A$ est majoré donc possède un maximum $N$ d'après la propriété fondamentale de $\mathbb{N}$. \\
On note $(S_n)$ et $(S_n')$ les sommes partielles associée à $\sum u_n$ et $\sum v_n$. \\
Pour $n \geq N$ : 
\begin{align*}
    S_n = S_n' + K \text{ où } K = \sum_{k\in A} (u_k - v_k) \text{ (constant)}
\end{align*}
Ainsi $(S_n)$ converge si et seulement si $(S_n')$ converge. 

\section{CN de convergence portant sur le terme général}
\begin{tcolorbox}[title=Théorème 27.12, title filled=false, colframe=orange, colback=orange!10!white]
    Si $\sum u_n$ converge, alors $(u_n)$ converge vers $0$. De manière équivalente, si $(u_n)$ ne tend pas vers $0$, la série $\sum u_n$ diverge.
\end{tcolorbox}

\noindent On suppose que $S_n \underset{n \to +\infty}{\longrightarrow} \ell \in \mathbb{R} \text{ ou } \mathbb{C}$. \\
\begin{align*}
    u_n = S_n - S_{n-1} &= \ell - \ell = 0
\end{align*}

\setsection{15}
\section{Théorème de comparaison des séries à termes positifs}
\begin{tcolorbox}[title=Théorème 27.16, title filled=false, colframe=orange, colback=orange!10!white]
    Soit $\sum u_n$ et $\sum v_n$ deux séries à termes positifs telles qu'il existe $N\in \mathbb{N}$ tel que pour tout $n\geq N$ : 
    \begin{align*}
        0\leq u_n \leq v_n
    \end{align*}
    Alors : 
    \begin{itemize}
        \item Si $\sum v_n$ converge, alors $\sum u_n$ converge aussi. 
        \item Si $\sum u_n$ diverge (vers $+\infty$ donc), alors $\sum v_n$ diverge aussi (vers $+\infty$ donc).
    \end{itemize}
    De plus, si la divergence est grossière pour $\sum u_n$, elle l'est aussi pour $\sum v_n$. 
\end{tcolorbox}

\noindent En utilisant les notations du (27.11), on peut supposer que : 
\begin{align*}
    \forall n \geq 0, 0\leq u_n \leq v_n
\end{align*}
Puis : 
\begin{align*}
    \forall n \geq 0, 0\leq S_n \leq S_n'
\end{align*}
On utilise alors le théroème de comparaison sur les suites. 

\setsection{19}
\section{Convergence absolue entraîne convergence}
\begin{tcolorbox}[title=Théorème 27.20, title filled=false, colframe=orange, colback=orange!10!white]
    Toute série réelle ou complexe absolument convergente est convergente. 
\end{tcolorbox}

\begin{itemize}
    \item On suppose que $(u_n)\in \mathbb{R}^{\mathbb{N}}$, avec $\sum |u_n|$ convergente. \\
    On pose, pour tout $n\in \mathbb{N}$ :
    \begin{align*}
        u_n^+ = \max(u_n, 0) \geq 0 \text{ et } u_n^- = \max(-u_n, 0) \geq 0
    \end{align*}
    Ainsi, $u_n = u_n^+ - u_n^-$. \\
    Or, pour tout $n$ : 
    \begin{align*}
        0 &\leq u_n^+ \leq |u_n| \\
        0 &\leq u_n^- \leq |u_n|
    \end{align*}
    Par comparaison des séries à termes positifs, $\sum u_n^+$ et $\sum u_n^-$ convergent et par linéarité (27.16) $\sum u_n$ converge. 

    \item On suppose que $(u_n)\in \mathbb{C}^{\mathbb{N}}$, avec $\sum |u_n|$ convergente. \\
    Alors : 
    \begin{align*}
        \forall n\in \mathbb{N}, |Re(u_n)| &\leq |u_n| \\
        |Im(u_n)| &\leq |u_n|
    \end{align*}
    Donc, $\sum Re(u_n)$ et $\sum Im(u_n)$ sont absolument convergentes (27.15) donc convergent, puis par combinaison linéaire (27.16) $\sum u_n$ converge. 
\end{itemize}

\setsection{22}
\section{Comparaison des séries par domination ou négligabilité}
\begin{tcolorbox}[title=Théorème 27.23, title filled=false, colframe=orange, colback=orange!10!white]
    Soit $\sum u_n$ une série à termes quelconques et $\sum v_n$ une série à termes positifs telles que $u_n = O(v_n)$ (ou $u_n = o(v_n)$). Alors : 
    \begin{itemize}
        \item La convergence de $\sum v_n$ entraîne la convergence absolue de $\sum u_n$. 
        \item La divergence de $\sum u_n$ (celle de $\sum |u_n|$ suffit) entraîne la divergence de $\sum v_n$.
    \end{itemize}
\end{tcolorbox}

\noindent On suppose $u_n = O(v_n)$ avec $v_n \geq 0$. 
\begin{itemize}
    \item On suppose que $\sum v_n$ converge. On a $|u_n| = O(v_n)$ donc à partir d'un certain rang : 
    \begin{align*}
        0 \leq |u_n| \leq M v_n
    \end{align*}
    D'après le théorème de comparaison par majoration des séries à termes positifs, $\sum |u_n|$ converge donc $\sum u_n$ converge.

    \item Si $\sum |u_n|$ diverge, par comparaison par minoration des séries à termes positifs, $\sum v_n$ diverge. 
\end{itemize}

\section{Comparaison des séries à termes positifs par équivalence}
\begin{tcolorbox}[title=Théorème 27.24, title filled=false, colframe=orange, colback=orange!10!white]
    Soit $\sum u_n$ et $\sum v_n$ deux séries à termes positifs. Si $u_n \underset{n\to +\infty}{\sim} v_n$, alors les séries $\sum u_n$ et $\sum v_n$ sont de même nature.
\end{tcolorbox}

\noindent Si $u_n \underset{n\to +\infty}{\sim} v_n$, alors $u_n \underset{n\to +\infty}{=} O(v_n)$ et $v_n \underset{n\to +\infty}{=} O(u_n)$. \\
On conclut avec (27.23). 

\section{Théorème de comparaison entre série et intégrale}
\begin{tcolorbox}[title=Théorème 27.25, title filled=false, colframe=orange, colback=orange!10!white]
    Soit $a\in \mathbb{R}$ et soit $f:[a; +\infty[\to \mathbb{R}$ une fonction décroissante et positive. Alors $\sum f(n)$ converge si et seulement si $\int_a^{+\infty} f(t) \,dt$ converge aussi (i.e. $\lim_{x\to +\infty} \int_{a}^{x} f(t) \,dt$ existe et est finie).
\end{tcolorbox}

\noindent D'après le TLM ($f\geq 0$), $\lim\limits_{x\to +\infty} \int_{a}^{x} f(t) \,dt$ existe dans $\mathbb{R}\cup \{+\infty\}$ et : 
\begin{align*}
    \lim_{x\to +\infty} \int_{a}^{x} f(t) \,dt &= \lim_{n\to +\infty} \int_{a}^{n} f(t) \,dt
\end{align*}
Soit $n_0\in \mathbb{N}$ avec $n_0 \geq a$. $\int_{a}^{+\infty} f(t) \,dt$ et $\int_{n_0}^{+\infty} f(t) \,dt$ sont de même nature. \\
Comme $f$ est décroissante, pour tout $n\geq n_0$ : 
\begin{align*}
    f(n+1) \leq \int_{n}^{n+1} f(t) \,dt \leq f(n)
\end{align*}
Donc par Chasles : 
\begin{align*}
    \underbrace{\sum_{k=n_0}^{n} f(k+1)}_{\sum\limits_{k=n_0}^{n+1} f(k) - f(n_0)} &\leq \int_{n_0}^{n+1} f(t) \,dt \leq \sum_{k=n_0}^{n} f(k)
\end{align*}
D'après le TLM : 
\begin{itemize}
    \item Si $\sum (f_n)$ converge, alors $\lim\limits_{n\to +\infty} \int_{n_0}^{n+1} f(t) \,dt\in \mathbb{R}_+$. 
    \item Si $\lim\limits_{n\to +\infty} \int_{n_0}^{n+1} f(t) \,dt\in \mathbb{R}_+$, alors $\sum (f_n)$ converge.
\end{itemize}

\section*{Exercice 1}
\begin{tcolorbox}[title=Exercice 27.1, title filled=false, colframe=darkgreen, colback=darkgreen!10!white]
    En utilisant le théorème de comparaison, déterminer la nature de la série de terme général $u_n = \left( \frac{1}{n} \right)^{1 + \frac{1}{n}}$. 
\end{tcolorbox}

\begin{align*}
    u_n &= \left( \frac{1}{n} \right)^{1 + \frac{1}{n}} \\
    &= \frac{1}{n} e^{o(1)} \\
    &\geq \frac{1}{2n} \text{ à partir d'un certain rang}
\end{align*}
Par comparaison des séries à termes positifs, $\sum u_n$ diverge.

\section*{Exercice 2}
\begin{tcolorbox}[title=Exercice 27.2, title filled=false, colframe=darkgreen, colback=darkgreen!10!white]
    En utilisant un théorème de comparaison par domination ou négligabilité, déterminer la nature de la série de terme général :
    \begin{align*}
        u_n &= \frac{e - \left( 1 + \frac{1}{n} \right)^n}{n^{\frac{3}{2}} - \lfloor n^{\frac{3}{2}} \rfloor + n}
    \end{align*}
\end{tcolorbox}

\begin{align*}
    u_n &= \frac{e - \left( 1 + \frac{1}{n} \right)^n}{n^{\frac{3}{2}} - \lfloor n^{\frac{3}{2}} \rfloor + n} \\
    &\underset{n\to +\infty}{=} \frac{e - \exp \left( n \ln \left( 1 + \frac{1}{n} \right) \right)}{O(1) + n} \\
    &\underset{n\to +\infty}{=} \frac{e - \exp \left( 1 - \frac{1}{2n} + o \left( \frac{1}{n} \right) \right)}{n + o(n)} \\
    &\underset{n\to +\infty}{=} \frac{e - e \times \exp \left( -\frac{1}{2n} + o \left( \frac{1}{n} \right) \right)}{n + o(n)} \\
    &\underset{n\to +\infty}{=} \frac{e - e \left( 1 - \frac{1}{2} + o \left( \frac{1}{n} \right) \right)}{n + o(n)} 
\end{align*}
Par comparaison par $\sim$, $\sum u_n$ est convergent. 

\setsection{28}
\section{Nature des séries de Riemann}
\begin{tcolorbox}[title=Théorème 27.29, title filled=false, colframe=orange, colback=orange!10!white]
    Soit $\alpha\in \mathbb{R}$. La série de Riemann $\sum \frac{1}{n^{\alpha}}$ converge si et seulement si $\alpha > 1$. 
\end{tcolorbox}

\begin{itemize}
    \item Si $\alpha < 0$, la divergence est grossière. 
    \item On a montré que $\sum \frac{1}{n}$ diverge.
    \item Si $\alpha\in ]0, 1]$ : 
    \begin{align*}
        0 < \frac{1}{n} < \frac{1}{n^{\alpha}}
    \end{align*}
    Donc $\sum \frac{1}{n^{\alpha}}$ diverge d'après le théorème de comparaison des séries à termes positifs. 
    \item Soit $\alpha > 1$, $t\mapsto \frac{1}{t^{\alpha}}$ est décoroissante et positive sur $[1, +\infty[$. \\
    Pour $x\geq 1$ : 
    \begin{align*}
        \int_{1}^{x} \frac{1}{t^{\alpha}} \,dt &= \left[ \frac{1}{(-\alpha + 1)t^{\alpha - 1}} \right]_1^x \\
        &= \frac{1}{(1 - \alpha)x^{\alpha - 1}} - \frac{1}{(1 - \alpha)} \\
        &\underset{x\to +\infty}{\longrightarrow} \frac{1}{1 - \alpha}
    \end{align*}
    Par comparaison série intégrale, $\sum \frac{1}{n^{\alpha}}$ converge.
\end{itemize}

\section{Nature des séries exponentielles}
\begin{tcolorbox}[title=Théorème 27.30, title filled=false, colframe=orange, colback=orange!10!white]
    Pour tout $x\in \mathbb{R}$, la série exponentielle $\sum\limits_{n\geq 0} \frac{x^n}{n!}$ est absolument convergente et sa somme vaut $e^x$. 
\end{tcolorbox}

\begin{itemize}
    \item Pour tout $x\in \mathbb{R}$ : 
    \begin{align*}
        \frac{x^n}{n!} = o \left( \frac{1}{n^2} \right)
    \end{align*}
    Par comparaison par domination à une série de Riemann de paramètre $2 > 1$, $\sum\limits_{n\geq 0} \frac{x^n}{n!}$ est absolument convergente. 

    \item Soit $x\in \mathbb{R}, \exp \in \mathcal{C}^{\infty}(\mathbb{R}, \mathbb{R})$, on applique la formule de Taylor avec reste intégral : \\
    Pour tout $n\in \mathbb{N}$ : 
    \begin{align*}
        e^x = \sum_{k=0}^{n} \frac{x^k}{k!} + \int_{0}^{x} \frac{(x-t)^n}{n!} e^t \,dt
    \end{align*}
    On pose $M = \max (1, e^x)$. 
    \begin{align*}
        \left| \int_{0}^{x} \frac{(x - t)^n e^t}{n!} \,dt \right| &\leq \pm \int_{0}^{x} \frac{(x - t)^n}{n!} M \,dt \\
        &= M \frac{|x|^{n+1}}{(n+1)!} \\
        &\underset{n \to +\infty}{\longrightarrow} 0
    \end{align*}
\end{itemize}

\setsection{31}
\section{Nature des séries de Bertrand - Hors Programme}
\begin{tcolorbox}[title=Propostion 27.32 - HP, title filled=false, colframe=lightblue, colback=lightblue!10!white]
    La \textbf{série de Bertrand de paramètre $(\alpha, \beta)\in \mathbb{R}^2$} est définie par $\sum \frac{1}{n^{\alpha} \ln^{\beta} n}$. Elle est convergente si et seulement si $(\alpha, \beta) > (1, 1)$ pour l'ordre lexicographique. Cela signifie : 
    \begin{itemize}
        \item si $\alpha > 1$, la série converge
        \item si $\alpha < 1$, la série diverge
        \item pour $\alpha = 1$ : 
        \begin{itemize}
            \item si $\beta > 1$, la série converge
            \item si $\beta \leq 1$, la série diverge
        \end{itemize}
    \end{itemize}
\end{tcolorbox}

\begin{itemize}
    \item Si $\alpha > 1$, alors pour tout $\beta\in \mathbb{R}$ : 
    \begin{align*}
        \frac{1}{n^{\alpha} \ln^{\beta} n} = o \left( \frac{1}{n^{\frac{\alpha + 1}{2}}} \right)
    \end{align*}
    Comme $\frac{1 + \alpha}{2} > 1$, par comparaison en $0$, $\sum \frac{1}{n^{\alpha} \ln^{\beta} n}$ converge.

    \item Si $\alpha < 1$, alors pour tout $\beta\in \mathbb{R}$ : 
    \begin{align*}
        \frac{1}{n^{\frac{\alpha + 1}{2}}} = o \left( \frac{1}{n^{\alpha \ln^{\beta} n}} \right)
    \end{align*}
    Comme $\frac{\alpha + 1}{2} < 1$, par comparaison en $0$, $\sum \frac{1}{n^{\alpha} \ln^{\beta} n}$ diverge.

    \item Si $\alpha = 1$. Pour $\beta = 1$, $\sum \frac{1}{n \ln n}$ diverge (comparaison série intégrale). \\
    Pour $\beta < 1$ : 
    \begin{align*}
        \frac{1}{n\ln n} < \frac{1}{n \ln^{\beta} n}
    \end{align*}
    $\sum \frac{1}{n\ln n}$ diverge donc par comparaison des séries à termes positifs, $\sum \frac{1}{n \ln^{\beta} n}$ diverge. \\
    Pour $\beta > 1$, $t\mapsto \frac{1}{t\ln^{\beta} t}$ est positive et décroissante sur $[2, +\infty[$. 
    \begin{align*}
        \int_{2}^{x} \frac{dt}{t\ln^{\beta} t} = \int_{2}^{x} \frac{1}{t} \times (\ln t)^{-\beta} \,dt = \left[ \frac{(\ln t)^{1 - \beta}}{1 - \beta} \right]_2^{x} \underset{x \to +\infty}{\longrightarrow} \frac{\ln(2)^{1 - \beta}}{\beta - 1}
    \end{align*}
    Par comparaison série intégrale, $\sum \frac{1}{n \ln^{\beta} n}$ converge. 
\end{itemize}


\end{document}