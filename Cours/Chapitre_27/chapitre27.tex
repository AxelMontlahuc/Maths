\documentclass[../main.tex]{subfiles}

\begin{document}
\setcounter{chapter}{26}
\chapter{Séries numériques}
\tableofcontents
\clearpage

\setsection{5}
\section{Série géométrique}
\begin{tcolorbox}[title=Théorème 27.6, title filled=false, colframe=orange, colback=orange!10!white]
    Soit $a\in \mathbb{C}$. La série $\sum a^n$ converge si et seulement si $|a|<1$. Dans ce cas : 
    \begin{align*}
        \sum_{n=0}^{+\infty} a^n = \frac{1}{1-a}
    \end{align*}
\end{tcolorbox}

\noindent Soit $n\in \mathbb{N}$. 
\begin{align*}
    S_n = \sum_{k=0}^{n} a^k &= \frac{1-a^{n+1}}{1-a} \text{ ($a\neq 1$)} \\
    &\underset{n \to +\infty}{\longrightarrow} \frac{1}{1-a} \text{ ($|a|<1$)}
\end{align*}
La série converge et $\sum\limits_{n\geq 0} a^n = \frac{1}{1-a}$. 

\setsection{10}
\section{Deux séries de termes généraux égaux presque partout}
\begin{tcolorbox}[title=Propostion 27.11, title filled=false, colframe=lightblue, colback=lightblue!10!white]
    Si $(u_n)$ et $(v_n)$ ne diffèrent que d'un nombre fini de termes, alors $\sum u_n$ et $\sum v_n$ sont de même nature. 
\end{tcolorbox}

\noindent On note $A = \{ n\in \mathbb{N}, u_n \neq v_n \}$. Supposons $A\neq \emptyset$. \\
D'après les hypothèses, $A$ est majoré donc possède un maximum $N$ d'après la propriété fondamentale de $\mathbb{N}$. \\
On note $(S_n)$ et $(S_n')$ les sommes partielles associée à $\sum u_n$ et $\sum v_n$. \\
Pour $n \geq N$ : 
\begin{align*}
    S_n = S_n' + K \text{ où } K = \sum_{k\in A} (u_k - v_k) \text{ (constant)}
\end{align*}
Ainsi $(S_n)$ converge si et seulement si $(S_n')$ converge. 

\section{CN de convergence portant sur le terme général}
\begin{tcolorbox}[title=Théorème 27.12, title filled=false, colframe=orange, colback=orange!10!white]
    Si $\sum u_n$ converge, alors $(u_n)$ converge vers $0$. De manière équivalente, si $(u_n)$ ne tend pas vers $0$, la série $\sum u_n$ diverge.
\end{tcolorbox}

\noindent On suppose que $S_n \underset{n \to +\infty}{\longrightarrow} \ell \in \mathbb{R} \text{ ou } \mathbb{C}$. \\
\begin{align*}
    u_n = S_n - S_{n-1} &= \ell - \ell = 0
\end{align*}

\setsection{15}
\section{Théorème de comparaison des séries à termes positifs}
\begin{tcolorbox}[title=Théorème 27.16, title filled=false, colframe=orange, colback=orange!10!white]
    Soit $\sum u_n$ et $\sum v_n$ deux séries à termes positifs telles qu'il existe $N\in \mathbb{N}$ tel que pour tout $n\geq N$ : 
    \begin{align*}
        0\leq u_n \leq v_n
    \end{align*}
    Alors : 
    \begin{itemize}
        \item Si $\sum v_n$ converge, alors $\sum u_n$ converge aussi. 
        \item Si $\sum u_n$ diverge (vers $+\infty$ donc), alors $\sum v_n$ diverge aussi (vers $+\infty$ donc).
    \end{itemize}
    De plus, si la divergence est grossière pour $\sum u_n$, elle l'est aussi pour $\sum v_n$. 
\end{tcolorbox}

\noindent En utilisant les notations du (27.11), on peut supposer que : 
\begin{align*}
    \forall n \geq 0, 0\leq u_n \leq v_n
\end{align*}
Puis : 
\begin{align*}
    \forall n \geq 0, 0\leq S_n \leq S_n'
\end{align*}
On utilise alors le théroème de comparaison sur les suites. 

\setsection{19}
\section{Convergence absolue entraîne convergence}
\begin{tcolorbox}[title=Théorème 27.20, title filled=false, colframe=orange, colback=orange!10!white]
    Toute série réelle ou complexe absolument convergente est convergente. 
\end{tcolorbox}

\begin{itemize}
    \item On suppose que $(u_n)\in \mathbb{R}^{\mathbb{N}}$, avec $\sum |u_n|$ convergente. \\
    On pose, pour tout $n\in \mathbb{N}$ :
    \begin{align*}
        u_n^+ = \max(u_n, 0) \geq 0 \text{ et } u_n^- = \max(-u_n, 0) \geq 0
    \end{align*}
    Ainsi, $u_n = u_n^+ - u_n^-$. \\
    Or, pour tout $n$ : 
    \begin{align*}
        0 &\leq u_n^+ \leq |u_n| \\
        0 &\leq u_n^- \leq |u_n|
    \end{align*}
    Par comparaison des séries à termes positifs, $\sum u_n^+$ et $\sum u_n^-$ convergent et par linéarité (27.16) $\sum u_n$ converge. 

    \item On suppose que $(u_n)\in \mathbb{C}^{\mathbb{N}}$, avec $\sum |u_n|$ convergente. \\
    Alors : 
    \begin{align*}
        \forall n\in \mathbb{N}, |Re(u_n)| &\leq |u_n| \\
        |Im(u_n)| &\leq |u_n|
    \end{align*}
    Donc, $\sum Re(u_n)$ et $\sum Im(u_n)$ sont absolument convergentes (27.15) donc convergent, puis par combinaison linéaire (27.16) $\sum u_n$ converge. 
\end{itemize}


\end{document}