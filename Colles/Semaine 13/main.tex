\documentclass[titlepage, twoside]{report}

\usepackage[utf8]{inputenc}
\usepackage[T1]{fontenc}
\usepackage{geometry}
\usepackage{subfiles}
\usepackage[french]{babel}
\usepackage{amsmath}
\usepackage{amssymb}
\usepackage{stmaryrd}
\usepackage{dsfont}
\usepackage{tikz}
\usepackage{cancel}
\usepackage{fancyhdr}
\usepackage{tcolorbox}
\usepackage[dvipsnames]{xcolor}

\geometry{a4paper, left=20mm, right=20mm, top=20mm, bottom=20mm}

\setcounter{tocdepth}{3}

\definecolor{darkgreen}{RGB}{6, 64, 43}
\definecolor{lightblue}{RGB}{87, 185, 255}

\pagestyle{fancy}
\fancyhead
\fancyfoot

\fancyhead[L]{Programme de colle}
\fancyhead[R]{Axel Montlahuc}
\fancyfoot[C]{\thepage}

\makeatletter
\renewcommand{\tableofcontents}{%
  \@starttoc{toc}
}
\makeatother
\renewcommand{\thesection}{\Roman{section}}

\begin{document}
\chapter*{Programme de colle : semaine 13}
\tableofcontents

\section{Suites numériques}
\subsection{Questions de cours}
\subsubsection{Enoncer et démontrer la caractérisation séquentielle de la borne supérieure}
\begin{tcolorbox}[title=Théorème 14.41, title filled=false, colframe=orange, colback=orange!10!white]
    Soit $A$ une partie non vide de $\mathbb{R}$ et soit $M \in \mathbb{R}$. Alors $M$ est la borne supérieure (resp. inférieure) de $A$ si et seulement si $M$ majore (resp. minore) $A$ et s'il existe une suite d'éléments de $A$ qui converge vers $M$. 
\end{tcolorbox}

\boxed{\Rightarrow} \\
On suppose que $M = \sup A$. Donc $M$ majore $A$. \\
On rappelle que : 
\begin{align*}
    \forall \epsilon > 0, \exists a \in A, M - \epsilon < a
\end{align*}
Donc : 
\begin{align*}
    \forall n \in N, \exists a \in A, M - \frac{1}{n+1} < a_n \leq M \text{ ($M$ est un majorant)}
\end{align*}
D'après la suite $(a_n) \in A^{\mathbb{N}}$ étant ainsi définie, d'après le théorème d'encadrement : 
\begin{align*}
    \boxed{a_n \underset{n \to +\infty}{\longrightarrow} M}
\end{align*} \\

\boxed{\Leftarrow} \\
On choisit $(a_n) \in A^{\mathbb{N}}$ telle que : 
\begin{align*}
    a_n \underset{n \to +\infty}{\longrightarrow} M \text{ (majorant de $A$)}
\end{align*}
Soit $\epsilon > 0$. On choisit $a_n \in A$ tel que : 
\begin{align*}
    a_n \in ]M-\epsilon, M+\epsilon[
\end{align*}
Donc $M - \epsilon$ ne majore pas $A$. \\
Donc : 
\begin{align*}
    \boxed{M = \sup A}
\end{align*}

\subsubsection{Enoncer et démontrer le théorème de la limite monotone}
\begin{tcolorbox}[title=Théorème 14.50, title filled=false, colframe=orange, colback=orange!10!white]
    Si $u$ est une suite croissante et majorée (resp. décroissante et minorée), alors $u$ converge vers $\sup_{n\in \mathbb{N}}(u_n)$ (resp. vers $\inf_{n\in \mathbb{N}}(u_n)$). \\
    Si $u$ est une suite croissante et non majorée (resp. décroissante et non minorée) alors $u$ tend vers $+\infty$ (resp. vers $-\infty$). 
\end{tcolorbox}

\begin{itemize}
    \item On suppose $u$ croissante et majorée. \\
    L'ensemble $A = \{ u_n | n\in \mathbb{N} \}$ est non vide et majoré. Cet ensemble possède une borne supérieure notée $l$ (propriété fondamentale de $\mathbb{R}$). \\
    Soit $\epsilon > $. Comme $l - \epsilon < u_n$ ne majore pas $A$, on choisit $N \in \mathbb{N}$ tel que $l - \epsilon < u_n$. \\
    Or $(u_n)$ est croissante donc : 
    \begin{align*}
        \forall n \geq N, l - \epsilon < u_N \leq u_n \leq l
    \end{align*}
    Donc : 
    \begin{align*}
        \forall n \geq N, u_n \in ]l-\epsilon, l+\epsilon[
    \end{align*}
    Soit : 
    \begin{align*}
        \boxed{u_n \underset{n \to +\infty}{\longrightarrow} l}
    \end{align*}

    \item On suppose $u$ croissante et non majorée. \\
    Soit $A \in \mathbb{R}_+$. Soit $N \in \mathbb{N}$ tel que : 
    \begin{align*}
        u_N \geq A \text{ ($u$ non majorée)}
    \end{align*}
    Donc : 
    \begin{align*}
        \forall n \geq N, A \leq u_N \leq u_n \text{ ($u$ croissante)}
    \end{align*}
    Soit : 
    \begin{align*}
        \boxed{u_n \underset{n \to +\infty}{\longrightarrow} +\infty}
    \end{align*}
\end{itemize}

\subsubsection{Démontrer que deux suites adjacentes convergent vers la même limite}
\begin{tcolorbox}[title=Théorème 14.55, title filled=false, colframe=orange, colback=orange!10!white]
    Deux suites adjacentes convergent vers une limite commune. 
\end{tcolorbox}

Soit $u$ et $v$ deux suites adjacentes avec $u$ croissante et $v$ décroissante. \\
Soit $w = v - u$. Par opération, $w$ est décroissante. \\
Par hypothèse : 
\begin{align*}
    w_n \underset{n \to +\infty}{\longrightarrow} 0
\end{align*}
Donc $w \leq 0$, soit $u \leq v$. \\
La suite $u$ est donc majorée par $v_0$, et croissante donc convergente d'après le théorème de la limite monotone. \\
Pour les mêmes raisons, $v$ converge. \\
Or, par théorème d'opérations : 
\begin{align*}
    \lim_{n\to +\infty} v_n - \lim_{n\to +\infty} u_n = \lim_{n\to +\infty} (v_n - u_n) = 0
\end{align*}


\subsection{Exercices types}
\begin{tcolorbox}[title=Exercice 1, title filled=false, colframe=darkgreen, colback=darkgreen!10!white]
    Déterminer l'expression explicite de la suite de Fibonnaci, définie par
    \begin{align*}
        \begin{cases}
            \phi_0 = 0 \text{ et } \phi_1 = 1 \\
            \forall n \in \mathbb{N}, \phi_{n+2} = \phi_{n+1} + \phi_n
        \end{cases}
    \end{align*}
\end{tcolorbox}

\begin{tcolorbox}[title=Exercice 2, title filled=false, colframe=darkgreen, colback=darkgreen!10!white]
    On pose pour tout $n \in \mathbb{N}, u_n = \sqrt{n} - \lfloor \sqrt{n} \rfloor$.
    \begin{enumerate}
        \item Etudier $\lim\limits_{n\to + \infty} u_{n^2 + n}$. En déduire que la suite $(u_n)$ n'a pas de limite. 
        \item Soit $a \in \mathbb{N}$ et $b \in \mathbb{N}^*$ avec $a \leq b$. Etudier $\lim\limits_{n\to +\infty} u_{n^2b^2 + 2an}$. 
        \item Montrer que tout élément de $[0, 1]$ est la limite d'une certaine suite extraite de $(u_n)$. 
    \end{enumerate}
\end{tcolorbox}

\begin{tcolorbox}[title=Exercice 3, title filled=false, colframe=darkgreen, colback=darkgreen!10!white]
    \begin{enumerate}
        \item Montrer que $[3, +\infty[$ est stable par $x\mapsto \frac{2x^2 - 3}{x + 2}$. On note alors $(x_n)$ la suite définie par $x_0 = 5$ et pour tout $n \in \mathbb{N}, x_{n+1} = \frac{2x_n^2 - 3}{x_n + 2}$. 
        \item \begin{enumerate}
            \item Etudier la monotonie de $(x_n)$.
            \item En déduire $\lim\limits_{n\to +\infty} x_n$.
        \end{enumerate}
    \end{enumerate}
\end{tcolorbox}


\section{Limites et continuité}
\subsection{Questions de cours}
\subsubsection{Enoncer et démontrer le théorème de caractérisation séquentielle de la limite d'une fonction}
\begin{tcolorbox}[title=Théorème 15.34, title filled=false, colframe=orange, colback=orange!10!white]
    Soit $f:X\to \mathbb{R}$ une fonction et $a \in \overline{X}$ et $\ell \in \overline{\mathbb{R}}$. Sont équivalentes : 
    \begin{enumerate}
        \item $\lim\limits_a f = \ell \Leftrightarrow \forall u_n \to a, \lim f(u_n) = \ell \text{ ($= f(\lim u_n)$)}$
        \item Pour toute suite $(u_n)$ de limite $a$ à valeurs dans $X$, la suite $(f(u_n))$ a pour limite $\ell$. 
    \end{enumerate}
\end{tcolorbox}

$\boxed{1 \Rightarrow 2}$ \\
On suppose que $\lim\limits_a f = \ell$. \\
Soit $(u_n) \in X^{\mathbb{N}}$ avec $u_n \underset{n \to +\infty}{\longrightarrow} a$. \\
Soit $V \in \mathcal{V}(\ell)$. On choisit $U \in \mathcal{V}(a)$ tel que : 
\begin{align*}
    f(U \cap X) \subset V \text{ ($\lim\limits_a f = \ell$)}
\end{align*}
Comme $u_n \underset{n \to +\infty}{\longrightarrow} a$, on choisit $N \in \mathbb{N}$ tel que : 
\begin{align*}
    \forall n \geq N, u_n \in U \cap X
\end{align*}
Donc : 
\begin{align*}
    \forall n \geq N, f(u_n) \in V
\end{align*}
Donc : 
\begin{align*}
    f(u_n) \underset{n \to +\infty}{\longrightarrow} \ell
\end{align*} \\

$\boxed{1 \Leftarrow 2}$ \\
Par contraposée. On suppose que $f$ n'admet pas $\ell$ comme limite en $a$. Pour tout $n \in \mathbb{N}$, on note : 
\begin{align*}
    V_n = \begin{cases}
        ]a - \frac{1}{n+1}, a + \frac{1}{n+1}[ \text{ si $a \in \mathbb{R}$} \\
        [n, +\infty[ \text{ si $a = +\infty$} \\
        ]-\infty, -n] \text{ si $a = -\infty$}
    \end{cases}
\end{align*}
Par définition, il existe $W \in \mathcal{V}(\ell)$ tel que pour tout $V \in \mathcal{V}(a)$, il existe $x \in V \cap X$ et $f(x) \neq W$. \\
Pour tout $n \in \mathbb{N}$, on choisit $x_n \in V_n \cap X$ tel que $f(x_n) \neq W$. \\
Par construction : 
\begin{align*}
    (x_n) \in X^\mathbb{N}, x_n \underset{n \to +\infty}{\longrightarrow} a \text{ et } f(x_n) \cancel{\underset{n \to +\infty}{\longrightarrow}} \ell
\end{align*}

\subsubsection{Enoncer et démontrer le théorème de Heine}
\begin{tcolorbox}[title=Théorème 15.65, title filled=false, colframe=orange, colback=orange!10!white]
    Une fonction continue sur un segment est uniformément continue sur ce segment. 
\end{tcolorbox}

\noindent \underline{Rappel :}
\begin{align*}
    C^0(I) &: \forall x \in I, \forall \epsilon > 0, \exists \eta > 0, \forall y \in I, |x-y| < \eta \Rightarrow |f(x) - f(y)| < \epsilon \\
    Cu(I) &: \forall \epsilon > 0, \exists \eta > 0, \forall (x,y) \in I^2, |x-y| < \eta \Rightarrow |f(x) - f(y)| < \epsilon
\end{align*}
On raisonne par l'absurde. Soit $f$ continue sur $[a,b]$ mais non uniformément continue sur $[a,b]$. \\
On choisit $\epsilon$ tel que : 
\begin{align*}
    \forall \eta > 0, \exists (x, y) \in [a,b]^2, |x - y| < \eta \text{ et } |f(x) - f(y)| \geq \epsilon
\end{align*}
Ainsi, pour tout $b \in \mathbb{N}^*$, on choisit un couple $(x_n, y_n) \in [a,b]^2$ tel que : 
\begin{align*}
    |x_n - y_n| < \frac{1}{n} \text{ et } \underbrace{|f(x_n) - f(y_n)|}_{(*)} \geq \epsilon
\end{align*}
En particulier $(x_n)$ est bornée donc d'après le théorème de Bolzano-Weierstrass, on en extrait $(x_{\varphi(n)})$ suite convergente vers $\ell$. \\
D'après le TCILPPL, $\ell \in [a,b]$. \\
Comme : 
\begin{align*}
    \forall n \in \mathbb{N}, |x_{\varphi(n)} - y_{\varphi(n)}| < \frac{1}{\varphi(n)} \underset{n \to +\infty}{\longrightarrow} 0
\end{align*}
Alors : 
\begin{align*}
    y_{\varphi(n)} \underset{n \to +\infty}{\longrightarrow} \ell
\end{align*}
Par continuité : 
\begin{align*}
    f(x_{\varphi(n)}) \underset{n \to +\infty}{\longrightarrow} f(\ell) \text{ et } f(y_{\varphi(n)}) \underset{n \to +\infty}{\longrightarrow} f(\ell)
\end{align*}
Donc par opération : 
\begin{align*}
    |f(x_{\varphi(n)}) - f(y_{\varphi(n)})| \underset{n \to +\infty}{\longrightarrow} 0
\end{align*}
Absurde d'après $(*)$. 

\subsubsection{Démontrer que l'image continue d'un compact est compact. Démontrer qu'une fonction continue sur un intervalle est injective si et seulement si elle est strictement monotone}
\begin{tcolorbox}[title=Lemme 15.68, title filled=false, colframe=orange, colback=orange!10!white]
    L'image continue d'un compact est compact. 
\end{tcolorbox}

\noindent Soit $I$ un segment, donc un intervalle. \\
Comme $f$ est continue sur $I$, $f(I)$ est un intervalle (TVI v3). \\
Montrons que $f(I)$ est compact. \\
Soit $(y_n) \in f(I)^{\mathbb{N}}$. Pour tout $n \in \mathbb{N}$, soit $x_n \in I$ tel que : 
\begin{align*}
    y_n = f(x_n)
\end{align*}
Or $I$ est compact (15.67), on choisit : 
\begin{align*}
    x_{\varphi(n)} \underset{n \to +\infty}{\longrightarrow} \ell \in I
\end{align*}
$y_{\varphi(n)} \underset{n \to +\infty}{\longrightarrow} f(\ell)$ car $f$ est continue sur $I$. 

\begin{tcolorbox}[title=Théorème 15.72, title filled=false, colframe=orange, colback=orange!10!white]
    Soit $I$ un intervalle et $f$ une fonction continue sur $I$. Alors $f$ est injective si et seulement si $f$ est strictement monotone. 
\end{tcolorbox}

$\boxed{\Leftarrow}$ \\
RAS \\ \\

$\boxed{\Rightarrow}$ \\
Supposons $f$ non strictement monotone. \\
On peut supposer qu'il existe alors : 
\begin{align*}
    x < y < z
\end{align*}
tels que $f(x) < f(y)$ et $f(z) < f(y)$. \\
Soit :
\begin{align*}
    \lambda = \frac{f(y) + \max(f(y), f(z))}{2} &\in ]f(x), f(y)[ \\
    &\in ]f(z), f(y)[
\end{align*}
Par continuité de $f$ sur les intervalles $]x, y[$ et $]y, z[$, il existe $\alpha \in ]x, y[$ et $\beta \in ]y, z[$ tels que : 
\begin{align*}
    f(\alpha) = \lambda = f(\beta)
\end{align*}
Donc $f$ n'est pas injective. 


\subsection{Exercices types}

\end{document}