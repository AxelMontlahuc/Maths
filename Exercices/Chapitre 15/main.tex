\documentclass{report}

\usepackage{amsmath}
\usepackage{amssymb}
\usepackage{stmaryrd}
\usepackage{geometry}
\usepackage{tkz-tab}
\usepackage{tcolorbox}

\geometry{a4paper, left=20mm, right=20mm, top=20mm, bottom=20mm}

\begin{document}
\setcounter{chapter}{15}
\setcounter{section}{1}
\section{Exercice}
\begin{align*}
    0 \leq x \left\lfloor \frac{1}{x} \right\rfloor \leq x \times \frac{1}{x} \\
    \text{donc } 0 \leq x \left\lfloor \frac{1}{x} \right\rfloor \leq 1 \\
\end{align*}
\begin{enumerate}
    \item Quand $x \in ]-1, 0[ \cup ]0, 1[$, on remarque que $\lim\limits_x \left\lfloor \frac{1}{x} \right\rfloor \geq \lim\limits_x \frac{1}{x}$. \\
    Ainsi, par théorème d'opérations on a :
    \begin{align*}
        \lim_{x\to 0} x \times \frac{1}{x} \leq \lim_{x\to 0} x \times \left\lfloor \frac{1}{x} \right\rfloor \leq \lim_{x\to 0} x \times \frac{1}{x}
    \end{align*}
    Donc d'après le théorème d'encadrement : 
    \begin{align*}
        \boxed{\lim_{x\to 0} x \left\lfloor \frac{1}{x} \right\rfloor = 1}
    \end{align*}

    \item Quand $x > 1$, $\left\lfloor \frac{1}{x} \right\rfloor = 0$. \\
    Donc : 
    \begin{align*}
        \forall x > 1, x \left\lfloor \frac{1}{x} \right\rfloor = 0
    \end{align*}
    Donc : 
    \begin{align*}
        \boxed{\lim_{x\to +\infty} x \left\lfloor \frac{1}{x} \right\rfloor = 0} 
    \end{align*}
\end{enumerate}

\setcounter{section}{5}
\section{Exercice}
\begin{enumerate}
    \item Soit $x > 0$, d'après le TLM : 
    \begin{align*}
        \lim_{x^-} f &\geq f(x) \geq \lim_{x^+} f \\
        \text{ et } \lim_{t\to x^-} tf(t) &\leq xf(x) \leq \lim_{t\to x^+} tf(t) \\
        \text{donc } x \lim_{x^-} f &\leq xf(x) \leq x \lim_{x^+} f \text{ (théorème d'opérations + TLM)} \\
        \text{ et } \lim_{x^-} f &\geq f(x) \geq \lim_{x^+} f \text{ ($x > 0$)} \\
        \text{donc } &\boxed{\lim_{x^-} f = f(x) = \lim_{x^+} f}
    \end{align*}

    \item Soit $x \leq y$. \\
    On a : 
    \begin{align*}
        f(x) - f(y) &= \int_{0}^{\frac{\pi}{2}} \frac{\overbrace{\sqrt{\cos^2 t + y^2 \sin^2 t} - \sqrt{\cos^2 t + x^2 \sin^2 t}}^{\geq 0 \text{ car } u\mapsto \sqrt{\cos^2 t + u^2 \sin^2 t} \text{ est croissante sur } \mathbb{R}_+}}{\underbrace{\sqrt{\cos^2 t + x^2 \sin^2 t} \sqrt{\cos^2 t + y^2 \sin^2 t}}_{> 0}} \,dt
    \end{align*}
    Donc $f$ est décroissante. \\
    Par ailleurs : 
    \begin{align*}
        xf(x) - yf(y) &= \int_{0}^{\frac{\pi}{2}} \frac{x\sqrt{\cos^2 t + y^2 \sin^2 t} - y\sqrt{\cos^2 t + x^2 \sin^2 t}}{\Delta} \,dt \\
        &= \int_{0}^{\frac{\pi}{2}} \frac{\overbrace{\cos^2 (t) (x^2 - y^2)}^{\leq 0}}{\underbrace{\Delta[x\sqrt{\ldots} + y\sqrt{\ldots}}_{> 0}]} \,dt \\
        &\leq 0
    \end{align*}
\end{enumerate}

\setcounter{section}{7}
\section{Exercice}
\begin{enumerate}
    \item \begin{enumerate}
        \item On a :
        \begin{align*}
            \left( \varphi_{n+1}(x) + \frac{1}{n+1} \right) - \left( \varphi_n (x) + \frac{1}{n} \right) &= \sum_{k=0}^{n+1} \frac{1}{(k+1)(k+x)} + \frac{1}{n+1} - \sum_{k=0}^{n} \frac{1}{(k+1)(k+x)} - \frac{1}{n} \\
            &= \frac{1}{(n+2)(n+1+x)} + \frac{1}{(n+1)n} \\
            &= \frac{n(n+1) - (n+2)(n+1+x)}{\underbrace{n(n+1)(n+2)(n+1+x)}_{\geq 0}}
        \end{align*}
        En outre : 
        \begin{align*}
            n \leq n+1 \text{ et } n+1 \leq n+1+x
        \end{align*}
        Ainsi : 
        \begin{align*}
            n(n+1) \leq (n+2)(n+1+x) &\Leftrightarrow n(n+1) - (n+2)(n+1+x) \\
            &\Leftrightarrow \left( \varphi_{n+1}(x) + \frac{1}{n+1} \right) - \left( \varphi_n (x) + \frac{1}{n} \right) \leq 0 \\
            &\Leftrightarrow \text{La suite est décroissante. }
        \end{align*}

        \item 
    \end{enumerate}
\end{enumerate}

\setcounter{section}{9}
\section{Exercice}
Soit $x \in \mathbb{R}$. 
\begin{align*}
    f(x)^2 = 1 &\Leftrightarrow f(x) = 1 \text{ ou } f(x) = -1 \\
    &\Leftrightarrow f:x\mapsto 1 \text{ ou } f:x\mapsto -1 \text{ (car $f\in \mathcal{C}$)}
\end{align*}


\end{document}