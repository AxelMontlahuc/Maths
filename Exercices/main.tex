\documentclass{report}

\usepackage{amsmath}
\usepackage{amssymb}
\usepackage{stmaryrd}
\usepackage{geometry}

\geometry{a4paper, left=20mm, right=20mm, top=20mm, bottom=20mm}

\begin{document}

\setcounter{chapter}{10}
\setcounter{section}{14}
\section{Exercice}
\begin{align*}
    x \mapsto x^{-1} \text{ est un morphisme de } G &\Leftrightarrow \forall (x,y) \in G^2, (xy)^{-1} = x^{-1}y^{-1} \\
\end{align*}
Or : 
\begin{align*}
    (xy)^{-1} &= y^{-1}x^{-1} \\
    \Leftrightarrow x^{-1}y^{-1} &= y^{-1}x^{-1}
\end{align*}
Donc $G$ est commutatif.

\setcounter{chapter}{11}
\setcounter{section}{0}
\section{Exercice}
\begin{align*}
    AB \text{ est symétrique } &\Leftrightarrow AB = ^t(AB) \\
    &\Leftrightarrow AB = ^tB \times ^tA \\
    &\Leftrightarrow AB = BA
\end{align*}

\setcounter{section}{3}
\section{Exercice}
\underline{Analyse :} \\
On suppose que :
\begin{align*}
    X + tr(X)A &= B \\
    \text{Donc } X &= B - tr(X)A \\
    \text{Donc } tr(X) &= tr(B) - tr(X)tr(A) \\
    \text{Donc } tr(X) &=
    \begin{cases}
        \frac{tr(B)}{1+tr(A)} \text{ si } tr(A) \neq -1\\
        tr(X) + tr(B) \text{ si } tr(A) = -1
    \end{cases} \\
    \text{Donc } X &= B - \frac{tr(B)}{1+tr(A)}A
\end{align*} 
\underline{Synthèse :} \\
On pose $X = B - \frac{tr(B)}{1+tr(A)}A$ \\

\setcounter{section}{14}
\section{Exercice}
\begin{align*}
    MX = 0 &\Leftrightarrow \exists (x,y,z) \in \mathbb{R}^3,
    \begin{pmatrix}
        1 & -2 & 1 \\
        0 & 1 & 3 \\
        -1 & 4 & 7
    \end{pmatrix}
    \begin{pmatrix}
        x \\
        y \\
        z
    \end{pmatrix}
    = 0 \\
    &\Leftrightarrow 
    \begin{cases}
        x - 2y + z = 0 \\
        y + 3z = 0 \\
        -x + 4y + 7z = 0
    \end{cases} \\
    &\Leftrightarrow 
    \begin{cases}
        x - 2y + z = 0 \\
        y + 3z = 0 \\
        2y + 8z = 0
    \end{cases} \\
    &\Leftrightarrow
    \begin{cases}
        x - 2y + z = 0 \\
        y + 3z = 0 \\
        2z = 0
    \end{cases} \\
    &\Leftrightarrow
    \begin{cases}
        x = 0 \\
        y = 0 \\
        z = 0
    \end{cases}
\end{align*}

\setcounter{chapter}{12}
\setcounter{section}{0}
\section{Exercice}
On remarque que $2^{10} \equiv 1 \pmod{11}$ et que $3^5 \equiv 1 \pmod{11}$. \\
Donc $2^{123} \equiv 2^3 \equiv 8 \pmod{11}$ et $3^{121} \equiv 3^1 \equiv 3 \pmod{11}$. \\ 
Donc $\boxed{2^{123} + 3^{121} \equiv 0 \pmod{11}}$. 

\section{Exercice}
On raisonne par disjonction de cas : \\
\begin{tabular}{|c|c|c|c|}
    \hline
    $n \pmod{6}$ & $n+2 \pmod{6}$ & $7n-5 \pmod{6}$ & $n(n+2)(7n-5) \pmod{6}$ \\
    \hline
    0 & 2 & 1 & 0 \\
    \hline
    1 & 3 & 2 & 0 \\
    \hline
    2 & 4 & 3 & 0 \\
    \hline
    3 & 5 & 4 & 0 \\
    \hline
    4 & 0 & 5 & 0 \\
    \hline
    5 & 1 & 0 & 0 \\
    \hline
\end{tabular}
\\
\noindent Donc $\boxed{\forall n \in \mathbb{Z}, n(n+2)(7n-5) \equiv 0 \pmod{6}}$.

\section{Exercice}
\begin{enumerate}
    \item On cherche une puissance cyclique de $3 \pmod{25}$. \\
    \begin{align*}
        3 &\equiv 3 \pmod{25} \\
        3^2 &\equiv 9 \pmod{25} \\
        3^3 &\equiv 2 \pmod{25} \\
        3^4 &\equiv 6 \pmod{25} \\
        3^5 &\equiv 18 \pmod{25} \\
        3^6 &\equiv 4 \pmod{25} \\
        3^7 &\equiv 12 \pmod{25} \\
        3^8 &\equiv 11 \pmod{25} \\
        3^9 &\equiv 8 \pmod{25} \\
        3^{10} &\equiv 24 \equiv -1 \pmod{25} \\
    \end{align*}
    Donc $\boxed{3^{2189} \equiv 3^{2180} \times 3^{9} \equiv (3^{10})^{218} \times 3^9 \equiv (-1)^{218} \times 8 \equiv 8 \pmod{25}}$.

    \item On cherche une puissance cyclique de $55 \pmod{8}$. \\
    \begin{align*}
        55 &\equiv 7 \pmod{8} \\
        55^2 &\equiv 1 \pmod{8} \\
    \end{align*}
    Donc $\boxed{55^{970321} \equiv 55^1 \equiv 7 \pmod{8}}$.

    \item On cherche une puissance cyclique de $1234^{4312} \pmod{7}$ et de $4321^{1234} \pmod{7}$. \\
    \begin{align*}
        1234^1 \equiv 2 \pmod{7} &\text{ et } 4321^1 \equiv 2 \pmod{7} \\
        1234^2 \equiv 4 \pmod{7} &\text{ et } 4321^2 \equiv 4 \pmod{7} \\
        1234^3 \equiv 1 \pmod{7} &\text{ et } 4321^3 \equiv 1 \pmod{7}
    \end{align*}
    Donc $1234^{4312} \equiv 1234^{3 \times 1437 + 1} \equiv 2 \pmod{7}$ et $4321^{1234} \equiv 4321^{3 \times 411 + 1} \equiv 2 \pmod{7}$. \\
    Donc $\boxed{1234^{4321} + 4321^{1234} \equiv 4 \pmod{7}}$.
\end{enumerate}

\section{Exercice}
\begin{enumerate}
    \item Soit $n \in \mathbb{N}$. \\
    On remarque que $4 | 100$, donc $\forall k \geq 2 \in \mathbb{N}, 4 | a_k$. \\
    Et :
    \begin{align*}
        4|n &\Leftrightarrow k | \sum\limits_{k=0}^{r} a_k \times 10^k \\
        &\Leftrightarrow k|\sum\limits_{k=2}^{r} (a_k \times 10^k) + a_1 \times 10 + a_0 \\
    \end{align*}
    Or comme on sait que $k|\sum\limits_{k=2}^{r} (a_k \times 10^k)$, nécessairement,  $k|(a_0 + a_1 \times 10)$. 
\end{enumerate}

\setcounter{section}{7}
\section{Exercice}
\begin{enumerate}
    \item Soit $a \geq 2$ et $n \geq 2$. 
    $$a^n - 1 = (a - 1) \sum_{k = 0}^{n-1} a^k$$
    On a donc :
    \begin{align*}
        a^n - 1 \in \mathbb{P} &\text{ donc } (a - 1) \sum_{k = 0}^{n-1} a^k \in \mathbb{P} \\
        &\text{ donc } a - 1 = 1 \text{ et } \sum_{k = 0}^{n-1} = a^n - 1 \text{ ou } a - 1 = a^n - 1 \text{ et } \sum_{k = 0}^{n-1} a^k = 1 \\
        &\text{ donc } a - 1 = 1 \text{ et } \sum_{k=0}^{n-1} a^n - 1 \\
        &\text{ donc } a = 2 
    \end{align*} 

    \item On raisonne par l'absurde : supposons que $n \not \in \mathbb{P}$. Alors $n = ab$ avec $a > 1$ et $b > 1$. 
     Donc on a : 
     \begin{align*}
         2^n - 1 \in \mathbb{P} &\Leftrightarrow 2^{ab} - 1 \in \mathbb{P} \\
         &\Leftrightarrow (2^a - 1)(\sum_{k=0}^{b-1} 2^{ak}) \in \mathbb{P}
     \end{align*}
     Absurde. \\
     Donc, nécessairement, $n$ est premier. 
\end{enumerate}

\setcounter{section}{9}
\section{Exercice}
\begin{enumerate}
    \item \begin{enumerate}
        \item Soit $k \in \mathbb{N}$ tel que $k \equiv 3 \pmod{4}$. $k$ est donc impair. \\
        On note $(p_i)_{i \in \mathbb{N}}$ les diviseurs premiers de $k$. \\
        Nécessairement, pour tout $i \in \mathbb{N}$ on a $p_i \equiv 1 \pmod{4}$ ou $p_i \equiv 3 \pmod{4}$ ($k$ est impair, donc $p_i$ est impair). \\
        On raisonne par l'absurde : supposons que $\forall i \in \mathbb{N}, p_i \equiv 1 \pmod{4}$. \\ 
        Alors $\prod\limits_{i = 0}^{n} p_i^{\alpha_i} \equiv 1 \pmod{4} \Leftrightarrow k \equiv 1 \pmod{4}$. \\
        Absurde. \\
        Donc tout entier naturel congru à $3$ modulo $4$ possède au moins un diviseur premier congru à $3$ modulo $4$. 

        \item On raisonne par l'absurde : on suppose que l'ensemble des nombres premiers congrus à $3$ modulo $4$ est fini. On note cet ensemble $\mathbb{P}_3$. 
        $$\mathbb{P}_3 = \left\{ p_1, p_2, \ldots, p_n \right | n \in \mathbb{N} \}$$
        Soit $n = |\mathbb{P}_3|$. 
        On remarque que pour tout $k \in \llbracket 1, n \rrbracket$, 
        $$4 \times \prod_{k=1}^{n} p_k - 1 \equiv 3 \pmod{4}$$
        n'est pas divisible par $p_k$ et n'appartient pas à $\mathbb{P}_3$. \\
        Or cette quantité possède forcément au moins un diviseur premier congru à $3$ modulo $4$ (cf. $\text{(a)}$). \\
        Absurde. \\
        Donc il existe une infinité de nombres premiers congrus à $3$ modulo $4$. 
    \end{enumerate}

    \item De la même manière, on montre que tout entier congru à $5$ modulo $6$ possède au moins un diviseur congru à $5$ modulo $6$. \\
    De la même manière, on raisonne par l'absurde en supposant que l'ensemle des nombres premiers congrus à $5$ modulo $6$ noté $\mathbb{P}_5$ est fini. \\
    On remarque qu'il existe un facteur premier $p \equiv 5 \pmod{6} \not \in \mathbb{P}_5$ qui divise $6 \prod\limits_{k = 1}^{|\mathbb{P}_5|} p_k - 1$. \\
    Absurde. \\ 
    Donc il existe une infinité de nombres premiers congrus à $5$ modulos $6$. 
\end{enumerate}

\setcounter{section}{10}
\section{Exercice}

\begin{enumerate}
    \item Soit $p \in \mathbb{P}$, $y \in \llbracket 1, p-1 \rrbracket$. \\
    \begin{align*}
        \exists ! x \in \llbracket 1, p-1 \rrbracket, xy \equiv 1 \pmod{p} \\
    \end{align*}
\end{enumerate}


\setcounter{section}{11}
\section{Exercice}
Soit $(a,b,c) \in \mathbb{N}^3$. \\
On suppose que $b \wedge c = 1$. \\
Donc : 
\begin{align*}
    \forall p \in \mathbb{P}, v_p(a \wedge (bc)) &= \min(v_p(a), v_p(bc)) \\
    &= \min(v_p(a), v_p(b) + v_p(c)) \\
    &= \min(v_p(a), v_p(b), v_p(c)) \text{ (car $b$ et $c$ sont premiers entre eux)}\\
    &= v_p(a \wedge b, a \wedge c)
\end{align*}
Donc $\boxed{a \wedge (bc) = a \wedge b, a \wedge c}$.

\section{Exercice}
Soit $(a,b) \in (\mathbb{Z}^*)^2$. \\
On suppose que $a^2 | b^2$. \\
On a : 
\begin{align*}
    \forall p \in \mathbb{P}, v_p(a^2) \leq v_p(b^2) &\text{ donc } \forall p \in \mathbb{P}, 2v_p(a) \leq 2v_p(b) \\
    &\text{ donc } \forall p \in \mathbb{P}, v_p(a) \leq v_p(b) \\
    &\text{ donc } a | b
\end{align*}

\section{Exercice}
Soit $(a,b) \in (\mathbb{N}^*)^2$. \\
\begin{align*}
    (a \wedge b)^n = a^n \wedge b^n &\Leftrightarrow \forall p \in \mathbb{P}, v_p((a \wedge b)^n) = v_p(a^n \wedge b^n) \\
    &\Leftrightarrow \forall p \in \mathbb{P}, n \times v_p(a \wedge b) = \min(v_p(a)^n, v_p(b)^n) \\
    &\Leftrightarrow \forall p \in \mathbb{P},n \times \min(v_p(a), v_p(b)) = \min(n \times v_p(a), n \times v_p(b)) \\
\end{align*}

\section{Exercice}
\begin{enumerate}
    \item Soit $(a,b) \in (\mathbb{N}^*)^2$ et $k \geq 2$ entier. \\
    On suppose que $a \wedge b = 1$ et que $\exists \lambda \in \mathbb{N}, ab = \lambda^k$. \\
    On note $\lambda$ l'entier tel que $ab = \lambda^k$. \\
    \begin{align*}
        ab = \lambda^k &\Leftrightarrow v_{\lambda}(ab) = v_\lambda(\lambda^k) \\
        &\Leftrightarrow v_\lambda(a) + v_\lambda(b) = k
    \end{align*}
    Or si $\lambda \neq 1$ : $$v_\lambda(a \wedge b) = 0 = \min(v_\lambda(a), v_\lambda(b))$$
    Donc, par disjonction de cas : 
    \begin{itemize}
        \item si $\lambda = 1$, alors $ab = 1^k$ et on a bien $a = 1^k$ et $b = 1^k$.
        \item si $\lambda \neq 1$, alors : 
        \begin{align*}
            v_\lambda(a) + v_\lambda(b) = k &\Leftrightarrow \max(v_\lambda(a), v_\lambda(b)) = k \\
            &\Leftrightarrow \max(a, b) = \lambda^k \text{ et } \min(a,b) = 1^k \text{ ($\forall d \neq \lambda \in \mathbb{P}, d \not | \lambda$, $v_d(ab) = 0$)}
        \end{align*}
    \end{itemize}

    \item Le résultat ne persiste pas pour $(a,b) \in \mathbb{Z}^2$ :  \\
    On choisit $a$ et $b$ négatifs tels que $a \wedge b = 1$ et $ab = \lambda^k$. \\
    Ainsi, il n'existe pas de $n \in \mathbb{N}$ tel que $a = n^k$ (car $a$ est négatif).

\end{enumerate}

\section{Exercice}
\begin{enumerate}
    \item \underline{Première méthode} \\
    Soit $p \in \mathbb{P}$ et $n \in \mathbb{N}$. \\
    On note $A_k = \left\{ q \in \llbracket 1, n \rrbracket, p^q | k \right\}$ et $a_k = |A_k|$. \\
    On note $V_l = \left\{ k \in \llbracket 1, n \rrbracket, v_p(k) = l \right\}$ et on note $m_l = |V_l|$. \\
    Lien entre $V_k, A_k$ : \\
    $$V_k = A_k \backslash A_{k+1}$$
    Par ailleurs, $A_{k+1} \subset A_k$. 
    $$m_k = a_k - a_{k+1}$$
    On a : 
    \begin{align*}
        v_p(n!) &= \sum_{k=1}^{n} v_p(k) \\
        &= \sum_{l \geq 0} l \times m_l \text{ (définition de $V_l$)} \\
        &= \sum_{l \geq 0} l(a_l - a_{l+1}) \\
        &= \sum_{l \geq 0} la_l - \sum_{l \geq 0} la_{l+1} \\
        &= \sum_{l \geq 1} la_l - \sum_{l \geq 1} (l-1)a_l \\
        &= \sum_{l \geq 1} a_l
    \end{align*}
    On explicite le cardinal de $A_l$. \\
    Déterminer le nombre de $k, 1 \leq kp^l \leq n$. \\
    Soit :
    $$\frac{1}{p^l} \leq k \leq \frac{n}{p^l}$$
    Soit : 
    $$1 \leq k \leq \frac{n}{p^l}$$
    Il y en a $\lfloor \frac{n}{p^l} \rfloor$. \\ \\

    \noindent \underline{Deuxième méthode} \\
    On peut montrer que : 
    $$\left\lfloor \frac{\left\lfloor \frac{a}{b} \right\rfloor}{c} \right\rfloor = \left\lfloor \frac{a}{bc} \right\rfloor$$
    En particulier : 
    $$\left\lfloor \frac{\left\lfloor \frac{n}{p^k} \right\rfloor}{p} \right\rfloor = \left\lfloor \frac{n}{p^{k+1}} \right\rfloor$$
    On raisonne par récurrence forte. \\
    \underline{Initialisation :} \\
    On vérifie que ça fonctionne pour $n=0$. \\ \\
    \noindent \underline{Hérédité :}
    \begin{align*}
        v_p((n+1)!) &= v_p(n+1) + v_p(n!) \\
        &= v_p(n+1) + \sum_{k=1}^{+\infty} \left\lfloor \frac{n}{p^k} \right\rfloor \\
        &= \sum_{1 \leq i \leq n+1} v_p(i) \\
        &= \sum_{1 \leq i \leq n+1, p|i} v_p(i) \text{ (si $p \not | i$, alors $v_p(i) = 0$)} \\
        &= \sum_{1 \leq kp \leq n+1} v_p(p \times k) \\
        &= \sum_{1 \leq k \leq \frac{n+1}{p}} v_p(p \times k) \\
        &= \sum_{1 \leq k \leq \frac{n+1}{p}} (v_p(k) + 1) \\
        &= \left\lfloor \frac{n+1}{p} \right\rfloor + \sum_{1 \leq k \leq \frac{n+1}{p}} v_p(k) \\
        &= \left\lfloor \frac{n+1}{p} \right\rfloor + v_p \left( \left\lfloor \frac{n+1}{p}! \right\rfloor \right) \\
        &= \left\lfloor \frac{n+1}{p} \right\rfloor + \sum_{k \geq 1} \left\lfloor \frac{\left\lfloor \frac{n}{p} \right\rfloor}{p^k} \right\rfloor \\
        &= \left\lfloor \frac{n+1}{p} \right\rfloor + \sum_{k \geq 1} \left\lfloor \frac{n+1}{p^{k+1}} \right\rfloor \\
        &= \sum_{k \geq 1} \left\lfloor \frac{n+1}{p^k} \right\rfloor
    \end{align*}

    \item \begin{align*}
        v_2(100!) &= \sum_{j \geq 1} \left\lfloor \frac{100}{2^k} \right\rfloor \\
        &=  \left\lfloor \frac{100}{2} \right\rfloor + \left\lfloor \frac{100}{4} \right\rfloor + \left\lfloor \frac{100}{8} \right\rfloor + \left\lfloor \frac{100}{16} \right\rfloor + \left\lfloor \frac{100}{32} \right\rfloor + \left\lfloor \frac{100}{64} \right\rfloor \\
        &= 50 + 25 + 12 + 6 + 3 + 1 \\
        &= 97 \\
        v_5(100)! &= \left\lfloor \frac{100}{5} \right\rfloor + \left\lfloor \frac{100}{25} \right\rfloor \\
        &= 20 + 4 \\
        &= 24
    \end{align*}
    $100!$ s'achève donc par $\min(v_2(100!), v_5(100!)) = 24$. 
\end{enumerate}

\setcounter{chapter}{13}
\setcounter{section}{0}
\section{Exercice}

Soit $P = \sum\limits_{k=0}^{+\infty} a_k X^k$ un polynôme de $\mathbb{K}[X]$. \\

\noindent On suppose que $P^{-1}$ existe. Alors : 
\begin{align*}
    PP^{-1} = 1 &\Leftrightarrow \deg(PP^{-1}) = 0 \\
    &\Leftrightarrow \deg{P} + \deg{P^{-1}} = 0  \text{ ($\mathbb{K}$ est un corps, donc est intègre)}\\
    &\Leftrightarrow \deg{P} = 0
\end{align*}

\noindent Ainsi, Comme $\mathbb{K}$ est un corps, on a : 
$$\forall P \neq 0 \in \mathbb{K}[X], \deg{P} = 0, P \in U(\mathbb{K}[X])$$


\section{Exercice}
\begin{align*}
    P_n &= (1 + X + X^2 + \dots + X^n)^2 \\
    &= \left( \sum_{k=0}^{n} X^k \right)^2 \\
    &= \left( \frac{1 - X^{n+1}}{1 - X} \right)^2 \\
    &= \frac{X^{2(n+1)} - 2X^{n+1} + 1}{X^2 - 2X + 1}
\end{align*}

\begin{align*}
    Q_n &= (1+X) \times (1+X^2) \times (1+X^4) \times \dots \times (1+X^{2^n}) \\
    &= \prod_{k=0}^{n} (1 + X^{2^k})
\end{align*}

\setcounter{chapter}{14}
\setcounter{section}{5}
\section{Exercice}

\begin{enumerate}
    \item Soit $n \in \mathbb{N}$. Lorsque $a = 1$, on a la relation $u_{n+1} = u_n + b$. 
    Ainsi, $(u_n)$ est une suite arithmétique d'expression : 
    $$u_n = u_0 + nb$$

    \item \begin{enumerate}
        \item Pour $n \in \mathbb{N}$ : 
        \begin{align*}
            v_n &= u_n + \lambda \\
            \text{donc } v_{n+1} &= u_{n+1} + \lambda \\
            &= a u_{n} + b + \lambda
        \end{align*}
        On remarque que pour $\lambda = \frac{b}{a-1}$ avec $a \neq 1$, on a :
        \begin{align*}
            v_{n+1} &= a u_n + b + \frac{b}{a - 1} \\
            &= \frac{(a-1) (a) u_n + (a-1)b + b}{a - 1} \\
            &= a \times \frac{(a-1) u_n + b}{a-1} \\
            &= a \left( u_n + \frac{b}{a - 1} \right) \\
            &= a v_n
        \end{align*}
        Donc pour $\lambda = \frac{b}{a-1}$, $(v_n)$ est géométrique. 

        \item \begin{align*}
            v_n &= u_n + \frac{b}{a-1} \\
            \text{donc } u_n &= v_n - \frac{b}{a-1} \\
            &= \left( u_0 + \frac{b}{a-1} \right) \times a^n - \frac{b}{a-1} \\
            &= u_0 a^n + (a^n - 1) \frac{b}{a-1}
        \end{align*}
    \end{enumerate}
\end{enumerate}

\setcounter{section}{1}
\section{Exercice}
\begin{enumerate}
    \item Soit $k \geq 2 \in \mathbb{N}$. \\
    \begin{align*}
        \frac{1}{k^2} \leq \frac{1}{k-1} - \frac{1}{k} &\Leftrightarrow \frac{1}{k^2} \leq \frac{1}{k(k-1)} \\
        &\Leftrightarrow \frac{1}{k} \leq \frac{1}{k-1} \\
        &\Leftrightarrow k \geq k-1 \text{ ($x \mapsto \frac{1}{x}$ est décroissante)}
    \end{align*}

    \item On suppose que $S_n$ converge. Ainsi, pour $n \in \mathbb{N}$, on a d'une part : 
    \begin{align*}
        \frac{1}{k^2} \leq \frac{1}{k-1} - \frac{1}{k}
        &\Leftrightarrow S_n \leq 1 + \sum_{k=2}^{n} \left( \frac{1}{k-1} - \frac{1}{k} \right) \\
        &\Leftrightarrow S_n \leq 1 + 1 - \frac{1}{n} \text{ (télescopage)} \\
        &\Leftrightarrow \lim_{n \to +\infty} S_n \leq \lim_{n \to +\infty} 2 + \frac{1}{n} \text{ (Hypothèse)}\\
        &\Leftrightarrow \lim_{n \to +\infty} S_n \leq 2
    \end{align*}
    D'autre part, la suite est strictement croissante, donc d'après le théorème de la limite monotone, $(u_n)$ converge. 
\end{enumerate}


\end{document}